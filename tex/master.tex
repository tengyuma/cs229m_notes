%% filename: amsbook-template.tex
%% version: 1.1
%% date: 2014/07/24
%%
%% American Mathematical Society
%% Technical Support
%% Publications Technical Group
%% 201 Charles Street
%% Providence, RI 02904
%% USA
%% tel: (401) 455-4080
%%      (800) 321-4267 (USA and Canada only)
%% fax: (401) 331-3842
%% email: tech-support@ams.org
%% 
%% Copyright 2006, 2008-2010, 2014 American Mathematical Society.
%% 
%% This work may be distributed and/or modified under the
%% conditions of the LaTeX Project Public License, either version 1.3c
%% of this license or (at your option) any later version.
%% The latest version of this license is in
%%   http://www.latex-project.org/lppl.txt
%% and version 1.3c or later is part of all distributions of LaTeX
%% version 2005/12/01 or later.
%% 
%% This work has the LPPL maintenance status `maintained'.
%% 
%% The Current Maintainer of this work is the American Mathematical
%% Society.
%%
%% ====================================================================

%    AMS-LaTeX v.2 driver file template for use with amsbook
%
%    Remove any commented or uncommented macros you do not use.

\documentclass[oneside, openany]{book}
\usepackage{amsfonts,bm,amsthm,amsmath,bbm,amssymb,mathtools}
\usepackage{fullpage}
\usepackage{tikz, pgfplots} % added for Lecture 2	
\usepackage{caption, subcaption}
\usepackage{float}  % added for Lecture 8
\usepackage[ruled,vlined,linesnumbered]{algorithm2e}  % added for Lecture 15
\usepackage{esdiff}
\usepackage{booktabs}  % added for Lecture 15
\usepackage{hyperref}
\hypersetup{linktocpage}
\usepackage{natbib}
\renewcommand{\cite}[1]{\citep{#1}}



\newtheorem{theorem}{Theorem}[chapter]
\newtheorem{lemma}[theorem]{Lemma}

\theoremstyle{definition}
\newtheorem{definition}[theorem]{Definition}
\newtheorem{example}[theorem]{Example}
\newtheorem{xca}[theorem]{Exercise}
\newtheorem{corollary}[theorem]{Corollary}  % added for Lecture 5
\newtheorem{proposition}{Proposition}[section]  % added for Lecture 6

\theoremstyle{remark}
\newtheorem{remark}[theorem]{Remark}

\numberwithin{section}{chapter}
\numberwithin{equation}{chapter}

%    For a single index; for multiple indexes, see the manual
%    "Instructions for preparation of papers and monographs:
%    AMS-LaTeX" (instr-l.pdf in the AMS-LaTeX distribution).
\makeindex
\def\lectureformat{0}
\usepackage{color}
\usepackage{lipsum}

\DeclareMathOperator*{\Exp}{\mathbb{E}}
\DeclareMathOperator*{\argmin}{\textup{argmin}}
\DeclareMathOperator*{\argmax}{\textup{argmax}}
\newcommand{\E}{\mathbb{E}}

\newcommand{\err}{\ell_{\textup{0-1}}}
\newcommand{\thetaerm}{\theta_{\textup{ERM}}}
\newcommand{\hatL}{\widehat{L}}
\newcommand{\tilO}{\widetilde{O}}
\newcommand{\iid}{\overset{\textup{iid}}{\sim}}

\newcommand{\norm}[1]{\|#1\|}
\newcommand{\Norm}[1]{\left\|#1\right\|}


\newcommand{\al}[1]{
	\begin{align}
	#1
	\end{align}
}


\renewcommand{\sp}[1]{^{(#1)}}

\newcommand{\cA}{\mathcal A}
\newcommand{\cB}{\mathcal B}
\newcommand{\cC}{\mathcal C}
\newcommand{\cD}{\mathcal D}
\newcommand{\cE}{\mathcal E}
\newcommand{\cF}{\mathcal F}
\newcommand{\cG}{\mathcal G}
\newcommand{\cH}{\mathcal H}
\newcommand{\cI}{\mathcal I}
\newcommand{\cJ}{\mathcal J}
\newcommand{\cK}{\mathcal K}
\newcommand{\cL}{\mathcal L}
\newcommand{\cM}{\mathcal M}
\newcommand{\cN}{\mathcal N}
\newcommand{\cO}{\mathcal O}
\newcommand{\cP}{\mathcal P}
\newcommand{\cQ}{\mathcal Q}
\newcommand{\cR}{\mathcal R}
\newcommand{\cS}{\mathcal S}
\newcommand{\cT}{\mathcal T}
\newcommand{\cU}{\mathcal U}
\newcommand{\cV}{\mathcal V}
\newcommand{\cW}{\mathcal W}
\newcommand{\cX}{\mathcal X}
\newcommand{\cY}{\mathcal Y}
\newcommand{\cZ}{\mathcal Z}

\newcommand{\bbB}{\mathbb B}
\newcommand{\bbS}{\mathbb S}
\newcommand{\bbR}{\mathbb R}
\newcommand{\bbZ}{\mathbb Z}
\newcommand{\bbI}{\mathbb I}
\newcommand{\bbQ}{\mathbb Q}
\newcommand{\bbP}{\mathbb P}
\newcommand{\bbE}{\mathbb E}
\newcommand{\bbN}{\mathbb N}

\newcommand{\R}{\bbR}
\pgfplotsset{compat=1.17}
\begin{document}
	
	\frontmatter
	
	\title{Lecture Notes for Machine Learning Theory (CS229M/STATS214)}
	
	%    Remove any unused author tags.
	
	%    author one information
	\author{Instructor: Tengyu Ma
	}
	%\address{}
	%\curraddr{}
	%\email{}
	%\thanks{}
	
	%    author two information
	%\author{}
	%\address{}
	%\curraddr{}
	%\email{}
	%\thanks{}
	
	%\subjclass[2010]{Primary }
	
	%\keywords{}
	
	%\date{}
	
	%\begin{abstract}
	%\end{abstract}
	
	
	\maketitle
	
	%    Dedication.  If the dedication is longer than a line or two,
	%    remove the centering instructions and the line break.
	%\cleardoublepage
	%\thispagestyle{empty}
	%\vspace*{13.5pc}
	%\begin{center}
	%  Dedication text (use \\[2pt] for line break if necessary)
	%\end{center}
	%\cleardoublepage
	
	%    Change page number to 6 if a dedication is present.
	%\setcounter{page}{1}
	
	\tableofcontents
	
	%    Include unnumbered chapters (preface, acknowledgments, etc.) here.
	%\include{}
	\mainmatter
	\let\sec\section
	\let\subsec\subsection
	\let\subsubsec\subsubsection
	
	\chapter*{Acknowledgments}
	\setcounter{page}{5}
	This monograph is a collection of scribe notes for the course CS229M/STATS214 at Stanford University. The materials in Chapter \ref{chap:supervised}--\ref{chap:gen-bounds} are mostly based on Percy Liang's lecture notes~\cite{percynotes}, and Chapter~\ref{chap:OL} is largely based on Haipeng Luo's lectures~\cite{haipengnotes}. Kenneth Tay contributed significantly to the revision of these notes as a teaching assistant for the course. The original contributor to the scribe notes are Stanford students including but not limited to Anusri Pampari, Gabriel Poesia, Alexander Ke, Trenton Chang, Brad Ross, Robbie Jones, Yizhou Qian, Will Song, Daniel Do, Spencer M. Richards, Thomas Lew, David Lin, Jinhui Wang, Rafael Rafailov, Aidan Perreault, Kevin Han, Han Wu, Andrew Wang, Rohan Taori, Jonathan Lee, Rohith Kuditipudi, Kefan Dong, Roshni Sahoo, Sarah Wu, Tianyu Du, Xin Lu, Soham Sinha, Kevin Guo, Jeff Z. HaoChen, Carrie Wu, Kaidi Cao, and Ruocheng Wang.  The notes will be updated every year with new materials. The reference list is far from complete.
	
	
	\chapter{Supervised Learning Formulations}\label{chap:supervised}
	% reset section counter

\setcounter{section}{0}

\metadata{1}{Anusri Pampari and Gabriel Poesia}{Jan 11th, 2021}


In this chapter, we will set up the standard theoretical formulation of supervised learning and introduce the \textit{empirical risk minimization} (ERM) paradigm. The setup will apply to almost the entire monograph \tnotelong{to be updated}and the ERM paradigm will be the main focus of Chapter~\ref{chap:asymp}, \ref{chap:conc}, and \ref{chap:uc}. 

\sec{Supervised learning}\label{lec1:sec:sup-learn}
In supervised learning, we have a dataset where each data point is associated with a label, and we aim to learn from the data a function that maps data points to their labels. The learned  function can be used to infer the labels of test data points. More formally, suppose the data points, also called inputs,  belong to some input space $\cX$ (e.g. images of birds), and labels belong to the output space $\cY$ (e.g. bird species). Suppose we are interested in a specific joint probability distribution $P$ over $\cX \times \cY$ (e.g. images of birds in North America), from which we draw a \emph{training set}, i.e we draw a a set of $n$ independent and identically distributed (i.i.d.) data points $\{(x^{(i)}, y^{(i)}\}_{i=1}^n$ from $P$. The goal of supervised learning is to learn a mapping (i.e. a function) from $\cX$ to $\cY$ using the training data. Any such function $h : \cX \rightarrow \cY$ is called a \emph{predictor} (also \emph{hypothesis} or \emph{model}).

Given two predictors, how do we decide which is better? For that, we define a \emph{loss function} over the predictions. There are several ways to define loss functions: for now, define a loss function $\ell$ as a function $\ell : \cY \times \cY \rightarrow \R$. Intuitively, the loss function takes two labels, the prediction made by a model $\hat{y}$ and the true label $y$, and gives a number that captures how different the two labels are. We assume $\ell$ is non-negative, i.e $\ell(\hat{y}, y) \geq 0$. Then, the loss of a model $h$ on an example $(x, y)$ is $\ell(h(x), y)$, i.e. the difference (as measured by $\ell$) between the prediction made by $h$ and the true label.


With these definitions, we are able to formalize the problem of supervised learning. Precisely, we seek to find a model $h$ that minimizes what we call the expected loss (or population loss or expected risk or population risk):
\al{
    L(h) \defeq{} \Exp_{(x, y)\sim p} [\ell(h(x), y)].
}


Note that $L$ is nonnegative because $\ell$ is nonnegative. Typically, the loss function is designed so that the best possible loss is zero when $\hat{y}$ matches $y$ exactly. Therefore, the goal is to find $h$ such that $L(h)$ is as close to zero as possible. % The best possible $h$ would have , we would find an $h$ with expected loss $0$, since that's the best possible we can do.

\paragraph{Examples: regression and classification problems.}

Here are two standard types of supervised learning problems based on the properties of the output space:

\begin{itemize}
    \item In the problem of \emph{regression}, predictions are real numbers ($\cY = \R$). We would like predictions to be as close as possible to the real labels. A classical loss function that captures this is the squared error, $\ell(\hat{y}, y) = (\hat{y} - y)^2$.
    \item In the problem of \emph{classification}, predictions are in a discrete set of $k$ unordered classes $\cY = [k] = \{1, \cdots, k \}$. One possible classification loss is the $0-1$ loss: $\ell(\hat{y}, y) = \mathbbm{1}(\hat{y} \neq y)$, i.e. $0$ if the prediction is equal to the true label, and $1$ otherwise.
\end{itemize}

\paragraph{Hypothesis class.}

So far, we said we would like to find \emph{any function} that minimizes population risk. However, in practice, we do not have a way of optimizing over arbitrary functions. Instead, we work within a more constrained set of functions $\cH$, which we call the \emph{hypothesis family} (or \emph{hypothesis class}). Each element of $\cH$ is a function $h : \cX \rightarrow \cY$. Usually, we choose a set $\cH$ that we know how to optimize over (e.g. linear functions, or neural networks).

Given one particular function $h \in \cH$, we define the \emph{excess risk} of $h$ with respect to $\cH$ as the difference between the population risk of $h$ and the best possible population risk inside $\cH$:

$$E(h) \defeq{} L(h) - \inf_{g\in\cH} L(g).$$

Generally we need more assumptions about a specific problem and hypothesis class to bound absolute population risk, hence we focus on bounding the excess risk.

Usually, the family we choose to work with can be parameterized by a vector of parameters $\theta \in \Theta$. In that case, we can refer to an element of $\cH$ by $h_\theta$, making that explicit. An example of such a parametrization of the hypothesis class is $\cH = \{ h: h_\theta(x) = \theta^\top x, \theta \in \mathbb{R}^d \}$.

\sec{Empirical risk minimization}

Our ultimate goal is to minimize population risk. However, in practice we do not have access to the entire population: we only have a \emph{training set} of $n$ data points, drawn from the same distribution as the entire population. While we cannot compute population risk, we can compute \emph{empirical risk}, the loss over the training set, and try to minimize that. This is, in short, the paradigm known as \emph{empirical risk minimization} (ERM): we optimize the training set loss, with the hope that this leads us to a model that has low
population loss. From now on, with some abuse of notation, we often write $\ell(h_\theta(x),y)$ as $\ell((x,y),\theta)$ and use the two notations interchangeably.  Formally, we define the empirical risk of a model $h$ as:
\al{
\hatL(h_\theta) \defeq{} \frac{1}{n} \sum_{i=1}^n \ell(h_\theta(x^{(i)}), y^{(i)}) = \frac{1}{n} \sum_{i=1}^n \ell((x^{(i)}, y^{(i)}), \theta).
}
\emph{Empirical risk minimization} is the method of finding the minimizer of $\hatL$, which we call $\hat{\theta}$:
\al{
    \label{lec1:eqn:erm}
    \hat{\theta} \defeq{} \argmin_{\theta\in\Theta} \hatL(h_\theta).
}
Since we are assuming that our training examples are drawn from the same distribution as the whole population, we know that empirical risk and population risk are equal
\emph{in expectation} (over the randomness of the training dataset):
\al{
    \Exp_{(x^{(i)}, y^{(i)}) \iid P}\ \hatL(h_\theta) &= \Exp_{(x^{(i)}, y^{(i)}) \iid P} \frac{1}{n} \sum_{i=1}^n \ell(h_\theta(x^{(i)}), y^{(i)}) \\
    &= \frac{1}{n} \sum_{i=1}^n \Exp_{(x^{(i)}, y^{(i)}) \iid P} \ell(h_\theta(x^{(i)}), y^{(i)}) \\
    &= \frac{1}{n} \cdot{} n \cdot{} \Exp_{(x^{(i)}, y^{(i)}) \iid P} \ell(h_\theta(x^{(i)}), y^{(i)}) \\
    &= L(h_\theta).
}


This is one reason why it makes sense to use empirical risk: it is an unbiased estimator of the population risk.

The key question that we seek to answer in the first part of this course is: \textbf{what guarantees do we have on the excess risk for the parameters learned by ERM?} The hope with ERM is that minimizing the training error will lead to small testing error. One way to make this rigorous is by showing that the ERM minimizer's excess risk is bounded.

	
	\chapter{Asymptotic Analysis}\label{chap:asymp}
	% reset section counter
\setcounter{section}{0}

%\metadata{lecture ID}{Your names}{date}
\metadata{2}{Alexander Ke and Trenton Chang}{Jan 13th, 2021}

This chapter briefly reviews the classical asymptotic results and techniques, which assume the number of training examples $n$ goes to infinity while all other (hyper-)parameters are treated as constants. It is not clear whether these results and techniques are still applicable to modern machine learning, where the number of parameters can often be larger than the number of examples. None of the later chapters depend on this chapter, so readers who are only interested in deep learning can feel free to skip it.  

Nevertheless, I still include these results in this text because the key ideas of using Taylor expansion and law of large numbers are so elegant and foundational that most machine learning theorists should probably be aware of them. 
The rest of the monograph will work in the non-asymptotic setting and one of the key technical difference will be that Taylor expansion cannot be applied and that the concentration inequalities need to be strengthened. 
Due to limited space, we will focus on the core ideas at the expense of some mathematical rigor. The conceptual ideas and techniques of this chapter is self-contained, whereas a fully rigorous treatment of these concepts require additional mathematical language that is beyond the scope of this monograph. We refer the readers to~\citep{vaart_1998} for more in-depth presentation of this line of work. 




%In this chapter, we use an asymptotic approach (i.e. assuming number of training samples $n \to \infty$) to achieve a bound on the ERM. 
%We then instantiate these results to the case where the loss function is the maximum likelihood  and discuss the limitations of asymptotics. 
%(In future chapters we will assume finite $n$ and provide a non-asymptotic analysis.)

\sec{Asymptotics of empirical risk minimization}

The goal of the asymptotic analysis of ERM is to bound from above the excess risk in the following form:
\al{
    L(\hat{\theta}) - \inf_{\theta \in \Theta} L(\theta) \leq \frac{c}{n} + o\left(\frac{1}{n}\right)\,. 
    \label{lec1:eqn:erm-bound}
}
Here $c$ is a problem-dependent constant that is independent of $n$, and the $o(\cdot)$ notation hides all dependencies except for $n$. The equation above shows that as the number of training examples $n$ increases, the excess risk of ERM decreases at the rate of $\frac{1}{n}$.

Let $\{(x^{(1)},y^{(1)}), \cdots, (x^{(n)},y^{(n)})\}$ be the training data and let $\cH = \{ h_\theta: \theta \in \R^p \}$ be the parameterized family of hypothesis functions. Let $\hat{\theta}$ be the empirical risk minimizer as defined in~\Cref{lec1:eqn:erm:theta}. Let $\theta^{*}$ be the minimizer of the population risk $L$ (i.e., $\theta^{*} = \argmin_\theta L(\theta)$, )which is assumed to be unique. The theorem below characterize the excess risk $L(\hat{\theta}) - L(\theta^{*})$, which is a random variable (where the randomness comes from the training dataset). 

\begin{theorem}[Informally stated] \label{lec1:thm:asymp}
Assume that (a) the consistency of the ERM estimator $\hat{\theta}$ in the sense that $\hat{\theta}  \overset{p}{\to} \theta^{*}$ as $n \to \infty$, (b) the Hessian of the population loss at $\theta^*$, $\nabla^{2}L(\theta^{*})$, is full rank, and  (c) other appropriate regularity conditions hold.\footnote{$X_n \overset{p}{\to} X$ implies that for all $\epsilon > 0$, $\bbP \left (\norm{X_n - X} > \epsilon \right ) \to 0$, while $X_n \overset{d}{\to} X$ implies that $\bbP(X_n \leq t) \to \bbP(X \leq t)$ at all points $t$ for which $\bbP(X \leq t)$ is continuous. These two notions of convergence are known as convergence in probability and convergence in distribution, respectively. These concepts are not essential to this course, but additional information can be found by reading the Wikipedia \href{https://en.wikipedia.org/wiki/Convergence_of_random_variables}{article} on convergence of random variables.} 
Then,
\begin{enumerate}
    \item The estimated model $\hat{\theta}$ converges to the $\theta^*$ with a $1/\sqrt{n}$ rate in the sense that $\sqrt{n} (\hat{\theta} - \theta^{*}) = O_P(1)$, that is, for every $\epsilon > 0$, there is an $M$ such that $\forall n, \bbP (\| \sqrt{n} (\hat{\theta} - \theta^{*}) \|_2 > M) \le \epsilon$.\footnote{In other words, the sequence of random variables $\{ \sqrt{n} (\hat{\theta} - \theta^{*}) \}$ indexed by $n$ is ``bounded in probability" . } 
    \item  The scaled parameter error $\sqrt{n}(\hat{\theta}-\theta^{*})$ has asymptotic normality distribution in the sense that as $n\rightarrow \infty$, 
    \al{\sqrt{n}(\hat{\theta}-\theta^{*}) \overset{d}{\to} \mathcal{N} \left(0, (\nabla^{2}L(\theta^{*}))^{-1}\Cov(\nabla \ell((x,y), \theta^*)) (\nabla^{2}L(\theta^{*}))^{-1} \right) \,.}
     \item The excess loss decays with an $1/n$ rate
    \al{n (L(\hat{\theta}) - L(\theta^{*})) = O_P(1)\,.
    	\label{eqn:3}
    }
     Moreover, 
     \begin{align}
     \lim_{n \to \infty} \Exp \left[ n (L(\hat \theta) - L(\theta^*)) \right] = \frac12 \tr\left( \nabla^2 L(\theta^*)^{-1} \Cov(\nabla \ell ((x, y), \theta^*) \right)\,.\label{eqn:4}
     \end{align}
    \item The scaled excess loss has a $\chi^2$ distribution: 
    \begin{align}
    n (L(\hat{\theta}) - L(\theta^{*})) \overset{d}{\to} \frac{1}{2} ||S||_{2}^{2} \,.
    \end{align}
    where $S \sim \mathcal{N} \left(0, (\nabla^{2}L(\theta^{*}))^{-1/2}\Cov(\nabla \ell((x,y), \theta^*)) (\nabla^{2}L(\theta^{*}))^{-1/2}\right)$.
\end{enumerate}
\end{theorem}
%\textbf{Remark:} In the theorem above, Parts 1 and 3 only show the rate or order of convergence, while Parts 2 and 4 define the limiting distribution for the random variables.
%is a powerful conclusion because once we know that $\sqrt{n}(\hat \theta  - \theta^*)$ is (asymptotically) Gaussian, we can easily work out the distribution of the excess risk. 
%If we believe in our assumptions and $n$ is large enough such that we can assume $n \to \infty$, this allows us to analytically determine quantities of interest in almost any scenario (for example, if our test distribution changes). 

The key takeaway is that our parameter error $\hat{\theta} - \theta^*$ decreases on the order of $1/\sqrt{n}$ and the excess risk decreases on the order of $1/n$. We note that the twice-differentiable loss function $L$ plays a crucial role in the rate here---e.g., if the loss for a regression problem is $|\hat{y}-y|$ instead of the squared loss, the excess rate will not have a $1/n$ rate. 

Theorem \ref{lec1:thm:asymp} is powerful because the characterization of the distribution of  $\sqrt{n}(\hat \theta  - \theta^*)$ can lead us to almost any property of $\hat{\theta}$ of interest. For example, in principle, one can characterize the expected loss of the estimator $\hat{\theta}$ on a test distribution different from the distribution $P$. On the other hand, the theorem requires taking the limit as $n$ goes to infinity while ignoring the dependencies on other problem parameters (such as dimension). This significantly limits the applicability of the theorem in modern machine learning settings (see Section~\ref{sec:limit-asymp} for more discussion). 
%While we will not discuss the regularity assumptions in Theorem~\ref{lec1:thm:asymp} in great detail, we note that the assumption that $L$ is twice differentiable is crucial. 

\subsec{Key proof ideas} 

We will sketch a proof Theorem \ref{lec1:thm:asymp} by applying the following two key ideas. 
\begin{enumerate}
    \item[1] \textbf{Taylor expansion.} We will derive a formula for $\hat{\theta}$ and the excess risk by Taylor-expanding the empirical loss $\nabla \hatL(\theta)$ (and its gradient )around the reference point $\theta^{*}$.
    \item[2] \textbf{Law of large number and central limit theorem.} We will simplify and characterize various empirical quantities, e.g.,  $\hatL(\theta)$ and $\hatL^2(\theta)$ by the law of large numbers and central limit theorems. For exmaple, we will use the facts that $\hatL(\theta) \overset{p}{\to} L(\theta)$, $\nabla\hatL(\theta) \overset{p}{\to} \nabla L(\theta)$   and  $\nabla^{2}\hatL(\theta) \overset{p}{\to} \nabla^{2} L(\theta)$ as $n \to \infty$, and that $\nabla \hatL(\theta)-\nabla L(\theta)$ is asymptotically normal. 
\end{enumerate}
 
To prepare us with more detail, we first state the the central limit theorem for a sum of i.i.d. random variables.

\begin{theorem}[Central Limit Theorem] \label{lec1:thm:CLT}
Let $X_1, \cdots, X_n$ be i.i.d. $d$-dimensional random variables and $\widehat{X}=\frac{1}{n} \sum_{i=1}^{n} X_i$ and the covariance matrix $\Sigma = \Exp[X_iX_i^\top]\in \R^{d\times d}$ is finite. Then, as $n \to \infty$, we have
\begin{enumerate}
    \item $\widehat{X} \overset{p}{\to} \Exp[X]$, and
    \item $\sqrt{n} (\widehat{X}-\Exp[X]) \overset{d}{\to} \mathcal{N}(0,\Sigma)$. %In particular, $\sqrt{n} (\widehat{X}-\Exp[X]) = O_P(1)$.
\end{enumerate}
\end{theorem}

We also state a lemma that asserts the linear transformation of a Gaussian random variable is still Gaussian and computes its covariance. %The second part claims that quadratic form of Gaussian random variable has a $\chi^2$ distribution.  
\begin{lemma}\label{lec1:lem:dist}
%\quad\quad
%    \begin{enumerate}
If $Z \sim N(0, \Sigma)$ and $A$ is a deterministic matrix, then $AZ \sim N(0, A \Sigma A^\top)$. \tnote{todo second part of the old lemma}
%\subsec{Main proof}        
%        \item If $Z \sim N(0, \Sigma^{-1})$ and $Z \in \bbR^p$, then $Z^\top \Sigma Z \sim \chi^2(p)$, where $\sim \chi^2(p)$ is the chi-squared distribution with $p$ degrees of freedom.
%    \end{enumerate}
\end{lemma}
With these preparations, we will sketch a proof for parts 1 and 2 of Theorem~\ref{lec1:thm:asymp}. 

By definition, the gradient of the empirical risk evaluated at the empirical risk minimizer, that is, $\nabla \hatL(\hat{\theta})$, is equal to $0$. From the Taylor expansion of $\nabla \hatL$ around $\theta^*$, we have that 
\begin{align}
    0 = \nabla \hatL(\hat{\theta}) = \nabla \hatL(\theta^*) + \nabla^2 \hatL(\theta^*)(\hat{\theta} - \theta^*) + O(\|\hat{\theta} - \theta^*\|^2_2)\perm\text{\footnotemark}
\end{align}

Rearranging the equation above,\footnotetext{Technically, the big-O notation here and in the sequel should be $O_P(\cdot)$ because $\hat{\theta}$ is a random variable. However, we omit this distinction in this proof sketch.} gives a solution for $\hat{\theta}-\theta^*$, 
\al{
 \hat{\theta}-\theta^{*} = -(\nabla^{2}\hatL(\theta^{*}))^{-1} \nabla \hatL(\theta^{*}) + O(||\hat{\theta}-\theta^{*}||_{2}^{2}). \label{lec1:eqn:branch}} 

Multiplying by $\sqrt{n}$ on both sides (so that we deal with quantities on the constant order), we have
 \al{
\sqrt{n} (\hat{\theta}-\theta^{*}) &= -(\nabla^{2}\hatL(\theta^{*}))^{-1} \sqrt{n} (\nabla \hatL(\theta^{*})) + O(\sqrt{n} ||\hat{\theta}-\theta^{*}||_{2}^{2}) \\
&\approx -(\nabla^{2}\hatL(\theta^{*}))^{-1} \sqrt{n} (\nabla \hatL(\theta^{*})). \label{lec1:eqn:interm}}

 
Applying the Central Limit Theorem (Theorem~\ref{lec1:thm:CLT}) using $X_i = \nabla \ell ((x^{(i)}, y^{(i)}), \theta^*)$ and $\widehat{X} = \nabla \hatL(\theta^*)$, and noticing that $\Exp[\nabla \hatL(\theta^{*})] = \nabla L(\theta^{*})$, we have
$\sqrt{n} (\nabla \hatL(\theta^{*}) - \nabla L(\theta^{*})) \overset{d}{\to} \mathcal{N}(0,\Cov(\nabla \ell((x, y), \theta^{*}))).$ 
 
Note that $\nabla L(\theta^{*}) = 0$ because $\theta^{*}$ is the minimizer of  $L$. Thus, we have \al{\sqrt{n} \cdot\nabla \hatL(\theta^{*}) \overset{d}{\to} \mathcal{N}(0,\Cov(\nabla \ell((x, y), \theta^{*})))\perm}
By the law of large numbers, we have $\nabla^2 \hatL(\theta^*) \stackrel{p}{\rightarrow} \nabla^2 L(\theta^*)$. We use these results to replace the empirical quantity in \Cref{lec1:eqn:interm} by the population quantity (technically this is an application of Slutsky's theorem), we have
\al{
\sqrt{n} (\hat{\theta}-\theta^{*}) &\overset{d}{\to} \nabla^{2}L(\theta^{*})^{-1} \mathcal{N}(0,\Cov(\nabla \ell((x,y),\theta^{*}))) \\
&\stackrel{d}{=} \mathcal{N} \left( 0,\nabla^{2}L(\theta^{*})^{-1}\Cov(\nabla \ell((x,y), \theta^{*})) \nabla^{2}L(\theta^{*})^{-1} \right),
}
where the second step is due to Lemma~\ref{lec1:lem:dist}. This proves Part 2 of Theorem~\ref{lec1:thm:asymp}.

Part 1 follows directly from Part 2 by the fact a sequence of random variables that converges to a fixed Gaussian distribution is also bounded in probability.

We now turn to proving Parts 3 and 4 of Theorem~\ref{lec1:thm:asymp} which uses Part 2. A Taylor expansion of $L$ at the reference point $\theta^*$ gives
\begin{equation}
L(\hat \theta) = L(\theta^*) 
+ \langle \nabla L(\theta^*), \hat \theta - \theta^* \rangle 
+ \frac12 \langle \hat \theta - \theta^*, \nabla^2 L(\theta^*) (\hat \theta - \theta^*) \rangle + o(\|\hat \theta - \theta^*\|_2^2).
\end{equation}
Since $\theta^*$ is the minimizer of the population loss $L$, we have $\nabla L(\theta^*) = 0$ and the linear term $\langle \nabla L(\theta^*), \hat \theta - \theta^* \rangle$ is equal to 0. Rearranging and multiplying by $n$, we can write
\begin{align}
n (L(\hat \theta) - L(\theta^*)) &= \frac{n}{2} \langle \hat \theta - \theta^*, \nabla^2 L(\theta^*) (\hat \theta - \theta^*) \rangle + o(\|\hat \theta - \theta^*\|_2^2) \\
&\approx \frac12 \langle \sqrt n(\hat \theta - \theta^*), \nabla^2 L(\theta^*) \sqrt n (\hat \theta - \theta^*) \rangle \\
&= \frac12 \left\|\nabla^2 L(\theta^*)^{1/2} \sqrt n(\hat \theta - \theta^*) \right\|_2^2,
\end{align}
where the last equality follows from the fact that for any vector $v$ and positive semi-definite matrix $A$ of appropriate dimensions, the inner product $\langle v, Av\rangle = v^\top Av = \lVert A^{1/2}v \rVert_2^2$. Let $S = \nabla^2 L(\theta^*)^{1/2} \sqrt n(\hat \theta - \theta^*)$, i.e., the random vector inside the norm. By Part 2, we know the asymptotic distribution of $\sqrt n(\hat \theta - \theta^*)$ is Gaussian, and by Lemma~\ref{lec1:lem:dist} we know $S$ also has a Gaussian distribution: 
\begin{align}
    S %&\sim \nabla^2 L(\theta^*)^{1/2} \cdot \cN \left(0, \nabla^2 L(\theta^*)^{-1} \Cov(\nabla \ell ((x, y), \theta^*)) \nabla^2 L(\theta^*)^{-1} \right) \\
    &\stackrel{d}{=} \cN \left(0, \nabla^2 L(\theta^*)^{-1/2} \Cov(\nabla \ell ((x, y), \theta^*)) \nabla^2 L(\theta^*)^{-1/2} \right).
\end{align}
This proves Part 4, and Part 3, \Cref{eqn:3} follows directly from the definition of the $O_P$ notation. 

Finally, we derive \Cref{eqn:4} by using the fact that the trace operator is invariant under cyclic permutations,  that $\Exp [S] = 0$, and some regularity conditions,
\begin{align}
    \lim_{n \to \infty} \Exp \left[ n (L(\hat \theta) - L(\theta^*)) \right] &= \frac12 \Exp\left[ \|S\|_2^2 \right] = \frac12 \Exp \left[ \tr(S^\top S) \right] \\
    &= \frac12 \Exp \left[ \tr(S S^\top) \right]  = \frac12 \tr \left(\Exp[S S^\top] \right) \\
    &= \frac12 \tr \left( \Cov(S) \right) \\
    &= \frac12 \tr\left( \nabla^2 L(\theta^*)^{-1} \Cov(\nabla \ell ((x, y), \theta^*)) \right).
\end{align}

\subsec{Well-specified case}

Theorem \ref{lec1:thm:asymp} is quite general without any assumptions of a probabilistic model of the data distribution $P$. In many applications, we assume a probabilistic model of our data and use the log-likelihood as the loss function. This is often referred to as the well-specified case in this context, and it allows some simplification of the conclusions of Theorem~\ref{lec1:thm:asymp} with nice interpretations. 


Formally, suppose that we have a family of probability distributions $P_\theta$, parameterized by $\theta \in \Theta$. We assume the data distribution $P$ belongs to this family, that is, there exists $\theta_*$ such that $P = P_{\theta_*}$. This is known as the well-specified case. Theorem \ref{lec1:thm:asymp} implies the following Theorem \ref{lec2:thm:applied} as a direct corollary.

\begin{theorem}
\label{lec2:thm:applied}
    In addition to the assumptions of Theorem~\ref{lec1:thm:asymp}, suppose there exists a parametric model $P(y \mid x; \theta)$, $\theta \in \Theta$, such that $\{ y\sp{i} \mid x\sp{i} \}_{i=1}^n \sim P( y\sp{i} \mid x\sp{i} ; \theta_*)$ for some $\theta_* \in \Theta$. Assume that we perform maximum likelihood estimation (MLE), i.e., our loss function is the negative log-likelihood $\ell((x\sp{i}, y\sp{i}), \theta) = - \log P( y\sp{i} \mid x\sp{i} ; \theta)$. As before, let $\hat\theta$ and $\theta^*$ denote the minimizers of empirical loss and population loss, respectively. Then, we have
    \al{
    \label{lec2:eqn:applied1}
        \theta^* = \theta_*,
    }
    \al{
    \label{lec2:eqn:applied2}
        \Exp \left[ \nabla \ell ((x, y), \theta^*) \right] = 0,
    }
    \al{
    \label{lec2:eqn:applied3}
        \Cov \left( \nabla \ell ((x, y), \theta^*) \right) = \nabla^2 L(\theta^*), \text{ and}
    }
    \al{
    \label{lec2:eqn:applied4}
        \sqrt n (\hat \theta - \theta^*) \overset d \to \cN (0, \nabla^2 L(\theta^*)^{-1}).
    }
\end{theorem}

\textbf{Remark 1:} You may also have seen \eqref{lec2:eqn:applied4} in the following form: under the maximum likelihood estimation (MLE) paradigm, the MLE is asymptotically efficient as it achieves the Cramer-Rao lower bound. That is, the parameter error of the MLE estimate converges in distribution to $\mathcal{N}(0, \mathcal{I}(\theta)^{-1})$, where $\mathcal{I}(\theta)$ is the Fisher information matrix (in this case, equivalent to the risk Hessian $\nabla^2 L(\theta^*)$)~\cite{rice2006mathematical}.

\textbf{Remark 2:} \eqref{lec2:eqn:applied3} is also known as Bartlett's identity~\cite{percynotes}.

Although the proofs were not presented in live lecture, we include them here.

\begin{proof}
From the definition of the population loss,
\begin{align}
    L(\theta) &= \Exp \left[ \ell((x\sp{i}, y\sp{i}), \theta) \right]\\
    &= \Exp \left[ - \log P(y \mid x; \theta) \right] \\
    &= \Exp \left[ - \log P(y \mid x; \theta) + \log P(y \mid x; \theta_*) \right] + \Exp \left[ - \log P(y \mid x; \theta_*) \right] \\
    &= \Exp \left[ \log \frac{P(y \mid x; \theta_*)}{P(y \mid x; \theta)} \right] + \Exp \left[ - \log P(y \mid x; \theta_*) \right].
\end{align}
Notice that the second term is a constant which we will express as $\cH(y \mid x; \theta_*)$. We expand the first term using the tower rule (or law of total expectation):
\begin{align}
    L(\theta) &= \Exp \left[ \Exp \left[ \log \frac{P(y \mid x; \theta_*)}{P(y \mid x; \theta)} \biggr\vert x \right] \right] + \cH(y \mid x; \theta_*).
\end{align}
The term in the expectation is just the KL divergence between the two probabilities, so 
\begin{align}
    L(\theta) &= \Exp \left[ \KL \left( y \mid x; \theta_* \| y \mid x; \theta \right) \right] + \cH(y \mid x; \theta_*) \\
    &\geq \cH(y \mid x; \theta_*),
\end{align}
since KL divergence is always non-negative. Since $\theta_*$ makes the KL divergence term 0, it minimizes $L(\theta)$ and so $\theta_* \in \argmin_\theta L(\theta)$. However, the minimizer of $L(\theta)$ is unique because of consistency, so  we must have $\argmin_\theta L(\theta) = \theta^*$ which proves (\ref{lec2:eqn:applied1}).

For \eqref{lec2:eqn:applied2}, recall $\nabla L(\theta^*) = 0$, so we have
\begin{equation}
0 = \nabla L(\theta^*) = \nabla \Exp \left[ \ell((x\sp{i}, y\sp{i}), \theta^*) \right] = \Exp \left[ \nabla \ell((x\sp{i}, y\sp{i}), \theta^*) \right],
\end{equation}
where we can switch the gradient and expectation under some regularity conditions.

To prove \eqref{lec2:eqn:applied3}, we first expand the RHS using the definition of covariance and express the marginal distributions as integrals:
\begin{align}
    \Cov \left( \nabla \ell ((x, y), \theta^*) \right) &= \Exp \left[ \nabla \ell ((x, y), \theta^*) \nabla \ell ((x, y), \theta^*)^\top \right] \\
    &= \int P(x) \left( \int P(y \mid x; \theta^*) \nabla \log P( y\sp{i} \mid x\sp{i} ; \theta^*) \nabla \log P( y\sp{i} \mid x\sp{i} ; \theta^*)^\top dy \right) dx \\
    &= \int P(x) \left( \int \frac{\nabla P(y \mid x; \theta^*) \nabla P(y \mid x; \theta^*)^\top}{P(y \mid x; \theta^*)}dy \right) dx.
\end{align}
Now we expand the LHS using the definition of the population loss and differentiate repeatedly:
\begin{align}
    \nabla^2 L(\theta^*) &= \Exp \left[- \nabla^2 \log P(y \mid x; \theta^*) \right] \\
    &= \int P(x) \left( \int - \nabla^2 P(y \mid x; \theta^*) + \frac{\nabla P(y \mid x; \theta^*) \nabla P(y \mid x; \theta^*)^\top}{P(y \mid x; \theta^*)}dy  \right) dx.
\end{align}
Note that we can express 
\begin{equation} \int \nabla^2 P(y \mid x; \theta^*) dy = \nabla^2 \int P(y \mid x; \theta^*) dy = \nabla 1  = 0 \end{equation}
so we find
\begin{equation} \nabla^2 L(\theta^*) = \int P(x) \left( \int \frac{\nabla P(y \mid x; \theta^*) \nabla P(y \mid x; \theta^*)^\top}{P(y \mid x; \theta^*)}dy \right) dx = \Cov \left( \nabla \ell ((x, y), \theta^*) \right). \end{equation}

Finally, \eqref{lec2:eqn:applied4} follows directly from Part 2 of Theorem~\ref{lec1:thm:asymp} and \eqref{lec2:eqn:applied3}.
\end{proof}

Using similar logic to our proof of Part 4 and 5 of Theorem~\ref{lec1:thm:asymp}, we can see that $n (L(\hat \theta) - L(\theta^*)) \overset d \to \frac12 \|S\|_2^2$ where $S \sim N(0, I)$. Since a chi-squared distribution with $p$ degrees of freedom is defined as a sum of the squares of $p$ independent standard normals, it quickly follows that $2n (L(\hat \theta) - L(\theta^*)) \sim  \chi^2(p)$, where $\theta \in \R^p$ and $n \to \infty$. We can thus characterize the excess risk in this case using the propertes of a chi-squared distribution:

\al{
    \lim_{n \to \infty} \Exp \left[ L(\hat \theta) - L(\theta^*) \right] = \frac{p}{2n}.
}

\sec{Limitations of asymptotic analysis}\label{sec:limit-asymp}

One limitation of asymptotic analysis is that our bounds often obscure dependencies on higher order terms. As an example, suppose we have a bound of the form
	\al{
		\frac{p}{2n} + o\left(\frac{1}{n}\right).
		\label{lec2:eqn:spicy_bound}
	}
(Here $o(\cdot)$ treats the parameter $p$ as a constant as $n$ goes to infinity.) 
We have no idea how large $n$ needs to be for asymptotic bounds to be ``reasonable." Compare two possible versions of \eqref{lec2:eqn:spicy_bound}: 
\begin{align}
    \frac{p}{2n} + \frac{1}{n^2} \quad \text{vs.} \quad \frac{p}{2n} + \frac{p^{100}}{n^2}.
\end{align}
Asymptotic analysis treats both of these bounds as the same, hiding the polynomial dependence on $p$ in the second bound. Clearly, the second bound is significantly more data-intensive than the first: we would need $n > p^{50}$ for $\frac{p^{100}}{n^2}$ to be less than one. Since $p$ represents the dimensionality of the data, this may be an unreasonable assumption.

This is where non-asymptotic analysis can be helpful. Whereas asymptotic analysis uses large-sample theorems such as the central limit theorem and the law of large numbers to provide convergence guarantees, non-asymptotic analysis relies on concentration inequalities to develop alternative techniques for reasoning about the performance of learning algorithms.


	
	\chapter{Concentration Inequalities}\label{chap:conc}
	% reset section counter
\setcounter{section}{0}

%\metadata{lecture ID}{Your names}{date}
\metadata{3}{Brad Ross and Robbie Jones}{Jan 20, 2021}

In this chapter, we take a little diversion and develop the notion of \emph{concentration inequalities}. Assume that we have independent random variables $X_1, \ldots, X_n$. We will develop tools to show results that formalize the intuition for these statements:
\begin{enumerate}
    \item $X_1 + \ldots + X_n$ concentrates around $\Exp[X_1 + \ldots + X_n]$.
    \item More generally, $f(X_1, \ldots, X_n)$ concentrates around $\Exp[f(X_1, \ldots, X_n)]$.
\end{enumerate}
These inequalities will be used in subsequent chapters to bound several key quantities of interest.

As it turns out, the material from this chapter constitutes arguably the important mathematical tools in the entire course. No matter what area of machine learning one wants to study, if it involves sample complexity, some kind of concentration result will typically be required. Hence, concentration inequalities are some of the most important tools in modern statistical learning theory.

\sec{The big-O notation}

Throughout the rest of this course, we will use ``big-O" notation in the following sense: every occurrence of $O(x)$ is a placeholder for some function $f(x)$ such that for every $x$, $|f(x)| \leq Cx$ for some absolute/universal constant $C$. In other words, suppose $O(n_1),\dots, O(n_k)$ occur in a statement, it means that \textbf{there exists} absolute constants $C_1,\dots, C_k > 0$ and functions $f_1,\dots, f_k$ satisfying $|f_i(x)|\le C_ix$ for all $x$, such that after replacing each occurrence $O(n_i)$ by $f_i(n_i)$,  the statement is true.  (The difference with traditional ``big-O" notation is that we do not need to send $n \to \infty$ in order to define ``big-O".)

Also, for any $a, b \geq 0$, we will let $a \lesssim b$ mean that there is some absolute constant $c > 0$ such that $a \leq cb$.

\sec{Chebyshev's inequality} 

We begin by considering an arbitrary random variable $Z$ with finite variance. One of the most famous results characterizing its tail behavior is the following theorem:

\begin{theorem}[Chebyshev's inequality]
    Let $Z$ be a random variable with finite expectation and variance. Then
    \al{
        \Pr[|Z - \Exp[Z]| \geq t] \leq \frac{\Var(Z)}{t^2}, \quad \forall t > 0.
        \label{lec3:eqn:chebyshev}
    }
\end{theorem}

Intuitively, this means that as we approach the tails of the distribution of $Z$, the density decreases at a rate of at least $1 / t^2$. Moreover, for any $\delta \in (0, 1]$, by plugging in $t = \sd(Z) / \sqrt{\delta}$ to \eqref{lec3:eqn:chebyshev} we see that 
    \al{
        \Pr\left[|Z - \Exp[Z]| \leq \frac{\sd(Z)}{\sqrt{\delta}}\right] \geq 1 - \delta.
        \label{lec3:eqn:chebyshevdelta}
    }
    
Unfortunately, it turns out that Chebyshev's inequality is a rather weak concentration inequality. To illustrate this, assume $Z \sim \cN(0, 1)$. We can show (using the Gaussian tail bound derived in Problem 3(c) in Homework 0) that
\al{
    \Pr\left[|Z - \Exp[Z]| \leq \sd(Z)\sqrt{2 \log (2 / \delta)}\right] \geq 1 - \delta.
    \label{lec3:eqn:normaltailbound}
}
for any $\delta \in (0, 1]$. In other words, the density at the tails of the normal distribution is decreasing at an exponential rate, while Chebyshev's inequality only gives a quadratic rate. The discrepancy between \eqref{lec3:eqn:chebyshevdelta} and \eqref{lec3:eqn:normaltailbound} is made more apparent when we consider inverse-polynomial $\delta = \frac{1}{n^c}$ for some parameter $n$ and degree $c$ (we will see concrete instances of this setup in future chapters). Then the tail bound for the normal distribution \eqref{lec3:eqn:normaltailbound} implies that
\al{
    |Z - \Exp[Z]| \leq \sd(Z) \cdot \sqrt{\log{O\left(n^c\right)}} = \sd(Z) \cdot O\left(\sqrt{\log{n}}\right) \quad w.p. \; 1 - \delta,
}
while Chebyshev's inequality gives us the weaker result
\al{
    |Z - \Exp[Z]| \leq \sd(Z) \cdot \sqrt{O(n^c)} = \sd(Z) \cdot O(n^{c / 2})  \quad w.p. \; 1 - \delta.
}

Chebyshev's inequality is actually optimal without further assumptions, in the sense that there exist distributions with finite variance for which the bound is tight. However, in many cases, we will be able to improve the $1/t^2$ rate of tail decay in Chebyshev's inequality to an $e^{-t}$ rate. In the next two sections, we will demonstrate how to construct tail bounds with exponential decay rates.

\sec{Hoeffding's inequality}\label{lec2:subsec:hoeffding}

We next provide a brief overview of Hoeffding's inequality, a concentration inequality for bounded random variables with an exponential tail bound:

\begin{theorem}[Hoeffding's inequality]
    Let $X_1, X_2, \dots, X_n$ be independent real-valued random variables drawn from some distribution, such that $a_i \leq X_i \leq b_i$ almost surely. Define $\bar{X} = \frac{1}{n}\sum_{i=1}^n X_i$, and let $\mu = \E [\bar{X}]$. Then for any $\varepsilon > 0$,
    \al {
    \Pr \left[ |\bar{X} - \mu | \leq \varepsilon \right] \geq 1 - 2  \exp\left(\frac{-2n^2\varepsilon^2}{\sum_{i=1}^n (b_i - a_i)^2}\right). \label{lec2:eqn:hoeffding}
    }
\end{theorem}

Note that the demoninator within the exponential term, $\sum_{i=1}^n (b_i - a_i)^2$, can be thought of as an upper bound or proxy for the variance $\Var(X_i)$. In fact, under the independence assumption, we can show
\begin{align}
    \Var\left(\bar{X} \right) &= \frac{1}{n^2}\sum_{i=1}^n \Var(X_i) \leq \frac{1}{n^2}\sum_{i=1}^n (b_i - a_i)^2.
\end{align}

Let $\sigma^2 = \frac{1}{n^2}\sum_{i=1}^n (b_i - a_i)^2$. If we take $\varepsilon = O(\sigma \sqrt{\log{n}}) = \sigma \sqrt{c \log n}$ so that $\varepsilon$ is bounded above by some large (i.e., $c \geq 10$) multiple of the standard deviation of the $X_i$'s times $\sqrt{\log{n}}$, we can substitute this value of $\varepsilon$ into \eqref{lec2:eqn:hoeffding} to reach the following conclusion: 
\begin{align}
    \Pr \left[ |\bar{X} - \mu| \leq \varepsilon \right] &\geq 1 - 2\exp\left(\frac{-2 \varepsilon^2}{\sigma^2}\right)\\
    &= 1 - 2 \exp(-2 c \log n)\\
    &= 1 - 2 n^{-2c}
\end{align}

We can see that as $n$ grows, the right-most term tends to zero such that $\Pr[|\bar{X} - \mu| \leq \varepsilon]$ very quickly approaches 1. Intuitively, this result tells us that, with high probability, the sample mean $\bar{X}$ will not be ``much farther" from the population mean $\mu$ by some sublogarithmic ($\sqrt{c \log n}$) factor of the standard deviation.\footnote{This is with the caveat, of course, that $\sigma$ is not exactly standard deviation but a loose upper bound on standard deviation.} Thus, we can restate the above claim we reached as follows:

\begin{remark}
    For sufficiently large $n$, $|\bar{X} - \mu | \leq O(\sigma \sqrt{\log{n}})$ with high probability.
\end{remark}

\begin{remark}\label{lec2:rem:hoeffding}
    If, in addition, we have $a_i = -O(1)$ and $b_i = O(1)$, then $\sigma^2 = O \left( \frac{1}{n}\right)$, and $|\bar{X} - \mu | \leq O\left(\sqrt{\frac{\log n}{n}}\right) = \tilO\left(\frac{1}{\sqrt{n}}\right)$.\footnote{$\tilO$ is analogous to Big-$O$ notation, in that $\tilO$ hides logarithmic factors. That is; if $f(n) = O(\log n)$, then $f(n) = \tilO(1)$.}
\end{remark}

Remark~\ref{lec2:rem:hoeffding} provides a compact form of the Hoeffding bound that we can use when the $X_i$ are bounded almost surely. 

So far, we have only shown how to construct exponential tail bounds for bounded random variables. Since requiring boundedness in $[0, 1]$ (or $[a, b]$ more generally) is limiting, it is worth asking what types of distributions permit such an exponential tail bound. The following section will explore such a class of random variables: \emph{sub-Gaussian} random variables.

\sec{Sub-Gaussian random variables}

We begin by defining the class of sub-Gaussian random variables by way of a bound on their moment generating functions. After estabishing this definition, we will see how this bound guarantees the exponential tail decay we desire.

\begin{definition}[Sub-Gaussian Random Variables]
    A random variable $X$ with finite mean $\mu$ is \textit{sub-Gaussian} with parameter $\sigma$ if
    \al{
        \Exp \left[ e^{\lambda(X - \mu)} \right] \leq e^{\sigma^2\lambda^2 / 2}, \quad \forall\lambda\in\R.
        \label{lec3:eqn:subgassdefn}
    }
    We say that $X$ is $\sigma$-sub-Gaussian and say it has \emph{variance proxy} $\sigma^2$.
\end{definition}

\begin{remark}\label{lec3:rem:mgf_strong}
    As it turns out, \eqref{lec3:eqn:subgassdefn} is quite a strong condition, requiring that infinitely many moments of $X$ exist and do not grow too quickly. To see why, assume without loss of generality that $\mu = 0$ and take a power series expansion of the moment generating function:
    \al{
        \Exp[\exp(\lambda X)] = \Exp\left[\sum_{k = 0}^\infty \frac{(\lambda X)^k}{k!}\right] = \sum_{k = 0}^\infty\frac{\lambda^k}{k!}\Exp[X^k].
    }
    A bound on the moment generating function then is a bound on infinitely many moments of $X$, i.e. a requirement that the moments of $X$ are all finite and grow slowly enough to allow the power series to converge. Though a proof of this result is beyond the scope of this monograph, Proposition 2.5.2 in \cite{vershynin2018high} shows that \eqref{lec3:eqn:subgassdefn} is equivalent to $\Exp \left [|X|^p \right ]^{1/p} \lesssim \sqrt{p}$ for all $p \geq 1$.
\end{remark}

\noindent Although \eqref{lec3:eqn:subgassdefn} is not a particularly intuitive definition, it turns out to imply exactly the type of exponential tail bound we want:

\begin{theorem}[Tail bound for sub-Gaussian random variables]\label{lec3:thm:subgausstail}
    If a random variable $X$ with finite mean $\mu$ is $\sigma$-sub-Gaussian, then
    \al{ 
        \Pr[|X - \mu| \geq t] \leq 2 \exp \left( -\frac{t^2}{2\sigma^2} \right), \quad \forall t \in \R.
        \label{lec3:eqn:subgausstail}
    }
\end{theorem}

\begin{proof}
Fix $t > 0$. For any $\lambda > 0$,
\al{
    \Pr[X - \mu \geq t] &= \Pr[\exp(\lambda (X - \mu)) \geq \exp(\lambda t)]  \\
    &\leq \exp(-\lambda t)\Exp[\exp(\lambda (X - \mu))] && \text{(by Markov's inequality)}  \\
    &\leq \exp(-\lambda t)\exp(\sigma^2\lambda^2/2) && \text{(by \eqref{lec3:eqn:subgassdefn})} \\
    &= \exp(-\lambda t + \sigma^2\lambda^2/2). \label{lec3:eqn:non_opt_tail_bound}
}
Because the bound \eqref{lec3:eqn:non_opt_tail_bound} holds for any choice of $\lambda > 0$ and $\exp(\cdot)$ is monotonically increasing, we can optimize the bound \eqref{lec3:eqn:non_opt_tail_bound} by finding $\lambda$ which minimizes the exponent $-\lambda t + \sigma^2 \lambda^2/2$. Differentiating and setting the derivative equal to zero, we find that the optimal choice is $\lambda = t/\sigma^2$, yielding the one-sided tail bound
\al{\label{lec3:eqn:opt_tail_bound_right}
    \Pr[X - \mu \geq t] \leq \exp\left(-\frac{t^2}{2\sigma^2}\right).
}
Going through the same line of reasoning but for $-X$ and $-t$, we can also show that for any $t > 0$,
\al{\label{lec3:eqn:opt_tail_bound_left}
    \Pr[X - \mu \leq -t] \leq \exp\left(-\frac{t^2}{2\sigma^2}\right).
}

We can then obtain \eqref{lec3:eqn:subgausstail} by applying the union bound:
\al{
    \Pr[|X - \mu| \geq t] = \Pr[X - \mu \geq t] + \Pr[X - \mu \leq -t] \leq 2\exp\left(-\frac{t^2}{2\sigma^2}\right).
}
\end{proof}

\begin{remark}[Tail bound implies sub-Gaussianity]\label{lec3:rem:tail_bound_remark}
    In addition to being a necessary condition for sub-Gaussianity (Theorem \ref{lec3:thm:subgausstail}), the tail bound \eqref{lec3:eqn:subgausstail} for sub-Gaussian random variables is also a sufficient condition up to a constant factor. In particular, if a random variable $X$ with finite mean $\mu$ satisfies \eqref{lec3:eqn:subgausstail} for some $\sigma > 0$, then $X$ is $O(\sigma)$-sub-Gaussian. Unfortunately, the proof of this reverse direction is somewhat more involved, so we refer the interested reader to Theorem 2.6 and its proof in Section 2.4 of \cite{wainwright2019high} and Proposition 2.5.2 in \cite{vershynin2018high} for details. While the tail bound is the property we ultimately care about most when studying sub-Gaussian random variables, the definition in \eqref{lec3:eqn:subgassdefn} is a more technically convenient characterization, as we will see in the proof of Theorem \ref{lec3:thm:sum_sub_gaussian}.
\end{remark}

\begin{remark}
    Note that in light of Remark \ref{lec3:rem:mgf_strong}, the tail bound \eqref{lec3:eqn:normaltailbound} requires all central moments of $X$ to exist and not grow too quickly. In contrast, Chebyshev's inequality (and more generally any polynomial variant of Markov's inequality $\Pr[|X-\mu| \geq t] = \Pr[|X-\mu|^k \geq t^k] \leq t^{-k}\E[|X-\mu|^k]$) only requires that the second central moment $\E[(X-\mu)^2]$ (more generally, the $k$th central moment $\E[|X - \mu|^k]$) is finite to yield a tail bound. If infinite moments exist however, it turns out that $\inf_{k \in \mathbb{N}} t^{-k}\E[|X-\mu|^k] \leq \inf_{\lambda > 0} \exp(-\lambda t) \Exp[\exp(\lambda (X-\lambda))]$, i.e. the optimal polynomial tail bound is tighter than the optimal exponential tail bound (see Exercise 2.3 in \cite{wainwright2019high}). As we will see shortly though, using exponential functions of random variables allows us to prove results about sums of random variables more conveniently. This ``tensorization'' property is why most researchers use exponential tail bounds in practice.
\end{remark}

Having defined and derived exponential tail bounds for sub-Gaussian random variables, we can now accomplish the first of the goals we set out at the beginning of the chapter: show that under certain conditions, namely independence and sub-Gaussianity of $X_1, \dotsc, X_n$, the sum $Z = \sum_{i = 1}^n X_i$ concentrates around $\Exp[Z] = \Exp[\sum_{i = 1}^n X_i]$.

\begin{theorem}[Sum of sub-Gaussian random variables is sub-Gaussian]\label{lec3:thm:sum_sub_gaussian}
    If $X_1, \ldots, X_n$ are independent sub-Gaussian random variables with variance proxies $\sigma_1^2, \ldots, \sigma_n^2$, then $Z = \sum_{i = 1}^n X_i$ is sub-Gaussian with variance proxy $\sum_{i = 1}^n \sigma_i^2$. As a consequence, we have the tail bound
    \al{
        \Pr[|Z - \Exp[Z]| \geq t] \leq 2\exp\left(-\frac{t^2}{2\sum_{i = 1}^n \sigma_i^2}\right),
    }
    for all $t \in \R$.
\end{theorem}

\begin{proof}
Using the independence of $X_1, \ldots, X_n$, we have that for any $\lambda \in \R$:
 \al{
    \Exp \left[ \exp \left\{\lambda(Z - \Exp[Z]) \right\} \right] &= \Exp\left[\prod_{i = 1}^n \exp \left\{\lambda(X_i - \Exp[X_i]) \right\}\right] \\
    &= \prod_{i = 1}^n \Exp \left[ \exp \left\{\lambda(X_i - \Exp[X_i]) \right\} \right] \\
    &\leq \prod_{i = 1}^n \exp \left( \frac{\lambda^2\sigma_i^2}{2} \right) \\
    &= \exp \left( \frac{\lambda^2 \sum_{i = 1}^n\sigma_i^2}{2} \right),
 }
 so $Z$ is sub-Gaussian with variance proxy $\sum_{i = 1}^n \sigma_i^2$. The tail bound then follows immediately from \eqref{lec3:eqn:subgausstail}.
\end{proof}

The proof above demonstrates the value of the moment generating functions of sub-Gaussian random variables: they factorize conveniently when dealing with sums of independent random variables.

\subsec{Examples of sub-Gaussian random variables}

We now provide several examples of classes of random variables that are sub-Gaussian, some of which will appear repeatedly throughout the remainder of the course.

\begin{example}[Rademacher random variables]
    A \textit{Rademacher random variable} $\epsilon$ takes a value of 1 with probability $1/2$ and a value of $-1$ with probability $1/2$. To see that $\epsilon$ is $1$-sub-Gaussian, we follow Example 2.3 in \cite{wainwright2019high} and upper bound the moment generating function of $\epsilon$ by way of a power series expansion of $\exp(\cdot)$:
    \al{
        \Exp[\exp(\lambda \epsilon)] &= \frac{1}{2}\left\{\exp(-\lambda) + \exp(\lambda)\right\} \\
        &= \frac{1}{2}\left\{\sum_{k = 0}^\infty \frac{(-\lambda)^k}{k!} + \sum_{k = 0}^\infty \frac{\lambda^k}{k!}\right\} \\
        &= \sum_{k = 0}^\infty \frac{\lambda^{2k}}{(2k)!} && \text{(for odd $k$, $(-\lambda)^k + \lambda^k = 0$)} \\
        &\leq 1 + \sum_{k = 1}^\infty \frac{\left(\lambda^2\right)^{k}}{2^k k!} && \text{($2^k k!$ is every other term of $(2k)!$)} \\
        & = \exp(\lambda^2/2),
    }
    which is exactly the moment generating function bound \eqref{lec3:eqn:subgassdefn} required for $1$-sub-Gaussianity.
\end{example}

\begin{example}[Random variables with bounded distance to mean]\label{lec3:ex:rand_var_bound_dist_to_mean}
    Suppose a random variable $X$ satisfies $|X - \Exp[X]| \leq M$ almost surely for some constant $M$. Then $X$ is $O(M)$-sub-Gaussian.
\end{example}
We now provide an even more general class of sub-Gaussian random variables that subsume the random variables in Example \ref{lec3:ex:rand_var_bound_dist_to_mean}:
\begin{example}[Bounded random variables]
    \label{lec3:ex:bounded_rand_var_subg}
    If $X$ is a random variable such that $a \leq X \leq b$ almost surely for some constants $a, b \in \R$, then
    \begin{equation*}
        \Exp\left[e^{\lambda(X - \Exp[X])}\right] \leq \exp \left[ \frac{\lambda^2(b - a)^2}{8} \right],
    \end{equation*}
    i.e., $X$ is sub-Gaussian with variance proxy $(b - a)^2/4$. (We will prove this in Question 2(a) of Homework 1.) Note that combining the $(b - a)/2$-sub-Gaussianity of i.i.d. bounded random variables $X_1, \dotsc, X_n$ and Theorem \ref{lec3:thm:sum_sub_gaussian} yields a proof of Hoeffding's inequality.
\end{example}

\begin{example}[Gaussian random variables]
If $X$ is Gaussian with variance $\sigma^2$, then $X$ satisfies \eqref{lec3:eqn:subgassdefn} with equality. In this special case, the variance and the variance proxy are the same.
\end{example}

\sec{Concentrations of functions of random variables}
We now introduce some important inequalities related to the second of our two goals, namely showing that for independent $X_1, \dotsc, X_n$ and certain functions $f$, $f(X_1, \dotsc, X_n)$ concentrates around $\Exp[f(X_1, \dotsc, X_n)]$.

\begin{theorem}[McDiarmid's inequality]
    Suppose $f : \R^n \to \R$ satisfies the \emph{bounded difference condition}: there exist constants $c_1, \ldots, c_n \in \R$ such that for all real numbers $x_1, \ldots, x_n$ and $x_i'$,
    \al{\label{lec3:eqn:mcdiarmid_fn_cond}
        |f(x_1, \ldots, x_n) - f(x_1, \ldots, x_{i - 1}, x_i', x_{i + 1}, \ldots, x_n)| \leq c_i.
    }
    (Intuitively, \eqref{lec3:eqn:mcdiarmid_fn_cond} states that $f$ is not overly sensitive to arbitrary changes in a single coordinate.) Then, for any independent random variables $X_1, \ldots, X_n$,
    \al{
        \Pr \left[ f(X_1, \ldots, X_n) - \Exp[f(X_1, \ldots, X_n)] \geq t \right] \leq \exp\left(-\frac{2t^2}{\sum_{i = 1}^n c_i^2}\right). \label{lec3:eqn:mcdiarmid_bound}
    }
    Moreover, $f(X_1, \ldots, X_n)$ is $O\left(\sqrt{\sum_{i = 1}^n c_i^2}\right)$-sub-Gaussian.
\end{theorem}

\begin{remark}
    Note that McDiarmid's inequality is a generalization of Hoeffding's inequality with $f(x_1, \dotsc, x_n) = \sum_{i = 1}^n x_i$ and $a_i \leq x_i \leq b_i$. 
    \tnoteimp{This remark seems to be a bit problematic. It's a generalization of Hoeffding when $f(x) = \sum x_i$ and $a_i \le x_i\le b_i$? }
\end{remark}

\begin{proof}
    The idea of this proof is to take the quantity $f(X_1,\dots,X_n) - \Exp[f(X_1,\dots,X_n)]$ and break it into manageable components by conditioning on portions of the sample. To this end, we begin by defining:
	\begin{align*}
	 	Z_0 &= \Exp \left[f(X_1,\dots,X_n) \right] &&\text{constant}\\
	 	Z_1 &= \Exp \left[f(X_1,\dots,X_n) \lvert X_1 \right] &&\text{a function of $X_1$} \\
        &\cdots \\
        Z_i &= \Exp \left [f(X_1,\dots,X_n) | X_1,\dots,X_i \right] &&\text{a function of $X_1,\dots,X_i$} \\
	 	&\cdots \\
        Z_n &= f(X_1,\dots,X_n)
	\end{align*}
    Using the law of total expectation, we show also that the expectation of $Z_i$ equals $Z_0$ for all $i$.
    \begin{align*}
        \Exp [Z_i] &= \Exp \left [ \Exp \left [f(X_1,\dots,X_n) | X_1,\dots,X_i \right] \right] \\
        &= \Exp[f(X_1,\dots,X_n)] \\
        &= Z_0
    \end{align*}
    The fact that $\E[D_i] = 0$, where $D_i = Z_i - Z_{i - 1}$, is an immediate corollary of this result. Next, we observe that we can rewrite the quantity of interest, $Z_n - Z_0$, as a telescoping sum in the increments $Z_i - Z_{i - 1}$:
    \begin{align*}
        Z_n - Z_0 &= (Z_n - Z_{n - 1}) + (Z_{n - 1} - Z_{n - 2}) + \cdots + (Z_1 - Z_0) \\
        &= \sum_{i = 1}^n D_i
    \end{align*} 
    Next, we show that conditional on $X_1,\dots,X_{i - 1}$, $D_i$ is a bounded random variable. First, observe that:
    \begin{align*}
        A_i = \inf_x \Exp \left[ f(X_1,\dots,X_n) | X_1,\dots,X_{i - 1}, X_i = x \right] - \Exp \left[ f(X_1,\dots,X_n) | X_1,\dots,X_{i - 1} \right] \\
        B_i = \sup_x \Exp \left [ f(X_1,\dots,X_n) | X_1,\dots,X_{i = 1}, X_i = x \right] - \Exp \left[ f(X_1,\dots,X_n) | X_1,\dots,X_{i - 1} \right]
    \end{align*}
    It is clear from their definition that $A_i \leq D_i \leq B_i$. Furthermore, by independence of the $X_i$'s, we have that:
    \begin{align*}
        B_i - A_i &\leq \sup_{x_{1:i - 1}} \sup_{x, x'} \int \left (f(x_1,\dots,x_{i - 1}, x, x_{i + 1},\dots,x_n) - f(x_1,\dots,x_{i - 1}, x', x_{i + 1},\dots,x_n)\right) dP(x_{i + 1},\dots,x_n) \\
        &\leq c_i
    \end{align*}
    Using this bound, the properties of conditional expectation, and Example~\ref{lec3:ex:bounded_rand_var_subg}, we can now prove that that $Z_n - Z_0$ is $O\left(\sqrt{\sum_{i = 1}^n c_i^2}\right)$-sub-Gaussian.
    \begin{align*}
        \Exp \left[e^{\lambda(Z_n - Z_0)} \right] &= \Exp \left[e^{\lambda \sum_{i = 1}^n (Z_i - Z_{i - 1})} \right] \\
        &= \Exp \left[ \Exp \left[e^{\lambda (Z_n - Z_{n - 1})} \biggr\lvert X_1,\dots,X_{n - 1} \right]e^{\lambda \sum_{i = 1}^{n - 1} (Z_i - Z_{i - 1})} \right] \\
        &\leq e^{\lambda^2 c_n^2/8} \Exp \left[ e^{\lambda \sum_{i = 1}^{n - 1} (Z_i - Z_{i - 1})} \right] \\
        &\cdots \\
        &\leq e^{\lambda^2 (\sum_{i = 1}^n c_i^2)/8}
    \end{align*}
    The final inequality given in \eqref{lec3:eqn:mcdiarmid_bound} follows by Theorem~\ref{lec3:thm:subgausstail}.
\end{proof}

A more general version of McDiarmid's inequality comes from Theorem 3.18 in~\cite{vanhandel2016high}. The setup for this theorem requires defining the \emph{one-sided differences} of a function $f : \R^n \to \R$:
\al{
    D_i^-{f(x)} &= f(x_1, \ldots, x_n) - \inf_z f(x_1, \ldots, x_{i - 1}, z, x_{i + 1}, \ldots, x_n) \\
    D_i^+{f(x)} &= \sup_z f(x_1, \ldots, x_{i - 1}, z, x_{i + 1}, \ldots, x_n) - f(x_1, \ldots, x_n).
}
These two quantities are functions of $x \in \R^n$, and hence can be interpreted as describing the sensitivity of $f$ \emph{at a particular point}. (Contrast this with the bounded difference condition \eqref{lec3:eqn:mcdiarmid_fn_cond}, which bounds the sensitivity of $f$ universally over all points.) For convenience, define
\al{
    d^+ &= \Norm{\sum_{i = 1}^n |D_i^+{f}|^2}_\infty = \sup_{x_1, \ldots, x_n}\sum_{i = 1}^n[|D_i^+{f(x_1, \ldots, x_n)}]^2 \\
    d^- &= \Norm{\sum_{i = 1}^n |D_i^-{f}|^2}_\infty = \sup_{x_1, \ldots, x_n}\sum_{i = 1}^n [D_i^-{f(x_1, \ldots, x_n)}]^2.
}
\begin{theorem}[Bounded difference inequality, Theorem 3.18 in~\cite{vanhandel2016high}]
    Let $f : \R^n \to \R$, and let $X_1, \ldots, X_n$ be independent random variables. Then, for all $t \geq 0$,
    \al{
        \Pr[f(X_1, \ldots, X_n) \geq \Exp[f(X_1, \ldots, X_n)] + t] &\leq \exp\left(-\frac{t^2}{4d^-}\right) \\
        \Pr[f(X_1, \ldots, X_n) \leq \Exp[f(X_1, \ldots, X_n)] - t] &\leq \exp\left(-\frac{t^2}{4d^+}\right).
    }
\end{theorem}

\subsec{Bounds for Gaussian random variables}
Unfortunately, the bounded difference condition (\ref{lec3:eqn:mcdiarmid_fn_cond}) is often only satisfied by bounded random variables or a bounded function. To get similar concentration inequalities for unbounded random variables, we need some other special conditions. The following inequalities assume that the random variables have the standard normal distribution.

\begin{theorem}[Gaussian Poincar\'{e} inequality, Corollary 2.27 in~\cite{vanhandel2016high}]
    Let $f : \R^n \to \R$ be smooth. If $X_1, \ldots, X_n$ are independently sampled from $\cN(0, 1)$, then
    \al{
        \Var(f(X_1, \ldots, X_n)) \leq \Exp \left[ \norm{\nabla{f}(X_1, \ldots, X_n)}_2^2 \right].
    }
\end{theorem}

Before introducing the next theorem, we recall that a function $f : \R^n \to \R$ is \emph{$L$-Lipschitz} with respect to the $\ell_2$-norm if there exists a non-negative constant $L \in \R$ such that for all $x, y \in \R^n$,
\al{
    |f(x) - f(y)| \leq L\norm{x - y}_2.
}
We emphasize that $L$ is universal for all points in $\R^n$.

\begin{theorem}[Theorem 2.26 in~\cite{wainwright2019high}]
    Suppose $f : \R^n \to \R$ is $L$-Lipschitz with respect to Euclidean distance, and let $X = (X_1, \ldots, X_n)$, where $X_1, \ldots, X_n \iid \cN(0, 1)$. Then for all $t \in \R$,
    \al{
        \Pr[|f(X) - \Exp[f(X)]| \geq t] \leq 2\exp\left(-\frac{t^2}{2L^2}\right).
    }
In particular, $f(X)$ is sub-Gaussian.
\end{theorem}

	
	\chapter{Generalization Bounds via Uniform Convergence}\label{chap:uc}
	\input{collection/04-01-uniform.tex}
	% reset section counter
%\setcounter{section}{0}

%\metadata{lecture ID}{Your names}{date}
\metadata{5}{Will Song}{Jan 27th, 2021}

\sec{Rademacher complexity}

\subsec{Motivation for a new complexity measure}

Recall that our goal is to bound the \textit{excess risk} $L(\hat{h}) - L(h^*)$, where $L$ is the expected loss (or population loss), $\hat{h}$ is our estimated hypothesis and $h^*$ is the hypothesis in the hypothesis class $\cH$ which minimizes the expected loss. We previously showed that to do so it suffices to upper bound $\sup_{h\in \cH} (L (h) - \hatL(h))$. (Note: we often call $L(\hat{h}) - \hatL(\hat{h})$ the \textit{generalization gap} or \textit{generalization error}.)

In the previous sections, we derived bounds for the generalization gap in two cases:
\begin{enumerate}
	\item If the hypothesis class $\cH$ is finite,
	\begin{equation}\label{lec5:eqn:bound-finite}
	L(\hat h) - \hat L(\hat h) \leq \tilde O \l( \sqrt{\frac{\log |\cH|}{n}} \r).
	\end{equation}
	\item If the hypothesis class $\cH$ is $p$-dimensional,
	\begin{equation}\label{lec5:eqn:bound-p}
	L(\hat h) - \hat L(\hat h) \leq \tilde O \l( \sqrt{\frac{p}{n}} \r).
	\end{equation}
\end{enumerate} 
Both of these bounds have a $\frac{1}{\sqrt{n}}$-dependency on $n$, which is known as the ``slow rate". The terms in the numerator ($\log |\cH|$ and $p$ resp.) can be thought of as complexity measures of $\cH$.

The bound \eqref{lec5:eqn:bound-p} is not precise enough: it depends solely on $p$ and is not always optimal. For example, this would be a poor bound if the hypothesis class $\cH$ has very high dimension but small norm. One specific example is for the following two hypothesis classes:
$$ \{\theta : \|\theta\|_1 \leq B\} \qquad \text{vs.} \qquad \{\theta : \|\theta\|_2 \leq B\},$$
\eqref{lec5:eqn:bound-p} would give both hypothesis classes the same bound of $\tilde O \l( \sqrt{\frac{p}{n}} \r)$. Intuitively, we should take into account the norms to prove a better bound.

With the complexity measure to be introduced, we will prove a bound of the form
\begin{align}
    L(\hat h) - \hat L(\hat h) \leq \tilde O\l(\sqrt{\frac{\text{Complexity}(\Theta)}{n}}\r).
\end{align}

This complexity measure will depend on the distribution $P$ over $\cX \times \cY$ (the input and output spaces), and hence takes into account how easy it is to learn $P$. If $P$ is easy to learn, then this complexity measure will be small even if the hypothesis space is big.

One of the practical implications of having such a complexity measure is that we can restrict the hypothesis space by regularizing the complexity measure (assuming it is something we can evaluate and train with). If we successfully find a low complexity model, then this generalization bound guarantees that we have not overfit.

\subsec{Definitions}

In uniform convergence, we sought a high probability bound for $\sup_{h \in H}(L(h) - \hat L (h))$. Here we have a weaker goal: we try to obtain an upper bound for its expectation instead, i.e.
\begin{equation}
\Exp\l[ \sup_{h \in H}(L(h) - \hat L (h)) \r] \leq \text{ upper bound}. \label{lec5:eq:generror}
\end{equation}
The expectation is over the randomness in the training data $\{(x^{(i)}, y^{(i)}\}_{i=1}^n$.\footnote{Though we might like to pull the $\sup$ outside of the $\Exp$ operator, and bound the expectation of the excess risk (a far simpler quantity to deal with!), in general, the $\sup$ and $\Exp$ operators do not commute. In particular, $\Exp\left [\sup_{h \in \cH} (L(h) - \hat{L}(h)) \right ] \geq \sup_{h \in \cH} \Exp \left[ L(h) - \hat{L} (h) \right]$.}

To do so, we first define \textit{Rademacher complexity}.

\begin{definition}[Rademacher complexity] \label{lec5:dfn:rc}
Let $\cF$ be a family of functions mapping $Z \mapsto \bbR$, and let $P$ be a distribution over $Z$. The \textit{(average) Rademacher complexity} of $\cF$ is defined as 
\begin{align}
    R_n(F) \triangleq \Exp_{z_1, \dots, z_n \iid P} \l[ 
    \Exp_{\sigma_1, \dots, \sigma_n \iid\{ \pm 1 \}} \l[ \sup_{f\in F} \frac{1}{n} \sum^n_{i=1} \sigma_i f(z_i) \r] \r], \label{lec5:eqn:Rn}
\end{align}
where $\sigma_1, \dots, \sigma_n$ are independent \textit{Rademacher random variables}, i.e. each taking on the value of $1$ or $-1$ with probability $1/2$.
\end{definition}

\begin{remark}
For applications to empirical risk minimization, we will take $\cZ = \cX \times \cY$. However, Definition \ref{lec5:dfn:rc} holds for abstract input spaces $\cZ$ as well.
\end{remark}

\begin{remark}
Note that $R_n(\cF)$ is also dependent on the measure $P$ of the space, so technically it should be $R_{n,P}(\cF)$, but for brevity, we refer to it as $R_n(\cF)$.
\end{remark}

An interpretation is $R_n(\cF)$ is the maximal possible correlation between outputs of some $f \in \cF$ (on points $f(z_1), \dots, f(z_n)$) and random Rademacher variables $ (\sigma_1, \dots, \sigma_n).$ Essentially, functions with more random sign outputs will better match random patterns of Rademacher variables and have higher complexity (greater ability to mimic or express randomness).

The following theorem is the main theorem involving Rademacher complexity:

\begin{theorem} \label{lec5:thm:thm1}
    \begin{align}
       \Exp_{z_1, \dots, z_n \iid P} \l[ \sup_{f\in F} \l[ \frac{1}{n} \sum^n_{i=1} f(z_i) -  \Exp_{z\sim P} [f(z)] \r]\r] \leq 2 R_n(\cF). \label{lec5:eqn:thm1}
    \end{align}
\end{theorem}

\begin{remark}
We can think of $\frac{1}{n} \sum^n_{i=1} f(z_i)$ as an empirical average and $\Exp_{z\sim P} [f(z)]$ as a population average.
\end{remark}
\noindent\textit{Why is Theorem \ref{lec5:thm:thm1} useful to us?} We can set $\cF$ to be the family of loss functions, i.e.
\begin{equation}
\cF = \l\{ z = (x,y) \in \cZ \mapsto \ell((x,y),h) \in \bbR : h \in \cH \r\}.
\end{equation} 
This is the family of losses induced by the hypothesis functions in $\cH$. We also define the function class $-\cF$ as $\{-f : f \in \cF\}$. It should be obvious from this definition that $R_n(\cF) = R_n(-\cF)$ since $\sigma_i \stackrel{d}{=} -\sigma_i$ for all $i$. Then, letting $z_i = (x^{(i)}, y^{(i)})$,
\begin{align}
    \Exp\l[ \sup_{h \in \cH}\l( L(h) - \hat L (h) \r) \r] &= \Exp_{\{(x^{(i)}, y^{(i)})\}} \l[ \sup_{h \in \cH} \l[L(h) - \frac{1}{n} \sum^n_{i=1} \ell((x^{(i)}, y^{(i)}, h)) \r] \r] \\
    &= \Exp_{\{z_i\}} \l[\sup_{f \in \cF} \l(\Exp[f(z)] - \frac{1}{n} \sum^n_{i=1} f(z_i) \r)\r] \\
    &= \Exp_{\{z_i\}} \l[\sup_{f \in -\cF} \l(\frac{1}{n} \sum^n_{i=1} f(z_i) - \Exp[f(z)] \r)\r] \\
    &\leq 2 R_n(-\cF) = 2R_n(\cF)
\end{align}
where the last step follows by Theorem \ref{lec5:thm:thm1}. 

Thus, $2R_n(\cF)$ is an upper bound for the generalization error. In this context, $R_n(\cF)$ can be interpreted as how well the loss sequence $\ell((x^{(1)}, y^{(1)}), h), \dots \ell((x^{(n)}, y^{(n)}), h)$ correlates with $\sigma_1, \dots, \sigma_n$.
\begin{example}
Consider the binary classification setting where $y \in \{\pm 1\}$. Let $\ell_{0-1}$ denote the zero-one loss function. Note that
\begin{equation}\label{lec5:eqn:01}
    \ell_{0-1}((x,y), h) = \mathbf{1}\{h(x) \neq y\} = \frac{1-yh(x)}{2}.
\end{equation}

Hence,
\begin{align}
    R_n(\cF) &= \Exp_{\{(x^{(i)}, y^{(i)})\}, \sigma_i} \l[ \sup_{h \in \cH} \frac{1}{n}\sum^n_{i=1} \ell_{0-1}((x^{(i)}, y^{(i)}),h)\sigma_i \r] &(\text{by definition}) \\
    &= \Exp_{\{(x^{(i)}, y^{(i)})\}, \sigma_i} \l[ \sup_{h \in \cH} \frac{1}{n}\sum^n_{i=1} \l(\frac{-h(x^{(i)})y^{(i)}+1}{2}\r)\sigma_i \r] &(\text{by } \eqref{lec5:eqn:01}) \\
    &= \frac{1}{2} \Exp_{\{(x^{(i)}, y^{(i)})\}, \sigma_i} \l[ \frac{1}{n}\sum^n_{i=1}\sigma_i + \sup_{h \in \cH} \frac{1}{n}\sum^n_{i=1} -h(x^{(i)})y^{(i)}\sigma_i \r] &(\sup \text{only over } \cH) \\
    &= \frac{1}{2} \Exp_{\{(x^{(i)}, y^{(i)})\}, \sigma_i} \l[\sup_{h \in \cH} \frac{1}{n}\sum^n_{i=1} -h(x^{(i)})y^{(i)}\sigma_i \r] &(\Exp [\sigma_i] = 0) \\
    &=\frac{1}{2} \Exp_{\{(x^{(i)}, y^{(i)})\}, \sigma_i} \l[\sup_{h \in \cH} \frac{1}{n}\sum^n_{i=1} h(x^{(i)})\sigma_i \r] &(-y_i \sigma_i \stackrel{d}{=} \sigma_i) \\
    &= \frac{1}{2}R_n(\cH). &(\text{by definition})
\end{align}

In this setting, $R_n(\cF)$ and $R_n(\cH)$ are the same (except for the factor of 2). $R_n(\cH)$ has a slightly more intuitive interpretation: it represents how well $h \in \cH$ can fit random patterns.

\textbf{Warning!} $R_n(\cF)$ is not always the same as $R_n(\cH)$ in other problems.
\end{example}

\begin{remark}
Rademacher complexity is invariant to translation. This property manifests in the previous example when the $+1$ in the $\l(\frac{-h(x^{(i)})y^{(i)}+1}{2}\r)$ term essentially vanishes in the computation.
\end{remark}

Let us now prove Theorem \ref{lec5:thm:thm1}.

\begin{proof}[Proof of Theorem \ref{lec5:thm:thm1}]
We use a technique called \textit{symmetrization}, which is a very important technique in probability theory. We first fix $z_1, \dots, z_n$and draw $ z_1', \dots z_n' \iid P$. Then we can rewrite the term in the expectation on the LHS of \eqref{lec5:eqn:thm1}:
\begin{align}
    \sup_{f \in \cF} \l( \frac{1}{n} \sum^n_{i=1} f(z_i) - \Exp[f] \r) &= \sup_{f \in \cF} \l( \frac{1}{n} \sum^n_{i=1} f(z_i) - \Exp_{z_1',\dots, z_n'} \l[ \frac{1}{n} \sum^n_{i=1} f(z_i')\r] \r) \\
    &= \sup_{f \in \cF} \l( \Exp_{z_1',\dots, z_n'} \l[\frac{1}{n} \sum^n_{i=1} f(z_i) -  \frac{1}{n} \sum^n_{i=1} f(z_i')\r] \r)\\
    &\leq \Exp_{z_1',\dots, z_n'} \l[\sup_{f \in \cF} \l( \frac{1}{n} \sum^n_{i=1} f(z_i) -  \frac{1}{n} \sum^n_{i=1} f(z_i')\r)\r]. \label{lec5:eqn:thm1-pf1}
\end{align}

The last inequality is because in general,
\begin{align}
    \sup_u \l(\Exp_v[g(u,v)]\r) \leq \sup_u \l( \Exp_v\l[\sup_{u'}g(u',v)\r]\r) = \Exp_v \l[\sup_u (g(u,v))\r]
\end{align}
since the $\sup$ over $u$ becomes vacuous after we replace $u$ with $u'$.

Now, if we take the expectation over $z_1, \dots, z_n$ for both sides of \eqref{lec5:eqn:thm1-pf1},
\begin{align}
    \Exp_{z_1, \dots, z_n} \l[\sup_{f \in \cF} \l( \frac{1}{n} \sum^n_{i=1} f(z_i) - \Exp[f] \r) \r] 
    &\leq \Exp_{z_i} \l[ \Exp_{z_i'} \l[\sup_{f \in \cF} \l( \frac{1}{n} \sum^n_{i=1} \l(f(z_i) -  f(z_i')\r)\r)\r]\r]\\
    &= \Exp_{z_i,z_i'} \l[ \Exp_{\sigma_i} \l[\sup_{f \in \cF} \l( \frac{1}{n} \sum^n_{i=1} \sigma_i\l(f(z_i) -  f(z_i')\r)\r)\r]\r] \label{lec5:eqn:thm1-pf2} \\
 &\leq \Exp_{z_i,z_i', \sigma_i} \l[\sup_{f \in \cF} \l( \frac{1}{n} \sum^n_{i=1} \sigma_i f(z_i)\r)+\sup_{f \in \cF} \l( \frac{1}{n} \sum^n_{i=1} -\sigma_i f(z_i')\r)\r] \\
    &= 2R_n(\cF),
\end{align}
where \eqref{lec5:eqn:thm1-pf2} is because $\sigma_i(f(z_i) - f(z_i')) \stackrel{d}{=} f(z_i) - f(z_i')$ since $f(z_i) - f(z_i')$ has a symmetric distribution. The last equality holds since $-\sigma_i \overset{d}{=} \sigma_i$ and $z_i, z_i'$ are drawn iid from the same distribution. 
\end{proof}

Here is an intuitive understanding of what Theorem \ref{lec5:thm:thm1} achieves. Consider the quantities on the LHS and RHS of \eqref{lec5:eqn:thm1}:
\begin{align*}
    \sup_{f\in \cF} \l(\frac{1}{n} \sum_{i=1}^n f(z_i) - \Exp[f(z)]\r) \qquad \text{v.s.} \qquad \sup_{f\in \cF} \l(\frac{1}{n} \sum_{i=1}^n \sigma_i f(z_i)\r).
\end{align*}
First, we removed $\Exp[f(z)]$, which is hard to control quantitatively since it is deterministic. Second, we added more randomness in the form of Rademacher variables. This will allow us to shift our focus from the randomness in the $z_i$'s to the randomness in the $\sigma_i$'s. In the future, our bounds on the Rademacher complexity will typically only depend on the randomness from the $\sigma_i$'s.

\subsec{Dependence of Rademacher complexity on $P$}
For intuition on how Rademacher complexity depends on the distribution $P$, consider the extreme example where $P$ is a point mass, i.e. $z = z_0$ almost surely. Assume that $-1 \leq f(z_0) \leq 1$ for all $f \in \cF$. Then
\begin{align}
    \Exp_{z_1, \dots, z_n \sim P} \l[ \sup_{f \in \cF} \frac{1}{n} \sum^n_{i=1} \sigma_i f(z_i)\r]
    &= \Exp_{\sigma_1, \dots, \sigma_n} \l[ \sup_{f \in \cF} \frac{1}{n}f(z_0) \sum^n_{i=1} \sigma_i \r] \\
    &\leq \Exp_{\sigma_1, \dots, \sigma_n} \l[ \l| \frac{1}{n} \sum^n_{i=1} \sigma_i \r|\r] &(\text{since } f(z_0) \in [-1,1]) \\
    &\leq \Exp_{\sigma_i} \l[ \l( \frac{1}{n} \sum^n_{i=1} \sigma_i \r)^2\r]^\frac{1}{2} &(\text{Jensen's Inequality}) \\
    &= \frac{1}{n}\l( \Exp_{\sigma_i, \sigma_j} \l[ \sum^n_{i, j=1} \sigma_i\sigma_j \r] \r)^\frac{1}{2}\\
    &= \frac{1}{n}\l( \Exp_{\sigma_i} \l[ \sum^n_{i=1} \sigma_i^2 \r] \r)^\frac{1}{2} \\
    &= \frac{1}{n} \cdot \sqrt{n} = \frac{1}{\sqrt{n}}.
\end{align}
This bound does not depend on $\cF$ (except that it is bounded). This example illustrates that a bound on the Rademacher complexity can sometimes only depend on the (known) distribution of the Rademacher random variables.

\sec{Empirical Rademacher complexity}

In the previous section, we bounded the expectation of $\sup_{f\in F} \l[ \frac{1}{n} \sum^n_{i=1} f(z_i) -  \Exp_{z\sim P} [f(z)] \r]$. This expectation is taken over the training examples $z_1, \dots, z_n$. In many instances we only have one training set, and do not have access to many training sets. Thus, the bound on the expectation does not give a guarantee for the one training set that we have. In this section, we seek to bound the quantity itself with high probability.

\begin{definition}[Empirical Rademacher complexity]
Given a dataset $S = \{z_1, \dots, z_n\}$, the \textit{emprical Rademacher complexity} is defined as
\begin{equation}
R_S(\cF) \overset{\Delta}{=} \Exp_{\sigma_1,\dots, \sigma_n} \l[ \sup_{f\in \cF} \frac{1}{n} \sum^n_{i=1} \sigma_i f(z_i) \r].
\end{equation}
$R_S(\cF)$ is a function of both the function class $\cF$ and the dataset $S$.
\end{definition}

Note that, as the name suggests, the expectation of the empirical Rademacher complexity is the Rademacher complexity:
\begin{align}
    R_n(\cF) = \underset{S=\{z_1,\dots, z_n\}}{\underset{z_1, \dots, z_n \iid P}\Exp}\l[ R_S(\cF) \r].
\end{align}


Here is the theorem involving empirical Rademacher complexity:

\begin{theorem}\label{lec5:thm:thm2}
    Suppose for all $f \in \cF$, $0 \leq f(z) \leq 1$. Then, with probability at least $1-\delta$,
    \begin{align}
        \sup_{f\in \cF} \l[ \frac{1}{n} \sum^n_{i=1} f(z_i) - \Exp[f(z)] \r] \leq 2 R_S(F) + 3\sqrt{\frac{\log{(2/\delta)}}{2n}}.
    \end{align}
\end{theorem}

\begin{proof}
For conciseness, define
\begin{equation} g(z_1, \dots, z_n) \triangleq \sup_{f\in F} \l[ \frac{1}{n} \sum^{n}_{i=1} f(z_i) - \Exp[f(z)]\r]. \end{equation}

We prove the theorem in 4 steps.

\textbf{Step 1:} We bound $g$ using McDiarmid's Inequality. To use McDiarmid's inequality, we check that the bounded difference condition holds:
\begin{align}
    g(z_1, \dots, z_n) - g(z_1, \dots, z_i', \dots, z_n)
    &\leq \sup_{f\in \cF} \l[ \frac{1}{n} \sum^{n}_{j=1} f(z_j) \r] - \sup_{f\in \cF} \l[ \l(\frac{1}{n} \sum^{n}_{j=1, j \neq i} f(z_j)\r) + \frac{f(z_i')}{n} \r]  \\
    &\leq \sup_{f\in \cF} \l[ \frac{1}{n} \sum^{n}_{j=1} f(z_j) - \l(\frac{1}{n} \sum^{n}_{j=1, j \neq i} f(z_j)\r) - \frac{f(z_i')}{n} \r] \label{lec5:eqn:thm2-pf1} \\
    &= \sup_{f\in \cF}\l[ \frac{1}{n} \l( f(z_i) - f(z_i') \r) \r] \\
    &\leq \frac{1}{n}. \label{lec5:eqn:thm2-pf2}
\end{align}
\eqref{lec5:eqn:thm2-pf1} holds because in general, $\sup_f A(f) - \sup_f B(f) \leq \sup_f [A(f) - B(f)]$, and \eqref{lec5:eqn:thm2-pf2} holds since $f$ is bounded by $[0,1]$. We can thus apply McDiarmid's Inequality with parameters $c_1 = \dots = c_n = 1/n$:
\begin{align}
    \Pr\l[ g(z_1, \dots, z_n) \geq \Exp_{z_1,\dots, z_n \iid P}[g] + \epsilon \r] \leq \exp{\l( \frac{-2\epsilon^2}{\sum^n_{i=1} c_i^2 }\r)} = \exp(-2n\epsilon^2).
\end{align}

\textbf{Step 2:} We apply Theorem \ref{lec5:thm:thm1} to get 
\begin{align}
 \Exp_{z_1,\dots, z_n \iid P}[g] \leq 2R_n(\cF).
\end{align}

\textbf{Step 3:} Define
\begin{equation} \tilde g (z_1, \dots, z_n) = R_S(\cF) \triangleq \Exp_{\sigma_i}\l[\sup_{f\in \cF} \frac{1}{n} \sum^n_{i=1} \sigma_i f(z_i)\r]. \end{equation}

Using a similar argument to that of Step 1, we show that $\tilde g$ satisfies the bounded difference condition:
\begin{align}
    &\tilde g(z_1, \dots, z_n) - \tilde g(z_1, \dots, z_i', \dots, z_n) \nonumber \\
    &\leq \Exp_{\sigma_i} \l[\sup_{f\in F} \l[ \frac{1}{n} \sum^{n}_{j=1} \sigma_j f(z_j) \r] - \sup_{f\in F} \l[ \l(\frac{1}{n} \sum^{n}_{j=1, j \neq i} \sigma_j f(z_j)\r) + \frac{1}{n} \sigma_if(z_i')\r]\r]\\
    &\leq \Exp_{\sigma_i}\l[\sup_{f\in F} \l(\frac{1}{n} \sigma_i(f(z_i) - f(z_i'))\r)\r] \\
    &\leq \frac{1}{n},
\end{align}
since the term inside the $\sup$ is always upper bounded by 1. We can thus apply McDiarmid's Inequality with parameters $c_1 = \dots = c_n = 1/n$:
\begin{align}
    \Pr\l[ \tilde g - \Exp[\tilde g] \geq \epsilon \r] \leq \exp(-2n \epsilon^2), \quad\text{and}\quad
    \Pr\l[ \tilde g - \Exp[\tilde g] \leq -\epsilon \r] \leq \exp(-2n \epsilon^2).
\end{align}

\textbf{Step 4:} We set $\delta$ such that $\exp(-2n \epsilon^2) = \delta/2$. (This implies that $\epsilon = \sqrt{\frac{\log(2/\delta)}{2n}}$.) Then, with probability $\geq 1 - \delta$,
\begin{align}
    \sup_{f\in \cF} \l[ \frac{1}{n} \sum^n_{i=1} f(z_i) - \Exp[f]\r] = g &\leq \Exp[g] + \epsilon &\text{(Step 1)} \\
    &\leq 2R_n(\cF) + \epsilon &\text{(Step 2)} \\
    &\leq 2(R_S(\cF) + \epsilon) + \epsilon &\text{(Step 3)} \\
    &= 2R_S(\cF) + 3\epsilon,
\end{align}
as required.
\end{proof}

Setting $\cF$ to be a family of loss functions bounded by $[0,1]$ in Theorem \ref{lec5:thm:thm2} gives the following corollary:
\begin{corollary}\label{lec6:cor:ggap-rsbound}
Let $\cF$ to be a family of loss functions $\cF = \l\{ (x,y) \mapsto \ell((x,y),h): h \in \cH \r\}$ with $\ell((x,y), h) \in [0,1]$ for all $\ell$, $(x,y)$ and $h$. Then, with probability $1-\delta,$ the generalization gap is
    \begin{equation}\label{lec6:eqn:ggap-rsbound}
        L(h) - \hat L(h) \leq 2R_S(F) + 3\sqrt{\frac{\log(2/\delta)}{2n}} \quad \text{for all } h\in \cH.
    \end{equation}
\end{corollary}

\begin{remark}
If we want to bound the generalization gap by the average Rademacher complexity instead, we can replace the RHS of \eqref{lec6:eqn:ggap-rsbound} with $2R_n(\cF) + \sqrt{\frac{\log(2/\delta)}{2n}}$.
\end{remark}

\paragraph{Interpretation of  Corollary \ref{lec6:cor:ggap-rsbound}.}
\sloppy It is typically the case that $O\l(\sqrt{\frac{\log (2/\delta)}{n}}\r) \ll R_S(\cF)$ and $O\l(\sqrt{\frac{\log (2/\delta)}{n}}\r) \ll R_n(\cF)$. This is the case because $R_S(\cF)$ and $R_n(\cF)$ often take the form $\frac{c}{\sqrt{n}}$ where $c$ is a big constant depending on the complexity of $\cF$, whereas we only have a logarithmic term in the numerator of $O\l(\sqrt{\frac{\log (2/\delta)}{n}}\r)$. As a result, we can view the $3\sqrt{\frac{\log (2/\delta)}{n}}$ term in the RHS of Corollary \ref{lec6:cor:ggap-rsbound} as negligible. Another way of seeing this is noting that a $\tilO \left( \frac{1}{\sqrt{n}} \right)$ term is necessary even for the concentration bound of a single function $h\in\cH$. Previously, we bounded $L(h)-\hat{L}(h)$ using a union bound over $h\in\cH$, which necessarily needs to be larger than $\tilO \left(\frac{1}{\sqrt{n}} \right)$. As a result, the $O\l(\sqrt{\frac{\log (2/\delta)}{n}}\r)$ term is not significant.

%\subsec{Empirical Rademacher complexity viewed in the output/function space}
%Assume we have a fixed dataset $S = \{z_1, \dots, z_n\}$. Since $z_1,\dots, z_n$ is fixed, each function $f\in\cF$ corresponds to a single output $(f(z_1),\dots,f(z_n))\in \R^n$. Hence, we can express the set of outputs for every function $f\in\cF$ as
%\begin{align}
%    Q_\cF = \left\{ \begin{pmatrix}f(z_1), \dots, f(z_n) \end{pmatrix} \mid f\in\cF \right\}.
%\end{align}
%
%Now we can mathematically re-express the empirical Rademacher complexity as an inner product:
%\begin{align}
%R_S(\cF) &= \E_{\sigma_1,\dots, \sigma_n} \l[ \sup_{f\in \cF} \frac{1}{n} \sum^n_{i=1} \sigma_i f(z_i) \r] \\
%&= \E_{\sigma_1,\dots, \sigma_n} \l[ \sup_{v\in Q} \frac{1}{n}\langle\sigma, v\rangle \r],
%\end{align}
%where $\sigma=(\sigma_1,\dots,\sigma_n)$. (See Figure \ref{lec6:fig:rs-innerprod} for an illustration of this idea.) This perspective will be helpful later on when proving bounds on the empirical Rademacher complexity.




\subsec{Rademacher complexity is translation invariant}
A useful fact is that both empirical Rademacher complexity and average Rademacher complexity are translation invariant. (This is not obvious when thinking of how translation affects the picture in Figure \ref{lec6:fig:rs-innerprod}.)

\begin{proposition}
Let $\cF$ be a family of functions mapping $Z \mapsto \R$ and define $\cF' = \{f'(z) = f(z) + c_0\mid f\in \cF\}$ for some $c_0\in\R$. Then $R_S(\cF) = R_S(\cF')$ and $R_n(\cF) = R_n(\cF')$.
\end{proposition}

\begin{proof}
We will prove here that empirical Rademacher complexity is translation invariant.
\begin{align}
R_S(\cF') &= \E_{\sigma_1,\dots, \sigma_n} \l[ \sup_{f'\in \cF'} \frac{1}{n} \sum^n_{i=1} \sigma_i f(z_i) \r] \\
&= \E_{\sigma_1,\dots, \sigma_n} \l[ \sup_{f\in \cF} \frac{1}{n} \sum^n_{i=1} \sigma_i (f(z_i)+c_0) \r] \\
&= \E_{\sigma_1,\dots, \sigma_n} \l[ \frac{1}{n} \sum^n_{i=1} \sigma_i c_0 + \sup_{f\in \cF} \frac{1}{n} \sum^n_{i=1} \sigma_i f(z_i) \r] \\
&= \E_{\sigma_1,\dots, \sigma_n} \l[\sup_{f\in \cF} \frac{1}{n} \sum^n_{i=1} \sigma_i f(z_i) \r] = R_S(\cF), \label{lec6:eqn:rs-translation}
\end{align}
where \eqref{lec6:eqn:rs-translation} follows because $\E_{\sigma_1,\dots,\sigma_n} \frac{1}{n}\sum_{i=1}^n \sigma_i c_0 = 0$, since the $\sigma_i$'s are Rademacher random variables.
\end{proof}


	 % reset section counter
%\setcounter{section}{0}
\metadata{8}{David Lin and Jinhui Wang}{Feb.~8th, 2021}

\sec{Covering number upper bounds Rademacher complexity}
In Chapter \ref{chap:gen-bounds}, we will prove Rademacher complexity bounds that hinge on elegant, ad-hoc algebraic manipulations that may not extend to more general settings. Here, we consider a more fundamental approach for proving empirical Rademacher complexity bounds based on coverings of the output space. The trade-off is generally more tedium.

The first important observation is that for purposes of computing the \textbf{empirical} Rademacher complexity on samples $z_1, ..., z_n$, 
\al{
    R_S(\cF) = \Exp_\sigma \sbr{\sup_{f \in \cF} \frac 1 n \sum_{i=1}^n \sigma_i f(z_i)},
}
we only care about each function's $f \in \cF$ behavior on $\{z_1, ..., z_n\}$. Hence, we can forget the rest of the input space and characterize $f \in \cF$ by its outputs $(f(z_1),\dots, f(z_n))$. Thus, there is a paradigm shift from the space of all functions $\cF$ to the \textit{output space}
\begin{equation}
\cQ \triangleq \cbr{ \begin{pmatrix} f(z_1), \dots, f(z_n) \end{pmatrix}^\top: f\in \cF} \subseteq \R^n, \label{lec6:eqn:shattercoef}
\end{equation}
which may be drastically smaller than $\cF$. Correspondingly, the empirical Rademacher complexity can be rewritten as a maximization over the output space $\cQ$ instead of the function space $\cF$: 
\al{
    R_S(\cF) &= \Exp_\sigma \sbr{\sup_{v\in \cQ} \frac 1 n \inprod{\sigma, v}}.
}
In other words, the complexity of $\cF$ can be also interpreted as how much the vectors in $Q$ can be correlated with a random vector $\sigma.$ See Figure \ref{lec6:fig:rs-innerprod} for an illustration of this idea. One can also view $\Exp_\sigma \sbr{\sup_{v\in \cQ} \frac 1 n \inprod{\sigma, v}}$ as a complexity measure for the set $Q$. If we replace $\sigma$ by a Gaussian vector with spherical covariance, then the corresponding quantity (without the $\frac 1 n$ scaling), $\Exp_{g\sim N(0,I)} \sbr{\sup_{v\in \cQ} \inprod{g, v}}$, is often referred to as the Gaussian complexity of the set $Q$. (It turns out that Gaussian complexity and Rademacher complexity are closely related.)

Another corollary of this is that the empirical Rademacher complexity only depends on the functionality of $\cF$ but not on the exact parameterization of $\cF$ . For example, suppose we have two parameterizations $\cF = \left\{f(x)=\sum \theta_{i} x_{i} \mid \theta \in \mathbb{R}^{d}\right\}$ and $\cF' = \left\{f(x)=\sum \theta_{i}^{3} \cdot w_{i} x_{i} \mid \theta \in \R^{d}, w \in \mathbb{R}^{d}\right\}$. Since $Q_\cF$ and $Q_{\cF'}$ are the same, we see that $R_S(\cF) = R_S(\cF')$ since our earlier expression for $R_S(\cF)$ only depends on $\cF$ through $Q_\cF$. 

\begin{figure}[ht!]
	\begin{center}
		\includegraphics[width=.5\textwidth]{figures/remark2.png}
	\end{center}
	\caption{We can view empirical Rademacher complexity as the expectation of the maximum inner product between $\sigma$ and $v\in Q$.}
	\label{lec6:fig:rs-innerprod}
\end{figure}

\paragraph{Rademacher complexity of finite hypothesis classes.} In practice, we cannot directly evaluate the Rademacher complexity, so we instead bound its value using quantities that are computable. Given finite $|\cQ|$, we often rely on the following bound, which is also known as Masssart's finite lemma: 
\begin{proposition}
    Let $\cF$ be a collection of functions mapping $Z \mapsto \mathbb{R}$ and let $\cQ$ be defined as in \eqref{lec6:eqn:shattercoef}. Assume that $\sigma^2_n = n^{-1} \Exp \left[\max_{f \in \cF } \sum_{i = 1}^n f(X_i)^2 \right] < \infty$. Then,
    \begin{align}
        R_S(\cF) \leq \sqrt{\frac{2 \sigma^2_n \log |\cQ|}{n}}
    \end{align}
    \label{lec6:prop:massartlemma}
\end{proposition}
We prove a (slightly) simplified version of this result in Problem 3(c) of Homework 2, so we omit the proof of Massart's lemma here. In practice, we rarely apply Massart's lemma directly since $|\cQ|$ is typically infinite. In the sequel, we discuss alternative approaches to bounding the Rademacher complexity that are appropriate for this setting.

\paragraph{Bounding Rademacher complexity using $\epsilon$-covers.}
When $|\cQ|$ is infinite, we can use the same discretization trick that we used to prove the generalization bound for an infinite-hypothesis space. Instead of trying to cover the parameter space, we try to cover the output space. To this end, we firstly recall a few definitions concerning $\epsilon$-covers.

\begin{definition}
$\cC$ is an \emph{$\epsilon$-cover} of $\cQ$ with respect to metric $\rho$ if for all $v\in \cQ$, there exists $v'\in \cC $ such that $\rho(v,v')\le \epsilon$.
\end{definition}

\begin{definition}
The \emph{covering number} is defined as the minimum size of an $\epsilon$-cover, or explicitly:
$$N(\epsilon, \cQ, \rho) \overset \triangle = (\text{min size of $\epsilon$-cover of $\cQ$ w.r.t.\ metric $\rho$}).$$
\end{definition}

The standard metric we will use is $\rho(v,v') = \frac 1 {\sqrt{n}} \norm{v-v'}_2$, with the leading coefficient inserted for convenience.

\begin{remark}
While we want to consider $\epsilon$-covers over $\cQ$, the notation in the literature refers to them as $\epsilon$-covers of the function class $\cF$ using the metric $\rho = L_2(p_n)$, i.e.
\begin{equation}
\rho(f,f') = \sqrt{ \frac 1 n \sum_{i=1}^n (f(z_i) - f'(z_i))^2 }
\end{equation}
If we take the corresponding $v,v'\in \cQ$, this is precisely $\rho(v,v') = \frac 1 {\sqrt{n}} \norm{v-v'}_2$.
\end{remark}

Equipped with the notion of $\epsilon$-covers, we can prove the following Rademacher complexity bound:

\begin{theorem}\label{lec8:thm:rc-covering-bd}
Let $\cF$ be a family of functions $Z \mapsto [-1,1]$. Then
\begin{equation}
R_S(\cF) \le \inf_{\epsilon > 0} \rbr{ \epsilon + \sqrt{ \frac {2\log N(\epsilon, \cF, L_2(P_n))} n } }. \label{lec8:eqn:rc-covering-bd}
\end{equation}
\end{theorem}

The $\epsilon$ term can be thought of as the discretization error, while the latter term is the term from Massart's lemma.

\begin{proof}
Fix any $\epsilon > 0$. Let $\cC$ be an $\epsilon$-cover $\cC$ of $\cQ$. Massart's lemma immediately gives the bound
\al{
R_S(\cC) \le \sqrt{ \frac {2\log |\cC|} n }.
}
For every point $v\in \cQ$, we can express it as $v=v'+z$, where $v'\in \cC$ and $z$ is small (specifically, $\frac 1 {\sqrt{n}} \norm{z}_2 \le \epsilon$). This gives
\al{
    \frac 1 n \inprod{v, \sigma} &= \frac 1 n \inprod{v',\sigma} + \frac 1 n \inprod{z, \sigma}\\
    &\le \frac 1 n \inprod{v', \sigma} + \frac 1 n \norm{z}_2 \norm{\sigma}_2 
        &&\text{(Cauchy-Schwarz)} \label{lec8:eqn:cs-step}\\
    &\le \frac 1 n \inprod{v', \sigma} + \epsilon.
        &&\text{(since $\norm{z}_2\le \sqrt{n}\epsilon$ and $\norm{\sigma}_2 \le \sqrt{n}$)}
}
Taking the expectation of the supremum on both sides of this inequality gives
\al{
    R_S(\cF) &= \Exp_\sigma \sbr{\sup_{v\in \cQ} \frac 1 n \inprod{v,\sigma} }\\
    &\le \Exp_\sigma \sbr{\sup_{v'\in \cC} \rbr{\frac 1 n \inprod{v',\sigma} + \epsilon}}\\ 
    &\le \sqrt{ \frac {2\log \abs{\cC}} n } + \epsilon. &\text{(Massart's lemma)}
}
Choosing $\cC$ to be a minimal $\epsilon$-cover allows us to set $|\cC| = N(\epsilon, \cF , L_2(p_n))$. Since the argument above holds for any $\epsilon > 0$, we can take the infimum over all $\epsilon$ to arrive at Equation \eqref{lec8:eqn:rc-covering-bd}, completing the proof.

\end{proof}

\sec{Chaining and Dudley's theorem}

While Theorem \ref{lec8:thm:rc-covering-bd} is useful, the bound in Equation \eqref{lec8:eqn:cs-step} is rarely tight as $z$ might not be perfectly correlated with $\sigma$. It is possible to obtain a stronger theorem by constructing a chained $\epsilon$-covering scheme. Specifically, when we decompose $v=v'+z$, we can construct a finer-grained covering of the ball $B(v',\epsilon)$, and then we can decompose $z$ into smaller components and so on (see Figure \ref{lec8:fig:chained cover} for an illustration).

\begin{figure}[H]
    \centering
    \includegraphics[scale = 0.4]{figures/chaining.png}
    \caption{Illustration of a chained cover. Within the $\epsilon$-ball containing the discretization error $z$, we find a finer $\epsilon'$-cover and obtain a smaller error $z'$ from discretizing $z$.}
    \label{lec8:fig:chained cover}
\end{figure}

Using this method of chaining, we can obtain the following (stronger) result:

\begin{theorem}[Dudley's Theorem]
If $\cF$ is a function class from $Z$ to $\R$, then

\begin{equation}\label{lec9:eqn:dudley}
    R_S(\mathcal{F})\leq 12\int_{0}^{\infty}\sqrt{\frac{\log N(\epsilon, \mathcal{F}, L_2({P_n}))}{n}}d\epsilon.
\end{equation}

\end{theorem}

We can interpret this bound as removing the discretization error term by averaging over different scales of $\epsilon$. For a proof of this theorem, refer to Theorem 15 of \cite{percynotes}.

If $\mathcal{F}$ consists of functions bounded in $[-1,1]$, then, we have that for all $\epsilon > 1, N(\epsilon, \mathcal{F}, L_2({P_n}))=1$. (To see this choose $\{f\equiv 0\}$, which is a complete cover for $\epsilon>1$.) Hence, the limits of integration in \eqref{lec9:eqn:dudley} can be truncated to $[0,1]$:
    
    \begin{equation}
    R_S(\mathcal{F})\leq 12\int_{0}^{1}\sqrt{\frac{\log N(\epsilon, \mathcal{F}, L_2({P_n}))}{n}}d\epsilon,
    \end{equation}
    
since $\log N(\epsilon, \mathcal{F}, L_2(P_n))=0$ for $\epsilon >1$.

\subsec{Covering number regimes for which Dudley's theorem is finite}

Of course, the bound in \eqref{lec9:eqn:dudley} is only useful if the integral on the RHS is finite. Here are some setups where this is the case (we continue to assume that the functions in $\cF$ are bounded in $[-1, 1]$):

\begin{enumerate}
\item If $N(\epsilon, \mathcal{F}, L_2(P_n))\approx (1 / \epsilon)^R$ (ignoring multiplicative and additive constants), then we have $\log N(\epsilon, \mathcal{F}, L_2(P_n))\approx  R\log (1/\epsilon)$. We can plug this into the RHS of \eqref{lec9:eqn:dudley} to get
        
\begin{equation}
\int_{0}^{1}\sqrt{\frac{\log N(\epsilon, \mathcal{F}, L_2({P_n}))}{n}}d\epsilon = \int_{0}^1\sqrt{\frac{R\log(1/\epsilon)}{n}}d\epsilon \approx \sqrt{\frac{R}{n}}.
\end{equation}
            
\item If the covering number has the form $N(\epsilon, \mathcal{F}, L_2(P_n))\approx a^{R/\epsilon}$ for some $a$, then we have $\log N(\epsilon, \mathcal{F}, L_2(P_n)) \approx \frac{R}{\epsilon}\log a$. The bound in \eqref{lec9:eqn:dudley} becomes
        
\begin{align}
\int_0^1\!\!\sqrt{\frac{\log N(\epsilon, \mathcal{F}, L_2(P_n))}{n}}d\epsilon &\approx \int_0^1\!\!\sqrt{\frac{R}{n\epsilon}\log a}\, d\epsilon \\
&= \sqrt{\frac{R}{n}\log a} \int_0^1\!\!\sqrt{\frac{1}{\epsilon}}d\epsilon \\
&= \tilO \l(\sqrt{\frac{R}{n}}\r).
\end{align}
        
\item If the covering number has the form $N(\epsilon, \mathcal{F}, L_2(P_n))\approx a^{R/\epsilon^2}$, then $\log N(\epsilon, \mathcal{F}, L_2(P_n))\approx \frac{R}{\epsilon^2}\log a$. In this case we have:
        
\begin{equation}\int_0^1\sqrt{\frac{\log N(\epsilon, \mathcal{F}, L_2(P_n))}{n}}d\epsilon \approx \sqrt{\frac{R}{n}\log a} \underbrace{\int_0^1\frac{1}{\epsilon}d\epsilon}_{=\infty}=\infty,
\end{equation}

i.e. the bound in \eqref{lec9:eqn:dudley} is vacuous. This is because of the behavior of $\epsilon \mapsto 1/\epsilon^2$ near 0: the function goes to infinity too quickly for us to upper bound its integral. Fortunately, there is an ``improved'' version of Dudley's theorem that is applicable here:
        
\begin{theorem}[Improved Dudley's Theorem]\label{lec9:thm:better-dudley}
If $\cF$ is a function class from $Z$ to $\R$, then for any fixed cutoff $\alpha \geq 0$ we have the bound
\begin{equation}\label{lec9:eqn:better-dudley}
R_S(\mathcal{F})\leq 4\alpha + 12\int_{\alpha}^{\infty}\sqrt{\frac{\log N(\epsilon, \mathcal{F}, L_2({P_n}))}{n}}d\epsilon.      
\end{equation}
\end{theorem}
The proof of this theorem is similar to the proof of the original Dudley's theorem, except that the iterative covering procedure is stopped at the threshold $\epsilon = \alpha$ at the cost of the extra $4\alpha$ term above.
        
Theorem \ref{lec9:thm:better-dudley} allows us to avoid the problematic region around $\epsilon=0$ in the integral in \eqref{lec9:eqn:dudley}. If we let $\alpha = 1/poly(n)$, the bound in \eqref{lec9:eqn:better-dudley} becomes
        
\begin{align}
R_S(\mathcal{F}) &\leq \frac{1}{poly(n)} + \frac{\sqrt{R\log a}}{\sqrt{n}}\int_{\alpha}^1\frac{1}{\epsilon}d\epsilon \\
&= \frac{1}{poly(n)}  + \frac{\sqrt{R\log a}}{\sqrt{n}} \log(1/\alpha) \\
&= \tilO \l(\sqrt{\frac{R}{n}}\r).
\end{align}
\end{enumerate}

In summary, we have that $R_S(\mathcal{F}) \leq \tilO\l(\sqrt{\frac{R}{n}}\r)$ for these three dependencies on $\epsilon$: when $\log N(\epsilon, \mathcal{F}, L_2({P_n})) \approx R\log (1/\epsilon),\ \frac{R}{\epsilon} \log a,\text{ or } \frac{R}{\epsilon^2} \log a$ for some $a$. Note that if the dependence on $\epsilon$ is $1/\epsilon^c$ for $c > 2$, then even the improved Dudley's theorem does not help us. This is because the $\log(1/\alpha)$ term above becomes $\alpha^{1-c/2}$, and when $\alpha = 1/poly(n)$, this term leads to a bad dependence on $n$.

\subsec{Regimes where we can get covering number bounds}
The previous remarks discuss how strong our bounds on covering number need to be in order to get a useful result. Here we mention some situations in which we know how to obtain these covering number bounds:

\begin{enumerate}
\item Covering number and corresponding Rademacher complexity bounds for linear models are well-known, but fairly technical (see \cite{zhang2002}).
    
\item Covering numbers interact nicely with composition by Lipschitz functions. If $\phi$ is a $\rho$-Lipschitz function, then the following bound holds:
\begin{equation}\label{lec9:eqn:covering-num-lipschitz}
N(\epsilon/\rho, \mathcal{F}, L_2({P_n}))\geq N(\epsilon, \phi\circ\mathcal{F}, L_2({P_n})).
\end{equation}
    
This result is the analog of Talagrand's lemma for covering numbers. The proof follows easily if one considers $\phi$ as a change of measure: informally, the Lipschitz condition on $\phi$ means that a distance of $\epsilon/\rho$ in the original space $\mathcal{F}$ can be increased to at most $\epsilon$ in the space $\phi \circ \mathcal{F}$.

\item Using these results we can obtain a bound on the Rademacher complexity of a dense neural network. Consider a deep network
\begin{equation}
f(x) = W_r\sigma(W_{r-1}\sigma(\cdots \sigma(W_1x)\ldots),
\end{equation}

where $W_i$ are layer-wise weights and $\sigma$ is an activation function which is 1-Lipschitz. For this setup we have the following Rademacher complexity bound:
\begin{equation}
R_S (\cF) \leq \underbrace{\l(\prod_{i=1}^r\|W_i\|_{\textup{op}} \r)}_{\text{relatively large}} \cdot \underbrace{\l( \sum_{i=1}^r\frac{\|W_i^\top\|^{2/3}_{2,1}}{\|W_i\|_{\textup{op}}^{2/3}}\r)^{3/2}}_{\text{relatively small}}.
\end{equation}
        
Here $\|W\|_{\textup{op}}$ is the operator norm (or spectral norm) of $W$, and $\|W_i^\top\|_{2,1}$ denotes the sum of the $l_2$ norms of the rows of $W_i$. The second term is relatively small as it is a sum of matrix norms, and so the bound is dominated by the first term, which is a product of matrix norms. This first term comes from composition of Lipschitz functions as in \eqref{lec9:eqn:covering-num-lipschitz} above, since the Lipschitz constant of a linear operator is its spectral norm. The full details are presented in \cite{bartlett2017}.
\end{enumerate}


\sec{VC dimension and its limitations}
We will focus on classification and will be working within the framework of supervised learning stated in Chapter \ref{chap:supervised}. The labels belong to the output space $\mathcal{Y} = \{-1, 1\}$, each classifier is a function $h:\mathcal{X}\to\R$ for all $h \in \cH$, and the prediction is the sign of the output, i.e. $\hat{y} = \sgn(h(x))$. We will look at zero-one loss, i.e. $\err((x,y), h) = \mathbbm{1}(\sgn(h(x))\neq y)$. Note that we can re-express the loss function as
\begin{equation}
\err((x,y), h) = \frac{1-\sgn(h(x))y}{2}.
\end{equation}

The first approach is to reason directly about the Rademacher complexity of $\err$ loss, i.e. considering the family of functions $\cF = \left\{ z = (x, y) \mapsto \err((x, y), h) : h \in \cH \right\}$. Define $Q$ to be the set of all possible outputs on our dataset: $Q=\left\{\left(\sgn\left(h\left(x^{(1)}\right)\right), \dots, \sgn \left(h\left(x^{(n)}\right)\right)\right)\mid  h \in \cH \right\}$. Then, using our earlier remark about viewing the empirical Rademacher complexity as an inner product between $v\in Q$ and $\sigma$, we have
\begin{align}
R_S(\cF) &= \Exp_{\sigma_1,\dots, \sigma_n} \l[ \sup_{f\in \cF} \frac{1}{n} \sum^n_{i=1} \sigma_i \frac{1-\sgn(h(x^{(i)}))y_i}{2} \r] \\
&= \Exp_{\sigma_1,\dots, \sigma_n} \l[ \sup_{f\in \cF} \frac{1}{n} \sum^n_{i=1} \sigma_i \frac{\sgn(h(x^{(i)}))}{2} \r] \\
&= \frac{1}{2}\Exp_{\sigma_1,\dots, \sigma_n} \l[ \sup_{v\in Q} \frac{1}{n} \langle \sigma, v\rangle \r].
\end{align}

Notice that the supremum is now over $Q$ instead of $\cF$. If $n$ is sufficiently large, then it is typically the case that $|Q|>|\cF|$. To see why this is the case, note that each function $f$ corresponds to a single element in $Q$. However, as $n$ increases, $|Q|$ increases as well. For any particular $v\in Q$, notice that $\langle v, \sigma\rangle$ is a sum of bounded random variables, so we can use Hoeffding's inequality to obtain
\begin{equation}
\Pr\left[\frac{1}{n}\langle\sigma, v\rangle\geq t\right] \leq \exp (-n t^2 / 2).
\end{equation}
Taking the union bound over $v\in Q$, we see that 
\begin{equation}
\Pr\left[\exists v\in Q \text{ such that } \frac{1}{n}\langle\sigma, v\rangle \geq t\right] \leq |Q| \exp (-nt^2 / 2).
\end{equation}
Thus, with probability at least $1-\delta$, it is true that $\sup _{v \in Q} \frac{1}{n}\langle v, \sigma \rangle \leq \sqrt{\frac{2(\log|Q| + \log (2/\delta))}{n}}$. Similarly, we can show that $\Exp \left[ \sup _{v \in Q} \frac{1}{n}\langle v, \sigma \rangle \right] \leq O\l(\sqrt{\frac{\log|Q| + \log (2/\delta)}{n}}\r)$ holds.

The key point to notice here is that the upper bound on $R_S(\cF)$ depends on $\log |Q|$. \textit{VC dimension} is one way that we deal with bounding the size of $Q$ We will not delve into the details of this approach (for those interested, see Section 3.11 of \cite{percynotes}). VC dimension, however, has a number of limitations. For one, we will always end up with a bound that depends somehow on the dimension. For linear models, we obtain a bound $\log |Q| \lesssim d \log n$, corresponding to a bound on Rademacher complexity that looks like
\begin{equation}
R_S(\cF) \leq \tilO \left( \sqrt{\frac{d}{n}} \right),
\end{equation}
so we still have a $\sqrt{d}$ term. This will not be a good bound for high-dimensional models. For general models, we will arrive a bound of the form 
\begin{equation}
R_S(\cF) \leq \tilO \left( \sqrt{\frac{\text{\# of parameters}}{n}} \right).
\end{equation}
This upper bound only depends on the number of parameters in our model, and does not take into the account the scale and norm of the parameters. Additionally, this doesn't work with kernel methods since the explicit parameterization is possibly infinite-dimensional, and therefore this upper bound becomes useless.

These limitations motivation the use of margin theory, which does take into account the norm of parameters and provides a theoretical basis for regularization techniques such as $L_1$ and $L_2$ regularization.
	
	\chapter{Rademacher Complexity Bounds for Concrete Models and Losses}\label{chap:gen-bounds}
	% reset section counter
%\setcounter{section}{0}

%\metadata{lecture ID}{Your names}{date}
%\metadata{6}{Daniel Do}{February 1st, 2021}
%In this chapter, we will instantiate Rademacher complexity for two important hypothesis classes: linear models and two-layer neural networks. In the process, we will develop margin theory and use it to bound the generalization gap for binary classifiers.
%\tnote{todo}


In this chapter, we will instantiate the Rademacher complexity theory developed in Chapter~\ref{chap:conc} to linear classifiers, and obtain generalization bounds that depend on the margin and norm of the classifier.  In section~\ref{chap:generalization:sec:margin}, we will first present the margin approach for classification problems that are in principle applicable to general hypothesis class. In Section~\ref{lec7:sec:lin_models}, we will bound the Rademacher complexity of linear models and combine it with the margin approach to obtain generalization bounds for linear classifiers.

\sec{Margin-based approach  for classification problems} \label{chap:generalization:sec:margin}

%In order to motivate the definition of margin and its use in analyzing classification problems, we will first clarify some high-level formulation choice for classification problem on what are the hypothesis class and loss functions. 
To motivate the concept of margin and its role in analyzing classification problems, we will first clarify the choice of some key elements in the formulation of classification problems: the hypothesis class and the loss function. 
For example, in~\Cref{example:binary-classficiation-rc}, we considered linear classification, where the model outputs $\sgn(w^\top x \ge 0)$ as the prediction. The most direct approach is to define the family of hypothesis $\cH = \{x\mapsto \sgn(w^\top x \ge 0): w\in \R^d\}$, and the loss function $\ell: \{\pm1\}\times \{\pm 1\}\rightarrow \R$ is defined to be $\ell(\hat{y}, y) =\ind{\hat{y} \neq y}$. One drawback of this formulation is that  the loss function and the model are both discontinuous, which makes them difficult to analyze. More importantly, the hypothesis class is scale-invariant---for any $\alpha> 0$, we have $\sgn(w^\top x \ge 0) = \sgn(\alpha w^\top x \ge 0)$. This means that the formulation does not allow us to consider finer details about the classifier, e.g., the norm of $w$ and the ``confidence'' of the prediction. It's worth noting that $w^\top x$ is the logit in logistic regression, where the model outputs a probability distribution for the two classes. The confidence of the classifier on training examples apriori should somewhat affect the generalization of the classifier, but this formulation does not take it into account.

It turns out that an alternative formulation allows us to take the norm and ``confidence'' of the classifier into account.  We let the hypothesis class be the family of functions that outputs the \textit{real-valued} logits \textit{without the threshold}. For example, for linear classification, we let $\cH = \{x\mapsto w^\top x: w\in \R^d \}$ (or its subsets by restricting the norm of $w$.) We fold the threshold function in the definition of the loss, that is, given a model $h: \cX \rightarrow \R$, let the 0-1 loss for example $(x,y)$ be
\begin{align}
\ellbinary((x,y), h) = \ind{yh(x) \le 0 } \perm \label{eqn:17}
\end{align}
This way, norm constrain on $w$ will be able to affect the complexity of the hypothesis class. However, the loss function is still discontinuous. We will design a surrogate continuous loss function below for analysis.  

%Assume that we are in the same setting as in the previous section. A fundamental problem we face in this setting is that we do not have a continuous loss: everything is discrete in the output space. We need to find a way to reason about the scale of the output. An example of this is logistic regression: the logistic regression model outputs a probability, and when we compare it to the outcome (0 or 1), its closeness to the true output gives us a measure of how confident we are in the prediction.

Now we formalize the notion of margin, which plays an important role in the analysis. 
As shown in Figure \ref{lec6:fig:margin}, intuitively, the black line is a ``better'' or more confident decision boundary than the red line because datapoints are further away from the black boundary than the read boundary.  
We will define the margin of a classifier to be the minimum distance from any point to the decision boundary, and thus the black line has bigger margin than the red line. 
In this section, we will prove that the larger margin and smaller norm will lead to better generalization error. 

\begin{figure}[t]
    \begin{center}
  \includegraphics[width=0.5\textwidth]{figures/margin.png}
  \end{center}
  \caption{The red and black lines are two decision boundaries. The X's are positive examples and the O's are negative examples. The black line has a larger margin than the red line, and is intuitively a better classifier.}
  \label{lec6:fig:margin}
\end{figure}

%\subsec{Formalizing margin theory} \label{sec:formal_margin}
We start with a formal mathematical definition of the margin. We will assume throughout this section that the dataset $\cD = ((x\sp{1}, y\sp{1}), \dots, (x\sp{n}, y\sp{n}))$ is \textit{separable} by the hypothesis class, that is,  exists some $h_\theta\in\cH$ that fits the data with zero error: $\forall i, y^{(i)} = \sgn(h_\theta(x^{(i)}))$. This assumption can be relaxed or weakened, but we assume it to  make the derivation cleaner. The separability assumption is empirically often true because the model family is very expressive and the training error can be close to zero. 

The first definition is the margin of a single example. 
\begin{definition}[(Unnormalized) margin]
The \textit{(unnormalized) margin} of a model $h_\theta$ for example $(x, y)$ is defined as $\margin(x) = yh_\theta(x)$. Typically, margin is only defined on correctedly labeled examples with $\sgn(h_\theta(x)) = y$. Under separability condition, all the datapoints have well-defined margin $\margin(x)\geq 0$. 
\end{definition}

Then, we can define the margin for a training dataset. 
\begin{definition}[Minimum margin] Given a dataset $S = \{(x\sp{1}, y\sp{1}), \dots, (x\sp{n}, y\sp{n})\}$, the \textit{minimum margin} of model $h_\theta$ over the dataset is defined as $\gamma_{\min} \triangleq \min_{i\in [n]} y^{(i)}h_\theta(x^{(i)})$.
\end{definition}

Our final result bound from above the generalization gap by a function of the margin and the parameter norm. In the literature, there are various bounds that involve variants of the definition of the margin. %we could derive based on what margin we use. For this current setting we are using $\gamma_{\min}$, which is the minimum margin, 
For example, some more refined generalization bounds can depend on the margin averaged over the datasets~\citep{srebro2010optimistic,wei2020improved}.

The 0-1 loss in \Cref{eqn:17}  is a discontinuous function of the $\margin(x)$. We introduce a \textit{surrogate loss}, a loss function which approximates zero-one loss but takes the scale of the margin into account. The \textbf{margin loss}, also known as \textbf{ramp loss}, is defined as 
\begin{equation}
    \ell_\gamma(t) = \begin{cases} 
      0, & t\geq \gamma \\
      1, & t\leq 0 \\
      1-t/\gamma, & 0\leq t\leq \gamma
   \end{cases} \label{lec6:eqn:ramp_loss}
\end{equation}

\begin{figure}[ht!]
    \begin{center}
  \includegraphics[width=0.5\textwidth]{figures/margin_loss.png}
  \end{center}
  \caption{Plotted margin loss.\tnotelong{also draw the binary loss; have some label for $x$ axis and $y$ axis}}
  \label{lec6:fig:marginloss}
\end{figure}
Here $\gamma$ is a tunable parameter that will be chosen in the analysis. Figure \ref{lec6:fig:marginloss} plots the ramp loss. For convenience, define $\ell_\gamma((x,y), h) \triangleq \ell_\gamma(yh(x))$. We can view $\ell_\gamma$ as a continuous version of $\err$ that is more sensitive to the scale of the margin on $[0,\gamma]$. Notice that $\err$ is always less than or equal to the $\ell_\gamma$ when $\gamma\geq 0$, i.e.
\begin{equation}
    \err((x,y), h) = \ind{yh(x) < 0}\leq \ell_\gamma(yh(x)) =\ell_\gamma ((x,y), h)
\end{equation}
holds for all $(x,y)\sim P$. Taking the expectation over $(x,y)$ on both sides of this inequality, we see that
\begin{equation}
    L(h) = \Exp_{(x,y)\sim P} \left[ \err((x,y), h) \right] \leq \Exp_{(x,y)\sim P} \left[ \ell_\gamma ((x,y), h) \right].
\end{equation}

Therefore, the population 0-1 loss is bounded by the population margin loss, and so it is sufficient to upperbound the latter, which we will do below. 
Formally, define the population and empirical versions margin losses as:
\begin{equation}
L_\gamma(h) = \Exp_{(x,y)\sim P}\l[ \ell_\gamma((x,y), h)\r], \quad \hat{L}_\gamma(h) = \sum_{i=1}^n\l [\ell_\gamma((x^{(i)},y^{(i)}), h)\r]\perm
\end{equation}
By the machinery of Rademacher complexity (e.g., Corollary \ref{lec6:cor:ggap-rsbound}) we can bound the generalization of the margin loss by the Rademacher complexity of the margin loss: with probability at least $1-\delta$, 
\begin{equation}
L_\gamma(h) - \hat{L}_\gamma(h)\leq 2R_S(\cF) + 3\sqrt{\frac{\log (2/\delta)}{2n}}\,,
\end{equation}
where $\cF = \{(x,y)\mapsto \ell_\gamma((x,y), h)\mid h\in\cH\}$ in the family of margin losses. Note that if we set $\gamma\leq \gamma_{\min}$, then $\forall i, y^{(i)}h(x^{(i)})\geq \gamma_{\min}\ge \gamma$ which implies that $\ell_\gamma((x^{(i)}, y^{(i)}), h)=0$. That is, the training margin loss $\hat{L}_{\gamma}(h) = 0$. 
Therefore, in order to bound $L_\gamma(h)$, it suffices to bound $R_S(\cF)$.
%This follows because by definition of $\gamma_{\min}$, $y^{(i)}h(x^{(i)})\geq \gamma_{\min}$ for any $(x^{(i)}, y^{(i)})\in \cD$. As a result, $\ell_\gamma((x^{(i)}, y^{(i)}), h) = \ell_\gamma(y^{(i)}h(x^{(i)})) = 0$ holds. 

It's only worthwhile going these steps via the margin loss if the Rademacher complexity of the margin loss family $\cF$ can be more amenable to theoretical analysis, which we wil perform next.  We first introduce a tool, called \textit{Talagrand's lemma} or \textit{Lipschitz contraction lemma} that can be used to deal with any Lipschitz loss function or generally Lipschitz composition. We will use to bound $R_S(\cF)$ in terms of $R_S(\cH)$ to remove any dependence on the loss function. 
 
\begin{lemma}[Talagrand's contraction lemma] \label{lec6:lem:talagrand_lemma}
Let $\phi:\R\to\R$ be a $\kappa$-Lipschitz function. Then \begin{equation}
    R_S(\phi\circ \cH)\leq \kappa R_S(\cH),
\end{equation} 
where $\phi\circ\cH = \{z\mapsto \phi(h(z))\mid h\in\cH\}$.
\end{lemma}

We apply the Talagrand's lemma with $\phi(t) = \ell_\gamma(t)$, which is $\frac{1}{\gamma}$-Lipschitz. We can express $\cF$ as $\cF=\ell_\gamma\circ\cH'$ where $\cH' = \{(x,y)\to yh(x)\mid h\in\cH\}$ is the family of the margin (which is a slightly variant of the model family due to the multiplication with the $y$). 
\begin{align}
R_S(\cF) &\leq \frac{1}{\gamma}R_S(\cH') \\
&= \frac{1}{\gamma}\Exp_{\sigma_1,\dots, \sigma_n} \l[ \sup_{h\in \cH} \frac{1}{n} \sum^n_{i=1} \sigma_i y^{(i)}h(x^{(i)}) \r] \\
&= \frac{1}{\gamma}\Exp_{\sigma_1,\dots, \sigma_n} \l[ \sup_{h\in \cH} \frac{1}{n} \sum^n_{i=1} \sigma_i h(x^{(i)})  \r] && \text{bc. $y^{(i)}\sigma_i$ has the same distribution as $\sigma_i$ }\\
&= \frac{1}{\gamma}R_S(\cH).
\end{align}

Putting this all together, we have shown that for $\gamma = \gamma_{\min}$,
\begin{align}
\Err(h) \leq L_\gamma(h) &\leq 0 + O \left( \frac{R_S(\cH)}{\gamma} \right) + O \left( \sqrt{\frac{\log (2 / \delta)}{2n}} \right) \\
&= O \left( \frac{R_S(\cH)}{\min_i y\sp{i} h(x\sp{i}) } \right) + \underbrace{O \left( \sqrt{\frac{\log (2 / \delta)}{2n}} \right)}_{\textup{low order term}}.
\end{align}

%In other words, for training data of the form $S = \{(x\sp{i},y\sp{i})\}_{i=1}^n \subset \mathbb{R}^d \times \{-1,1\}$, a hypothesis class~$\mathcal{H}$ and 0-1 loss, we can derive a bound of the form
%\begin{equation}\label{lec7:eqn:generalization_loss}
%    \text{g} \leq O\left(\frac{R_S(\mathcal{H})}{\gamma_{\mathrm{min}}}\right) + \text{low-order term},
%\end{equation}
%where $\gamma_\mathrm{min}$ is the minimum margin achievable on~$S$ over those hypotheses in $\cH$ that separate the data, and $R_S(\cH)$ is the empirical Rademacher complexity of $\cH$. 
%Such bounds state that simpler models will generalize better beyond the training data, particularly for data that is strongly separable.

%\begin{remark} \label{lec7:rmk:union_bound_margin}
%Consequently, the $\gamma$ we choose to define the hypothesis class is random, which is not a valid choice when thinking about Rademacher complexity! 
	
\noindent{\bf Caveat and Fixes.}
We note that the argument above has a caveat and is not entirely correct. 
In the context of standard Rademacher complexity machinery (e.g., Corollary \ref{lec6:cor:ggap-rsbound}), the dataset $S$ is a  random variable, whereas the loss function $\ell$ and hypothesis class $\cH$ (and the class of losses $\cF$) are  fixed object that cannot depend on the dataset $S$. The technical reason for this requirement is that we need the $\ell_\gamma((x\sp{i}, y\sp{i}), h)$'s to be independent with each other to apply concentration inequalities. This is not true when $\gamma$ depends on the dataset $S$. Therefore, choosing the margin loss $\ell_\gamma$ with $\gamma$ depending on the dataset is not allowed when we invoke Corollary~\ref{lec6:cor:ggap-rsbound}. 

%Technically we cannot apply Talagrand's lemma with a random $\kappa$ (which we took to be $1/\gamma$). 
%Also, when we use concentration inequalities, we implicitly assume that the $\ell_\gamma((x\sp{i}, y\sp{i}), h)$ are independent of each other. That is not the case if $\gamma$ is dependent on the data.

Fortunately, there is a simple fix for this caveat, and in general any similar issues involving a single variable or a constant number of variables could be addressed by an additional layer of union bound as sketched below. 
%We sketch out how one might address this issue below. 
The main idea is to do another union bound over the choice of $\gamma$ so that we have uniform convergence of the set of $\gamma$.  
We choose a grid of $\gamma$ in the log space, that is, define  $\Gamma = \left\{ 2^k: k \in [-B, B] \right\}$ for some sufficiently large $B$ that contains $2B+1$ choices of $\gamma$'s. 
For every fixed $\gamma \in \Gamma$, with probability greater than $1 - \delta/(2B+1)$, we have
\begin{align}
\Err(h) \leq \hatL_\gamma (h) + O \left( \frac{R_S(\cH)}{\gamma} \right) + O \left( \sqrt{\frac{\log \frac{2B+1}{\delta}}{n}} \right).
\end{align}
The, we taking a union bound over all $\gamma \in \Gamma$. Then, with probability greater than $1 - \delta$, for all $\gamma \in \Gamma$, 
\begin{align}
    \Err(h) \leq \hatL_\gamma (h) + O \left( \frac{R_S(\cH)}{\gamma} \right) + O \left( \sqrt{\frac{\log \frac{2B+1}{\delta}}{n}}\right). \label{lec7:eqn:unionboundmargin}
\end{align}
\tnote{sopped editing here}
Last, choose the largest $\gamma \in \Gamma$ such that $\gamma \leq \gamma_{\min}$. Then, for this value of $\gamma$, our desired bound directly follows from the bound in \eqref{lec7:eqn:unionboundmargin}. Namely, we have that $\hatL_{\gamma} (h) = 0$ and $O \left( \frac{R_S(\cH)}{\gamma} \right) = O \left( \frac{R_S(\cH)}{\gamma_{\min}} \right)$. The additional term, $\tilO\left ( \sqrt{\frac{\log B}{n} }\right )$, is the price exacted by the uniform convergence argument required to correct the heuristic bound given in \eqref{lec7:eqn:generalization_loss}.

%\end{remark}

	% reset section counter
%\setcounter{section}{0}

%\metadata{lecture ID}{Your names}{date}
\metadata{7}{Spencer M.~Richards and Thomas Lew}{Feb.~3rd, 2021}

\sec{Linear models}\label{lec7:sec:lin_models}

\subsec{Linear models with weights bounded in $\ell_2$ norm}
We begin with the Rademacher complexity of linear models using weights with bounded $\ell_2$ norm.

\begin{theorem}\label{lec7:thm:l2-thm}
    Let $\mathcal{H} = \{x \mapsto \inprod{w,x} \mid w \in \R^d, \Norm{w}_2 \le B\}$ for some constant $B > 0$. Moreover, assume $\Exp_{x \sim P}\sbr{\Norm{x}_2^2} \leq C^2$, where $P$ is some distribution and $C > 0$ is a constant. Then
    \begin{align}
        R_S(\mathcal{H}) &\le \frac{B}{n} \sqrt{\sum_{i=1}^n \Norm{x\sp{i}}_2^2},  \label{lec7:eqn:linear-sample}
        \intertext{and}
        R_n(\mathcal{H}) &\le \frac{BC}{\sqrt{n}}.  \label{lec7:eqn:linear}
    \end{align}
\end{theorem}

Generally speaking, there are two methods with which we can bound the Rademacher complexity of a model. The first method, which we used in Chapter \ref{chap:uc}, consists of discretizing the space of possible outputs from our hypothesis class, then using a union bound or covering number argument to bound the Rademacher complexity of the model. While this method is powerful and generally applicable, it yields bounds that depend on the logarithm of the cardinality of this discretized output space, which in turn depends on the number of data points~$n$. In the proof below, we will instead use a more elegant, albeit limited technique which does not rely on discretization of the output space.

\begin{proof}
We start with the proof of \eqref{lec7:eqn:linear-sample}. By definition,
\begin{align}
    R_S(\mathcal{H}) 
    &= \E_\sigma\sbr{ \sup_{\Norm{w}_2 \le B} \frac{1}{n} \sum_{i=1}^n\sigma_i \inprod{w,x\sp{i}} }
    \\&= \frac{1}{n} \E_\sigma\sbr{ \sup_{\Norm{w}_2 \le B} \inprod{w,\sum_{i=1}^n\sigma_i x\sp{i}} }
    \\&= \frac{B}{n} \E_\sigma\sbr{ \Norm{\sum_{i=1}^n \sigma_i  x\sp{i}}_2 }
        &&\text{($\textstyle\sup_{\Norm{w}_2 \le B} \langle w,v\rangle =B\Norm{v}_2$)}
    \\&\leq \frac{B}{n} \sqrt{ \E_\sigma\sbr{\Norm{ \sum_{i=1}^n \sigma_i x\sp{i} }_2^2} }
        &&\text{(Jensen's ineq. for $\alpha \mapsto \alpha^2$)} 
    \\&= \frac{B}{n} \sqrt{ \E_\sigma \sbr{\sum_{i=1}^n \rbr{\sigma_i^2 \Norm{x\sp{i}}_2^2 + \inprod{\sigma_ix\sp{i},\sum_{j \ne i}^n \sigma_j x\sp{j}} }} }
    \\&= \frac{B}{n} \sqrt{\sum_{i=1}^n \Norm{x\sp{i}}_2^2}.
        &&\text{($\sigma_i$ indep. and $\E[\sigma_i]=0$)}
\end{align}
This completes the proof of \eqref{lec7:eqn:linear-sample} for the empirical Rademacher complexity. The bound on the average Rademacher complexity in \eqref{lec7:eqn:linear} follows from taking the expectation of both sides to get
\begin{equation}
    R_n(\mathcal{H}) = \E\sbr{ R_S(\mathcal{H}) }
    = \frac{B}{n} \E\sbr{ \sqrt{\sum_{i=1}^n \Norm{x\sp{i}}_2^2} }
    \le \frac{B}{n} \sqrt{ \sum_{i=1}^n \E\sbr{\Norm{x\sp{i}}_2^2} }
    \le \frac{BC}{\sqrt{n}},
\end{equation}
where the first inequality is another application of Jensen's inequality, and the second follows from the assumption $\Exp_{x \sim P}\sbr{\Norm{x}_2^2} \leq C^2$.

\end{proof}

We observe that both the empirical and average Rademacher complexities scale with the upper $\ell_2$-norm bound $\Norm{w}_2 \le B$ on the parameters~$w$, which motivates regularizing the model. However, smaller weights in the model may reduce the margin $\gamma_\mathrm{min}$, which in turn hurts generalization according to \eqref{lec7:eqn:generalization_loss}.

\begin{remark}
Note that if we scale the data by some multiplicative factor, the bound on empirical Rademacher complexity $R_S(\cH)$ will scale accordingly. However, at the same time we expect the margin to scale by the same multiplicative factor, so the bound on the generalization gap in \eqref{lec7:eqn:generalization_loss} does not change. This lines up with our intuition that the bound should not depend on the scaling of the data.
\end{remark}

\subsec{Linear models with weights bounded in $\ell_1$ norm}
Now, we consider linear models again, except we restrict the $\ell_1$-norm of the parameters and assume an $\ell_\infty$-norm bound on the data.

\begin{theorem}\label{lec7:thm:l1-thm}
    Let $\mathcal{H} = \cbr{x \mapsto \inprod{w,x} \mid w \in \R^d, \Norm{w}_1 \le B}$ for some constant $B > 0$. Moreover, assume $\Norm{x\sp{i}}_\infty \leq C$ for some constant $C > 0$ and all points in $S = \{x\sp{i}\}_{i=1}^n \subset \R^d$. Then
    \begin{equation}
        R_S(\mathcal{H}) \leq BC\sqrt{\frac{2\log(2d)}{n}}.
    \end{equation}
\end{theorem}

To prove the theorem, we will need Massart's lemma, which provides a bound for the Rademacher complexity of a finite hypothesis class.

    \begin{lemma}[Massart's lemma]
        Suppose $\mathcal{Q} \subset \R^n$ is finite and contained in the $\ell_2$-norm ball of radius $M\sqrt{n}$ for some constant $M > 0$, i.e.,
        \begin{equation}
            \mathcal{Q} \subset \{v \in \R^n \mid \Norm{v}_2 \leq M\sqrt{n} \}.
        \end{equation}
        Then, for Rademacher variables $\sigma = (\sigma_1,\sigma_2,\dots,\sigma_n) \in \R^n$,
        \begin{equation}
            \E_\sigma \left[ \sup_{v\in \mathcal{Q}} \frac{1}{n}\inprod{\sigma,v} \right] \leq M\sqrt{\frac{2\log|\mathcal{Q}|}{n}}.
        \end{equation}
        As a corollary, if $\mathcal{F}$ is a set of real-valued functions satisfying
        \begin{equation}
            \sup_{f\in\mathcal{F}} \frac{1}{n}\sum_{i=1}^n f(z\sp{i})^2 \leq M^2,
        \end{equation}
        over some data $S = \{z\sp{i}\}_{i=1}^n$, then
        \begin{align}
            R_S(\mathcal{F}) \leq M\sqrt{\frac{2\log|\mathcal{F}|}{n}}, \quad\text{and}\quad
            R_n(\mathcal{F}) \leq M\sqrt{\frac{2\log|\mathcal{F}|}{n}}.
        \end{align}
    \end{lemma}

We will not prove Massart's lemma in detail. The intuition is to use concentration inequalities to bound $\frac{1}{n}\inprod{\sigma, v}$ for fixed $v$, then to use a union bound over the elements $v \in \mathcal{Q}$.

We will now prove Theorem \ref{lec7:thm:l1-thm}:

\begin{proof}[Proof of Theorem \ref{lec7:thm:l1-thm}]
    By definition,
    \begin{align}
        R_S(\mathcal{H}) &= \E_\sigma\sbr{ \sup_{\Norm{w}_1 \le B} \frac{1}{n} \sum_{i=1}^n\sigma_i \inprod{w,x\sp{i}} } \\
        &= \frac{1}{n} \E_\sigma\sbr{ \sup_{\Norm{w}_1\le B} \inprod{w,\sum_{i=1}^n\sigma_i x\sp{i}} } \\
        &= \frac{B}{n} \E_\sigma\sbr{ \Norm{\sum_{i=1}^n \sigma_i  x\sp{i}}_\infty  },
    \end{align}
    
    where the last equality is because $\sup_{\Norm{w}_1 \leq B}\inprod{w,v} = B\Norm{v}_\infty$, i.e., the $\ell_\infty$-norm is the dual of the $\ell_1$-norm, which is a consequence of H\"older's inequality. However, the $\ell_\infty$-norm is difficult to simplify further. Instead, we use the fact that $\sup_{\Norm{w}_1 \leq 1} \inprod{w,v}$ for any $v \in \R^d$ is always attained at one of the vertices $\mathcal{W} = \bigcup_{i=1}^d \{-e_i,e_i\}$, where $e_i \in \R^d$ is the $i$-th coordinate unit vector. Defining the restricted hypothesis class $\bar{\mathcal{H}} = \{x \mapsto \inprod{w,x} \mid w \in \mathcal{W}\} \subset \mathcal{H}$, this yields
    \begin{align}
        R_S(\mathcal{H}) &= \frac{1}{n} \E_\sigma\sbr{ \sup_{\Norm{w}_1 \le B} \inprod{w,\sum_{i=1}^n\sigma_i x\sp{i}} } \\
        &= \frac{B}{n} \E_\sigma\sbr{ \max_{w\in\mathcal{W}} \inprod{w,\sum_{i=1}^n\sigma_i x\sp{i}} } \\
        &= BR_S(\bar{\mathcal{H}}).
    \end{align}
    
    In particular, the model class $\bar{\mathcal{H}}$ is bounded and finite with cardinality $|\bar{\mathcal{H}}| = 2d$. This suggests using Massart's lemma to complete the proof. To do so, we need to confirm that $\mathcal{\bar{H}}$ is bounded with respect to the $\ell_2$-metric. Indeed, since the inner product of $x\sp{i}$ with a coordinate vector $e_j$ just selects the $j$-th coordinate of $x\sp{i}$, for any $w \in \mathcal{W}$ we have
    \begin{equation}
        \frac{1}{n}\sum_{i=1}^n \inprod{w,x\sp{i}}^2 \leq \frac{1}{n}\sum_{i=1}^n \Norm{x\sp{i}}^2_\infty \leq \frac{1}{n}\sum_{i=1}^n C^2 = C^2,
    \end{equation}
    where the last inequality uses the assumption $\Norm{x_i}_\infty \leq C$. So $\bar{\mathcal{H}}$ is bounded in the $\ell_2$-metric and finite, thus by Massart's Lemma we have
    \begin{equation}
        R_S(\mathcal{H}) = B R_S(\bar{\mathcal{H}}) \leq BC\sqrt{\frac{2\log|\bar{\mathcal{H}}|}{n}} = BC\sqrt{\frac{2\log(2d)}{n}},
    \end{equation}
    which completes the proof.
\end{proof}

\subsec{Comparing the bounds for different $\cH$}

First, we note that for this hypothesis class of linear models, it is possible to obtain an upper bound proportional to $\sqrt{d/n}$ using the VC~dimension, which grows quickly with the data dimension~$d$. Our bound is better since it does not have as strong of a dependence on~$d$, and accounts for the norms of our model parameters and the data.

In the two subsections above, we considered two different hypothesis classes of linear models, each restricting different norms. In both cases, the bound on the average Rademacher complexity depended on the product of the norm bound on the parameters $w$ and the norm bound on each data point $x$. To determine which choice of hypothesis class is better, consider the bounds
    \begin{equation*}
        \Norm{w}_2\Norm{x}_2 \quad\text{vs.}\quad \Norm{w}_1\Norm{x}_\infty
    \end{equation*}
    and see how they compare in different settings. We consider 3 settings here:
    
    \begin{itemize}
    \item Suppose $w$ and $x$ are random variables with $w_i$ and $x_i$ close to the set of values $\{-1,1\}$. Then we have
    \begin{equation*}
        \sqrt{d}\cdot \sqrt{d} \quad\text{vs.}\quad d\cdot 1.
    \end{equation*}
    In this case, there is no difference in using either linear hypothesis class.
    
    \item If we additionally suppose $w$ is sparse with at most $k$ non-zero entries, then we have
    \begin{equation*}
        \sqrt{k}\cdot\sqrt{d} \quad\text{vs.}\quad k\cdot 1.
    \end{equation*}
    So for $d \gg k$, we have $\sqrt{kd} \gg k$ and thus $\ell_1$-norm regularization leads to a better complexity bound when $w$ is suspected to be sparse. Indeed, $\sqrt{d}\Norm{x}_\infty \approx \Norm{x}_2$ when the entries of $x$ are somewhat uniformly distributed, and so in the sparse case we have
    \begin{equation}
        \Norm{w}_2\Norm{x}_2 \geq \sqrt{d}\Norm{w}_2\Norm{x}_\infty \geq \Norm{w}_1\Norm{x}_\infty. 
    \end{equation}
    
    \item On the other hand, if $w$ is dense in the sense that $\Norm{w}_2\approx {\sqrt{d}}\Norm{w}_1$ (i.e., if all entries in $w$ are close to each other in magnitude), then
    \begin{equation}
        \Norm{w}_2\Norm{x}_2 \leq \frac{1}{\sqrt{d}}\Norm{w}_1 \cdot \sqrt{d} \Norm{x}_\infty \leq \Norm{w}_1\Norm{x}_\infty.
    \end{equation}
    In this case, it makes sense to regularize the $\ell_2$-norm instead.
    \end{itemize}
    
    In practice, other multiplicative factors enter the generalization bound, so regularizing both the $\ell_1$- and $\ell_2$-norms of the model parameters $w$ is preferable.

    Continuing with this rough style of analysis, for the hypothesis class with restricted $\ell_2$-norm, we can write the bound on the generalization gap in \eqref{lec7:eqn:generalization_loss} as
    \begin{equation}
        \text{generalization loss} \lesssim \frac{\Norm{w}_2\Norm{x}_2}{\sqrt{n}\gamma_{\mathrm{min}}} + \text{low-order term}.
    \end{equation}
    The presence of $\Norm{w}_2/\gamma_{\mathrm{min}}$ motivates both the minimum norm and the maximum margin formulations of the Support Vector Machine (SVM) problem as good methods to improve generalization performance of binary classifiers.

%*****************************************************************************
\sec{Two-layer neural networks}
We now compute a bound for the Rademacher complexity of two-layer neural networks.  Throughout this section, we use the following notation:
\begin{itemize}
    \item $\theta = (w, U)$ are the parameters of the model with $w \in \R^m$ and $U \in \R^{m \times d}$, where $m$ denotes the number of hidden units. We use $u_i\in\R^d$ to denote the $i$-th row of $U$ (written as a column vector).
    \item $\phi(z) = \max(z, 0)$ is the ReLU activation function applied element-wise.
    \item $f_\theta(x) = \inprod{w,\phi(Ux)} = w^\top \phi(Ux)$ is the model.
    \item $\{ (x\sp{i}, y\sp{i}) \}_{i=1}^n$ is the training set, with $x\sp{i}\in\R^d$ and $y\sp{i}\in\R$.
\end{itemize}
We start with a somewhat weak bound which introduces the technical tools we need to derive tighter bounds subsequently.

\begin{theorem}\label{lec7:thm:thm_3}
    For some constants $B_w > 0$ and $B_u > 0$, let
    \begin{equation}
        \mathcal{H} = \cbr{ f_\theta \mid \Norm{w}_2 \leq B_w,\ \Norm{u_i}_2 \leq B_u,\ \forall i \in \{1,2,\dots,m\} }, \label{lec7:eqn:thm_3}
    \end{equation}
    and suppose $\E\sbr{\Norm{x}_2^2} \leq C^2$. Then
    \begin{align}
        R_n(\mathcal{H}) \le 2 B_w B_u C\sqrt{\frac{m}{n}}.
    \end{align}
\end{theorem}

This bound is not ideal as it depends on the number of neurons~$m$. Empirically, it has been found that the generalization error does \emph{not} increase monotonically with~$m$. As more neurons are added to the model, thereby giving it more expressive power, studies have shown that generalization is improved \cite{belkin2019}. This contradicts the bound above, which states that more neurons leads to worse generalization. We also note that the theorem can be generalized straightforwardly to the setting where the $w$ and $U$ are jointly constrained in the sense that we set $\mathcal{H} = \cbr{ f_\theta \mid \Norm{w}_2\cdot \left(\max_i\Norm{u_i}_2\right) \leq B}$ and obtain the generalization bound $        R_n(\mathcal{H}) \le 2 B C\sqrt{\frac{m}{n}}.$ However, the $\sqrt{m}$ dependency still exists under this formulation of $\cH$. 
Nevertheless, we now derive this bound.

\begin{proof}
    By definition,
    \begin{align}
        R_S(\mathcal{H}) 
        &= \E_\sigma\sbr{ \sup_\theta \frac{1}{n} \sum_{i=1}^n \sigma_i \inprod{w,\phi(Ux\sp{i})} }
        \\&= \frac{1}{n} \E_\sigma\sbr{ \sup_{U : \Norm{u_j}_2 \leq B_u} \sup_{\Norm{w}_2 \leq B_w} \inprod{w,\sum_{i=1}^n \sigma_i \phi(Ux\sp{i})} }
        \\&= \frac{B_w}{n}\E_\sigma\sbr{ \sup_{U : \Norm{u_j}_2 \leq B_u} \Norm{ \sum_{i=1}^n \sigma_i \phi(Ux\sp{i})}_2 }
            &&\text{($\textstyle\sup_{\Norm{w}_2\leq B}\inprod{w,v} = B\Norm{v}_2$)}
        \\&\leq \frac{B_w\sqrt{m}}{n}\E_\sigma\sbr{ \sup_{U : \Norm{u_j}_2 \leq B_u} \Norm{ \sum_{i=1}^n \sigma_i \phi(Ux\sp{i})}_\infty }
            &&\text{($\Norm{v}_2 \leq m\Norm{v}_\infty$)}
        \\&= \frac{B_w\sqrt{m}}{n}\E_\sigma\sbr{ \sup_{U : \Norm{u_j}_2 \leq B_u} \max_{1\leq j\leq m} \abs{ \sum_{i=1}^n \sigma_i \phi(u_j^\top x\sp{i})} } 
        \\&= \frac{B_w\sqrt{m}}{n}\E_\sigma\sbr{ \sup_{\Norm{u}_2 \leq B_u} \abs{ \sum_{i=1}^n \sigma_i \phi(u^\top x\sp{i})} }
        \\&\leq \frac{2B_w\sqrt{m}}{n}\E_\sigma\sbr{ \sup_{\Norm{u}_2 \leq B_u} \sum_{i=1}^n \sigma_i \phi(u^\top x\sp{i}) }
            &&\text{(by Lemma \ref{lec8:lemma:absfortwo})} \label{lec7:eqn:nn-proof1}
        \\&\leq \frac{2B_w\sqrt{m}}{n}\E_\sigma\sbr{ \sup_{\Norm{u}_2 \leq B_u} \sum_{i=1}^n \sigma_i u^\top x\sp{i} }, \label{lec7:eqn:nn-proof2}
    \end{align}
    where the last inequality follows by applying the contraction lemma (Talagrand's lemma) and observing that the ReLU function is $1$-Lipschitz. (Observe that the expectation in \eqref{lec7:eqn:nn-proof1} is the Rademacher complexity for $\{ x \mapsto \phi(u^\top x) \mid \Norm{u}_2 \leq B_u \}$: this is the family that we are applying the contraction lemma to.)
    
    We now observe that the expectation in \eqref{lec7:eqn:nn-proof2} is the Rademacher complexity of the family of linear models $\{x \mapsto \inprod{u,x} \mid \Norm{u}_2\leq B_u\}$. Thus, applying Theorem~\ref{lec7:thm:l1-thm} yields
    \begin{equation}
        R_S(\mathcal{H}) \leq \frac{2B_w\sqrt{m}}{n}B_u\sqrt{\sum_{i=1}^n \Norm{x\sp{i}}_2^2}.
    \end{equation}
    
    Taking the expectation of both sides and using similar steps to those in the proof of Theorem~\ref{lec7:thm:l1-thm} gives us
    \begin{align}
        R_n(\mathcal{H})  &= \E\left[ R_S(\mathcal{H})\right] \\
        &\leq \frac{2B_wB_u\sqrt{m}}{n} \E\sbr{\sqrt{\sum_{i=1}^n \Norm{x\sp{i}}_2^2}} \\
        &\leq \frac{2B_wB_u\sqrt{m}}{n} C\sqrt{n} \\
        &= 2 B_w B_u C\sqrt{\frac{m}{n}},
    \end{align}
    which completes the proof.
    
\end{proof}

This upper bound is undesirable since it grows with the number of neurons $m$, contradicting empirical observations of the generalization error decreasing with $m$.

%*****************************************************************************

\subsec{Refined bounds}
\newcommand{\boundsforcomp}{B}
Next, we look at a finer bound that results from defining a new complexity measure. A recurring theme in subsequent proofs will be the functional invariance of two-layer neural networks under a class of rescaling transformations. The key ingredient will be the \textit{positive homogeneity} of the ReLU function, i.e.
\begin{equation}
\alpha \phi(x) = \phi(\alpha x) \qquad \forall \alpha > 0.
\end{equation}
This implies that for any $\lambda_i > 0$ ($i = 1, \dots, m$), the transformation $\theta = \{(w_i, u_i)\}_{1 \leq i \leq m} \mapsto \theta' = \{(\lambda_i w_i,  u_i / \lambda_i )\}_{1 \leq i \leq m}$ has no net effect on the neural network's functionality (i.e. $f_{\theta} = f_{\theta'}$) since 
\begin{equation}
w_i\cdot \phi \left(u_i^\top x\sp i \right) = (\lambda_i w_i) \cdot \phi\l(\l( \frac{u_i}{\lambda_i}\r)^\top x\sp i\r).   
\end{equation}
In light of this, we devise a new complexity measure $C(\theta)$ that is also invariant under such transformations and use it to prove a better bound for the Rademacher complexity. This positive homogeneity property is absent in the complexity measure used in the hypothesis class \eqref{lec7:eqn:thm_3} of Theorem \ref{lec7:thm:thm_3}.

\begin{theorem}\label{lec8:thm:thm-improved-nn-rc}
$\operatorname{Let} C(\theta)=\sum_{j=1}^{m}\left|w_{j}\right|\left\|u_{j}\right\|_{2},$ and for some constant $\boundsforcomp>0$ consider the hypothesis class
\begin{equation}
\mathcal{H}=\left\{f_{\theta} \mid C(\theta) \leq \boundsforcomp\right\}. \label{eqn:H}
\end{equation}
If $\left\|x\sp{i}\right\|_{2} \leq C$ for all $i \in\{1, \ldots, n\},$ then
\begin{equation}
R_{S}(\mathcal{H}) \leq \frac{2 \boundsforcomp C}{\sqrt{n}}.
\end{equation}
\end{theorem}

\begin{remark}
	Compared to Theorem~\ref{lec7:thm:thm_3}, this bound does not explicitly depend on the number of neurons $m$. Thus, it is possible to use more neurons and still maintain a tight bound if the value of the new complexity measure $C(\theta)$ is reasonable. In contrast, the bound of Theorem \ref{lec7:thm:thm_3} explicitly grows with the total number of neurons. In fact, Theorem~\ref{lec8:thm:thm-improved-nn-rc} is strictly stronger than Theorem~\ref{lec7:thm:thm_3} as elaborated below. Note that 
	\begin{align}
		\sum |w_j|\|u_j\|_2 &\le \left(\sum |w_j|^2\right)^{1/2} \left(\sum\|u_j\|_2^2\right)^{1/2} \tag{by Cauchy-Schwarz inequality} \\
		& \le \|w\|_2 \cdot \sqrt{m} \cdot \max_{j}\|u_j\|_2
	\end{align}
	Therefore, if we consider $\cH^1 = \{f_\theta \mid \sum |w_j|\|u_j\|_2\le B'\}$ and $\cH^2 = \{f_\theta \mid \|w\|_2 \cdot \sqrt{m} \cdot \max_{j}\|u_j\|_2 \le B'\}$, then either Theorem~\ref{lec8:thm:thm-improved-nn-rc} on $\cH^1$ or Theorem~\ref{lec7:thm:thm_3} on $\cH^2$ gives the same generalization bound $O(B'/\sqrt{n})$ but $\cH^1 \supset \cH^2$. 
	
	Moreover, Theorem~\ref{lec8:thm:thm-improved-nn-rc} is stronger as we have more neurons---this is because the hypothesis class $\cH$ as defined in~\eqref{eqn:H} is bigger as $m$ increases. Because of this, it's possible to obtain a generalization guarantee that decreases as $m$ increases, as shown in Section~\ref{sec:gen-bounds:decreasing-in-m}. 
	
%	For example, consider solving the constrained problem
%	\begin{equation}
%	\rho_m = \min_\theta C(\theta) 
%	\quad \text{such that}\quad 
%	\text{$f_\theta$ fits the data  $\{(x\sp{i}, y\sp{i})\}_{i=1}^n$.}
%	\end{equation}
%	In this case, $\rho_m$ monotonically decreases as the number of neurons $m$ increases. Indeed, models with more parameters necessarily include models with a lower number of parameters and thus those of lower complexity.  As a result, it is possible to obtain lower complexity models by increasing the number of parameters $m$.
\end{remark}

\begin{proof}[Proof of Theorem~\ref{lec8:thm:thm-improved-nn-rc}]
Due to the positive homogeneity of the ReLU function $\phi$, it will be useful to define the $\ell_2$-normalized weight vector $\bar{u}_j := u_j / \norm{u_j}_2$ so that $\phi\left(u_j^\top x\right) = \norm{u_j}_2 \cdot \phi(\bar{u}_j^\top x)$. The empirical Rademacher complexity satisfies
\allowdisplaybreaks
\al{
R_S(\cH) &= \frac{1}{n}\E_{\sigma}\left[ \sup_{\theta} \sum_{i=1}^n \sigma_i f_{\theta}\left(x\sp{i}\right) \right] \\
&= \frac{1}{n}\E_{\sigma}\left[ \sup_{\theta} \sum_{i=1}^n \sigma_i \left[\sum_{j=1}^m w_j \phi\left(u_j ^ T x\sp{i}\right) \right] \right] &&\text{(by dfn of $f_\theta$)} \\
&=  \frac{1}{n}\E_{\sigma}\left[ \sup_{\theta} \sum_{i=1}^n \sigma_i \left[\sum_{j=1}^m w_j \norm{u_j}_2  \phi\left(\bar{u}_j ^ T x\sp{i}\right) \right] \right]  
    && \text{(by positive homogeneity of $\phi$)}\\
&= \frac{1}{n}\E_{\sigma}\left[ \sup_{\theta}  \sum_{j=1}^m w_j \norm{u_j}_2 \left[ \sum_{i=1}^n \sigma_i  \phi\left(\bar{u}_j ^ T x\sp{i}\right) \right] \right] \\ 
&\leq \frac{1}{n}\E_{\sigma}\left[ \sup_{\theta}  \sum_{j=1}^m |w_j| \norm{u_j}_2 \max_{k \in [n]}\left| \sum_{i=1}^n \sigma_i  \phi\left(\bar{u}_k ^ T x\sp{i}\right) \right| \right] && \l(\because \sum_j \alpha_j \beta_j \leq \sum_j |\alpha_j| \max_{k} |\beta_k|\r) \\ 
&\leq \frac{\boundsforcomp}{n} \E_{\sigma}\sbr{ \sup_{\theta = (w, U)} \max_{k \in [n]} \left| \sum_{i=1}^n \sigma_i  \phi\left(\bar{u}_k ^ T x\sp{i}\right) \right| } && \text{($\because C(\theta) \leq \boundsforcomp$)} \\
&=  \frac{\boundsforcomp}{n} \E_{\sigma}\sbr{ \sup_{\bar{u}: \norm{\bar{u}}_2 = 1} \left| \sum_{i=1}^n \sigma_i  \phi\left(\bar{u} ^ T x\sp{i}\right) \right| } \\
&\le \frac{\boundsforcomp}{n} \E_{\sigma}\sbr{ \sup_{\bar{u}: \norm{\bar{u}}_2 \le 1} \left| \sum_{i=1}^n \sigma_i  \phi\left(\bar{u} ^ T x\sp{i}\right) \right| } \\
&\le \frac{2\boundsforcomp}{n}  \E_{\sigma}\sbr{ \sup_{\bar{u}: \norm{\bar{u}}_2 \le 1} \sum_{i=1}^n \sigma_i  \phi\left(\bar{u} ^ T x\sp{i}\right) } && \text{(see Lemma \ref{lec8:lemma:absfortwo})} \\
&= 2\boundsforcomp R_S(\cH '),
}
where $\cH' = \l\{x \mapsto \phi(\bar{u}^\top x) :  \bar{u} \in \mathbb{R}^d, \norm{\bar{u}}_2 \leq 1 \r\}$. By Talagrand's lemma, since $\phi$ is $1$-Lipschitz, $R_S(\cH') \leq R_S(\cH'')$ where  $\cH'' = \l\{x \mapsto \bar{u}^\top x :  \bar{u} \in \mathbb{R}^d, \norm{\bar{u}}_2 \leq 1 \r\}$ is a linear hypothesis space. Using $R_S(\cH'') \leq \frac{C}{\sqrt{n}}$ by Theorem \ref{lec7:thm:l2-thm} then concludes the proof.

\end{proof}

We complete the proof by deriving the Lemma \ref{lec8:lemma:absfortwo} used in the second last inequality. Notably, the lemma's assumption holds in the current context, since
\al{
\sup_{\theta} \langle \sigma, f_{\theta}(x) \rangle = \sup_{\bar{u}: \norm{\bar{u}}_2 \leq 1} 
\sum_{i=1}^n \sigma_i \phi \l(\bar{u}^\top x\sp i \r)  \geq 0.
}
since one can take $\bar{u} = 0$ for any $\sigma = (\sigma_1, \dots, \sigma_n)$.

\begin{lemma}\label{lec8:lemma:absfortwo}
Let $\sigma = (\sigma_1, ..., \sigma_n)$ and $f_{\theta}(x) = \l(f_{\theta}\l(x\sp{1}\r), ...,  f_{\theta}\l(x\sp{n} \r)\r)$. Suppose that for any $\sigma \in \{\pm 1\}^n$, $\sup_{\theta} \langle \sigma, f_{\theta}(x) \rangle \geq 0$. Then, 
\begin{equation}
\mathbb{E}_{\sigma}\l[ \sup_{\theta}  \l | \langle \sigma, f_{\theta}(x) \rangle \r|  \r] \leq 2 \mathbb{E}_{\sigma}\l[ \sup_{\theta}  \langle \sigma, f_{\theta}(x) \rangle   \r].
\end{equation}
\end{lemma}

\begin{proof}
Letting $\phi$ be the ReLU function, the lemma's assumption implies that $\sup_{\theta} \phi\left(\langle \sigma, f_{\theta}(x) \rangle\right) = \sup_{\theta}\langle \sigma, f_{\theta}(x) \rangle$ for any $\sigma \in \{\pm 1\}^n$. Observing that $|z| = \phi(z) + \phi(-z)$, 
\begin{align}
\sup_{\theta} \abs{\inprod{ \sigma, f_{\theta}(x) }}%
&= \sup_{\theta} \left[ \phi \l(\inprod{ \sigma, f_{\theta}(x) } \r) + \phi \l(\inprod{-\sigma, f_{\theta}(x) } \r)\right] \\
&\le \sup_{\theta}  \phi \l(\inprod{ \sigma, f_{\theta}(x) } \r) +  \sup_{\theta}  \phi \l(\inprod{-\sigma, f_{\theta}(x) } \r)  \\
&= \sup_{\theta} \inprod{ \sigma, f_{\theta}(x) } +  \sup_{\theta}  \inprod{-\sigma, f_{\theta}(x) }. 
\end{align}
Taking the expectation over $\sigma$ (and noting that $\sigma \overset d = -\sigma$), we get the desired conclusion.
\end{proof}



\sec{More implications and discussions on neural networks}
In this section, we discuss practical implications of the refined neural network bound. 

\subsec{Connection to $\ell_2$ regularization}\label{sec:gen-bounds:impliciation}

Recall that margin theory yields
\begin{equation}
\text{for all } \theta, \quad \Err(\theta) \leq \frac{2R_S(\cH)}{\gammamin} + \tilO\l(\sqrt{\frac{\log \l( 2 / \delta \r)}{n}}\r), \label{lec8:eqn:margin-bound}
\end{equation}
with probability at least $1 -\delta$. Thus, Theorem \ref{lec8:thm:thm-improved-nn-rc} motivates us to minimize $\frac{R_S(\cH)}{\gammamin}$ by regularizing $C(\theta)$. Concretely, this can be formulated as the optimization problem 
\al{
\text{minimize} & \qquad C(\theta) = \sum_{j=1}^m |w_j|\cdot \norm{u_j}_2 \nonumber \tag{I} \label{lec8:eqn:opt1} \\ 
\text{subject to} & \qquad \gammamin(\theta)\ge 1, \nonumber
}
or equivalently,
\al{
\text{maximize} & \qquad \gammamin(\theta) \nonumber \tag{II} \label{lec8:eqn:opt2} \\ 
\text{subject to} & \qquad C(\theta)\le 1. \nonumber
}

At first glance, the above seems orthogonal to techniques used in practice. However, it turns out that the optimal neural network from \eqref{lec8:eqn:opt1} is functionally equivalent to that of the new problem:
\al{
\text{minimize} & \qquad C_{\ell_2}(\theta) = \frac{1}{2}\sum_{j=1}^m |w_j|^2 + \frac{1}{2}\sum_{j=1}^m \norm{u_j}_2^2 \nonumber \tag{I*} \label{lec8:eqn:opt1star} \\ 
\text{subject to} & \qquad \gammamin(\theta)\ge 1. \nonumber
}
This is a simple consequence of the positive homogeneity of $\phi$. For any scaling factor $\lambda=(\lambda_1, \dots, \lambda_m)\in \R_+^m$, the rescaled neural network $\theta_\lambda := \{(\lambda_i w_i, u_i/\lambda_i)\}$ has the same functionality as the original neural network $\theta = \{w_i, u_i \}$ (i.e. it achieves the same $\gammamin$). Thus, 
\al{
\min_{\theta} C_{\ell_2}(\theta) &= \min_{\theta} \min_{\lambda} \rbr{ \frac{1}{2}\sum_{j=1}^m \lambda_j^2 |w_j|^2 + \frac{1}{2}\sum_{j=1}^m \lambda_j^{-2}\norm{u_j}_2^2 }\\
&= \min_{\theta}  \sum_{j=1}^m |w_j|\cdot \norm{u_j}_2 \\
&= \min_{\theta}  C(\theta)
}
where we have used the equality case of the AM-GM inequality, attainable by $\lambda_j^* = \sqrt{\frac{\norm{u_j}_2}{|w_j|}}$, in the second step. This equality case also shows that $\theta^* = \{(w_i, u_i ) \}$ is the optimal solution of \eqref{lec8:eqn:opt1} if and only if $\hat{\theta}^* = \theta_{\lambda^*}$ is the optimal solution of \eqref{lec8:eqn:opt1star}---proving that $\hat{\theta}^*$ and $\theta^*$ are functionally equivalent since they only differ by a positive scale factor. 

This connects our $C(\theta)$ regularization to $\ell_2$-norm penalties that are more prevalent in practice. In retrospect, we see this equivalence is essentially due to the positive homogeneity of the neural network which ``homogenizes'' any inhomogeneous objective such as $C_{\ell_2}$. Hence, we can just deal with $C(\theta)$ which is transparently homogeneous.

\subsec{Generalization bounds that are decreasing in $m$} \label{sec:gen-bounds:decreasing-in-m}

Next, we show that the generalization bound given by Theorem \ref{lec8:thm:thm-improved-nn-rc} does not deteriorate with the network width (number of neurons) $m$, which is consistent with experimental results. To this end, the perspective of \eqref{lec8:eqn:opt2} enables us to isolate all dependencies of $m$ in $\gammamin$. Letting $\widehat \theta_m$ denote the minimizer of program \eqref{lec8:eqn:opt2} with width $m$ and defining optimal value $\gamma_m^* = \gammamin\l(\widehat \theta_m\r)$, we can rewrite the margin bound \eqref{lec8:eqn:margin-bound} as 
\begin{equation}
L(\widehat \theta_m) \le \frac{4C}{\sqrt{n}} \cdot \frac{1}{\gamma_m^*} + \text{(lower-order terms)},
\end{equation}
where all dependencies on $m$ are now contained in $\gamma_m^*$. Hence, to show that this bound does not worsen as $m$ grows, we just have to show that $\gamma_m^*$ is non-decreasing in $m$. This is intuitively the case since a neural network of width $m+1$ contains one of width $m$ under the same complexity constraints. The following theorem formalizes this hunch:

\begin{theorem}
Let $\gamma_m^*$ be the minimum margin obtained by solving \eqref{lec8:eqn:opt2} with a two-layer neural network of width $m$. Then $\gamma_m^* \leq \gamma_{m+j}^*$ for all positive integers $j$.
\end{theorem}

\begin{proof}
Suppose $\theta = \{(w_i, u_i)\}_{1 \leq i \leq m}$ is a two-layer neural network of width $m$ satisfying $C(\theta)\le 1$. Then we may construct a neural network $\widetilde \theta = \{(\tilde w_i, \tilde u_i)\}_{1 \leq i \leq m+1}$ of width $m+1$ by simply taking
\al{
(\widetilde w_i, \widetilde u_i) = \begin{cases}
(w_i, u_i) & i\le m, \\
(0,0) & \text{otherwise}
\end{cases}
}
$\widetilde \theta$ is functionally equivalent to $\theta$ and $C(\widetilde \theta) = C(\theta) \le 1$. This means maximizing $\gammamin$ over $\{C(\widetilde \theta): \widetilde \theta\text{ of width }m+1\}$ should give no lower of a value than the maximum of $\gammamin$ over $\{C(\theta): \theta\text{ of width }m\}$.
\end{proof}

\subsec{Equivalence to an $\ell_1$-SVM in $m \to \infty$ limit}

Since $\gamma_m^*$ is non-decreasing in $m$, quantity 
\begin{equation}
\gamma_\infty ^* = \lim_{m\to \infty } \gamma_m^*
\end{equation}
is well-defined. The next interesting fact is that in this $m \to \infty$ limit, $\gamma_{\infty}^*$ of the two-layer neural network is equivalent to the minimum margin of an $\ell_1$-SVM. As a brief digression, we recap the formulation of $\ell_p$-SVMs and discuss the importance of $\ell_1$-SVMs in particular.

Since a collection of data points with binary class labels may not be a priori separable, a \textit{kernel model} first transforms an input $x$ to $\Phi(x)$ where $\Phi: \mathbb{R}^d \to \mathcal{G}$ is known as the \textit{feature map}. The model then seeks a separating hyperplane in this new (extremely high-dimensional) feature space $\mathcal{G}$, parameterized by a vector $\mu$ pointing from the origin to the hyperplane. The prediction of the model on an input $x$ is then a decision score that quantifies $\Phi(x)$'s displacement with respect to the hyperplane:
\begin{equation}
g_{\mu, \Phi}(x) := \l\langle \mu, \Phi(x) \r\rangle.
\end{equation}
Motivated by margin theory, it is desirable to seek the maximum-margin hyperplane under a constraint on $\mu$ to guarantee the generalizability of the model. In particular, a kernel model with an $\ell_p$-constraint seeks to solve the following program:
\al{
\text{maximize} & \qquad \gamma_{min} \coloneqq \min_{i \in [n]} y\sp{i}\langle \mu, \Phi(x\sp{i}) \rangle \\ 
\text{subject to} & \qquad \norm{\mu}_p \le 1. \nonumber
}
Observe that both the prediction and optimization of the feature model only rely on inner products in $\mathcal{G}$. The ingenuity of the SVM is to choose maps $\Phi$ such that $K(x, x') = \l\langle \Phi(x), \Phi(x') \r\rangle$ can be directly computed in terms of $x$ and $x'$ in the original space $\mathbb{R}^d$, thereby circumventing the need to perform expensive inner products in the large space $\mathcal{G}$. Remarkably, this ``kernel trick'' enables us to even operate in an implicit, infinite-dimensional $\mathcal{G}$. 

The case of $p=1$ is particularly useful in practice as $\ell_1$-regularization generally produces sparse feature weights (the constrained parameter space is a polyhedron and the optimum tends to lie at one of its vertices). Hence, $\ell_1$-regularization is an important feature selection method when one expects only a few dimensions of $\cG$ to be significant. Unfortunately, the $\ell_1$-SVM is not kernelizable due to the kernel trick relying on $\ell_2$-geometry, and is hence infeasible to implement. However, our next theorem shows that a two-layer neural network can approximate a particular $\ell_1$-SVM in the $m \to \infty$ limit (and in fact, for finite $m$). For the sake of simplicity, we sacrifice rigor in defining the space $\mathcal{G}$ and convey the main ideas.

\begin{theorem}\label{lec8:thm:thm8.5}
Define the feature map $\phirelu: \mathbb{R}^d \to \mathcal{G}$ such that $x$ is mapped to $\phi(u^\top x)$ for all vectors $u$ on the $d-1$-dimensional sphere $\mathcal{S}^{d-1}$. Informally, 
$$\phirelu(x) := \begin{bmatrix} \vdots \\ \phi(u^\top x) \\ \vdots \end{bmatrix}_{u\in S^{d-1}}$$
is an ``infinite-dimensional vector'' that contains an entry $\phi(u^\top x)$ for each vector $u \in \mathcal{S}^{d-1}$, and we let $\phirelu(x)[u]$ denote the ``$u$''-th entry of this vector. Noting that $\mathcal{G}$ is the space of functions which can be indexed by $u \in S^{d-1}$, the inner product structure on $\mathcal{G}$ is defined by $\langle f, g \rangle = \int_{S^{d-1}} f[u]g[u] du$.

Under this set-up, we have
\begin{equation}
\gamma_{\infty}^* = \gamma_{\ell_1}^*,
\end{equation}
where $\gamma_{\ell_1}^*$ is the minimum margin of the optimized $\ell_1$-SVM with $\Phi = \phirelu$.
\end{theorem}

\begin{proof}

We will only prove the $\gamma_{\infty}^* \leq \gamma_{\ell_1}^*$ direction. (The $\gamma_{\infty}^* \geq \gamma_{\ell_1}^*$ direction requires substantial functional analysis.)

Suppose $\gamma_\infty^*$ is obtained by network weights $(w_1,w_2, \cdots), (u_1, u_2, \cdots)$ where $w_i\in \R, u_i\in \R^d$ (there is a slight subtlety here to be rectified later). Define renormalized versions of $\{w_i\}$ and $\{u_i\}$:
\begin{equation}
\widetilde w_i := w_i\cdot \norm{u_i}_2, \qquad \overline u_i := \frac {u_i} {\norm{u_i}_2}.   
\end{equation}
Note that $\{(\widetilde w_i, \overline u_i)\}$ has the same functionality (and also the same complexity measure $C(\theta)$, margin, etc.) as that of $\{(w_i,u_i)\}$, but now $\overline u_i$ has unit $\ell_2$-norm (i.e. $\bar{u}_i \in \mathcal{S}^{d-1}$). Thus, $\phi(\overline u_i ^\top x)$ can be treated as a feature in $\cG$ and we can construct an equivalent $\ell_1$-SVM (denoted by $\mu$) such that $\widetilde w_i$ is the coefficient of $\mu$ associated with that feature. Since $\widetilde w_i$ must only be ``turned on' at $\overline u_i $, we have 
\al{
\mu[u] = \sum_{i \in \mathcal{S}^{d-1}} \tilde{w}_i \delta(u - \overline u_i),
}
where $ \delta(u)$ is the Dirac-delta function. Given this $\mu$, we can check that the SVM's prediction is
\al{
g_{\mu, \phirelu}(x) &= \int_{S^{d-1}} \mu[u] \phirelu(x)[u] du \\
&= \int_{S^{d-1}}   \sum_{i \in \mathcal{S}^{d-1}} \tilde{w}_i \delta(u - \overline u_i) \phi\left(\overline u ^\top x\right) du \\
&= \sum_{i \in \mathcal{S}^{d-1}}  \tilde{w}_i \phi\left(\overline u_i ^\top x\right) ,
}
which is identical to the output $f_{\{(\widetilde w_i, \overline u_i)\}}(x)$ of the neural network. Furthermore, 
\al{
\norm{\mu}_1 =  \sum_{i=1}^{\infty} |\widetilde w_i| = \sum_{i=1}^{\infty} |w_i|\cdot \norm{u_i}_2 \leq 1,
}
where the last equality holds because $\{(\widetilde w_i, \overline u_i)\}$ satisfies the constraints of \eqref{lec8:eqn:opt2}. This shows that our constructed $\mu$ satisfies the $\ell_1$-SVM constraint. Thus, $\gamma_{\infty}^* \leq \gamma_{\ell_1}^*$ since the functional behavior of the optimal neural network is contained in the search range of the SVM.

\end{proof}

\begin{remark}
How well does a finite dimensional neural network approximate the infinite-dimensional $\ell_1$ network? Proposition B.11 of \cite{wei2020regularization} shows that you only need $n+1$ neurons. Another way to say this is that $\{\gamma_m\}$ stabilizes once $m=n+1$:
\begin{equation}
\gamma_1^* \le \gamma_2^* \le \dots \le \gamma_{n+1}^* = \gamma_\infty^*.
\end{equation}
The main idea of the proof is that if we have a neural net $\theta$ with $(n+2)$ neurons, then we can always pick a simplification $\theta'$ with $(n+1)$ neurons such that $\theta,\theta'$ agree on all $n$ datapoints.

As an aside, this result also resolves the issue in our partial proof. A priori, $\gamma_{\infty}^*$ may not necessarily be attained by a set of weights $\{(\widetilde w_i, \overline u_i)\}$ but the above shows that it is achievable with just $n+1$ non-zero indices.

\end{remark}
	\sec{Deep neural nets (via covering number)}\label{sec:deep_nets}
In Section~\ref{lec9:sec:cover_to_radem}, we discuss how strong our bounds on covering number need to be in order to get a useful result. 
Here we describe some situations in which we know how to obtain these covering number bounds for concrete models such as linear models and neural networks. 

\subsec{Preparation: covering number for linear models}
First, consider the following covering number bound for linear models:

\begin{theorem}[\cite{zhang2002}] \label{lec9:thm:univariate_rad}
Suppose $x^{(1)}, \cdots, x^{(n)} \in \mathbb{R}^d$ are $n$ data points, and $p, q$ satisfies $1/p + 1/q = 1$ and $2 \le p \le \infty$. Assume that $||x^{(i)}||_p \le C$ for all $i$. Let:
\begin{align}
    \cF_q = \{x \mapsto w^\top x : ||w||_q \le B\}
\end{align}
and let $\rho = L_2(P_n)$. Then, $\log N(\epsilon, \cF_q, \rho) \le \l [\frac{B^2C^2}{\epsilon^2}\r ] \log_2 (2d + 1)$. When $p = 2, q = 2$, we further obtain that:
\begin{align}
    \log N(\epsilon, \cF_2, \rho) \le \l [\frac{B^2C^2}{\epsilon^2} \r ] \log_2 (2 \min (n, d ) + 1)
\end{align}
\end{theorem}
\begin{remark}
Applying \eqref{lec9:eqn:rademacherbound_three} to the covering number bound derived above with $R = B^2C^2$, we conclude that the Rademacher complexity of this class of linear models satisfies
\begin{align}
    R_S(\cF_q) &\le \tilO{\left( \frac{BC}{\sqrt{n}} \right)}.
\end{align} 
We also prove this result without relying on Dudley's theorem in Theorem~\ref{lec7:thm:l2-thm}.
\end{remark}
Next, we consider multivariate linear functions as they are building blocks for multi-layer neural networks. Let $M = (M_1, \cdots, M_n) \in \mathbb{R}^{m \times n}$ and $\norm{M}_{2,1} = \sum_{i = 1}^n \norm{M_i}_2$. Then, $\norm{M^\top}_{2,1}$ denotes the sum of the $\ell_2$ norms of the rows of $M$. 
\begin{theorem}\label{lec9:thm:multivariate_rad}
Let $\cF = \{x \to Wx : W \in \mathbb{R}^{m \times d}, ||W^\top||_{2, 1} \le B\}$ and let $C = \sqrt{\frac{1}{n} \sum_{i = 1}^n ||x^{(i)}||_2^2}$. Then, 
\begin{equation}
\log N(\epsilon, \cF, L_2(P_n)) \le \l [\frac{c^2B^2}{\epsilon^2} \r ] \ln (2dm).
\end{equation}
\end{theorem}
\begin{remark}
    In some sense, Theorem~\ref{lec9:thm:multivariate_rad} arises from treating each dimension of the multivariate problem independently. We can view the linear layer as applying $m$ different linear functions. Explicitly, if $W = \begin{pmatrix} w_1^\top \\ \vdots \\ w_m^\top \end{pmatrix}$ and $Wx = \begin{pmatrix} w_1^\top x \\ \vdots \\ w_m^\top x \end{pmatrix}$, then as we expect, $\norm{W^\top}_{2,1} = \sum \norm{w_i}_2$.
\end{remark}


\subsec{Deep neural networks}
In this lecture, we discuss a bound on the Rademacher complexity of a dense neural network. We set up notation as follows: $W_i$ denotes the linear weight matrix at the $i$-th layer of the neural network, we have a total of $r$ layers, and $\sigma$ is the activation function which is 1-Lipschitz (for example, ReLU, softmax, or sigmoid). If the input is a vector $x$, the neural network's output can be represented as follows:

\begin{align}
    f(x) = W_r\sigma(W_{r-1}\sigma(\cdots \sigma(W_1x)\ldots),
\end{align}
Using this notation, we establish an upper bound on the Rademacher complexity of a dense neural network.

\begin{theorem}[\cite{bartlett2017}]
\label{lec10:thm:dnn_rademacher}
Suppose that $\forall i, \norm{x^{(i)}}_2 \leq c$ and let
\begin{align}
    \cF = \{h_\theta : \norm{W_i}_{\text{op}} \leq \kappa_i, \norm{W_i^T}_{2,1} \leq b_i\}.
\end{align}
Then,
\begin{equation}
    R_S (\cF) \leq \frac{c}{\sqrt{n}} \cdot \underbrace{\left(\prod_{i=1}^r \kappa_i \right)}_{\text{relatively large}} \cdot \underbrace{\left( \sum_{i=1}^r\frac{b_i^{2/3}}{\kappa_i^{2/3}}\right)^{3/2}}_{\text{relatively small}}.
\end{equation}
\end{theorem}

\begin{remark}
    We use $\norm{W}_{\textup{op}}$ to denote the operator norm (or spectral norm) of $W$, and $\norm{W_i^\top}_{2,1}$ denotes the sum of the $l_2$ norms of the rows of $W_i$. We note that $f(x) = Wx$ is Lipschitz with a Lipschitz constant of $\norm{W}_{\textup{op}}$. This is because $\norm{f(x)-f(y)}_2 = \norm{Wx-Wy}_2 \leq \norm{W}_{\textup{op}}\norm{x-y}_2$, since $\norm{W}_{\textup{op}} = \max_{x:\norm{x}_2=1}\norm{Wx}_2$. The second term is relatively small as it is a sum of matrix norms, and so the bound is dominated by the first term, which is a product of matrix norms.
\end{remark}

\begin{remark}
    As a corollary of the above theorem, we also get a bound on the generalization error for the margin loss of the following form:
    \begin{equation}
        \texttt{generalization error} \leq \tilde{O}\left(\frac{1}{\gamma_{\min}} \cdot \frac{1}{\sqrt{n}}\cdot c \cdot \left(\prod_{i=1}^r\norm{W_i}_{\textup{op}} \right) \cdot {\left( \sum_{i=1}^r\frac{\norm{W_i^\top}^{2/3}_{2,1}}{\norm{W_i}_{\textup{op}}^{2/3}}\right)^{3/2}}  \right),
    \end{equation}
    where $\gamma_{\min}$ denotes the margin.
\end{remark}
	
First, we motivate the proof by presenting the main idea, and then work through each part of the proof. The main ideas of the proof can be summarized as follows:
    
\begin{itemize}
    \item At a high level, we want to show that the covering number $N(\epsilon, \cF, \rho)$ for a dense neural network is $\leq \frac{R}{\epsilon^2}$. Proving this would enable us to apply the Localized Dudley Theorem to get a bound on the Rademacher Complexity.
    \item To bound the covering number for a dense neural network, we use $\epsilon$-covers to cover each layer of $f$ separately, and then combine them to prove that there exists an $\epsilon$-cover of the original function $f$. 
    \item To combine the epsilon covers of each layer, we use the Lipschitzness of each layer.
    \item We control and approximate the error propagation that is introduced through discretizing each layer using $\epsilon_i$-coverings in order to get a reasonable final $\epsilon$.
\end{itemize}

As a prelude to the proof, let us think of each layer of $f$ as a separate function $f_i$. Note that $f_i$ corresponds to a matrix multiplication and an activation layer. Thus, $f$ can be written as 
\begin{align}
    f = f_r \circ f_{r-1} \circ \cdots \circ f_1,
\end{align}
where each $f_i$ is $\kappa_i$-Lipschitz. Let us also assume, for simplicity, that $f_i(0) = 0$. 

We now derive the $\epsilon$-covering of $f(x)$ in three steps:
\begin{enumerate}
    \item We take points as input to the $i^{th}$ layer, and devise an $\epsilon_i$ covering of the space spanned by each point passed through the function $f_i$.
    \item We use the $N(\epsilon_i, \cF_i, \rho)$ points of the $\epsilon_i$ covering as an input to the $(i+1)$-th layer, where $\cF_i$ is the function family of the $i$-th layer and $\rho = L_2(p_n)$.
    \item Finally, the $\epsilon_r$ covering (i.e covering of the last layer) will be the $\epsilon$ covering of $f(x)$.
\end{enumerate}

Before stating a critical lemma, we establish some definitions. First, define $c_i$ such that all the inputs to the $i^{th}$ layer, $z_1, z_2, \dots, z_n$ are bounded as $\norm{z_j} \leq c_{i-1}$, and therefore $c_0 = c$. We also assume that there exists a function $g$ such that $\log N (\epsilon_i, \cF_i, \rho) \leq g\left(\epsilon_i, c_{i-1}\right)$. Then, we can state the following lemma:

\begin{lemma}
    There exists an $\epsilon$ cover of $\mathcal{F} = \{f(x) : f(x) = f_r \circ f_{r-1} \circ \cdots \circ f_1\}$, where $\epsilon = \epsilon_r + \kappa_r\epsilon_{r-1} + \cdots + \kappa_r\kappa_{r-1}\dots\kappa_2\epsilon_1$, and
    \begin{align}
    \log N(\epsilon, \cF, \rho) \leq \sum_{i=1}^{r} g\left(\epsilon_i, c_{i-1}\right)
    \end{align}
    \label{lec10:lma:additive_cover}
\end{lemma}

\begin{proof}
We define $\cC_1$ to be an $\epsilon_1$-cover of $\cF_1$. For all $f_1' \in \cC_1$, construct $\cC_{2, f_1'}$ to $\epsilon_2$ cover the set $\cF_2 \circ f_1' = \left\{f_2\left(f_1'\left(X\right)\right) : f_2 \in \cF_{2,f_1'} \right\}$. Then, we define $\cC_2 = \cup_{f_1'\in \cC_1}\cC_{2,f_1'}$, and observe that $\cC_2$ a cover of $\cF_2 \circ \cF_1$ with $\epsilon = \epsilon_1 \times \kappa_2 + \epsilon_2$.

We now note that
\begin{align}
    \log \abs{\cC_{2, f_1'}} \leq g\left(\epsilon_2, c_1\right)
\end{align}
Taking a union over all $f_1'$, 
\begin{align}
    \abs{\cC_{2}} &\leq \abs{\cC_{1}} \exp\left(g\left(\epsilon_2, c_1\right)\right) \\
    \log\abs{\cC_{2}} &\leq \log\abs{\cC_{1}} + g\left(\epsilon_2, c_1\right) \leq g\left(\epsilon_1, c_0\right) + g\left(\epsilon_2, c_1\right)
\end{align}
Similarly, given $\cC_k$, for any $f_k' \circ f_{k-1}' \circ \cdots \circ f_1' \in \cC_k$, we construct a $\cC_{k+1, f_k', \dots, f_1'}$ that is an $\epsilon_{k+1}$-covering of $\cF_{k+1} \circ f_k' \circ \cdots \circ f_1'$. We choose $\cC_{k+1} = \cup_{f_i \in \cC_i, i \leq k} C_{k+1, f_k', \dots, f_1'}$. Then, we have
\begin{align}
    \log \abs{\cC_{k+1}} \leq g\left(\epsilon_{k+1}, c_k\right) + \cdots + g\left(\epsilon_1, c_0\right)
\end{align}
Next, we show that for the above construction, the radius of the cover for $\cF$ is
\begin{align}
    \epsilon = \sum_{i=1}^{r} \left(\epsilon_i \prod_{j=i+1}^{r}\kappa_{j}\right).
\end{align}
If we consider a function composition $f_r \circ \cdots \circ f_1 \in \cF_r \circ \cF_{r-1} \circ \cdots \circ \cF_1$, then we know that there exists an $f_1' \in \cC_1$ such that $\rho\left(f_1, f_1'\right) \leq \epsilon_1$. Similarly, we know there exists $f_{2, f_1'}' \in \cC_{2, f_1'}$ such that $\rho\left(f_2' \circ f_1', f_2\circ f_1' \right) \leq \epsilon_2$. Using the triangle inequality, we can say that 
\begin{align}
   \rho\left(f_2' \circ f_1', f_2 \circ f_1\right) &\leq \rho\left(f_2' \circ f_1', f_2 \circ f_1'\right) + \rho\left(f_2 \circ f_1', f_2 \circ f_1\right) \\ 
   &\leq \epsilon_2 + \rho\left(f_2 \circ f_1', f_2 \circ f_1\right) \\ 
   &\leq \epsilon_2 + \kappa_2 \rho\left(f_1', f_1\right) \\ 
   &\leq \epsilon_2 + \kappa_2\epsilon_1z
\end{align}
Using the same logic for a general $k$, we know that there exists $f_{k,f_1',\dots,f_{k-1}'}' \in \cC_k$ such that
\begin{align}
    \rho\left(f_k' \circ f_{k-1}' \circ \cdots \circ f_1', f_k \circ \cdots \circ f_1\right) &\leq \rho\left(f_k' \circ f_{k-1}'\circ \cdots \circ f_1', f_k \circ f_{k-1}'\circ \cdots \circ f_1' \right) \\ 
    &+ \rho\left(f_k \circ f_{k-1}'\circ f_{k-2}' \circ \cdots \circ f_1', f_k \circ f_{k-1}\circ f_{k-2}' \circ \cdots \circ f_1'\right) \\ &+ \cdots + \rho\left(f_k \circ f_{k-1}\circ \cdots \circ f_2 \circ f_1', f_k \circ f_{k-1}\circ \cdots \circ f_1\right) \\ & \leq \sum_{i=1}^{r} \left(\epsilon_i\prod_{j=i+1}^{r}\kappa_{j}\right)
\end{align}
\end{proof}

\begin{proof}[Proof of Theorem~\ref{lec10:thm:dnn_rademacher}]
We now apply Lemma~\ref{lec10:lma:additive_cover} to dense neural networks. Dense neural networks consist of a composition of layers, where each layer is a linear model composed with a 1-Lipschitz activation. Using Theorem~\ref{lec9:thm:multivariate_rad} along with the property that 1-Lipschitz functions will only contribute a factor of at most $1$ (Lemma~\ref{lec9:lma:talagrand}), the covering number of each layer can be bounded by:
\begin{align}
    g\left(\epsilon_i, c_{i-1}\right) = \tilde{O}\left(\frac{c_{i-1}^2b_i^2}{\epsilon_i^2}\right),
\end{align}
where $c_{i-1}^2$ is the norm of the inputs, $b_i^2$ is $\norm{W_i^T}_{2,1}$, and $\epsilon_i^2$ is the radius. From Lemma~\ref{lec10:lma:additive_cover}, we know that 
\begin{align}
    \log N(\epsilon, \cF, \rho) &= \tilde{O}\left(\sum_{i=1}^{r}\frac{c_{i-1}^2b_i^2}{\epsilon_i^2}\right) 
\end{align}
for
\begin{align}
    \epsilon &= \sum_{i=1}^{r} \left(\epsilon_i \prod_{j=i+1}^{r}\kappa_j\right)
\end{align}

We now have a bound on $N(\epsilon, \cF, \rho)$ that relies on $\epsilon_i$'s, but $N(\epsilon, \cF, \rho)$ should only be a function of $\epsilon$. Since we already know that $\epsilon = \sum_{i=1}^{r} \left(\epsilon_i \prod_{j=i+1}^{r}\kappa_j\right)$, we keep $\epsilon$ constant and optimize the upper bound of $N(\epsilon, \cF, \rho)$ over different choices of $\epsilon_i$. To find the optimal $\epsilon_i$, we will first find a lower bound on $N(\epsilon, \cF, \rho)$. We then choose $\epsilon_i$ so that this lower bound is achieved. Ultimately, our optimized $\epsilon_i$ yields a bound on the covering number of the following form: $\log\left(N\left(\epsilon, \cF, \rho\right)\right) \leq \frac{R}{\epsilon^2}$, where $R$ is some constant independent of $\epsilon$. 

We derive this lower bound using Holder's inequality, which states that
\begin{align}
    \langle a,  b \rangle \leq \|a\|_p \|b\|_q
\end{align}
when $\frac{1}{p} + \frac{1}{q} = 1$. Writing out the vectors $a, b$, we get that 
\begin{align}
    \sum_{i}a_ib_i \leq \left(\sum a_i^p\right)^{\frac{1}{p}}\left(\sum b_i^q\right)^{\frac{1}{q}}
\end{align}

Letting $\alpha_i^2 = c_{i-1}^2b_i^2, \beta_i = \prod_{j=i+1}^{r}\kappa_j$. By Holder's inequality, using $p = 3, q = \frac{3}{2}$, we get
\begin{align}
    \left(\sum_{i=1}^{r}\frac{\alpha_i^2}{\epsilon_i^2}\right)\left(\sum_{i=1}^{r}\beta_i\epsilon_i\right)^2 &\geq \left(\sum_{i=1}^{r}\left(\alpha_i\beta_i\right)^{\frac{2}{3}}\right)^{\frac{3}{2}}
\end{align}
\begin{align}
    \sum_{i=1}^{r}\frac{\alpha_i^2}{\epsilon_i^2} &\geq \frac{R}{\epsilon^2},
\end{align}
where $R = \left(\left(\sum_{i=1}^{r}\left(c_{i-1}b_i\prod_{j=i+1}^{r}\kappa_j\right)\right)^{\frac{2}{3}}\right)^{\frac{3}{2}}$. We note that equality holds when 
\begin{align}
    \epsilon_i = \left(\frac{c_{i-1}^2b_i^2}{\prod_{j=i+1}^{r}\kappa_j}\right)^{\frac{1}{3}} \cdot \underbrace{\frac{\epsilon}{\left(\sum_{i=1}^{r}\frac{b_i^{\frac{2}{3}}}{\kappa_i^{\frac{2}{3}}}\right)\prod_{i=1}^{r}\kappa_i^{\frac{2}{3}} }}_{\epsilon'} \label{eqn:lec10:holder_eps_defn}
\end{align}
Using this choice of $\epsilon_i$ and letting $\epsilon'$ equal the second factor in \eqref{eqn:lec10:holder_eps_defn} for notational convenience, we know that the log covering number is (up to a constant factor):
\al{
    \sum_{i=1}^r \frac{c_{i-1}^2b_i^2}{\epsilon_i^2} &= \sum_{i=1}^r \frac{c_{i-1}^2b_i^2(\kappa_{i+1}\cdots\kappa_r)^\frac{2}{3}}{c_{i-1}^\frac{4}{3}b_i^\frac{4}{3}(\epsilon')^2} \\
    &= \sum_{i=1}^r (c_{i-1}b_i\kappa_{i+1}\cdots\kappa_r)^\frac{2}{3}\frac{1}{(\epsilon')^2} \\
    &= c^\frac{2}{3}\sum_{i=1}^r \left(\frac{b_i}{\kappa_i}\right)^\frac{2}{3} \prod_{i=1}^r \kappa_i^\frac{2}{3} \frac{\left(c^\frac{2}{3}\left(\sum_{i=1}^r (\frac{b_i}{\kappa_i})^\frac{2}{3} \prod_{i=1}^r \kappa_i^\frac{2}{3}\right)\right)^2}{\epsilon^2} \\
    &= \left(c^\frac{2}{3}\sum_{i=1}^r \left(\frac{b_i}{\kappa_i}\right)^\frac{2}{3} \prod_{i=1}^r \kappa_i^\frac{2}{3}\right)^3\frac{1}{\epsilon^2} \\
    &= c^2\prod_{i=1}^r \kappa_i^2\left(\sum_{i=1}^r \left(\frac{b_i}{\kappa_i}\right)^\frac{2}{3}\right)^3\frac{1}{\epsilon^2}.
}
Since this log covering number is of the form $R / \epsilon^2$, we can apply \eqref{lec9:eqn:rademacherbound_three} and conclude that
\al{
    \mathcal{R}_S(\cF) \lesssim \sqrt\frac{R}{n}
}
Last, plugging in
\al{
    R = c^2\prod_{i=1}^r \kappa_i^2\left(\sum_{i=1}^r \left(\frac{b_i}{\kappa_i}\right)^\frac{2}{3}\right)^3
}
we obtain the desired result
\al{
    \mathcal{R}_S(\cF) \lesssim \frac{c}{\sqrt n}\prod_{i=1}^r \kappa_i\left(\sum_{i=1}^r \left(\frac{b_i}{\kappa_i}\right)^\frac{2}{3}\right)^\frac{3}{2}.
}

\end{proof}

\chapter{Data-dependent Generalization Bounds for Deep Nets}\label{sec:deep_nets_data_dependent}

In Theorem~\ref{lec10:thm:dnn_rademacher}, we proved the following bound on the Rademacher complexity of deep neural networks:
\begin{align}
    R_S(\cF) \leq \prod_{i = 1}^r \norm{W_i}_{\text{op}} \cdot \mathsf{poly}(\norm{W_i}).
\end{align}
This bound, however, suffers from multiple deficiencies. In particular, it grows exponentially in the depth, $r$, of the network and $\norm{W_i}_{\text{op}}$ measures the worst-case Lipschitz-ness of the network layers over the input space. As a consequence, the bound fails to accurately predict the good generalization properties of deep nets.

In this section, we obtain a tighter bound for $R_S(\cF)$ by modifying our approach to depend upon the realized Lipschitz-ness of the model on the training data. To further motivate this approach, we also note that stochastic gradient descent, i.e. the typical optimization method typically used to fit deep neural networks, prefers models that are more Lipschitz. This preference must be realized by the model \emph{on empirical data}, however, as no learning algorithm has access to the model's properties over the entire data space.

Ultimately, we aim to prove a tighter bound on the population loss that grows polynomially in the Lipschitz-ness of $f \in \cF$. Namely, given that $f_\theta$ is $\kappa$-Lipschitz on $X^{(1)},\dots,X^{(n)}$, we hope to show that:
\begin{align}
    L(\theta) \leq \mathsf{poly} (\kappa, \norm{\theta}).
\end{align}
\tnote{something wrong with the equation}

\paragraph{Uniform convergence with a data-dependent hypothesis class.}
Classical uniform convergence does not have a single consistent definition. So far in this course, given some complexity measure we denote as $\text{comp}(\cdot)$, our uniform convergence results always appear in one of the two following forms: 
\tnote{let's move whp to up front because this phrasing makes the order of the quantifier a bit tricky}
\begin{align}
    L(f) &\leq \frac{\text{comp}(\cF)}{\sqrt{n}} \text{ (w.h.p.) $\forall f \in \cF$} &&\text{(I)} \\
    L(f) &\leq \frac{\text{comp}(f)}{\sqrt{n}} \text{ (w.h.p.) $\forall f \in \cF$} &&\text{(II)}
\end{align}

\begin{remark}
    Most of the results we have obtained so far are of type I, e.g. $\text{comp}(\cF)$ is a Rademacher complexity. We obtain results of type II by considering a restricted set of functions $\cF_C = \{f : \text{comp}(f) \leq C\}$. We then apply a type I bound to $\cF_C$ and take a union bound over all $C$.
\end{remark}

Note, however, that neither of these approaches produce bounds that depend upon the data. By contrast, in the sequel, we will derive a new \textit{data-dependent} generalization bound. These bounds state that with high probability over the choice of the empirical data and for all functions $f$, 
\begin{align}
    L(f) \leq \text{comp}\left (f, \{x\sp{i}, y\sp{i}\}_{i = 1}^n\right)
\end{align}
A further advantage of this approach is that this bound can be used as a regularizer.

\begin{remark}
Although there is no universal consensus on the type of generalization bound we should derive, we can argue that there is no way to improve upon this bound. For example, one might try to use the input distribution $P$ to define our complexity measure, but if we allowed ourselves access to $P$, we could just define $\text{comp}(f, P) = \Exp_P[f(X)]$. In some sense, defining a generalization bound using the true distribution amounts to cheating, so it becomes difficult to define a distibution-dependent generalization bound in a principled way.
\end{remark}

In this new paradigm, we can no longer reduce bounds of type I to bounds of type II using a union bound. To see why, suppose that we have the hypothesis class
\begin{align}
    \cF = \{f: \text{comp}(f, \{x\sp{i}, y\sp{i}\}_{i=1}^n) \leq C)\}
\end{align}
If our complexity measure depends on the empirical data, then so does our hypothesis class, which makes $\cF$ itself a random variable. However, our theorems regarding Rademacher complexity require that the hypothesis class be fixed before we ever see the empirical data.

We get around this by changing the way we think about uniform convergence. Suppose that our new complexity measure is separable, i.e.
\begin{align}
    \text{comp}(f, \{x\sp{i}, y\sp{i}\}_{i=1}^n) = \sum_{i=1}^n g(f, x\sp{i}),
\end{align}
for some function $g$. Then we can consider an \textit{augmented loss}:
\begin{align}
    \tilde{\ell}(f) = \ell(f) \mathbf{1}(g(f, x\sp{i} \leq C))
\end{align}
Suppose we have a region of low complexity in our existing loss function. Because this region is random, so we cannot selectively apply universal convergence. However, we can use our new surrogate loss function $\tilde{\ell}$ in that region. By modifying the loss function in this way, we can still fix the hypothesis class ahead of time, allowing us to apply existing tools to $\tilde{\ell}(f)$. In the sequel, we introduce a particular surrogate ``margin'' that allows us to cleanly apply our previous results to a implicitly data-dependent hypothesis class \cite{wei2019data}.

\sec{All-layer margin}
We next introduce a new surrogate loss called the \textit{all-layer margin} that can also be thought of as a surrogate margin. This loss will essentially zero out high-complexity regions so that we may focus on low-complexity regions. Once we adopt this new loss function, we will be able to apply some of our earlier methods.

Let $f: \R^d \to \R$ be a classification model. Recall that the standard margin is defined as $y f(x)$, with $y$ in $\{-1, 1\}$. We will say that $g_f(x, y)$ is a \textit{generalized margin} if it satisfies
\begin{align}
    g_f(x, y) = \begin{cases}
0,& \text{ if } f(x)y \leq 0 \text{ (an incorrect classification)}\\
> 0,& \text{ if } f(x)y > 0 \text{ (a correct classification)}
\end{cases}.
\end{align}
That is, the generalized margin ``zeroes out" incorrect classifications.

We also define the \textit{$\infty$-covering number} $N_\infty(\epsilon, \cF)$ as the minimum cover size with respect to the metric $\rho = \norm{\cdot}_\infty$, Precisely,
\begin{equation}
\rho(f, f) = \sup_{x \in \mathcal{X}} |f(x) - f'(x)| = \|f - f'\|_\infty
\end{equation}
\begin{remark}
    Notice that $N_\infty(\epsilon, \cF) \geq N(\epsilon, \cF, L_2(P_n))$. This is because the $\ell_\infty$ norm is a more demanding measure of error: $f$ and $f'$ must be close on \textit{every} input, not just the empirical data. That is,
    \begin{equation}
    \sqrt{\frac{1}{n} \sum_{i=1}^n (f(x_i) - f'(x_i))^2} \leq \sup_{x \in \mathcal{X}} |f(x) - f'(x)|.
    \end{equation}
\end{remark}

\begin{lemma}
Suppose $g_f$ is a generalized margin. Let $\cG = \{g_f: f \in \mathcal{F}\}$. Suppose that for some $R$, $\log N_\infty(\epsilon, \cG) \leq \lfloor \frac{R^2}{\epsilon^2} \rfloor$ for all $\epsilon > 0$. (Recall that this is the worst regime that we can tolerate when working with Rademacher complexity.) Then, with probability greater than or equal to $1 - \delta$ over the randomness in the training data, for all $f$ in $\mathcal{F}$ that correctly predict all the examples,
\begin{equation}
L_{0/1} \leq \tilO \l (\frac{1}{\sqrt{n}} \cdot \frac{R}{\min g_f(x\sp{i}, y\sp{i})} \r ) + \tilO\l (\frac{1}{\sqrt{n}}\r ).
\end{equation}
\label{lec11:genmargin-lemma}
\end{lemma}

\begin{proof}
The high-level idea of our proof is to replace $\cF$ with $\cG$ before repeating a standard margin theory argument.

Let $\ell_\gamma$ be the ramp loss, which is 1 for negative values, 0 for values greater than $\gamma$, and a linear interpolation between 1 and 0 for values between 0 and $\gamma$. We define the surrogate loss as $\hat{L_\gamma(\theta)} = \frac{1}{n} \ell_\gamma(g_{f_\theta}(x\sp{i}, y\sp{i}))$, and the surrogate population loss as $\Exp[\ell_\gamma(g_{f_\theta}(x, y))]$. As before, when we used the Rademacher complexity to control the difference between these two functions, we have that
\begin{equation}
L_\gamma(\theta) - \hat{L}_\gamma(\theta) \leq R_S(\ell_\gamma \circ \cG) + \tilO\l (\frac{1}{\sqrt{n}}\r ).
\end{equation}
Next we observe that $\log N(\epsilon, \ell_\gamma \circ \cG, L_2(P_n)) \leq \log N(\epsilon\gamma, G, L_2(P_n))$; this is due to the $\frac{1}{\gamma}$-Lipschitzness of $\ell_\gamma$. The right hand side is also less than $\log N_\infty(\epsilon\gamma, G)$, which is in turn bounded above by $\lfloor \frac{R^2}{\epsilon^2 \gamma^2} \rfloor$ by assumption.
Then, using our results relating the log of the covering number to a bound on the Rademacher complexity (recall \ref{lec9:eqn:rademacherbound_three} and Theorem~\ref{lec9:thm:better-dudley}), we have that $R_S(\ell_\gamma \circ G) \leq \tilO(\frac{R}{\gamma \sqrt{n}})$.
Take $\gamma = \gamma_{\min} = \min_{i} g_\gamma(x\sp{i}, y\sp{i})$.\footnote{A caveat: because $\gamma$ is a random variable, proving this result rigorously requires taking a union bound over a discretized $\gamma$. We sketched out this argument more thoroughly in Remark~\ref{lec7:rmk:union_bound_margin}.} Then we have $\hat{L}_{\gamma_\text{min}} (\theta) \leq 0 + \tilO(\frac{R}{\sqrt{n} \cdot \gamma_\text{min}}) + \tilO(\frac{1}{\sqrt{n}})$, as desired.
\end{proof}
For which $g_f$ can we bound the covering number? If we take $g_f(x, y) = yf(x)$, then the covering number depends on the product $\norm{w_i}_{\text{op}}$, but we originally set out to do better than this. If we have a linear model $w^T x$, the normalized margin, $\frac{y \cdot w^T x}{\norm{w}}$, governs the generalization performance. But how do we normalize for more general models? 

For a deep neural net, a natural normalizer is the product of the Lipschitz constants of the layers. However, we do not want to normalize by a constant that depends only on the function class, so we take a different approach. We interpret the normalized margin as the minimum $\delta$ such that $w(x + \delta y) < 0$. In plain English, how can we find the minimum perturbation that gets our data point across the boundary?

To define this for all layers simultaneously, we define the \textit{all-layer margin}. We will consider a model in which we perturb all of the layers, not just the input (since perturbing that alone turns out not to suffice). We will consider perturbed models $\delta = (\delta_1, ..., \delta_r)$, where each $\delta_i$ is a vector. We incorporate these perturbations into our model as follows:
\begin{align}
    h_1(x, \delta) &= W_1 x + \delta_1 \cdot \norm{x}_2 \\
    h_2(x, \delta) &= \sigma(W_2 \cdot h_1(x, \delta)) + \delta_2 \cdot \norm{h_1(x, \delta)}_2 \\
    &\vdots \\
    f(x, \delta) = h_r(x, \delta) &= \sigma(W_r h_{r - 1}(x, \delta)) + \delta_r \cdot \norm{h_{r - 1}(x, \delta)}_2.
\end{align}
We can then ask: what was the smallest perturbation that changed our decision? That is, let
\begin{align}
    m_f(x, y) \defeq \min_\delta \sqrt{\sum_{i=1}^r ||\delta_i||_2^2} 
\end{align}
such that $f(x, \delta)y \leq 0$ (i.e., obtain incorrect predictions).

Informally, $m_f(x, y)$ is a measure of how hard it is to perturb the model $f$. $f$ can be hard to perturb for two reasons: $f$ is Lipschitz (in its intermediate layers), or $yf(x)$ is large. Even more informally, large margins imply confidence in our predictions, and so it becomes harder to change the model's mind.

We can now introduce our main result regarding the all-layer margin.
\begin{theorem}
With high probability,
\begin{equation}
L_{01}(f) \leq \tilO\l (\frac{1}{\sqrt{n}} \cdot \frac{\sum_{i=1}^n \norm{W_i}_{1, 1}}{\min_i} m_f(x\sp{i}, y\sp{i}\r ) + \tilO\l (\frac{r}{\sqrt{n}}\r ),
\end{equation}
where
$\norm{W_i}_{1, 1}$ is the sum of absolute values of entries of W.\footnote{It is not clear that this is the best bound we can hope for.}
\end{theorem}
In summary, robustness to perturbations in intermediate errors implies good generalization. We can show that this bound is strictly better than the ones we obtained previously; we have $\frac{1}{m_f(x, y)} \leq \frac{\prod \norm{W_i}_{\text{op}}}{f(x)}$.

To prove this theorem, it suffices to bound $N_\infty(\epsilon, \cG)$ by $O(\frac{\sum{||W_i||_{1, 1}}}{\epsilon^2})$. Then we can apply Lemma~\ref{lec11:genmargin-lemma}.

Let $\cF_i = \{ z \mapsto \sigma (W_i z) : ||W_i||_{1, 1} \leq \beta_i \}$. Then $\cF = \cF_r \circ \cF_{r-1} \circ \cdots \circ \cF_1$. Let $m \circ \cF$ denote $\{m_f : f \in \mathcal{F} \}$. Then
\begin{equation}
\log N_\infty\l (\sqrt{\sum_{i=1}^n \epsilon_i^2}, m \circ \cF \r ) \leq \sum_{i=1}^r \log N_\infty(\epsilon_i, \cF_i).
\end{equation}
where $N_\infty(\epsilon_i, \cF_i)$ is defined with respect to the input domain $\mathcal{X} = \{x : \norm{x}_2 \leq 1 \}$.

That is, we only have to find the covering number for each layer, and then we have the covering number for the (all-layer margin of the) composed function class. Notice the key difference: we used $m \circ \cF$ in the above, not $\cF$.

Then, the desired result follows assuming that the following ``decomposition lemma'' holds:

\begin{lemma}
    If $\log N_\infty(\epsilon_i, \cF_i) \leq \lfloor \frac{c_i^2}{\epsilon_i^2} \rfloor$, then if we take $\epsilon_i = \epsilon \cdot \frac{c_i}{\sqrt{\sum_i c_i^2}}$, we obtain:
    \begin{equation}
    \log N_\infty(\epsilon, m \circ \cF) \leq \frac{\sqrt{\sum_i c_i^2}}{\epsilon^2}.
    \end{equation}
\end{lemma}
This result gives the complexity of the composed model in terms of the complexity of the layers, with each $c_i$ given by $\norm{W_i}_{1, 1}$.

For linear models, we can show $N_\infty(\epsilon_i, \cF_i) \leq \tilO\l (\frac{\beta_i^2}{\epsilon^2} \r )$ (where $\beta_i$ is a bound on $\norm{W_i}_{1, 1}$), and this implies our main theorem.

\begin{proof}
Now we will prove a limited form of the decomposition lemma for affine models: $\cF_i = \{ z \mapsto \sigma(w_i z): \norm{w_i}_{1, 1} \leq \beta_i \}$. There are two crucial steps to this problem. First, we prove that $m_f(x, y)$ is 1-Lipschitz in $f$. That is, for all $\cF = \cF_r \circ \cF_{r-1} ... \circ \cF_1$ and $\cF' = \cF_r' \circ 
cF_{r-1}' ... \circ \cF_1'$, $|m_f(x, y) - m_{f'}(x, y)| \leq \sqrt{\sum_{i=1}^r \max_{\norm{x}_2 \leq 1} \norm{f_i(x) - f_i'(x)}_2}$. Notice that now we are working with a clean sum of differences, with no multipliers!

Second, we construct a cover: Let $U_1, \dots, U_r$ be $\epsilon_1, \dots, \epsilon_r$ covers of $\cF_1, ..., \cF_r$, respectively, such that $|U_i| = N_\infty(\epsilon_i, \cF_i)$. By definition, for all $f_i$ in $\cF_i$, there exists a $u_i \in U_i$ such that $\max_{\norm{x} \leq 1} \norm{f_i(x) - u_i(x)}_2 \leq \epsilon_i$. Take $U = U_r \circ U_{r-1} \circ \cdots \circ U_1 = \{u_r \circ u_{r-1} \circ \cdots \circ u_1 \}$ as the cover for $m \circ \cF$. Suppose we were given $f = f_r \circ ... \circ f_1 \in \cF$. Let $u_r, ..., u_1$ be the nearest neighbor of $f_r, ..., f_1$. Then
\begin{equation}
|m_f(x, y) - m_u(x, y)| \leq \sqrt{\sum_{i=1}^r \max_{||x|| \leq 1} ||f_i(x) - u_i(x)||_2^2}.
\end{equation}
Because $f$ and $u$ are close by construction, this is $\leq \sqrt{\sum_{i=1}^r \epsilon_i^2}$.

Having established the validity of our cover, we now return to our claim of 1-Lipschitz-ness in step 1 of this proof. We will show an upper bound for $m_{f'}(x, y) - m_f(x, y)$; the lower bound follows by symmetry.

Let $\delta_1^*, ..., \delta_r^*$ be the optimal choices of $\delta$ in defining $m_f(x, y)$. Our goal is to turn these into $\hat{\delta}_1, ..., \hat{\delta}_r$, a feasible solution of $m_{f'}(x, y)$. $m_{f'}(x, y) \leq \sqrt{\sigma \norm{\hat{\delta}_i}_2} \leftrightarrow \sqrt{\sigma||\delta_i||^2}$; the solution must be feasible for this bound on $m_{f'}(x, y)$ to hold.

We want to make $(f_1', \hat{\delta}_1, ..., \hat{\delta}_r$ behave the same way under perturbations as $(f_1, \delta_1^*, ..., \delta_r^*$. $f$ has parameters $W_1, ..., W_r$ and $f'$ has parameters $W_1', ..., W_r'$. Then,
\begin{align}
    h_1 &= W_1 x + \delta_1^* \norm{x}_2 \\
    h_2 &= \sigma(W_2 h_1) + \delta_2^* \norm{h_1}_2 \\
    &\vdots \\
    h_r &= \sigma(W_r h_{r - 1}) + \delta_r^* \norm{h_{r - 1}}_2
\end{align}
We want to imitate this by perturbing $f'$ in some way. In particular, let
\begin{equation}
    h_1 = W_1'x + \delta_1^* \norm{x}_2 + (W_1 - W_1')x,
\end{equation}
where the last term serves to compensate for the difference between $W_1$ and $W_1'$. Therefore $\hat{\delta}_1^* \defeq \delta_1^* + \frac{(W_1 - W_1')x}{\norm{x}_2}$.
We repeat this for every layer, e.g.,
\begin{equation}
    h_2 = \sigma(W_2' h_1) + \delta_2^*\norm{h_1} + \sigma(W_2 h_1) - \sigma(W_2' h_1),
\end{equation}
and $\delta_2 \cdot \norm{h} = h_2 - \sigma(W_2' h_1)$. So $\hat{\delta_2} = \delta_2^* + \frac{\sigma(w_2 h_1) - \sigma(w_2' h_1)}{\norm{h_1}_2}$. In general, 
\begin{align}
    \hat{\delta}_i = \delta_i^* + \frac{\sigma(W_ih_{i-1}) - \sigma(W_i' h_{i-1})}{\norm{h_{i-1}}_2}
\end{align} 

Then $\hat{\delta}_1, ..., \hat{\delta}_r$ on $f'$ are making the same predictions as $\delta_1, ..., \delta_r$ on $f'$. Next, we prove that this is a feasible solution.
\begin{align}
    m_{f'}(x, y) &\leq \sqrt{\sum ||\hat{\delta}_i||_2^2} \\
    &\leq \sqrt{\sum \norm{\delta_i}_2^2} + \sqrt{\sum \frac{\sigma(W_i h_{i-1}) - \sigma(W_i' h_{i-1})}{\norm{h_{i-1}}_2}}. 
\end{align} 
The last step follows by Minkowski's inequality, which states that $\sqrt{\sum \norm{a_i + b_i}_2^2} \leq \sqrt{\sum \norm{a_i}_2^2} + \sqrt{\sum \norm{b_i}_2^2}$. In this setting, this inequality can also be proved using Cauchy-Schwarz.

But $\sqrt{\sum \norm{\delta_i}_2^2}$ can be bounded above by $m_f(x, y) + \sqrt{\sum_{i=1}^r \max_{\norm{x} \leq 1} (\sigma(Wx)-\sigma(W'x))^2}$; dividing by the 2-norm here is the same as restricting the 2-norm to be 1. This equals $m_f(x, y) + \sqrt{\sum_{i=1}^r \max_{\norm{x} \leq 1} (f_i(x)-f_i'(x))^2}$, which is what we wanted to show.
\end{proof}

\tnote{maybe introduce some remarks below to make the paragraphs below a bit more structured}
We can compare the above with \cite{bartlett2017},
\begin{equation}
f(x, \delta) - f(x) \leq \norm{\delta_r} \cdot \norm{W_{r-1}}_{\text{op}} \cdots \norm{W_1}_{\text{op}} + \norm{W_r}_{\text{op}} \cdot \norm{\delta_{r-1}} \cdots \norm{W_1}_{\text{op}} + \cdots + \norm{W_r}_{\text{op}} \cdots \norm{W_2}_{\text{op}} \cdot \norm{\delta_1}.
\end{equation}
Ignoring minor details (e.g. dependency on $r$), and supposing that $y = 1$, if we want $f(x) > 0$ and $f(x + \delta) \leq 0$, then $\norm{\delta}$ must exceed $\frac{1}{\prod\norm{W_i}_{\text{op}}}$ in order to make up the difference. This says that, approximately, $\frac{m_f(x, y)}{y f(x)} \geq \frac{1}{\prod \norm{W_i}_{\text{op}}}$, so $\frac{1}{m_f(x, y)}$, the inverse margin, is $\frac{1}{yf(x)} \cdot \prod \norm{W_i}_{\text{op}}$. So, we conclude that the former is a tighter bound.

How \textit{much} better is our new bound? Empirically, it seems to be substantially better, since our empirical Lipschitzness is better than the worst-case bound. Another source of hope that this is better is that SGD prefers Lipschitz solutions and Lipschitzness on data points. The algorithm is (in a sense) minimizing Lipschitzness on a data point, making it better than the worst-case Lipschitzness on the entire domain, which likely accounts for the gap between the two bounds. (Implicitly, we are maximizing the all-layer margin.)

Moreover, SAM (sharpness-aware regularization) is a form of perturbation, but on the parameter $\theta$ itself rather than on the intermediate hidden parameters $h_i$. However, these two methods are related! If we consider the (single-example) loss $\frac{\partial \ell}{\partial W_i}$, it equals $\frac{\partial \ell}{\partial h_{i+1}} \cdot h_i^T$. Therefore the norm of the term on the left equals the product of the norms of the two terms of the right, relating the Lipschitzness of the parameters.

Finally, we can prove a more general version in which we do not need to worry about the minimum margin of the entire dataset, and instead find an average. We can show that the test error is bounded above by $\frac{1}{n}$ times $\sqrt{\frac{1}{n} \sum_{i=1}^n \frac{1}{m_f(x\sp{i}, y\sp{i})^2}}$ times the sum of complexities of each layer, plus a lower-order term.


	
	\chapter{Theoretical Mysteries in Deep Learning}
	% reset section counter
\setcounter{section}{0}

%\metadata{lecture ID}{Your names}{date}
\metadata{9}{Rafael Rafailov and Aidan Perreault}{Feb 10th, 2021}

We now turn to a high-level overview of deep learning theory. To begin, we outline a framework for classical machine learning theory, then discuss how the situation is different from deep learning theory.

\sec{Framework for classical machine learning theory}
At the risk of oversimplification, we can divide classical machine learning theory into three parts:

\begin{enumerate}
\item {\bf Approximation theory} attempts to answer whether there is any choice of parameters $\theta$ that achieves low population error. In other words, is the choice of hypothesis class good enough to approximate the ground truth function? Using notation from earlier in this course, the goal is to upper bound $L(\theta^*) = \min_{\theta \in \Theta} L(\theta).$
    
\item {\bf Statistical generalization} focuses on bounding the excess risk $L(\hat{\theta}) - L(\theta^*)$. In Chapter \ref{chap:uc} we obtained the following bound:
    
\begin{equation}
L(\hat{\theta})-L(\theta^*)\leq \underbrace{L(\hat{\theta})-\hat{L}(\hat{\theta})}_{\text{generalization error}} + |L(\theta^*)-\hat{L}(\theta^*)|.
\end{equation}
    
The first term here is the generalization error, which usually has an upper bound of the form $R(\theta)/\sqrt{n}$, where $R(\theta)$ is some complexity measure.\footnote{In earlier chapters, we defined the complexity of a hypothesis class, not of a specific parameter value. To reconcile these two approaches, think of $R$ as a measure of complexity (such as a norm) that we can then use to define a hypothesis class $\Theta$, i.e.~$\Theta = \{\theta' : R(\theta') \le R(\theta)\}$.} This is a demonstration of \textit{Occam's Razor}: the principle that simple (low-complexity) explanations generalize better. 
    
This statistical approach allows us to define a regularized loss  $\hat{L}_{\textup{reg}}(\theta)=\hat{L}(\theta)+\lambda R(\theta)$. Minimizing this loss gives us a solution $\hat{\theta}_\lambda$ which simultaneously has low training error and low complexity, which lets us bound both the training error and the generalization error. To summarize, in the classical setting, we can prove statements of the form
    
\begin{equation}\label{lec9:eqn:classical-guarantee}
\text{If }\hat{\theta}_\lambda \text{ minimizes } \hat{L}_{\textup{reg}},\text{ then } L(\hat{\theta}_\lambda) - L(\theta^*) \text{ is small.}
\end{equation}

\item {\bf Optimization} considers how to obtain the minimizer $\hat\theta$ or $\hat{\theta}_\lambda$ computationally. This usually involves convex optimization: if $\hat{L}$ or $\hat{L}_{\textup{reg}}$ is convex, then we have a polynomial-time algorithm to find the global minimum.
\end{enumerate}

While there are many tradeoffs to consider between these three components (for example, we may be able to find a loss function for which optimization is easy, but generalization becomes worse), it is still possible to study each area individually, then combine all three to get a result.

\sec{Deep learning theory and its differences}
The situation is not so simple for deep learning theory. Let us consider how this is the case for each of the three components described for classical machine learning theory

\begin{enumerate} 
\item {\bf Approximation theory:} Large neural net models are considered to be very expressive. That is, both the population loss $L(\theta^*)$ and the finite sample loss $\hat{L}(\hat\theta)$ can be made small. In fact, neural networks are \textit{universal approximators}; see for example \cite{hornik1991}. This can be a somewhat misleading statement as the definition of universal approximator allows for the size of the network to be impracticably large, but morally it seems to hold true in practice anyway.
        
This expressivity is possible because neural networks are usually highly \textit{over-parametrized}: they have many more parameters than samples. It is possible to prove that in this regime, the network can ``memorize'' the entire dataset and achieve approximately zero training error.
    
\item {\bf Statistical generalization:} Relatively weak regularization is used in practice. In many cases only weak $\ell_2$ regularization is used, i.e.
\begin{equation}
\widehat{L}_{\textup{reg}}(\theta)=\hat{L}(\theta)+\lambda\|\theta\|_2^2.
\end{equation}
    
The first interesting fact is that this regularized loss does not have a unique (approximate) global minimizer. This is due to overparametrization: there are so many degrees of freedom that there are many approximate global minimizers with approximately the same $\ell_2$ norm.
    
However, it turns out that these global minimizers are not equally good: many models which achieve zero training error may have very bad test error. Take, for example, using stochastic gradient descent (SGD) to learn a model to classify the dataset CIFAR-10. Consider two instantiations of this: one starting with a large learning rate and slowly decreasing it, and one with a small learning rate throughout. Even though both instantiations result in approximately zero training error, the former leads to much better test performance. \tnoteimp{add deep-learning-implicit-reg.png in the figure folder. perhaps can also include ``bad global min'' as a demonstration}

Therefore, the goal in deep learning theory is not just to find an arbitrary global minimum: we need to find the right global minimum. This contrasts sharply with \eqref{lec9:eqn:classical-guarantee} from the classical setting, where achieving a global minimum leads to good guarantees on generalization error. This means that \eqref{lec9:eqn:classical-guarantee} is simply not powerful enough to deal with deep learning, because it cannot distinguish between $\theta$'s with good test error and bad test error.

\item {\bf Optimization:} The discussion above means that optimization plays a significant role in generalization for deep learning. Different training algorithms have different ``implicit biases'', causing them to converge to different global minimizers. Understanding the implicit biases of algorithms is thus a central goal of deep learning theory. It is impossible to design a good optimization algorithm without also considering its impact on generalization. In fact, many algorithms for non-convex optimization have been proposed that work well for minimizing training loss, but because their implicit bias is different, they lead to worse test performance and are therefore not too useful.
    
Often these implicit biases can be interpreted as encouraging $\hat\theta$ to have low complexity in some sense. The deep learning analog of  \eqref{lec9:eqn:classical-guarantee} is that ``low complexity solutions generalize''. This means that we end up doing more work on the optimization front in order to understand the implicit bias of our algorithm, and then proving generalization bounds works similarly to the classical setting once we understand how our optimizer finds a low-complexity solution.
    
\end{enumerate}

To explain the success of deep learning, we will cover three tasks in the next two chapters:

\begin{enumerate}
    \item Prove that our optimization algorithm converges to an approximate global minimum, even though the objective function is non-convex. Our results here will mostly be for simplified models (e.g. linearized neural nets). (We will also show later that this can be accomplished separately from the other tasks using a special optimization setup (the ``NTK approach"). However, the generalization of this approach can be poor.)
    
    \item Show that the solution $\hat{\theta}$ has low complexity $R(\hat{\theta})\leq C$. We can only answer this question for some special cases of models (e.g. logistic regression, matrix factorization) and optimizers (e.g. gradient descent, label noise in SGD, dropout, learning rate).
    
    \item Show that for all $\theta$ such that $R(\theta)\leq C$ with $\hat{L}(\theta)\approx 0$, we have $L(\theta)$ is small. That is, we show that low-complexity solutions to the empirical risk problem generalize well. We will be working with more fine-grained complexity measures, and several of the tools we used in classical machine learning can still apply.
\end{enumerate}
	
	\chapter{Nonconvex Optimization}
	% reset section counter
\setcounter{section}{0}

%\metadata{lecture ID}{Your names}{date}
\metadata{10}{Kevin Han and Han Wu}{Feb 17th, 2021}

In the previous chapter, we outlined conceptual topics in deep learning theory and how the situation was different from classical machine learning theory. In particular, we described \textit{approximation theory}, \textit{statistical generalization} and \textit{optimization}. In this chapter, we will focus on optimization theory in deep learning. We will introduce some basics about optimization (Section~\ref{sec:optim_convergence}), discuss how we can make the notion ``all local minima are global minima'' rigorous, and walk through two examples where this is the case (Section~\ref{sec:two_optim_examples}). Finally, we introduce the neural tangent kernel approach which allows us to characterize of the loss of general neural networks near a specific initialization (or under specific parameterization).

\sec{Optimization landscape} \label{sec:optim_intro}

The big question that we have in mind is the following: many existing optimizers are designed for optimizing convex functions. \textbf{Why do they still work well empirically for non-convex functions?} We note that it is not true that these optimizers always work well with non-convex functions: there are still some very hard cases that give trouble (e.g. very deep feed-forward networks are still hard to fit because of issues like vanishing and exploding gradients). One possible reason is that the non-convex functions that we are minimizing in deep learning usually have some nice properties: see Figure \ref{lec10:fig:optimization} for an illustration.

\begin{figure}[ht!]
    \centering
    \includegraphics[scale = 0.5]{figures/landscape.png}
    \caption{Classification of different functions for optimization. The functions we optimize in deep learning seem to fall mostly within the middle cloud.}
    \label{lec10:fig:optimization}
\end{figure}

\begin{figure}[ht!]
    \centering
    \includegraphics[scale = 0.3]{figures/gradient_descent.png}
    \caption{Illustration of how gradient descent does not always find the global minimum. In the picture, gradient descent initialized at the blue point only makes it to the local minimum at the red point: it does not find the global minimum at the black point.}
    \label{lec10:fig:gradient_descent}
\end{figure}
Before diving into details, we first highlight some observations that will be important to keep in mind when discussing optimization in deep learning. Suppose $g(\theta)$ is the loss function. Recall that the \textit{gradient descent (GD)} algorithm would do the following:
\begin{enumerate}
    \item $\theta_0 \defeq$ initialization
    \item $\theta_{t + 1} = \theta_t - \eta\nabla g(\theta_t)$, where $\eta$ is the step size.
\end{enumerate}
Here are some observations to :
\begin{enumerate}
    \item[] \textit{Observation 1}: Gradient descent can find a global minimum for convex functions\footnote{A more precise version of this claim is that gradient descent can find a point that has function value arbitrary close to the global minimal value. } but cannot always find the global minimum for any general continuous functions (see Figure \ref{lec10:fig:gradient_descent} for an illustration).
    \item[] \textit{Observation 2}: Finding the global minimum of general non-convex functions is NP-hard.
%    \item[] \textit{Observation 3}: Gradient descent .
    \item[] \textit{Observation 3}: The objective function in deep learning is non-convex., but empirically gradient descent/stochastic gradient descent typically finds an approximate global minimum of loss function in deep learning.
\end{enumerate}

These observations motivate the following two-step plan:

\begin{enumerate}
    \item Identify a large set of functions that stochastic gradient descent/gradient descent can solve.
    \item Prove that some of the loss functions in machine learning problems belong to this set. (Most of the effort will be spent here.)
\end{enumerate}
\textbf{Basic idea:} Gradient descent can find local minimum $+$ all local minima of $f$ are also global $\Rightarrow$ Gradient descent can find global minima.

\sec{Efficient convergence to (approximate) local minima} \label{sec:optim_convergence}
Let $f$ be a twice-differentiable function. We start with the following definition:
\begin{definition} [Local minimum of a function]
We say that $x$ is a \textit{local minimum} of a function $f$ if there exists an open neighborhood $N$ around $x$ such that in $N$, the function values are at least $f(x)$.
\end{definition}

Note that if $x$ is a local minimum of $f$, then $\nabla f(x) = 0$ and $\nabla^2 f(x) \succeq 0$. However, as the next example shows, the reverse is not true. When $\nabla f(x) = 0$ and $\nabla^2 f(x)$ vanishes in some direction (i.e. merely positive semi-definite instead of being strictly positive definite), higher-order derivatives start to matter.

\begin{example}
\label{lec10:ex:counterexample}
Consider the function $f(x_1, x_2) = x_1^2 + x_2^3$. $(x_1, x_2) = (0, 0)$ satisfies $\nabla f(x) = 0$ and $\nabla^2 f(x)|_{(x_1, x_2) = (0, 0)} = \begin{bmatrix} 2 & 0 \\
0 & 0\end{bmatrix} \succeq 0$. However, if we move in the negative direction of $x_2$, we can decrease the function value. Hence, this example shows why $\nabla f(x) = 0$ and $\nabla^2 f(x) \succeq 0$ does not imply that $x$ is a local minimum.
\end{example}

It is generally not easy to verify if a point is a local minimum. In fact, we have the following theorem regarding the computational tractability:
\begin{theorem}
\label{lec10:thm:np_hard}
It is NP-hard to check whether a point is a local minimum or not \cite{murty1987}. In addition, Hillar and Lim \cite{hillar2013} show that a degree four polynomial is NP-hard to optimize.
\end{theorem}

\subsec{Strict-saddle condition}
Theorem~\ref{lec10:thm:np_hard} forces us to consider more specific types of functions to be able to obtain computational tractability. To this end, we define the following \textit{strict-saddle condition}:

\begin{definition} [Strict-saddle condition \cite{lee2016}]
For positive $\alpha, \beta, \gamma$, we say that $f: \R^d \mapsto \R$ is \textit{$(\alpha, \beta, \gamma)$-strict-saddle} if every $x \in \bbR^d$ satisfies one of the following:
\begin{enumerate}
    \item $\|\nabla f(x)\|_2 \geq \alpha$.
    \item $\lambda_{\min}(\nabla^2 f(x)) \leq -\beta$.
    \item $x$ is $\gamma$-close to a local minimum $x^*$ in Euclidean distance, i.e. $\|x - x^*\|_2 \leq \gamma$.
\end{enumerate}
\end{definition}

Intuitively speaking, this definition is saying if a point has zero gradient and positive semi-definite Hessian, it must be close to a local minimum, i.e. there is no pathological case like Example \ref{lec10:ex:counterexample}.

We have the following theorem for functions that satisfy strict-saddle condition:

\begin{theorem} [Informally stated]
If $f$ is $(\alpha, \beta, \gamma)$-strict-saddle for some positive $\alpha, \beta, \gamma$, then many optimizers (e.g. gradient descent, stochastic gradient descent, cubic regularization) can converge to a local minimum with $\epsilon$-error in Euclidean distance in time $poly \left(d, \frac{1}{\alpha}, \frac{1}{\beta}, \frac{1}{\gamma}, \frac{1}{\epsilon}\right)$.
\end{theorem}

Therefore, if all local minima are global minima and the function satisfies the strict-saddle condition, then optimizers can converge to a global minimum with $\epsilon$-error in polynomial time. (See Figure \ref{lec10:fig:strict-saddle} for an example of a function whose local minima are all global minima.) The next theorem expresses this concretely by being explicit about the strict-saddle condition:

\begin{theorem}
Suppose $f$ is a function that satisfies the following condition: $\exists  \ \epsilon_0, \tau_0, c > 0$ such that if $x \in \bbR^d$ satisfies $\|\nabla f(x)\|_2 \leq \epsilon < \epsilon_0$ and $\nabla^2 f(x) \succeq -\tau_0I$, then $x$ is $\epsilon^c$-close to a global minimum of $f$. Then many optimizers can converge to a global minimum of $f$ up to $\delta$-error in Euclidean distance in time $poly\left(\frac{1}{\delta}, \frac{1}{\tau_0}, d \right)$.
\end{theorem}

\begin{figure}[ht!]
    \centering
    \includegraphics[scale = 0.5]{figures/localmin.png}
    \caption{A two-dimensional function with the property that all local minima are global minima. It also satisfies the strict-saddle condition because all the saddle points have a strictly negative curvature in some direction.}
    \label{lec10:fig:strict-saddle}
\end{figure}

\sec{All local minima are global minima: two examples} \label{sec:two_optim_examples}
So far, we have focused on general results. Next, we give two concrete examples that have the property that all local minima are global minima: (i) principal components analysis (PCA)/matrix factorization/linearized neural nets, and (ii) matrix completion. \tnote{need some quick literature survey; Tengyu will add}%There is a rich literature on this topic and 

\subsec{Principal components analysis (PCA)}
Let matrix $M \in \bbR^{d \times d}$ be symmetric and positive semi-definite. Consider the problem of finding the best rank-1 approximation of the matrix $M$. The objective function here is non-convex:
\begin{equation}
    \min_{x \in \bbR^d}g(x) \triangleq \frac{1}{2}\|M - xx^\top \|_F^2.
\end{equation}

\begin{theorem}
All local minima of $g$ are global minima (even though $g$ is non-convex).
\end{theorem}

\begin{remark}
For $d = 1$, $g(x) = \frac{1}{2}(m - x^2)^2$ for some constant $m$. Figure~\ref{lec10:fig:pca_objective} below shows such an example. We can see that all local minima are indeed global minima.
\end{remark}

\begin{figure}[ht!]
    \centering
    \includegraphics[scale = 0.4]{figures/pca.png}
    \caption{Objective function for principal components analysis (PCA) when $d = 1$.}
    \label{lec10:fig:pca_objective}
\end{figure}

\begin{proof}

\textit{Step 1: Show that all stationary points must be eigenvectors.} From HW0, we know that $\nabla g(x) = -(M - xx^\top )x$, hence
\begin{equation}\label{lec10:eqn:pca-firstorder}
\nabla g(x) = 0 \implies Mx = \|x\|_2^2\cdot x,
\end{equation}
which implies that $x$ is an eigenvector of $M$ with eigenvalue $\|x\|_2^2$. From the Eckart–Young–Mirsky theorem we know the global minimum (i.e. the best rank-1 approximation) is the eigenvector with the largest eigenvalue.

\textit{Step 2: Show that all local minima must be eigenvectors of the largest eigenvalue.} We use the second order condition for this. For $x$ to be a local minimum we need $\nabla^2g(x) \succeq 0$, which means for any $v \in  \bbR^d$, 
\begin{equation}
\langle v, \nabla^2g(x) v \rangle \geq 0.
\end{equation}
To compute $\langle v, \nabla^2g(x) v \rangle$, we use the following trick: expand $g(x + v)$ into $g(x) + \text{linear term in } v + \text{quadratic term in } v$, then the quadratic term will be $\frac{1}{2}\langle v, \nabla^2g(x) v \rangle$ (see HW0 Problem 2d for an example). Using this trick, we get 

\begin{align}
    g(x+v) &= \frac{1}{2}\|M - (x+v)(x+v)^\top \|_F^2 \\
           &= \frac{1}{2}\|M-xx^\top\|_F^2 - \langle M-xx^\top , xv^\top + vx^\top\rangle + \frac{1}{2}\langle xv^\top + vx^\top , xv^\top + vx^\top \rangle \nonumber \\
          & \quad -\langle M-xx^\top, vv^\top\rangle + \text{higher order terms in }v.
\end{align}
Hence, we have 
\begin{align}
    \frac{1}{2}\langle v, \nabla^2g(x) v \rangle & = \frac{1}{2}\langle xv^\top + vx^\top, xv^\top + vx^\top \rangle
          -\langle M-xx^\top, vv^\top\rangle  \\
          &= \langle x, v\rangle^2 + \|x\|_2^2\|v\|_2^2 - v^ Mv + \langle x, v\rangle^2 \\
          & = 2\langle x, v\rangle^2 + \|x\|_2^2\|v\|_2^2 - v^\top Mv.
\end{align}

Picking $v = v_1$, the unit eigenvector with the largest eigenvalue (denoted $\lambda_1$), for $x$ to be a local minimum it must satisfy 
\begin{equation}
\langle v_1, \nabla^2g(x) v_1 \rangle = 2\langle x, v_1 \rangle^2 - v_1^\top Mv_1 + \|x\|_2^2 \geq 0.
\end{equation}

Note that by \eqref{lec10:eqn:pca-firstorder}, all our candidates for local minima are eigenvectors of $M$ so naturally we have two cases:
\begin{itemize}
\item \textit{Case 1: $x$ has eigenvalue $\lambda_1$}. Then x is the global minimum (by the Eckart–Young–Mirsky theorem).
\item \textit{Case 2: $x$ has eigenvalue $\lambda < \lambda_1$}. Then we know $x$ and $v_1$ are orthogonal (eigenvectors with different eigenvalues are always orthogonal), hence 
\begin{equation}
2\langle x, v_1 \rangle^2 - v_1^\top Mv_1 + \|x\|_2^2 = 0  -\lambda_1 + \lambda \geq 0,
\end{equation}
which implies $\lambda \geq \lambda_1$, a contradiction. 
\end{itemize}

In summary, if $x$ is a stationary point and $x$ is not a global minimum, then moving in the direction of $v_1$ would lead to second-order improvement and $x$ cannot be a local minimum. 
\end{proof}

	% reset section counter
%\setcounter{section}{0}

%\metadata{lecture ID}{Your names}{date}
\metadata{11}{Andrew Wang}{Feb 22nd, 2021}

\subsec{Matrix Completion \cite{ge2016}}
We consider rank-1 matrix completion for simplicity. Let $M = zz^T$ be a rank-1 symmetric and positive semi-definite matrix for some $z\in \bbR^d$. Given random entries of $M$, our goal is to recover the rest of entries. Formally, we have the following definitions:

\begin{definition}
Suppose $M\in \bbR^{d\times d}$ and $\Omega \subseteq [d] \times [d]$, we define $P_{\Omega}(M)$ to be the matrix obtained by zeroing out every entry outside $\Omega$. 
\end{definition}

\begin{definition}[Matrix Completion]
Suppose $M\in \bbR^{d\times d}$ and every entry of $M$ is included in $\Omega$ with probability $p$. The \textit{matrix completion task} is to recover $M$ (with respect to some loss functions) given the observation $P_{\Omega}(M)$.
\end{definition}

A nice real world example of matrix completion is when we have a matrix describing the user ratings for each item. We only observe a small portion of the entries as each customer only buys a small subset of the items. A good matrix completion algorithm is indispensable for a recommendation engine. 

\begin{remark}
We need $d$ parameters to describe a rank-1 matrix $M$ and the number of observations is roughly $pd^2$. Thus, for identifiability we need to work in the regime where $pd^2 > d$, i.e. $p \gg \frac{1}{d}$. 
\end{remark}

We define our non-convex loss functions to be 
\begin{align}
    \min_{x \in \bbR^d} f(x) & \triangleq \frac{1}{2}\sum_{(i,j)\in \Omega}(M_{ij}-x_ix_j)^2 \\
     & = \frac{1}{2}\|P_{\Omega}(M-xx^T)\|_F^2.
\end{align}

To really solve our problem we need some regularity condition on the ground truth vector $z$ (recall $M = zz^T$). \textit{Incoherence} is one such condition:
\begin{definition}[Incoherence]
Without loss of generality, assume the ground truth vector $z\in\bbR^d$ satisfies $\|z\|_2 = 1$. $z$ satisfies the \textit{incoherence condition} if $\|z\|_{\infty} \leq \frac{\mu}{\sqrt{d}}$, where $\mu$ is considered to be a constant or log in dimension $d$. 
\end{definition}

\begin{remark}
A nice counterexample to think about why such condition is necessary is when $z = e_1$ and $M = e_1 e_1^T$. All entries of $M$ are 0 except for a 1 in the top-left corner. There is no way to recover $M$ without observing the top-left corner.
\end{remark}

The goal is to prove that local minima of this objective function are close to a global minimum:

\begin{theorem}\label{lec11:thm:matrix-completion}
Assume $p = \dfrac{\textrm{poly}(\mu, \log d)}{d\epsilon^2}$ for some sufficient small constant $\epsilon$ and assume $z$ is incoherent. Then with high probability, all local minima of $f$ are $O(\sqrt{\epsilon})$-close to $+z$ or $-z$ (the global minima of $f$).
\end{theorem}

Before presenting the proof, we make some observations that will guide the proof strategy.

\begin{remark}
$f(x)$ can be viewed as a sampled version of the PCA loss function $g(x) = \frac{1}{2}\norm{M - xx^T}_F^2 = \frac{1}{2}\sum_{(i,j) \in [d]\times[d]} (M_{ij} - x_ix_j)^2$, in which we only observe a subset of the matrix entries. Thus, we would like to claim that $f(x) \approx g(x)$. However, matching the values of $f$ and $g$ is not sufficient to prove the theorem: even a small margin of error between $f$ and $g$ could lead to creation of many spurious local minima (see Figure~\ref{lec11:fig:matrix_completion_f_g} for an illustration). In order to ensure that the local minima of $f$ look like the local minima of $g$, we will need further conditions like $\nabla f(x) \approx \nabla g(x)$ and $\nabla^2 f(x) \approx \nabla^2 g(x)$.
\end{remark}

\begin{figure}
    \centering
    \includegraphics[width=2.5in]{figures/matrix-completion-f-g.png}
    \caption{Even if $f(x)$ and $g(x)$ are no more than $\epsilon$ apart at any given $x$, the local minima of $f$ may look dramatically different from the local minima of $g$.}
    \label{lec11:fig:matrix_completion_f_g}
\end{figure}

\begin{remark}
Key idea: concentration for scalars is easy. We can approximate a sum of scalars via a sample:
\begin{equation}
\sum_{(i,j) \in \Omega} T_{ij} \approx p\sum_{(i,j) \in [d]\times[d]} T_{ij},
\end{equation}
where we use $\approx$ to mean that
\begin{equation}
\Bigl| \sum_{(i,j) \in \Omega} T_{ij} - p\sum_{(i,j) \in [d]\times[d]} T_{ij} \Bigr| < \epsilon
\end{equation}
with high probability. This suggests the strategy of casting the estimation of our desired quantities in the form of estimating a scalar sum via a sample. In particular, we note that for any matrices $A$ and $B$,
\begin{equation}
\langle A, P_\Omega(B) \rangle = \sum_{(i,j) \in \Omega} A_{ij}B_{ij} \approx p\langle A, B \rangle.
\end{equation}
\end{remark}

To make use of this observation to understand the quantities of interest ($\nabla f(x)$ and $\nabla^2 f(x)$), we compute the bilinear and quadratic forms for $\nabla f(x)$ and $\nabla^2 f(x)$ respectively:
\begin{equation}
\langle v, \nabla f(x) \rangle = \langle v, P_\Omega(M-xx^T)x \rangle = \langle vx^T, P_\Omega(M-xx^T) \rangle,
\end{equation}
where we have used the fact that $\langle A,BC \rangle = \langle AC^T,B\rangle$. Also note that $vx^T$ is a rank-1 matrix and $M-xx^T$ is a rank-2 matrix.
\begin{align}
\langle v, \nabla^2 f(x) v \rangle &= \norm{P_\Omega(vx^T + xv^T)}_F^2 - 2\langle P_\Omega(M-xx^T), vv^T \rangle \\
&=  \langle P_\Omega(vx^T + xv^T), vx^T + xv^T \rangle - 2\langle P_\Omega(M-xx^T), vv^T \rangle,
\end{align}

where we have used the fact that $\norm{P_\Omega(A)}_F^2 = \langle P_\Omega(A), P_\Omega(A)\rangle = \langle P(\Omega(A), A\rangle$.

The key lemma that applies the scalar concentration to these matrix quantities is as follows:

\begin{lemma}
Let $\epsilon>0$, $p = \dfrac{\textrm{poly}(\mu, \log d)}{d\epsilon^2}$. Given that $A = uu^T, B=vv^T$ for some $u, v$ satisfying $\norm{u}_2 \leq 1$, $\norm{v}_2 \leq 1$, $\norm{u}_\infty \leq \mu / \sqrt{d}$, $\norm{v}_\infty \leq \mu / \sqrt{d}$, we have $|\langle P_\Omega(A), B \rangle/p - \langle A, B\rangle| \leq \epsilon$ w.h.p.
\label{lec11:lem:concentration_lemma}
\end{lemma}

If we can show that $g$ has no bad local minima via a proof that only uses $g$ via terms of the form $\langle v, \nabla g(x) \rangle$ and $\langle v, \nabla^2 g(x) v \rangle$, then by Lemma~\ref{lec11:lem:concentration_lemma} this proof will automatically generalize to $f$ by concentration.

Next, we prove some facts about $g$ and show the analogous proofs for $f$ that we will use in the proof of Theorem~\ref{lec11:thm:matrix-completion}.

\begin{lemma}[Connecting inner product and norm for $g$]\label{lec11:lem:inner-g}
If $x$ satisfies $\nabla g(x) = 0$, then $\langle x,z \rangle^2 = \norm{x}_2^4$.
\end{lemma}

\begin{proof}
\begin{align}
    \nabla g(x) = 0 &\implies \langle x, \nabla g(x) \rangle = 0 \\
   & \implies \langle x, (zz^T-xx^T)x \rangle = 0 & (\because \nabla g(x) = (M - xx^T)x) \\
   & \implies \langle x,z \rangle^2 = \norm{x}_2^4.
\end{align}
\end{proof}

\begin{lemma}[Connecting inner product and norm for $f$]\label{lec11:lem:inner-f}
Suppose $\norm{x}_\infty \leq 2\mu / \sqrt{d}$. If $x$ satisfies $\nabla f(x) = 0$, then $\langle x,z \rangle^2 \geq \norm{x}_2^4 - \epsilon$ with high probability.
\label{inner_prod_norm_f}
\end{lemma}

\begin{proof}
\begin{align}
    \nabla f(x) = 0 &\implies \langle x, \nabla f(x) \rangle = 0 \\
    & \implies \langle x, \nabla g(x) \rangle \approx \langle x, \nabla f(x) \rangle/p \pm \epsilon & \text{(by Lemma \ref{lec11:lem:concentration_lemma})} \\
   & \implies |\langle x, (zz^T-xx^T)x \rangle| \leq \epsilon & \text{w.h.p.} \\
   & \implies \langle x,z \rangle^2 \geq \norm{x}_2^4 - \epsilon & \text{w.h.p.}
\end{align}
\end{proof}

\begin{lemma}[Bound norm for $g$]\label{lec11:lem:bound-g}
    If $\nabla^2 g(x) \succeq 0$, then $\norm{x}_2^2 \geq 1/3$.
\end{lemma}

\begin{proof}
\begin{align}
    \nabla^2 g(x) \succeq 0
    &\implies \langle z, \nabla^2 g(x)z\rangle \geq 0 \\
    &\implies \norm{zx^T + xz^T}_F^2 - 2z^T(zz^T-xx^T)z \geq 0 \\
    &\implies 1 \leq \norm{x}_2^2 + 2\langle x,z \rangle^2 \leq \norm{x}_2^2 + 2\norm{x}_2^2 = 3\norm{x}_2^2 &\text{(by Cauchy-Schwarz)} \\
    &\implies \norm{x}_2^2 \geq 1/3.
\end{align}
\end{proof}

\begin{lemma}[Bound norm for $f$]\label{lec11:lem:bound-f}
    Suppose $\norm{x}_\infty \leq \mu / \sqrt{d}$. If $\nabla^2 f(x) \succeq 0$, then $\norm{x}_2^2 \geq 1/3 - \epsilon/3$ with high probability.
\end{lemma}
\begin{proof}
\begin{align}
    \nabla^2 f(x) \succeq 0
    &\implies \langle z, \nabla^2 f(x)z \rangle \geq 0 \\
    &\implies \langle z, \nabla^2g(x)z \rangle \geq -\epsilon & \text{w.h.p. (by Lemma \ref{lec11:lem:concentration_lemma})} \\
    &\implies 3\norm{x}_2^2 \geq 1-\epsilon & \text{w.h.p.} \\
    &\implies \norm{x}_2^2 \geq 1/3 - \epsilon/3 & \text{w.h.p.}
\end{align}
\end{proof}

\begin{lemma}[$g$ has no bad local minimum]
    All local minima of $g$ are global minima.
\end{lemma}

\begin{proof}
\begin{align}
    \nabla g(x) = 0
    & \implies \langle z, \nabla g(x) \rangle = 0 \\
    & \implies \langle z, (zz^T-xx^T)x \rangle = 0 \\
    & \implies \langle x,z\rangle (1-\norm{x}_2^2) = 0.
\end{align}
Since $|\langle x,z \rangle| \geq 1/3 \neq 0$ (by Lemma~\ref{lec11:lem:bound-g}), we must have $\norm{x}_2^2 = 1$. But then Lemma~\ref{lec11:lem:inner-g} implies $\langle x, z\rangle^2 = \norm{x}_2^4 = 1$, so $x = \pm z$ by Cauchy-Schwarz.
\end{proof}

We now prove Theorem~\ref{lec11:thm:matrix-completion}, restated for convenience:
\begin{theorem}[$f$ has no bad local minimum]
Assume $p = \dfrac{\textrm{poly}(\mu, \log d)}{d\epsilon^2}$. Then with high probability, all local minima of $f$ are $O(\sqrt{\epsilon})$-close to $+z$ or $-z$.
\end{theorem}

\begin{proof}
Observe that $\norm{x-z}_2^2 = \norm{x}_2^2 + \norm{z}_2^2 - 2\langle x,z \rangle \leq \norm{x}_2^2 + 1 - 2\langle x,z \rangle$. Our goal is to show that this quantity is small with high probability, hence we need to bound $\norm{x}_2^2$ and $\langle x,z \rangle$ w.h.p. Note that the following bounds in this proof are understood to hold w.h.p.
    
Let $x$ be such that $\nabla f(x) = 0$. For $\epsilon \leq 1/16$,
\begin{align}
\langle x,z \rangle^2 &\geq \norm{x}_2^4 - \epsilon &\text{(by Lemma~\ref{lec11:lem:inner-f})} \\
&\geq (1/3-\epsilon/3)^2 - \epsilon &\text{(by Lemma~\ref{lec11:lem:bound-f})} \\
&\geq 1/32. \label{lec11:eqn:xz-bound}
\end{align}

With this, we can get a bound on $\norm{x}_2^2$:
\begin{align}
\nabla f(x) = 0 &\implies \langle x, \nabla f(x) \rangle = 0 \\
&\implies |\langle z, \nabla g(x) \rangle| \leq \epsilon & \text{(by Lemma \ref{lec11:lem:concentration_lemma})} \\
&\implies |\langle x,z\rangle| \cdot |1-\norm{x}_2^2| \leq \epsilon &\text{(by dfn of $g$)} \\
&\implies |1-\norm{x}_2^2| \leq 32\epsilon = O(\epsilon) &\text{(by \eqref{lec11:eqn:xz-bound})} \\
&\implies \norm{x}_2^2 = 1 \pm O(\epsilon). \label{lec11:eqn:xnorm-bound}
\end{align}
    
Next, we bound $\langle x,z \rangle$:
\begin{align}
\langle x, z \rangle^2 &\geq \norm{x}_2^4 - \epsilon &\text{(by Lemma \ref{inner_prod_norm_f})} \\
&\geq (1-O(\epsilon))^2 - \epsilon &\text{(by \eqref{lec11:eqn:xnorm-bound})} \\
&= 1 - O(\epsilon).
\end{align}

Finally, we put these quantities together to bound $\norm{x-z}_2^2$. We have two cases:
    
\textbf{Case 1}: $\langle x,z\rangle \geq 1 - O(\epsilon)$. Then
\begin{align}
\norm{x-z}_2^2 &= \norm{x}_2^2 + \norm{z}_2^2 - 2\langle x,z \rangle \\
&\leq \norm{x}_2^2 + 1 - 2\langle x,z \rangle \\
&\leq 1 + O(\epsilon) + 1 - 2(1-O(\epsilon)) \\
&\leq O(\epsilon).
\end{align} 
    
Hence we conclude $x$ is $O(\sqrt{\epsilon})$-close to $z$.
    
\textbf{Case 2}: $\langle x,z\rangle \leq -(1 - O(\epsilon))$. Then by an analogous argument, $x$ is $O(\sqrt{\epsilon})$-close to $-z$.
\end{proof}

We have shown above that matrix completion of a rank-1 matrix has no spurious local minima. This proof strategy can be extended to handle higher-rank matrices and noisy matrices \cite{ge2016}. The proof also demonstrates a generally useful proof strategy: often, reducing a hard problem to an easy problem results in solutions that do not give much insight into the original problem, because the proof techniques do not generalize. It can often be fruitful to seek a proof in the simplified problem that makes use of a restricted set of tools that could generalize to the harder problem. Here we limited ourselves to only using $\langle v, \nabla g(x)\rangle$ and $\langle v, \nabla^2 g(x) v\rangle$ in the easy case; these quantities could then be easily converted to analogous quantities in $f$ via the concentration lemma (Lemma~\ref{lec11:lem:concentration_lemma}).

\sec{Other problems where all local minima are global minima}
We have now demonstrated that two classes of machine learning problems, rank-1 PCA and rank-1 matrix completion, have no spurious local minima and are thus amenable to being solvable by gradient descent methods. We now outline some major classes of problems for which it is known that there are no spurious local minima.

\begin{itemize}
    \item Principal component analysis (covered in previous lecture).
    \item Matrix completion (and other matrix factorization problems). On a related note, it has also been shown that linearized neural networks of the form $y = W_1W_2x$, where $W_1$ and $W_2$ are optimized separately, have no spurious local minima \cite{baldi1989neural}. It should be noted that linearized neural networks are not very useful in practice since the advantage of optimizing $W_1$ and $W_2$ separately versus optimizing a single $W=W_1W_2$ is not clear.
    \item Tensor decomposition. The problem is as follows:
    \begin{align}
        \text{maximize }\quad \sum_{i=1}^d \sum_{j=1}^d \sum_{k=1}^d \sum_{l=1}^d T_{ijkl} x_ix_jx_kx_l \quad \text{such that } \quad \norm{x}_2 = 1.
    \end{align}
    Additionally, constraints are imposed on the tensor $T$ to make the problem tractable. For example, one condition is that $T$ must be a low-rank tensor with orthonormal components \cite{ge2015}.
\end{itemize}
	% reset section counter
%\setcounter{section}{0}

%\metadata{lecture ID}{Your names}{date}
\metadata{13}{Justin Young and Josh Cho}{November 1, 2021}

\sec{The Neural Tangent Kernel (NTK) Approach}

In the previous sections, we studied non-convex optimization problems in which all local minima are global. Here we consider an objective for which we can identify particular regions of the input space in which all local minima are also global minima. \tnote{a bit better transition and positioning, e.g., specify that this only works for certain cases; also add some literature; Tengyu can help edit later}

To be more formal, we take an appropriate parameter initialization $\theta^0$ such that in a neighborhood around it, which we denote by $B(\theta^0)$, the loss function is convex and its global minimum is attained. Figure \ref{lec13:fig:NTKapproach} depicts a function and region for which this condition holds. 

\begin{figure}[ht]
    \centering
    \includegraphics[scale=0.5]{figures/ntk_initialization.png}
    \caption{Training loss around an initialized $\theta^0$. The dotted lines indicate $B(\theta^0)$, a region where the loss is convex, and where a global minimum exists.
    	\tnote{(a) have two equally good global min. (b) make the global min have y-value zero or near zero. }
    }
    \label{lec13:fig:NTKapproach}
\end{figure}


Given a nonlinear $f_\theta(x)$, we examine the Taylor expansion at $\theta^0$: 
\begin{align} 
    f_\theta(x) = \underbrace{f_{\theta^0}(x) + \langle \nabla_\theta f_{\theta^0}(x),\theta-\theta^0 \rangle}_{\defeq g_\theta(x)} + \text{ higher order terms}
\end{align} 

Note that $g_\theta(x)$ is an affine function in $\theta$, as $f_{\theta^0}(x)$ is a constant for fixed $x,\theta^0$. Similarly, defining $\Delta \theta = \theta-\theta^0$, we can say that $g_\theta(x)$ is linear in $\Delta \theta$. For convenience, we will sometimes choose $\theta^0$ such that $f_{\theta^0}(x) = 0$ for all $x$. It is easy to see why such an initialization exists. Consider splitting a two-layer neural network $f_{\theta}(x)$ with width $2m$ into two halves, each with $m$ neurons; the outputs of these two networks are then given by $\sum_{i=1}^m a_i \sigma (w_i^\top x)$ and $\sum_{i=1}^m -a_i \sigma (w_i^\top x)$, respectively. Here, $w_i$ can be randomly chosen so long as $W_i$ is the same in both halves, and $a_i$ can be randomly chosen as long as the other half is initialized with $-a_i$. Summing these two networks together yields $f_{\theta^0}(x) \equiv 0$ for all $x$.

Letting 
\begin{align}
    y' &= y- f_{\theta^0}(x) \\
    &= \inprod{\nabla_\theta f_{\theta^0}(x), \Delta \theta},
\end{align}
we observe that $\Delta \theta$ depends upon the parameter we evaluate the network at, while $\nabla_\theta f_{\theta^0}(x)$ can be thought of as a feature map since it is a fixed function of $x$ (given the architecture and $\theta^0$) that does not depend on $\theta$ whatsoever. We thus let $\phi(x) \triangleq \nabla_\theta f_{\theta^0}(x)$, which motivates the following definition: 

\begin{definition}[Neural Tangent Kernel]
For simplicity, we assume $f_{\theta^0}(x)=0$ so that $y=y'$. The \textit{neural tangent kernel} $K$ is given by  
\begin{align} 
    K(x,x') &= \inprod{\phi(x), \phi(x')} \\
    &= \inprod{\nabla_\theta f_{\theta^0}(x), \nabla_\theta f_{\theta^0}(x')}.
\end{align} 
\end{definition}
Here, the feature $\nabla_\theta f_{\theta^0}(x)$ is precisely the gradient of the neural network. This is where the ``tangent'' in Neural Tangent Kernel comes from. 

Instead of $f_\theta(x)$, suppose we use the approximation $g_\theta(x)$, which we recall is linear in $\theta$. The kernel method gives a linear model on top of features. When $\theta \approx {\theta^0}$, given a convex loss function $\ell$, we have 
\begin{align} 
    \underbrace{\ell (f_\theta(x),y)}_{\substack{\text{not} \\ \text{necessarily} \\ \text{convex}}} \approx \underbrace{\ell(g_\theta(x),y)}_{\text{convex}}.
\end{align} 
Convexity of the RHS follows from the fact that a convex function, $\ell$, composed with a linear function, $g_\theta$, is still convex. 

A natural question to ask is: how valid is this approximation? We devote the rest of this chapter to answering this question. First, we define the empirical loss: 
\begin{align}
    \hat{L}(f_\theta) & = \frac{1}{n}\sum_{i=1}^n \ell \left( f_\theta\big( x^{(i)} \big) , y^{(i)} \right) \\ 
    \hat{L}(g_\theta) & = \frac{1}{n}\sum_{i=1}^n \ell \left( g_\theta\big( x^{(i)} \big) , y^{(i)} \right).
\end{align} 
The key idea is that the Taylor approximation works for certain cases. We defer a more complete enumeration of these cases to a later section of this monograph. For now, we claim that this Taylor expansion is valid for the following situation\tnote{rephrase a bit? maybe "Here we outline the high-level approach to validate and use the Taylor expansion.."}. 
 We assume \tnote{reword -- we will show instead of assume} that there exists a neighborhood around $\theta^0$ called $B(\theta^0)$, such that we have the following:
\begin{enumerate}
    \item Accurate approximation: $f_\theta(x) \approx g_\theta(x)$, and $\hat{L}(f_\theta) \approx \hat{L}(g_\theta)$ for all $\theta \in B(\theta^0)$.
    \item It suffices to optimize in $B(\theta^0)$: There exists an approximate global minimum $\hat{\theta} \in B(\theta^0)$, so $\hat{L}(g_{\hat{\theta}}) \approx 0$. This is the lowest possible loss (because the loss is nonnegative), which implies we are close to the global minimum. Because of 1, this implies that $\hat{L}(f_{\hat{\theta}}) \approx 0$ as well. See Figure \ref{fig:ntkglobalmin} for an illustration.
    \item Optimizing $\hat{L} (f_\theta)$ is similar to optimizing $\hat{L}(g_\theta)$ and does not leave $B(\theta^0)$, i.e. everything is confined to this region. Intuitively, this last point to some extent is ``implied" by (1) and (2), but this claim still requires a formal proof. 
\end{enumerate}

\begin{figure}[h!]
    \centering
    \includegraphics[scale=0.5]{figures/ntk_global_min.png}
    \caption{Here, $\hat{L}(g_{\theta})$ and $\hat{L}(f_{\theta})$ are both plotted. At $\hat{\theta}$, we have reached the approximate global minimum where $\hat{L}(g_{\hat{\theta}}) \approx 0$, in turn implying also that $\hat{L}(f_{\hat{\theta}}) \approx 0$.}
    \label{fig:ntkglobalmin}
\end{figure}

Note (1), (2), and (3) can all be true in various settings. In particular, to attain all three, we will require: 
\begin{enumerate}[label=\alph*]
    \item[(a)] Overparametrization and/or a particular scaling of the initialized $\theta^0$. 
    \item[(b)] Small (or even zero) stochasticity, so $\theta$ never leaves $B(\theta^0)$. This condition is guaranteed by a small learning rate or full-batch gradient descent. 
\end{enumerate} 
Despite the limitations of the requirements of (a) and (b), the existence of such a region is still surprising. Given the loss landscape which could potentially be highly non-convex, it is striking to find a neighborhood where the loss function is convex (e.g. quadratic) with a global minimum. This suggests there is some flexibility in the loss landscape.  

To begin our formal discussion, we now prove (1) and (2). Let 
\begin{align}
    \phi^{(i)} = \phi\l(x\sp{i}\r) = \nabla_\theta f_{\theta^0}\l( x\sp{i} \r)
\end{align}
and 
\begin{align}
    \Phi = \begin{bmatrix} {\phi\sp{1}}^\top \\ \vdots \\ {\phi\sp{n}}^\top \end{bmatrix} \in \R^{n \times p}
\end{align}
where $p$ is the number of parameters. Taking the quadratic loss, we have
\begin{align}
    \hat{L}(g_\theta) = \frac{1}{n}\sum_{i=1}^n \left( y\sp{i} - \phi\l(x\sp{i} \r)^\top \Delta \theta \right)^2 = \frac{1}{n} \norm{\vec{y} - \Phi \cdot \Delta \theta}_2^2
\end{align} 
where $\vec{y} = \l[ y\sp{1}, \cdots, y\sp{n} \r]^\top \in \R^n$. Note that this looks a lot like linear regression, where $\Phi$ and $\Delta \theta$ are the analogues of the design matrix and parameter, respectively. We further assume that $y^{(i)} = O(1)$ and $\norm{y}_2 = O(\sqrt{n})$. Now, we can prove a lemma that addresses the second of the three conditions we described above, i.e. that it is sufficient to optimize in some small ball around $\theta^0$.

\begin{lemma}[for (2)] 
    Suppose we are in the setting where $p \geq n$, $\textup{rank}(\Phi) = n$, and $\sigma_{\min}(\Phi) = \sigma > 0$. Then, letting $\Delta \hat{\theta}$ denote the minimum norm solution, i.e. the nearest global minimum, of $\vec{y} = \Phi \Delta \theta$, we have 
    \begin{align} 
        \norm{\Delta \hat{\theta}}_2 \leq O(\sqrt{n} / \sigma)
    \end{align} 
\end{lemma}
\begin{remark} \label{lec13:rmk:intuitiononlemma} 
    The meaning of the bound on $\Delta \hat{\theta}$ becomes clear if we consider the ball given by 
    \begin{align}
        B_{\theta^0} = \{ \theta = \theta^0 + \Delta \theta: \norm{\Delta \theta}_2 \leq O(\sqrt{n}/\sigma )\}.
    \end{align} 
    In particular, notice that $B_{\theta^0}$ contains a global minimum, so this lemma characterizes how large the ball must be to contain a global minimum. 
    \end{remark} 
\begin{proof}
    Letting $\Phi^\dagger$ denote the Moore-Penrose pseudoinverse of $\Phi$, note that $\Delta \hat{\theta} = \Phi^\dagger \boldsymbol{y}$, and $\norm{\Phi^\dagger} _{\text{op}} = \frac{1}{\sigma_{\min} (\Phi)} = \frac{1}{\sigma}$.  A simple argument shows 
    \begin{align}
        \norm{\Delta \hat{\theta}}_2 &\leq \norm{\Phi^\dagger}_{\text{op}} \cdot \norm{\vec{y}}_2 \\
        &\leq O\left( \frac{1}{\sigma}\cdot \sqrt{n} \right),
    \end{align} 
    where the last inequality follows from the assumption that $\norm{\vec{y}}_2 \leq O(\sqrt{n})$. 
\end{proof}
Next, we prove a lemma that addresses the first of the three conditions we described above.
\begin{lemma}[for (1)] 
    \label{lec13:lma:accurate_approximation}
    Suppose $\nabla_\theta f_\theta(x)$ is $\beta$-Lipschitz in $\theta$, i.e. for every $x$, and $\theta, \theta'$, we have 
    \begin{align}
        \norm{\nabla_\theta f_{\theta} (x) - \nabla_{\theta} f_{\theta'}(x)}_2 \leq \beta \cdot \norm{ \theta - \theta'}_2.
    \end{align} 
    Then, 
    \begin{align} 
        \left| f_\theta(x) - g_\theta(x) \right| \leq O \left( \beta \norm{\Delta \theta}_2^2 \right|.
    \end{align}  
    \tnote{typo above}
    If we further restrict our choice of $\theta$ using $B_{\theta^0}$ as defined in Remark~\ref{lec13:rmk:intuitiononlemma}, we obtain that
    \begin{align} 
        | f_\theta(x) - g_\theta(x) | \leq O \left( \frac{\beta n }{\sigma^2 }\right), \quad \forall \theta \in B_{\theta^0}. \label{lec13:eqn:lemma1bound} 
    \end{align} 
\end{lemma}
\begin{proof}
    The proof comes from the following fact:  if $h(\theta)$ is such that $\nabla h(\theta)$ is $\beta$-Lipschitz (which if differentiable is equivalent to $\norm{\nabla^2 h(\theta)}_{\text{op}} \leq \beta$), then
    \begin{align}
        \bigg| \underbrace{h(\theta)}_{f_\theta(x)}  \underbrace{-h(\theta^0) - \inprod{\nabla h(\theta^0), \theta-\theta^0}}_{-g_\theta(x)}\bigg| \leq O\left( \beta \norm{\theta-\theta^0}_2^2 \right).
    \end{align} 
    \tnotelong{add a lemma in the toolbox section about this}
    As shown above, the proof is as simple as plugging in $f_\theta(x) = h(\theta)$ and $g_\theta(x)=h(\theta^0) + \inprod{\nabla h(\theta^0), \Delta \theta}$. 
\end{proof}

\begin{remark}
The lemma above bounds the approximation error. Intuitively, as you move farther away from $\theta^0$, the Taylor approximation gets worse; the approximation error is bounded above by a second order $\Delta \theta$ term.
\end{remark}

\begin{remark}
Note that if $f_\theta$ involves a $\text{relu}$ function, then $\nabla f_\theta$ is not continuous everywhere. This requires a technical fix outside the scope of our discussion.\footnote{A $\text{relu}$ function is continuous almost everywhere, so we can make some minor fixes and still use some modified notion of Lipschitzness to derive an upper bound.} \tnote{Tengyu will add a reference here}
\end{remark}

\subsec{Tightening our Approximation Error Bound}
By \eqref{lec13:eqn:lemma1bound}, we have now established a bound on our approximation error, but we have yet to analyze how good it is, as $\beta n /\sigma^2$ is not obviously either big or small. An important fact to notice is that $\beta/\sigma^2$ is not scaling invariant, so we can play with the scaling in order to drive this term to $0$. In particular, there are two notable cases where $\beta/\sigma^2 \to 0$: 
\begin{enumerate}
    \item  \textbf{Reparametrize with a scalar} \cite{chizat2018note}. Let $f_\theta(x) = \alpha \cdot \bar{f}_\theta(x)$ where $\bar{f}_\theta(x)$ is a standard neural net with fixed width and depth. We only vary $\alpha$, i.e. the scaling, and we see how things change accordingly. Fix an initial $\theta^0$, and let 
    \begin{align}
        \bar{\sigma} = \sigma_{\min}\left( \begin{bmatrix}  \nabla_\theta \bar{f}_{\theta^0} \big( x^{(i)} \big)^\top \\ \vdots \\ \nabla_\theta \bar{f}_{\theta^0} \big(x^{(i)} \big)^\top \end{bmatrix}\right).
    \end{align} 
    Furthermore, let $\bar{\beta}$ be the Lipschitz parameter of $\nabla_\theta \bar{f}_\theta(x)$ in $\theta$. A simple chain-rule gradient argument shows that scaling $\bar{f}_{\theta}$ by $\alpha$ also scales $\sigma$ and $\beta$ accordingly, i.e. $\sigma = \alpha \bar{\sigma}$, and $\beta = \alpha \bar{\beta}$. Some straightforward algebra yields 
    \begin{align} 
        \frac{\beta}{\sigma^2}= \frac{\bar{\beta}}{\bar{\sigma}^2} \cdot \frac{1}{\alpha} \to 0 \quad \text{as} \quad \alpha \to \infty.
    \end{align}
    Once $\alpha$ becomes big enough, then by Lemma~\ref{lec13:lma:accurate_approximation}, the approximation $|f_\theta(x) - g_\theta(x)| \leq O\left( \beta n / \sigma^2 \right)$ becomes very good.  
    \begin{remark}
    Note that this scaling does not scale the loss. Intuitively, while the function $f_\theta$ becomes sharper and non-smooth as $\alpha$ grows big (leading to higher approximation error), we note that we only need to travel a little bit away from $\theta^0$ given the $\beta$-Lipschitz constraint on the gradient of $f_{\theta}$. In particular, the neighborhood around $\theta^0$ shrinks faster than the approximation error grows.  
    \end{remark}
    \item \textbf{Overparametrize the model}. Early papers on the NTK take this approach. Consider a  two-layer network with $m$ neurons. 
    \begin{align}
        \hat{y} = \frac{1}{\sqrt{m}} \sum_{i=1}^m a_i \sigma(w_i^\top x )
    \end{align} 
    The scaling $1/\sqrt{m}$ will be for convenience, as we see momentarily. We make the following assumptions regarding the network and its inputs.
    \begin{align}
        W &= \begin{bmatrix} w_1^\top \\ \vdots \\ w_m^\top \end{bmatrix} \in \R^{m \times d} \\
        \sigma &\text{ is $1$-Lipschitz and twice-differentiable} \\
        a_i &\sim \{\pm 1\} \quad &\text{(not optimized)} \\
        w_i^0 &\sim \cN(0, I_d) \\
        \norm{x}_2 &= \Theta(1) \\
        \theta &= \text{vec}(W) \in \R^{dm} \quad &\text{(vectorized $W$)}
    \end{align} 
    We will assume $m \to \infty$, and in particular, for fixed $n,d$, $m = \textsf{poly}(n,d)$.
    
    Why do we use the $1/\sqrt{m}$ scaling? Note that $\sigma\big({w_i^0}^\top x\big) \approx 1$ because $\norm{x}_2 = \Theta(1)$ and $w_i^0$ is drawn from a spherical Gaussian. Thus, as some $a_i$ are positive and others are negative, $\left|\sum_{i=1}^m a_i \sigma \big({w_i^0}^\top x\big) \right| = \Theta \left( \sqrt{m} \right)$, and finally $f_{\theta^0} (x) = \Theta(1)$. 
    
    Now we analyze $\sigma$ and $\beta$. We let
    \begin{align}
        \sigma = \sigma_{\min} (\Phi) = \sqrt{\sigma_{\min} \left( \Phi \Phi^\top \right)}
    \end{align}
    where 
    \begin{align}
        \left( \Phi \Phi^\top \right)_{ij} = \inprod{\nabla_\theta f_{\theta^0} \big(x^{(i)} \big) , \nabla_\theta f_{\theta^0} \big(x^{(j)} \big)} \label{lec13:eqn:phifeature} 
    \end{align} 
    Note that the gradient with respect to $w_i$ is given by 
    \begin{align}  
        \frac{\partial f_\theta(x) }{\partial w_i} = \frac{1}{\sqrt{m}} \sigma'(w_i^\top x ) \cdot x 
    \end{align} 
    Now observe that
    \begin{align}
        \norm{\nabla f_\theta(x)}_2^2 & = \frac{1}{m}\sum_{i=1}^m \norm{\sigma'\big({w_i}^\top x \big) \cdot x }_2^2 \\ 
        & = \frac{1}{m}\norm{x}_2^2 \cdot \sum_{i=1}^m \left( \sigma' \big({w_i}^\top x \big) \right)^2 \\ 
        &\to \Exp_{w \sim \cN(0,I_d)} \left[ \sigma' \big( w^\top x \big)^2 \right] \cdot \norm{x}_2^2 \quad \text{as} \quad m\to\infty \\ 
        &= O(1) &\text{(not depending on $m$)}
    \end{align} 
    where the penultimate line follows from the law of large numbers, as $\frac{1}{m} \sum_{i=1}^m \left( \sigma'(w_i^\top x ) \right)^2$ can be interpreted as a mean. 
    
    Note that $\norm{\nabla_\theta f_{\theta^0} (x)}_2$ does not depend on $m$, so the inner product in \eqref{lec13:eqn:phifeature} also does not depend on $m$ either. As above, we can show 
    \begin{align} 
        \inprod{\nabla_\theta f_{\theta^0} (x), \nabla_\theta f_{\theta^0} (x')} & = \frac{1}{m}\inprod{ x,x'} \sum_{i=1}^m \sigma'(w^\top x) \sigma'(w^\top x')  \\ 
        & \to \Exp_{w \sim \cN(0,I_d)} \left[ \sigma'(w^\top x) \sigma'(w^\top x') \right] \inprod{ x, x'} \label{lec13:eqn:kernelcalc} 
    \end{align}
    
    \eqref{lec13:eqn:kernelcalc} implies that as $m \to \infty$, $\Phi \Phi^\top$ converges to a constant matrix denoted by 
    \begin{align}
        K^\infty = \lim_{m \to \infty} \Phi\Phi^\top 
    \end{align}
    This is precisely the NTK with $m=\infty$. 
    
    Though we omit the proof of this claim, it can be shown that $K^\infty$ is full rank. Then, let \begin{align}
        \sigma_{\min} \triangleq \sigma_{\min} (K^\infty) > 0.
    \end{align}
    We can show that 
    \begin{align}
        \sigma = \sigma_{\min} \left( \Phi \Phi^\top \right) > \frac{1}{2}\sigma_{\min} 
    \end{align} 
    Intuitively, $\Phi \Phi^\top \to K^\infty$, so the spectrum of the matrix should also converge. Thus, in some sense, we have shown that $\sigma$ is constant in the limit. 
    
    Now what about $\beta$? If we can show $\beta \to 0$ as $m \to \infty$, we are done. We begin by analyzing this key expression:  
    \begin{align}
        \nabla_\theta f_\theta(x) - \nabla_\theta f_{\theta'} (x) = \left[ \frac{1}{\sqrt{m}} \left( \sigma' \big( w_i^\top x \big) - \sigma' \big({w_i'}^\top x \big) \right) \cdot x \right]_{i=1}^m \label{lec13:eqn:lipschitzmatrix}
    \end{align}
    Note that \eqref{lec13:eqn:lipschitzmatrix} above consists of matrices, as $\theta$ is a vectorized matrix. Then,
    \begin{align}
        \norm{\nabla_\theta f_\theta(x) - \nabla_\theta f_{\theta'}(x)}_2^2 & = \frac{1}{m}\sum_{i=1}^m \norm{x}_2^2 \left( \sigma' \big(w_i^\top x \big) - \sigma' \big({w_i}'^\top x \big) \right)^2  \\ 
        & \leq O \left( \frac{1}{m}\sum_{i=1}^m \norm{ x}_2^2 \big( w_i^\top x - {w_i'}^\top x \big)^2 \right) \\ 
        & =  O \left( \frac{1}{m}\sum_{i=1}^m \norm{ w_i - w_i'}_2^2 \right) \\ 
        & = O \left(\frac{1}{m} \norm{ \theta - \theta' }_2^2 \right)
    \end{align} 
    The first line follows from the fact that $\frac{1}{\sqrt{m}} \left( \sigma' \big( w_i^\top x \big) - \sigma' \big({w_i'}^\top x \big) \right)$ is a scalar. The second line uses the assumption that $\sigma'$ is $O(1)$-Lipschitz. The third line uses Cauchy-Schwarz and the fact that $\norm{x}_2^2 \approx 1$. Taking the square root, we have that
    \begin{align} 
        \norm{\nabla_\theta f_\theta(x) - \nabla_\theta f_{\theta'}(x)}_2 \lesssim \frac{1}{\sqrt{m} } \norm{ \theta -\theta' }_2
    \end{align} 
    Thus, the Lipschitz parameter is $\beta = O(1/\sqrt{m})$. Thus, our key quantity $\beta/\sigma^2$ goes to $0$ as $m$ grows. Namely,
    \begin{align} 
        \frac{\beta}{\sigma^2} \approx \frac{1}{\sqrt{m} }\cdot \frac{1}{\sigma_{\min}^2} \to 0 \quad \text{as} \quad m\to\infty.
    \end{align} 
    Recall here that $\sigma_{\min}$ does not depend on $m$. Concretely, this result tells us that our function becomes more smooth (the gradient has a smaller Lipschitz constant) as we add more neurons. 
\end{enumerate}

\subsec{Optimizing $\hat{L}(g_\theta)$ vs. $\hat{L}(f_\theta)$}
We now discuss how to establish the last of the three conditions under which we claimed a Taylor approximation is reasonable. We need to show that  optimizing $\hat{L} (f_\theta)$ is similar to optimizing $\hat{L}(g_\theta)$. To do so, we require two steps:
\begin{enumerate}[label=\alph*]
    \item[(A)] Analyze optimization of $\hat{L}(g_\theta)$.
    \item[(B)] Analyze optimization of $\hat{L}(f_\theta)$ by re-using the proof in (A).
\end{enumerate}
There are two approaches in the literature for (A), which implies that there exist two approaches for (B) as well. 
\begin{enumerate}
    \item[(i)] We leverage the strong convexity of $\hat{L} (g_\theta)$, and then show an exponential convergence rate.\footnote{Recall that a differentiable function $f$ is strongly convex if 
    \begin{align} 
        f(y) \geq f(x) + \nabla f(x)^\top (y-x) + \frac{\mu}{2} \norm{y-x}_2^2
    \end{align} for some $\mu>0$ and all $x,y$.} 
    \item[(ii)] Instead of strong convexity, we use smoothness of $f_\theta$ (i.e. bounded second derivative). 
\end{enumerate}
We will only discuss the first of these two methods in this monograph.

\begin{remark} 
Suppose at any $\theta^t$, we take the Taylor expansion of $g_\theta$ at $\theta^t$ so 
\begin{align} 
    g_\theta^t(x) = g_{\theta^t} (x) + \inprod{ \nabla f_{\theta^t} (x),\theta-\theta^t } 
\end{align} 
Consider the gradient we are interested in taking: $\nabla \hat{L} ( f_{\theta^t})$. Notice that: \begin{align} 
    \nabla \hat{L} ( f_{\theta^t}) = \nabla \hat{L} ( g_{\theta^t}^t)
\end{align} 
This is really saying that $f_\theta$ and $g_\theta^t$ agree up to first-order at $\theta^t$. This implies that $L(f_\theta)$ and $L(g_\theta^t)$ also agree up to the first-order at $\theta^t$. This means that $T$ steps of gradient descent on $\hat{L}(f_\theta)$ is the same as performing online gradient descent\footnote{See Chapter~\ref{chap:OL}.} on a sequence of changing objectives $L(g_\theta^0), \ldots, L(g_\theta^T)$.
\end{remark} 
%	\subsec{Limitations of NTK}

The NTK approach has its limitations.
\begin{itemize}
    \item Empirically, optimizing $g_\theta(x)$ as described in the theory does not work as well as state-of-the-art (or even standard) deep learning methods. For example, using the NTK approach (i.e., taking the Taylor expansion and optimizing $g_{\theta}(x)$) with a ResNet generally does not perform as well as ResNet with best-tuned hyperparameters.
    
    \item The NTK approach requires a specific initialization scheme and learning rate which may not coincide with what is commonly used in practice.
    
    \item The analysis above was for gradient descent, while stochastic gradient descent is used in practice, introducing noise in the procedure. This means that NTK with stochastic gradient descent requires a small learning rate to stay in the initialization neighborhood. Deviating from the requirements can lead to leaving the initialization neighborhood.
\end{itemize}

One possible explanation for the gap between theory and practice is because NTK effectively requires a fixed kernel, so there is no incentive to select the right features. Furthermore, the minimum $\ell_2$-norm solution is typically dense. This is similar to the difference between sparse and dense combinations of features observed in the $\ell_1$-SVM/two-layer network versus the standard kernel method SVM (or $\ell_2$-SVM) analyzed previously.

To make these ideas more concrete, consider the following example \cite{wei2020regularization}. 
\begin{example}\label{lec12:ex:sparse123}
Let $x \in \R^d$ and $y \in \{-1, 1\}$. Assume that each component of $x$ satisfies $x_i \in \{ -1, 1\}$. Define the output $y = x_1x_2$, that is, $y$ is only a function of the first two components of $x$.

This output function can be described exactly by a neural network consisting of a sparse combination of the features (4 neurons to be exact):
\begin{align}
\hat y &= \frac{1}{2} \left[ \phirelu(x_1 + x_2) + \phirelu(-x_1 - x_2)  - \phirelu(x_1 - x_2) -  \phirelu(x_2 - x_1)  \right] \\
&= \frac{1}{2}\left( |x_1 + x_2| - |x_1 - x_2| \right) \label{lec12:eqn:ex1} \\
&= x_1x_2. \label{lec12:eqn:ex2}
\end{align}
\eqref{lec12:eqn:ex1} follows from the fact that $\phirelu(t) + \phirelu(-t) = |t|$ for all $t$, while \eqref{lec12:eqn:ex2} follows from evaluating the 4 possible values of $(x_1, x_2)$. Thus, we can solve this problem exactly with a very sparse combination of features.

However, if we were to use the NTK approach (kernel method), the network's output will always involve $\sigma(w^\top x)$ where $w$ is random so it includes all components of $x$ (i.e. a dense combination of features), and cannot isolate just the relevant features $x_1$ and $x_2$. This is illustrated in the following informal theorem:
\begin{theorem}
The kernel method with NTK requires $n = \Omega(d^2)$ samples to learn Example \ref{lec12:ex:sparse123} well. In contrast, the neural network regularized by $\sum_{j = 1}^m | u_j| \| w_j\|_2$ only requires $n = O(d)$ samples.
\end{theorem}
\end{example}
	
	\chapter{Implicit/Algorithmic Regularization Effect}
	% reset section counter
\setcounter{section}{0}

%\metadata{lecture ID}{Your names}{date}
\metadata{13}{Rohith Kuditipudi and Kefan Dong}{Mar 1st, 2021}

One of the miracles of modern deep learning is the phenomenon of \textit{algorithmic regularization} (also known as \textit{implicit regularization} or \textit{implicit bias}): although the loss landscape may contain infinitely many global minimizers, many of which do not generalize well, in practice our optimizer (e.g. SGD) tends to recover solutions with good generalization properties.

The focus of this chapter will be to illustrate algorithmic regularization in simple settings. In particular, we first show that gradient descent (with the right initialization) identifies the minimum norm interpolating solution in overparametrized linear regression. Next, we show that for a certain non-convex reparametrization of the linear regression task where the data is generated from a sparse ground-truth model, gradient descent (again, suitably initialized) approximately recovers a sparse solution with good generalization. Finally, we discuss algorithmic regularization in the classification setting, and how stochasticity can contribute to algorithmic regularization.

\sec{Algorithmic regularization in overparametrized linear regression}\label{lec13:sec:olr}
We prove that gradient descent initialized at the origin converges to the minimum norm interpolating solution (assuming such a solution exists). Let $X:= \l[x\sp{1},...,x\sp{n} \r]^\top \in \bbR^{n \times d}$ denote our data matrix and $y:= \l[y\sp{1},...,y\sp{n} \r]^\top \in \bbR^n$ denote our label vector, where $n < d$. Assume $X$ is full rank. Our goal is to find a weight vector $\beta$ that minimizes our empirical loss function $\hatL (\beta) := \frac{1}{2}||y - X\beta||_2^2$.

\subsec{Analysis of algorithmic regularization}
As we are in the overparametrized setting with $n < d$ and $X$ full rank, there exist infinitely many global minimizers that interpolate the data and hence achieve zero loss. In fact, the following lemma shows that the set of global minimizers forms a subspace.

\begin{lemma}\label{lec13:lem:soln-subspace}
Let $X^+$ denote the pseudoinverse\footnote{Since $X$ is full rank, $XX^\top$ is invertible and so we have $X^+ = X^\top (X X^\top)^{-1}$. Note that $X X^+ X = X$.} of $X$. Then $\beta$ is a global minimizer if and only if $\beta = X^+ y + \zeta$ for some $\zeta$ such that $\zeta \perp x_1,...,x_n$.
\end{lemma}

\begin{proof}
For any $\beta \in \R^d$, we can decompose it as $\beta = X^+ + \zeta$ for some $\zeta \in \R^d$. Since
\begin{equation}
X\beta = X (X^+ y + \zeta) = y + X\zeta,
\end{equation}

$\beta$ is a global minimizer if and only if $X\zeta = 0$, which happens if and only if $\zeta \perp x_1,...,x_n$.

\end{proof}

From Lemma~\ref{lec13:lem:soln-subspace}, we can derive an explicit formula for the minimum norm interpolant $\beta^\star := \arg \min_{\beta : \hatL(\beta) = 0} ||\beta||_2$.
\tnotelong{add some basics about SVD }
\begin{corollary}
$\beta^\star = X^+ y$.
\end{corollary}

\begin{proof}
Take any $\beta$ such that $\hatL(\beta) = 0$, and write $\beta = X^+ y + \zeta$. Then from the definition of $X^+$ and the fact that $X \zeta = 0$ (see the proof of Lemma~\ref{lec13:lem:soln-subspace}), we have 
\begin{align}
    ||\beta||_2^2 &= ||X^+ y||_2^2 + ||\zeta||_2^2 + 2 \langle X^+ y, \zeta \rangle \\
    &= ||X^+ y||_2^2 + ||\zeta||_2^2 + 2 \langle X^\top(X X^\top)^{-1} y, \zeta \rangle \\
    &= ||X^+ y||_2^2 + ||\zeta||_2^2 + 2 \langle (X X^\top)^{-1} y, X \zeta \rangle \\
    &= ||X^+ y||_2^2 + ||\zeta||_2^2 &\text{(because $X\zeta = 0$)} \\
    &\geq ||X^+ y||_2^2,
\end{align}
with equality if and only if $\zeta = 0$.

\end{proof}

Now, suppose we learn $\beta$ using gradient descent with initialization $\beta^0$, where at iteration $t$ we set $\beta^t = \beta^{t-1} - \eta \nabla \hatL(\beta^{t-1})$ for some learning rate $\eta$. Since $\hatL (\beta)$ is convex, we know from standard results in convex optimization that gradient descent will converge to a global minimizer for a suitably chosen learning rate $\eta$ (in particular, taking $\eta$ to be sufficiently small). Assuming $\beta^0 = 0$, we will in fact recover the minimum norm interpolating solution.
\begin{theorem}\label{lec13:thm:linear-main}
Suppose gradient descent on $\hatL(\beta)$ with initialization $\beta^0 = 0$ converges to a solution $\hat \beta$ such that $\hatL(\hat \beta) = 0$. Then $\hat \beta = \beta^\star$.
\end{theorem}

The main idea of the proof is that the iterates of gradient descent always lie in the span of the $x\sp{i}$'s (see Figure \ref{lec13:fig:1} for an illustration).

\begin{figure}
\centering
\includegraphics[width=.35\linewidth]{figures/subspace-global-min.png}
\caption{Visualization of proof intuition for Theorem~\ref{lec13:thm:linear-main}. The solution $\beta^\star$ is the projection of the origin onto the subspace of global minima.}
\label{lec13:fig:1}
\end{figure}

\begin{proof}
We first show via induction that $\beta^t \in \text{span}\l\{ x\sp{1}, \dots,x\sp{n} \r\}$ for all $t$. For the induction base case, note that $\beta^0 = 0 \in \text{span}\l\{ x\sp{1}, \dots,x\sp{n} \r\}$. Now suppose $\beta^{t-1} \in \text{span}\l\{ x\sp{1}, \dots,x\sp{n} \r\}$. Recall that $\beta^t = \beta^{t-1} - \eta \nabla \hatL(\beta^{t-1})$. Since left-multiplying any vector by $X^\top$ amounts to taking a linear combination of the rows of $X$, it follows that $\eta \nabla \hatL(\beta^{t-1}) = \eta X^\top(X\beta^{t-1} - y) \in \text{span}\l\{ x\sp{1}, \dots,x\sp{n} \r\}$, and so $\beta^t = \beta^{t-1} - \eta \nabla \hatL(\beta^{t-1}) \in \text{span}\l\{ x\sp{1}, \dots,x\sp{n} \r\}$. This proves the induction step.

Next, we show that $\hat \beta \in \text{span}\l\{ x\sp{1}, \dots,x\sp{n} \r\}$ and $\hatL(\hat \beta) = 0$ implies $\hat \beta = \beta^\star$. By definition, $\hat \beta \in \text{span}\l\{ x\sp{1}, \dots,x\sp{n} \r\}$ implies $\hat \beta = X^\top v$ for some $v \in \bbR^n$. Since $\hatL(\hat \beta) = 0$, we have $0 = X\hat \beta - y = X X^\top v - y$. This implies $v = (X X^\top)^{-1}y$, and so $\hat \beta = X^\top v = X^\top (X X^\top)^{-1} y = X^+ y = \hat \beta^\star$.
\end{proof}

\sec{Algorithmic regularization in non-linear models}
We give an example of algorithmic regularization in a non-convex version of the overparametrized linear regression task considered in the previous section.

Take $X$ and $y$ as defined in Section~\ref{lec13:sec:olr}. This time, our goal is to find a weight vector that minimizes our empirical loss function
\begin{equation}
\hatL(\beta) := \frac{1}{4n}\sum_{i=1}^n \left(y\sp{i} - f_\beta(x\sp{i})\right)^2, \label{lec13:eqn:hadamard_model_1}
\end{equation}
where $f_\beta(x):= \langle \beta \odot \beta, x\rangle$. (The operation $\odot$ denotes the Hadamard product: for $u,v \in \bbR^d$, $u \odot v \in \bbR^d$ is defined by $(u \odot v)_i := u_i v_i$ for $i = 1, \dots, d$.)

We assume $x\sp{1},...,x\sp{n} \iid \cN(0,I_{d \times d})$ and $y\sp{i} = f_{\beta^\star}(x\sp{i})$, where the ground truth vector $\beta^\star$ is $r$-sparse (i.e. $\|\beta^\star\|_0 = r$). For simplicity, we assume $\beta_i^\star = \mathbf{1} \{i \in S\}$ for some $S \subset [d]$ such that $|S| = r$. We again analyze the overparametrized setting, where this time $n \ll d$ but also $n \geq \widetilde \Omega(r^2)$.

\subsec{Main results of algorithmic regularization}
Note that while $f_\beta$ is still linear over $x$, our loss is no longer convex over $\beta$. (To see this, suppose $\beta \neq 0$ is a global minimizer. Then we have $\hatL(0) > \hatL(\beta) = \hatL(-\beta)$.) Thus, the effect of algorithmic regularization induced by gradient descent will be much different from the overparametrized linear regression setting. 

In the previous setting of linear regression, solutions with low $\ell_2$ norm are desirable as they tend to generalize well. In the present setting, we know our ground-truth parameter $\beta^\star$ is sparse. Thus, we want to learn a sparse solution $\hat \beta$, avoiding non-sparse solutions that may not generalize well. One approach to finding sparse solutions, called \textit{lasso regression}, is to minimize the $\ell_1$-regularized proxy loss
\begin{equation}
\sum_{i=1}^n \left(\langle \theta, x\sp{i} \rangle - y\sp{i} \right)^2 + \lambda \| \theta \|_1
\end{equation}
with respect to $\theta$, where $\theta = \beta \odot \beta$. However, it turns out that we can equivalently learn a sparse solution by running gradient descent from a suitable initialization on the original \textit{unregularized} loss.

To be specific, let $\beta^0=\alpha \mathbf{1} \in \R^d$ be the initialization where $\alpha$ is a small positive number. The update rule of gradient descent algorithm is given by $\beta^{t+1}=\beta^t-\eta\nabla \hatL(\beta^{t}).$ The next theorem shows that when $n=\widetilde{\Omega}(r^2)$, gradient descent on $\hatL(\beta)$ converges to $\beta^\star.$

\begin{theorem}\label{lec13:thm:non-linear-main}
Let $c$ be a sufficiently large universal constant. Suppose $n\ge cr^2\log^2(d)$ and $\alpha\le 1 / d^c$, then when $\dfrac{\log(d/\alpha)}{\eta}\lesssim T\lesssim \dfrac{1}{\eta\sqrt{d\alpha}},$ we have
\begin{equation}\label{lec13:eqn:non-linear-main}
    \l\|\beta^\top\odot\beta^\top-\beta^\star\odot\beta^\star \r\|_2^2\le O \l( \alpha\sqrt{d} \r).
\end{equation}

(Here, $T$ indexes the gradient descent steps.)
\end{theorem}

We make several remarks about Theorem~\ref{lec13:thm:non-linear-main} before presenting the proof.

\begin{remark}
In this problem we do not use $\beta^0=0$ as the initialization point because $\beta=0$ is a critical point, that is, $\nabla\hatL(0)=0$. Note that the lower bound on $T$ depends logarithimically on $1/\alpha$, so we can take $\alpha$ to be a small inverse polynomial on $d$ and the lower bound won't change much. Also, the upper bound depends polynomially on $1/\alpha$ (which is considered very big when $c$ is sufficiently large), so we do not need to use early stopping in a serious way.
\end{remark}

\begin{remark}
Theorem~\ref{lec13:thm:non-linear-main} is a simplified version of Theorem 1.1 in \cite{li2018algorithmic}.
\end{remark}

\begin{remark}
$\hatL(\beta)$ has many global minima. To see this, observe that the number of parameters is $d$ and the number of constraints to fit all the examples is $O(n)$ because there are only $n$ examples. Recall that for overparameterized model we have $d\gg n$; consequently, there exists many global minima of $\hatL(\beta)$.
\end{remark}

\begin{remark}
$\beta^\star$ is the min-norm solution in this case. That is,
    \begin{align}\label{lec13:eqn:opt}
        \beta^\star=\argmin \|\beta\|_2^2\qquad \text{s.t. }\hatL(\beta)=0.
    \end{align}
    Informally, this is because we can view $\beta\odot \beta$ as a vector $\theta\in \R^{d}$, which leads to $\|\beta\|_2^2 =\|\theta\|_1.$ Then in the $\theta$ space (and with a little abuse of notation), the optimization problem~\eqref{lec13:eqn:opt} becomes
    \begin{align}\label{lec13:eqn:opt-theta}
        \theta^\star=\argmin \|\theta\|_1 \qquad \text{s.t. }\hatL(\theta)=0,
    \end{align}
    which is a lasso regression, whose solution is sparse.
\end{remark}

\begin{remark}    
In this non-linear case and the linear case before, gradient descent with small initialization converges to minimum $\ell_2$-norm solution. Similarly, in the NTK regime, gradient descent converges to a solution that is very close to the initialization. Therefore, it seems conceivable that GD generally prefers global minima nearest to the initialization. However, we do not have a general theorem for this phenomenon (and the instructor also believes that this is not universally true without other conditions). 
\end{remark}

\subsec{Ground work for proof and the restricted isometry property}\label{lec13:sec:rip}

In this section we prepare the ground work for the proof of Theorem~\ref{lec13:thm:non-linear-main}.

We start by showing several basic properties about $\hatL(\beta)$. Note that for any fixed vector $v\in\R^{d}$ and $x\in \R^{d}$, when $x$ is drawn from $\cN(0,I)$, we have
\begin{equation}\label{lec13:eqn:gaussian-product}
    \Exp \l[\langle x, v\rangle^2 \r]=\Exp \l[ v^\top xx^\top v \r]=v^\top\Exp \l[ xx^\top \r]v=\|v\|_2^2.
\end{equation}

It follows that 
\begin{align}
    L(\beta)&=\frac{1}{4}\Exp_{x\sim \cN(0,I)} \l[(y-\langle \beta\odot\beta,x\rangle^2 \r] \\
    &=\frac{1}{4}\Exp_{x\sim \cN(0,I)} \l[\langle \beta^\star\odot\beta^\star-\beta\odot\beta,x\rangle^2 \r] &\text{(by definition of $y$)} \\
    &=\frac{1}{4} \l\| \beta^\star\odot\beta^\star-\beta\odot\beta \r\|_2^2.\label{lec13:eqn:loss-form} &\text{(by \eqref{lec13:eqn:gaussian-product})}
\end{align}
Note that \eqref{lec13:eqn:loss-form} is the metric that we use to characterize how close $\beta$ is to the ground-truch parameter $\beta^\star$ (see \eqref{lec13:eqn:non-linear-main}).

In the following lemma we show that $\hatL(\beta) \approx L(\beta)$ by uniform convergence. Generally speaking, uniform convergence of the loss function for all $\beta$ requires $n\ge \Omega(d)$ samples, so in our setting (where $n\ll d$) $\hatL(\beta) \approx L(\beta)$ does not always hold. However, since we assume $\beta^\star$ is sparse, the analysis only requires uniform convergence for sparse vectors.

\begin{lemma}\label{lec13:lem:RIP}
Assume $n\ge \widetilde\Omega(r^2)$. With high probability over the randomness in $x^{(1)},\cdots,x^{(n)}$, $\forall v$ such that $\|v\|_0\le r$ we have
\begin{equation}\label{lec13:eqn:RIP}
(1-\delta)\|v\|_2^2\le \frac{1}{n}\sum_{i=1}^{n}\langle v,x^{(i)}\rangle^2\le (1+\delta)\|v\|_2^2.
\end{equation}
\end{lemma}

Lemma~\ref{lec13:lem:RIP} is a special case of Lemma 2.2 in \cite{li2018algorithmic} so the proof is omitted here. We say the set $\l\{ x^{(1)},\cdots,x^{(n)} \r\}$ (or $X=[x^{(1)},\cdots,x^{(n)}]$) satisfies $(r,\delta)$\textit{-RIP condition} (\textit{restricted isometric property}) if \eqref{lec13:eqn:RIP} holds.

By algebraic manipulation, \eqref{lec13:eqn:RIP} is equivalent to 
\begin{align}\label{lec13:eqn:RIP-2}
(1-\delta)\|v\|_2^2\le v^\top \left(\frac{1}{n}\sum_{i=1}^{n}x^{(i)}(x^{(i)})^\top\right)v\le (1+\delta)\|v\|_2^2.
\end{align}
In other words, from the point of view of a sparse vector $v$ we have $\sum_{i=1}^{n}x^{(i)}(x^{(i)})^\top\approx I$. (Note however that $\sum_{i=1}^{n}x^{(i)}(x^{(i)})^\top$ is not close to $I_{d\times d}$ in other notions of closeness. For example, $\sum_{i=1}^{n}x^{(i)}(x^{(i)})^\top$ is not close to $I_{d\times d}$ in spectral norm. Another way to see this is that $\sum_{i=1}^{n}x^{(i)}(x^{(i)})^\top$ is a $d \times d$ matrix but only has rank $n \ll d$.)

As a result, with the RIP condition we have $\hatL(\beta)\approx L(\beta)$ if $\beta$ is sparse. With more tools we can also get $\nabla \hatL(\beta)\approx \nabla L(\beta)$. Let us define the set $S_r=\{\beta:\|\beta\|_0\le O(r)\}$, the set where we have uniform convergence of $\hatL$ and $\nabla \hatL$. Informally, as long as we are in the set $S_r$, $\hatL$ and $\nabla\hatL$ have similar behavior to their population counterparts. (Note, on the other hand, that there exists a dense $\beta\not\in S_r$ such that $\hatL(\beta)=0$ but $L(\beta)\gg 0.$)

The RIP condition also gives us the following lemma which will be needed for the proof of Theorem \ref{lec13:thm:non-linear-main}.

\begin{lemma}\label{lec14:lem:rip}
    Suppose $x^{(1)}, x^{(2)}, \dots x^{(n)}$ satisfy the $(r, \delta)$-RIP condition. Then, $\forall v, w$ such that $\Norm{v}_{0} \leq r$ and $\Norm{w}_{0} \leq r$, we have that
    \begin{align}
        \left| \frac{1}{n} \sum_{i=1}^{n} \langle x^{(i)}, v \rangle \langle x^{(i)}, w \rangle  - \langle v, w \rangle \right| &= \left|  v^{T} \l(\frac{1}{n} \sum_{i=1}^{n}  x^{(i)} (x^{(i)})^\top \r)  w  - \langle v, w \rangle \right| \\
        &\leq 4 \delta \Norm{v}_{2} \cdot \Norm{w}_{2}.
    \end{align} 
\end{lemma}

\tnotelong{To add proof of this lemma in the future.}
\begin{corollary}\label{lec14:cor:rip}
    Taking $w = e_1, \dots, e_d$ in Lemma~\ref{lec14:lem:rip}, we can conclude that
    \begin{align}
        \Norm{ \frac{1}{n} \sum_{i=1}^n \langle x^{(i)}, v\rangle x^{(i)} - v }_\infty &= \Norm{ \l(\frac{1}{n} \sum_{i=1}^n x^{(i)}(x^{(i)})^\top \r)v - v }_\infty \\
        &\leq 4\delta \Norm{v}_2.
    \end{align}
\end{corollary}

\subsec{Warm up: Gradient descent on population loss}

The main intuition for proving Theorem~\ref{lec13:thm:non-linear-main} is to leverage the uniform convergence when $\beta$ belongs to the set $S_r$ (see Figure~\ref{lec13:fig:uc-sr}). Note that the initialization $\beta^0$ is not exactly $r$-sparse, but taking $\alpha$ to be sufficiently small, $\beta^0$ is approximately $0$-sparse. The proof is decomposed into the following steps:

\begin{enumerate}
    \item Gradient descent on $L(\beta)$ converges to $\beta^\star$ without leaving $S_r$, and
    \item Gradient descent on $\hatL(\beta)$ is similar to gradient descent on $L(\beta)$ inside $S_r$.
\end{enumerate}

Combining the two steps we can show that gradient descent on $\hatL(\beta)$ does not leave $S_r$ and converges to $\beta^\star.$

\begin{figure}
\centering
\includegraphics[width=.7\linewidth]{figures/uc-sr.png}
\caption{Visualization of proof intuition for Theorem~\ref{lec13:thm:non-linear-main}.}
\label{lec13:fig:uc-sr}
\end{figure}

As a warm up, we prove the following theorem for gradient descent on $L(\beta).$
\begin{theorem}
For sufficiently small $\eta$, gradient descent on $L(\beta)$ converges to $\beta^\star$ in $\Theta\left(\dfrac{\log (1/ (\epsilon\alpha) )}{\eta}\right)$ iteration with $\epsilon$-error in $\ell_2$-distance.
\end{theorem}

\begin{proof}

Since
\begin{equation}
\nabla L(\beta) = (\beta\odot \beta-\beta^\star\odot\beta^\star)\odot\beta,
\end{equation}

the gradient descent step is
\begin{equation}
\beta^{t+1} = \beta^t - \eta (\beta^t \odot \beta^t -\beta^\star \odot \beta^\star)\odot\beta^t.
\end{equation}

Recall that $\beta^\star=\mathbf{1} \{i \in S \}$ and $\beta^0=\alpha \mathbf{1}$, and the update rule above decouples across the coordinates of $\beta^t$. Thus, we only need to show that $| \beta_i^\star - \beta^t | \leq \epsilon$ for the number of iterations stated in the Theorem.

\underline{Case 1: $i\in S$.} For $i \in S$, the update rule for coordinate $i$ is
\begin{align}
\beta_i^{t+1} &= \beta_i^t - \eta (\beta_i^t \cdot \beta_i^t - 1 \cdot 1) \cdot\beta_i^t \\ 
&= \beta_i^t - \eta \l[ \left(\beta_i^t\right)^2 - 1 \r] \beta_i^t.
\end{align}

Consider the following two cases:

\begin{itemize}
\item If $\beta_i^t\le 1/2$, we have
\begin{align}
\beta_i^{t+1}&=\beta_i^{t} \l[ 1+\eta \l(1- \l(\beta_i^t \r)^2 \r) \r] \\
&\ge \beta_i^t \l( 1+\frac{3}{4}\eta \r).
\end{align}

Consequently, $\beta_i^{t+1}$ grow exponentially, and it takes $\Theta\left(\dfrac{\log (1/\alpha)}{\eta}\right)$ iterations for $\beta_i^t$ to grow from $\alpha$ to at least $1/2.$\footnote{This is because $(1+\eta)^{1/\eta}\approx e$, so $(1+\eta)^{c/\eta}\approx e^{c}.$} This will bring us into the second case.
    
\item if $\beta_i^t\ge 1/2$, we have
\begin{align}
1-\beta_i^{t+1}&=1-\beta_i^t+\eta \l[ \l(\beta_i^t \r)^2-1 \r] \beta_i^t\\
&=1-\beta_i^{t}-\eta \l( 1-\beta_i^t \r) \l(1+\beta_i^t \r)\beta_i^t\\
&\le 1-\beta_i^t-\eta \l( 1-\beta_i^t \r)\beta_i^t &\text{(because $1+\beta_i^t\ge 1$)} \\
&= \l(1-\beta_i^t \r) \l( 1-\eta \beta_i^t \r) \\
&\le \l( 1-\beta_i^t \r) \l(1-\eta/2 \r). &\text{(because $\beta_i^t\ge 1/2$)}
\end{align}

Therefore it takes $\Theta\left(\dfrac{\log (1/\epsilon)}{\eta}\right)$ iterations to achieve $1-\beta_i^t\le \epsilon.$
\end{itemize}

\underline{Case 2: $i \notin S$.} For all $i \notin S$, we claim (informally) that it is sufficient to show that when $t \leq 1 / (10 \eta \alpha^{2})$, $\beta_{i}^{t} \leq 2\alpha$. This is because when $i \notin S$, $\beta_{i}$ stays small and will take many iterations before it even gets to $2\alpha$, which is close to $0$ since $\alpha$ is chosen to be small.

For a coordinate $i\notin S$, the gradient descent update for this problem becomes
\begin{align}
    \beta_i^{t+1} &= \left[ \beta^{t} - \eta (\beta^{t} \odot \beta^{t} - \beta^\star \odot \beta^\star) \odot \beta^{t} \right]_i \\
    &= \beta_i^{t} - \eta (\beta_i^{t} \cdot \beta_i^{t}) \cdot \beta_i^{t} & (\text{since } \beta_{i}^\star = 0 \ \forall i \notin S) \\
    &= \beta_i^{t} - \eta (\beta_i^{t})^{3}.
\end{align}

Since our initialization $\beta^{0}$ was small, the update to these coordinates will be even smaller because $(\beta_{i}^{t})^{3}$ is small. We can prove the desired claim using strong induction. Suppose $\beta_{i}^{s} \leq 2\alpha$ for all $s \leq t$ and $i \notin S$, and that $t+1 \leq 1 / (10\eta \alpha^{2})$. Then, for all $s \leq t$,
\begin{align}
\beta_{i}^{s+1} %&= \beta^{s}_{i} - \eta (\beta_{i}^{s})^{3} \\
    &= (1 - \eta (\beta_{i}^{s})^{2})\beta_{i}^{s} \\
    &\leq (1 + \eta (\beta_{i}^{s})^{2}) \beta_{i}^{s} \\
    &\leq (1 + 4\eta \alpha^{2}) \beta_{i}^{s}. & (\text{since } \beta_{i}^{s} \leq 2\alpha)
\end{align}

With strong induction, we can repeatedly apply this gradient update starting from $t=0$ to obtain
\begin{align}
    \beta_{i}^{t+1} &\leq \beta_{0} \cdot (1 + 4 \eta \alpha^{2})^t \\
    &\leq \beta_{0} ( 1 + 4 \eta \alpha^{2})^{\frac{1}{10 \eta \alpha^{2} }} \\
    &\leq \beta_{0} \exp \bigg(\frac{4\eta \alpha^{2}}{10 \eta \alpha^{2}} \bigg) \\
    &=  \beta_{0} \cdot e^{2/5} \\
    &\leq 2 \alpha,
 \end{align}
 which completes the inductive proof of the claim.

\end{proof}
	% reset section counter
%\setcounter{section}{0}

%\metadata{lecture ID}{Your names}{date}
\metadata{14}{Roshni Sahoo and Sarah Wu}{Mar 3rd, 2021}

\subsec{Proof of main result: Gradient descent on empirical loss}

Analyzing gradient descsent on the empirical risk $\empL$ is more complicated than analyzing gradient descent on the population risk, so we focus on the case when $\beta^\star$ is $1$-sparse, i.e. $r=1$. (When $r>1$, the main idea is the same but requires some more advanced analysis techniques.) 

Note that in our setup, i.e. when $x^{(1)} \ldots x^{(n)} \iid \mathcal{N}(0, I_{d\times d})$ and when $n\geq \widetilde{\Omega}(r/\delta^2)$, with high probability the data satisfy the $(r, \delta)$-RIP condition. It follows that when $r=1$ and $\delta = \tilO(1/\sqrt{n})$, the data are $(1, \delta)$-RIP. This will allow us to use the lemmas involving the RIP condition for the proof.

We restate the case of $r=1$ in the following theorem.
 
\begin{theorem} \label{lec14:thm:main}
Suppose $\eta \geq \widetilde{\Omega}(1).$ Then, gradient descent on $\empL$ with $t = \Theta \l(\frac{\alpha \log (1/\delta)}{\eta}\r)$ steps satisfies 
\begin{equation}
\Norm{\beta^{t} \odot \beta^{t} - \beta^\star \odot \beta^\star}_{2}^{2} \leq \tilO\l(\frac{1}{\sqrt{n}}\r).
\end{equation} 
\end{theorem}

\begin{remark}
Note that Theorem~\ref{lec14:thm:main} is a slightly weaker version of Theorem~\ref{lec13:thm:non-linear-main} for $r=1$, since the bound on the RHS depends on the number of examples and not the initialization $\alpha$. In Theorem~\ref{lec13:thm:non-linear-main}, we could take $\alpha$ as small as we like to drive the bound to zero; we cannot do this for Theorem~\ref{lec14:thm:main}.
\end{remark}

We proceed to prove Theorem~\ref{lec14:thm:main} with the follow steps:
\begin{enumerate}
\item Computing the gradient update $\nabla \empL$,
\item Dynamics analysis of noise $\zeta_t$, 
\item Dynamics analysis of signal $r_t$, and
\item Putting it all together.
\end{enumerate}

\underline{Computing the gradient update $\nabla \empL$}

WLOG, assume that $\beta^\star = e_{1}.$ We can decompose the gradient descent iterate $\beta^{t}$ as
\begin{equation}
    \beta^{t} = r_{t} \cdot e_{1} + \zeta_{t},
\end{equation}
where $\zeta_t \perp e_1$. The idea is to prove convergence to $\beta^\star$ by showing that (i) $r_{t} \rightarrow 1$ as $t \rightarrow \infty$, and (ii) $\norm{\zeta_{t}}_{\infty} \leq O(\alpha)$ for $t \leq \tilO\big(1/\eta).$ In other words, the \textit{signal} $r_{t}$ converges quickly to $1$ while the \textit{noise} $\zeta_t$ remains small for some number of initial iterations. One may be concerned that it is possible for the noise to amplify after many iterations, but we will not have to worry about this scenario if we can guarantee that $\beta^{t}$ converges to $\beta^\star$ first.

We can compute the gradient of $\empLt$ as follows. Since $y\sp{i} = \langle \beta^\star \odot \beta^\star, x\sp{i} \rangle$ and $\beta^{t} = r_{t}e_{1} + \zeta_{t} = r_{t}\beta^\star + \zeta_{t}$,
\begin{align}
    \nabla \empLt &= \frac{1}{n} \sum_{i=1}^{n} (\langle \beta^{t} \odot \beta^{t}, x\sp{i} \rangle - y\sp{i} ) x\sp{i} \odot \beta^{t} \\
    &= \frac{1}{n} \sum_{i=1}^{n} ( \langle \beta^{t} \odot \beta^{t} - \beta^\star \odot \beta^\star, x\sp{i} \rangle ) x\sp{i} \odot \beta^{t} \\
    &= \frac{1}{n} \sum_{i=1}^{n} \langle r_{t}^{2} \beta^\star \odot \beta^\star + \zeta_{t} \odot \zeta_{t} - \beta^\star \odot \beta^\star, x\sp{i}  \rangle x\sp{i} \odot \beta^{t} \\
    &= \underbrace{\frac{1}{n} \sum_{i=1}^{n} \Big\langle \big(r_{t}^{2} - 1\big) \beta^\star \odot \beta^\star + \zeta_{t} \odot \zeta_{t}, x\sp{i}  \Big\rangle x\sp{i}}_{m_t} \odot \beta^{t}.
\end{align}

To simplify the analysis, we can rearrange some of the terms that are part of the gradient. Define $m_{t} $ such that $\nabla \empLt = m_{t} \odot \beta^{t}.$ Also, let $X = \frac{1}{n} \sum_{i=1}^{n} x\sp{i}(x\sp{i})^\top.$ Then,
\begin{align}
    m_{t} &= \frac{1}{n} \sum_{i=1}^{n} \Big\langle \big(r_{t}^{2} - 1\big) \beta^\star \odot \beta^\star + \zeta_{t} \odot \zeta_{t}, \ x\sp{i}  \Big\rangle \ x\sp{i} \\
    &= \l( \frac{1}{n} \sum_{i=1}^{n} x\sp{i}\big(x\sp{i}\big)^\top \r) \l(r_{t}^{2} - 1\r) \cdot \l(\beta^\star \odot \beta^\star\r) + \l( \frac{1}{n} \sum_{i=1}^{n} x\sp{i}(x\sp{i})^\top \r) \l(\zeta_{t} \odot \zeta_{t}\r) \\
    &= \underbrace{X \big(r_{t}^{2} - 1\big) \cdot \big(\beta^\star \odot \beta^\star\big)}_{\text{part of } u_t} + \underbrace{X \big(\zeta_{t} \odot \zeta_{t}\big)}_{v_t}.
\end{align}

Now, define $u_{t} := (r_{t}^{2} - 1) (\beta^\star \odot \beta^\star) - X (r_{t}^{2} - 1) (\beta_{*} \odot \beta_{*})$ and $v_{t} := X \big(\beta_{t} \odot \beta_{t}\big)$. Then we can rewrite the gradient as

\begin{equation}
    \nabla \empLt = m_{t} \odot \beta^{t} = [(r_{t}^{2} -1) \beta^\star \odot \beta^\star - u_{t} + v_{t}] \odot \beta_{t}. \label{lec14:eqn:emp-gradient}
\end{equation}

Our goal is to show that both $u_t$ and $v_t$ are small, so that $\nabla \empLt$ is close to its population version $\nabla L(\beta^t)$. Observe that $X$ appears in both $u_{t}$ and $v_{t}$. This matrix is challenging to deal with mathematically because it does not have full rank (because $n < d$). Instead, we rely on the RIP condition to reason about the behavior of $X$: the idea is that $X$ behaves like the identity for sparse vector multiplication. Applying Corollary~\ref{lec14:cor:rip}, we can bound $\Norm{u_{t}}_{\infty}$ as
\begin{equation} \label{lec14:eqn:u-inf-norm}
    \Norm{u_{t}}_{\infty} \leq 4\delta \Norm{(r_t^2 - 1)  \beta^\star \odot \beta^\star }_{2} 
    \leq 4\delta ||\beta^\star \odot \beta^\star||_{2} \leq 4\delta.
\end{equation}

(In the second inequality, we assume that $|r_t| < 1$. We can do this because $r_t$ starts out at $\alpha$ which is small; if $r_t \geq 1$, then we are already in the regime where gradient descent has converged.) We can bound $\Norm{v_{t}}_{\infty}$ in a similar manner: since Corollary~\ref{lec14:cor:rip} implies $\Norm{v_t - \zeta_t \odot \zeta_t}_\infty \leq 4\delta \Norm{\zeta_{t} \odot \zeta_{t}}_{2}$,
\begin{align}
    \Norm{v_{t}}_{\infty} &\leq \Norm{\zeta_{t} \odot \zeta_{t}}_{\infty} + 4\delta \Norm{\zeta_{t} \odot \zeta_{t}}_{2} &(\text{by the triangle inequality}) \\
    &\leq \Norm{\zeta_{t}}_{\infty}^{2} + 4\delta \Norm{\zeta_{t} \odot \zeta_{t}}_{1} &(\text{since } \zeta_t \text{ very small}) \\
    &= \Norm{\zeta_{t}}_{\infty}^{2} + 4\delta \Norm{\zeta_{t}}_{2}^{2}. \label{lec14:eqn:v-inf-norm}
\end{align}

Note that the size of $v_t$ depends on the size of the noise $\zeta_t$. Thus, by bounding $\zeta_t$ (e.g. with a small initialization), we can ensure that $v_t$ is also small. (Ensuring bounds on $u_t$ is more difficult because it depends only on $\delta$.) In the next two subsections, we analyze the growth of $\zeta_t$ and $r_t$.

\underline{Dynamics analysis of $\zeta_t$}

First, we analyze the dynamics of the noise $\zeta_t$, which we want to ensure does not grow too fast.

\begin{lemma} \label{lec14:lem:dynamics_noise}
    For all $t\leq 1 / (c\eta\delta)$ with sufficiently large constant $c$, we have
    \begin{equation} \label{lec14:eqn:dynamics_noise}
        \Norm{\zeta_t}_\infty \leq 2\alpha, \quad \quad \Norm{\zeta_t}_2^2 \leq \frac{1}{2}, \quad \quad \text{and} \quad \Norm{\zeta_{t+1}}_\infty \leq \big(1 + O(\eta\delta)\big) \Norm{\zeta_t}_\infty.
    \end{equation}
\end{lemma}
Note that this result is weaker than what we were able to show for the population gradient (exponential growth with a small fixed rate), but we will ultimately show that the growth of the signal will be even faster.

\begin{proof}
Recall that the empirical gradient \eqref{lec14:eqn:emp-gradient} is $\nabla \hat{L}(\beta) = \big[(r_{t}^{2} - 1) \beta^\star \odot \beta^\star - u_{t} + v_{t} \big] \odot \beta^{t}$. Hence, the gradient update to $\beta^{t}$ is

\begin{align}
\beta^{t+1} &= \beta^{t} - \eta \l[\l(r_{t}^{2} - 1\r) \beta^\star \odot \beta^\star - u_{t} + v_{t} \r] \odot \beta^{t} \\
&= \underbrace{\beta^{t} - \eta \l(r_{t}^{2} - 1\r) \beta^\star  \odot \beta^\star \odot \beta^{t}}_{\text{GD update for population loss}} - \eta \l(- u_{t} + v_{t}\r) \odot \beta^{t}. \label{lec14:eqn:gd-update}
\end{align}
    
Recall that $\zeta_{t+1}$ is simply $\beta^{t+1}$ except for the first coordinate (where it has a zero instead of $r_{t+1}$), i.e. $\zeta_{t+1}$ is the projection of $\beta^{t+1}$ onto the subspace orthogonal to $e_1$. Hence,
\begin{align}
\zeta_{t+1} &= \l(I - e_{1} e_{1}^\top\r) \beta^{t+1} \\
&= \l(I - e_{1} e_{1}^\top\r) \cdot \beta^{t} - \eta \l(I - e_{1} e_{1}^\top\r) (v_{t} - u_{t}) \odot \beta^{t} &\text{(by \eqref{lec14:eqn:gd-update}, second term = 0)} \\
&= \zeta_{t} - \eta \l[\l(I - e_{1}e_{1}^{T}\r) (v_{t} - u_{t}) \odot \l(I - e_{1}e_{1}^{T}\r) \beta^{t}\r] &(\text{by distribution law for $\odot$}) \\
&= \zeta_t - \eta \underbrace{\l[ \l(I - e_{1}e_{1}^{T}\r) \l(v_t - u_t\r)\r]}_{\rho_t} \odot \zeta_t.
\end{align}
    
If we define $\rho_t$ such that $\zeta_{t+1} = \zeta_t - \eta \rho_t \odot \zeta_t$, then the growth of $\zeta_t$ is dictated by the size of $\rho_t$. We can bound this as
\begin{equation}
\Norm{\zeta_{t+1}}_{\infty} \leq (1 + \eta \Norm{\rho_{t}}_{\infty}) \Norm{\zeta_{t}}_{\infty}. \label{lec14:eqn:zeta-growth-bd}
\end{equation}

Now, we will prove the lemma by using strong induction on $t$. Suppose that the first two pieces of \eqref{lec14:eqn:dynamics_noise} hold for all iterations up to $t$. We can show that
\begin{align}
\Norm{\rho_{t}}_{\infty} &\leq \Norm{u_{t}}_{\infty} + \Norm{v_{t}}_{\infty}  \\
&\leq 4\delta + \Norm{\zeta_t}_{\infty}^{2} + 4\delta \Norm{\zeta_t}_{2}^{2} &(\text{by \eqref{lec14:eqn:u-inf-norm} and \eqref{lec14:eqn:v-inf-norm}}) \\
&\leq  4\delta + (2\alpha)^2 + 4\delta \cdot \frac{1}{2} &(\text{by the inductive hypothesis})\\
\label{lec14:eqn:diff-inf-norm}
&\leq 8\delta,
\end{align}
where the last step holds because we can take $\alpha$ to be arbitrarily small (e.g. $\alpha \leq \text{poly}(1/n) \leq O(\delta)$). Plugging this into \eqref{lec14:eqn:zeta-growth-bd}, we have
\begin{equation}
\Norm{\zeta_{t+1}}_\infty \leq (1 + 8\eta \delta) \Norm{\zeta_t}_{\infty} = \big(1 + O(\eta\delta)\big) \Norm{\zeta_t}_\infty,
\end{equation}
which proves the third piece of the lemma. Using this piece, we can show that
\begin{equation}
\Norm{\zeta_{t+1}}_{\infty} \leq \l(1 + 8 \eta \delta\r)^{t+1} \Norm{\zeta_{0}}_{\infty} \leq \big(1 + 8\eta \delta\big) ^{1/(c\eta \delta)} \cdot \alpha  \leq 2\alpha
\end{equation}
for a sufficiently large constant $c$, which proves the second piece. Finally, we show that
\begin{equation}
\Norm{\zeta_{t+1}}_{2}^2 \leq \big(1 + 8\eta \delta\big)^{t+1}\Norm{\zeta_{0}}_{2}^2 \leq \big(1 + 8\eta \delta)^{1/(c\eta \delta)} \cdot \alpha \sqrt{d} \leq \frac{1}{2},
\end{equation}
if $\alpha \leq \frac{1}{n^{O(1)}}$, which proves the first piece.

\end{proof}

\underline{Dynamics analysis of $r_t$}

Next, we analyze the dynamics of the signal $r_t$, which we want to show converges to 1.

\begin{lemma} \label{lec14:lem:dynamics_signal}
    For all $t\leq 1 / (c\eta\delta)$ with sufficiently large constant $c$, we have that
    \[ r_{t+1} = \big(1 + \eta\big( 1 - r_t^2 \big) \big) r_t + O\big(\eta\delta\big) r_t. \]
\end{lemma}
Note that the first term on the RHS is $r_{t+1}$ during gradient descent on the population loss, and the second term captures the error.

\begin{proof}
    Recall that the gradient descent update from the empirical gradient~\eqref{lec14:eqn:emp-gradient} is
    \begin{equation}
        \beta^{t+1} = \beta^t - \eta \big[\big(r_{t}^{2} -1\big) \beta^\star \odot \beta^\star - u_{t} + v_{t}\big] \odot \beta_{t}.
    \end{equation} 
    We have that
    \begin{align}
        r_{t+1} &= \big\langle \beta^{t+1}, e_1\big\rangle \\
        &= \big\langle \beta^t, e_1\big\rangle - \eta \big(r_{t}^{2} -1\big)\big\langle \beta^t, e_1\big\rangle - \eta \big\langle v_t-u_t, e_1\big\rangle \big\langle \beta^t, e_1 \big\rangle \\
        &= r_t - \eta \big(r_{t}^{2} -1\big) r_t - \eta \big\langle v_t-u_t, e_1\big\rangle r_t \\
        &= \Big(1 + \eta\big( 1 - r_t^2 \big) \Big) r_t + \eta \big\langle u_t-v_t, e_1\big\rangle r_t
    \end{align}
    so all we need to do is bound the second term as follows:
    \begin{align}
        |\eta \langle v_t - u_t, e_1\rangle r_t| &\leq \eta \cdot r_t \Norm{v_t-u_t}_\infty \\
        &\leq \eta \cdot r_t \cdot 8\delta &(\text{by \eqref{lec14:eqn:diff-inf-norm}}) \\
        &= O(\eta\delta) \cdot r_t.
    \end{align}
\end{proof}

\underline{Putting it all together}
Finally, we return to the proof of Theorem~\ref{lec14:thm:main}. By Lemma~\ref{lec14:lem:dynamics_signal}, we know that as long as $r_t \leq 1/2$ it will grow exponentially fast, since
\begin{equation}
    r_{t+1} \geq \Big(1 + \eta\big(1-r_t^2\big) - O(\eta\delta) \Big) \cdot r_t \geq \bigg(1 + \frac{\eta}{2}\bigg)\cdot r_t.
\end{equation} 
This implies that at some $t_0 = O\Big(\frac{\log (1/\alpha)}{\eta}\Big)$, we'll observe $r_{t_0} > 1/2$ for the first time. Consider what happens after this point.

\begin{itemize}
    \item When $1/2 < r_t \leq 1$, we have that
    \begin{align}
        1 - r_{t+1} &\leq 1 - r_t - \eta \big(1 - r_t^2\big) r_t + O(\eta\delta) \cdot r_t \\
        &\leq 1 - r_t - \frac{\eta \big(1 - r_t^2\big)}{2} + O(\eta\delta) \\
        &\leq 1 - r_t - \frac{\eta \big(1 - r_t\big)}{2} + O(\eta\delta) \\
        &= \bigg(1 - \frac{\eta}{2}\bigg) (1 - r_t) + O(\eta\delta).
    \end{align}
    Thus, we can achieve $1 - r_{t+1} \leq 2 \cdot \frac{O(n\delta)}{\eta/2} = O(\delta)$ in $\Theta\Big(\frac{\log(1/\delta)}{\eta}\Big)$ iterations.
    
    \item When $r_t > 1$, we can show in a similar manner that
    \begin{equation}
        r_{t+1} - 1 \leq (1 - \eta) (r_t - 1) + O(\eta\delta) \leq O(\delta),
    \end{equation} 
    implying that $r_t$ remains very close to 1 after the same order of iterations.
\end{itemize}

This completes the proof of Theorem~\ref{lec14:thm:main}, bounding the number of iterations needed for gradient descent on the empirical loss to converge to $\beta^*$.
\qed 

	\metadata{16}{Leah Reeder and Trevor Maxfield}{Nov 10th, 2021}

\sec{From Small to Large Initialization: a Precise Characterization} \tnote{please double check consistency of capitalization in section headers}

We have previously discussed how certain initializations of gradient descent converge to minimum-norm solutions. In the sequel, we characterize the effect of initialization more precisely---we will show that in a variant of the settings in Section~\ref{?} \tnote{section 9.2}, we can precisely compute the corresponding regularizer induced by any initialization. We will see that when the initialization is small, we obtain the bias towards minimum norm solution (in the parameter space used in optimization), whereas when the initialization is large, we are in the NTK regime (Section~\ref{?} \tnote{ntk section})where the implicit bias is towards the min norm solution under the NTK kernel. 

\subsection{Preparation: gradient flow}
To simplify the analysis, we will consider the concept of gradient flow, i.e. gradient descent with an infinitesimal learning rate.  This allows us omit the second order effect from the learning rate and simplify the analysis. 

We begin by recalling the gradient descent update formula. In our previous description of gradient descent, we indexed the updated parameters by $t = 1,2,\dots$. Anticipating our generalization to infinitesimal steps, we will index the updated parameters using parentheses instead of subscripts. In particular, the standard gradient descent update given a loss function $L(w)$ is
\al{
w(t+1) = w(t) - \eta \nabla L(w(t)).
}
If we scale the time by $\eta$ so that each update by gradient descent corresponds to a time step of size $\eta$ (rather than size 1), the update becomes
\al{
w(t + \eta) = w(t) - \eta \nabla L(w(t)).
}
Taking $\eta \to 0$ yields a differential equation, which can be thought of as a continuous process rather than discrete updates:
\al{
w(t+dt) = w(t) - dt \cdot \nabla L(w(t)).
}
This can also be written as:
\al{
\dot{w}(t) = -\nabla L(w(t) \quad \text{ with } \quad \dot{w}(t) = \frac{\partial w(t)}{\partial t}
}
This allows us to ignore the $\eta^2$ term (alternatively the $(dt^2)$ term), which will simplify some of the technical details that follow.

\subsec{Characterizing the implicit bias of initialization}
The results in this section are slight simplification of the recent paper by~\citet{woodworth2020kernel}. The model is a variant of the one we introduced in \eqref{lec13:eqn:hadamard_model_1}. Recalling that $x^{\odot 2} = x \odot x$, let
\al{
f_w(x) = \left(w_+^{\odot 2} - w_-^{\odot 2}\right)^\top x.
}
where $w_+, w_- \in \R^d$. Let $w$ denote the concatenation of the two parameter vectors, i.e. $= (w_+, w_-)$.  In \eqref{lec13:eqn:hadamard_model_1}, we defined $f_\beta(x) = (\beta \odot \beta)^\top x$; this model can only represent positive linear combinations of $x$.  By contrast, $f_w(x)$ can represent any linear model. Moreover, if we choose our initialization for $w$ such that $w_+(0) = w_-(0)$, we obtain $f_{w(0)}(x) \equiv 0$ for all $x$. Similar to our analysis of the NTK, this initialization will simplify the subsequent derivations.

For the loss, we define\tnote{a bit more formal language}
\al{
\hatL(w) = \frac{1}{2} \sum_{i=1}^n \left( y\sp{i} - f_w(x\sp{i})\right)^2
}
and consider the initialization
\al{
w_+(0) = w_-(0) = \alpha \cdot \vec{\mathbf{1}}
}
where $\vec{\mathbf{1}}$ denotes the all-ones vector. The analysis technique still applies to any general initializations as long as all the dimension are initialized to be non-zero, but the the initialization scale is the most important factor, and therefore we chose this simplification for the ease of exposition. 

Note that every $w = (w_+, w_{-})$ corresponds to a de facto linear function of $x$. We denote the resulting linear model as $\theta_w$:
\al{
\theta_w = w_+^{\odot 2} - w_-^{\odot 2}.
}
Note that $\theta_w^\top x = f_w(x)$. 

Let $w(\infty)$ denote the limit of the gradient flow, i.e.
\al{
w(\infty) = \lim_{t \to \infty} w(t).
}
Then, the converged linear model in the $\theta$ space is defined by $\theta_\alpha(\infty) = \theta_{w(\infty)}$---we are interested in understanding its properties.  For simplicity, we will omit the $\infty$ index and refer to this quantity as $\theta_\alpha$. We assume throughout that the limit exists and all corresponding regularity conditions are met.

Let
\al{
X = \begin{bmatrix} x^{(1)^\top} \\ \vdots \\ x^{(n)^\top} \end{bmatrix} \in \R^{n \times d} \quad \text{ and } \quad \hat{y} = \begin{bmatrix} y^{(1)} \\ \vdots \\ y^{(n)} \end{bmatrix}.
}
\tnote{should be $\vec{y}$}
And with this setup we can give the following theorem\tnote{more formal language}:
 % 18:30
\begin{theorem}[Theorem 1 in \cite{woodworth2020kernel}]\tnote{use citet}
	 \label{lec16:thm:interpolatingAlpha}
For any $0 < \alpha < \infty$, assume that we \tnote{/GF with initilaization..} converge to a solution that fits the data exactly: $X \theta_{\alpha} = \vec{y}$.\footnote{This assumption can likely be proved to be true and thus not required. Here we still include the condition because the original paper~\citet{woodworth2020kernel} assumed it.}  Then, the solution satisfies the following notion of minimum complexity:
\al{ 
\theta_\alpha = \argmin_\theta Q_\alpha(\theta)\\
 \quad \textup{ s.t. } \quad X \theta = y \label{lec16:eqn:constrained_complexity}
}
\tnote{$y$ should be $\vec{y}$; check other occurrences?}
where
\al{
Q_\alpha(\theta) = \alpha^2 \cdot \sum_{i=1}^n q\left(\frac{\theta_i}{\alpha^2} \right)
}
and
\al{
q(z) = 2 - \sqrt{4 + z^2} + z \cdot \textup{arcsinh}\left(\frac{z}{2}\right)
}
\end{theorem}
In words, Theorem~\ref{lec16:thm:interpolatingAlpha} claims that $\theta_\alpha$ is the minimum complexity solution for the complexity measure $Q_\alpha$.

%23 minutes.
\begin{remark}
In particular, when $\alpha \to \infty$ we have that 
\begin{align}
    q(\theta_i /\alpha^2) \asymp \theta_i^2/\alpha^4
\end{align}
and so 
\begin{align}
    Q_\alpha(\theta) \asymp \frac{1}{\alpha^2} \Norm{\theta}_2^2.
\end{align}
This means that if $\alpha \to \infty$ than the complexity measure $Q_\alpha$ is the $\ell_2$-norm, $||\theta||_2$.  If $\alpha \to 0$, then the complexity measure becomes
\al{
q\left(\frac{\theta_i}{\alpha^2}\right) &\asymp \frac{\left|\theta_i\right|}{\alpha^2} \log\left(\frac{1}{\alpha^2}\right) \quad\text{(by Taylor expansion)}
}
and so,
\al{
Q_\alpha\left(\theta\right) &\asymp \frac{\Norm{\theta}_1}{\alpha^2} \log\left(\frac{1}{\alpha^2}\right)
}
To summarize, for $\alpha \to \infty$, the constrained minimization problem we solve in \eqref{lec16:eqn:constrained_complexity} yields the minimum $\ell_2$-norm solution of $\theta$ (i.e. the $\ell_4$-norm for $w$).  When $\alpha \to 0$, solving \eqref{lec16:eqn:constrained_complexity} yields the minimum $\ell_1$-norm $\theta$ (which is the $\ell_2$-norm for $w$).  For $0 < \alpha < \infty$, we obtain some interpolation of $\ell_1$ and $\ell_2$ regularization of the optimum.
\end{remark}

%27.30 minutes
\begin{remark}
Note that when $\alpha \to 0$, the intuition is similar to what we had observed in previous analyses; in particular, the solution is the global minimum closest to the initialization.  Note however, that when $\alpha \neq 0$, the solution discovered by gradient descent will not \textit{exactly} correspond to the solution closest to the initialization.
\end{remark}

\begin{remark}
When $\alpha \to \infty$, we claim that the model optimization is in the neural tangent kernel (NTK) regime.  Recall that we had two parameters, $(\sigma, \beta)$, that determined if we could treat the optimization problem as a kernel regression. Further recall that $\sigma$ denotes the minimum singular value of $\Phi$ and $\beta$ is the Lipschitzness of the gradient. Let us now compute $\sigma$ and $\beta$ for large $\alpha$ initializations of our model.

For $w_-(0) = w_+(0) = \alpha \vec{\mathbf{1}}$,
\al{
\nabla f_{w(0)}(x) = 2 \begin{bmatrix} W_{+}(0) \cdot x \\ -W_{-}(0) \odot x \end{bmatrix} = 2 \alpha \begin{bmatrix} x \\ -x \end{bmatrix}
}
\tnote{litter letter for W in equation above; other occurrences?}
by the chain rule.  It is clear then that both $\sigma$ and $\beta$ linearly depend on $\alpha$.  This implies that
\al{
\frac{\beta}{\sigma^2} \to 0 \quad \text{ as } \alpha \to \infty
}
since the denominator is $O(\alpha^2)$, while the numerator is $O(\alpha)$.  In particular, the features used in this kernel method are:
\al{
\phi(x) = \nabla f_{w(0)} (x) = 2 \alpha \begin{bmatrix} x \\ - x \end{bmatrix}
}
The neural tangent kernel perspective then gives an alternative proof of this complexity minimization result for $\alpha \to \infty$. In the NTK regime, the solution (to our convex problem) is always the minimum $\ell_2$-norm solution for the feature matrix, which in this case equals $\begin{bmatrix} x \\ - x \end{bmatrix}$. 

Note that practice tends not to follow the assumptions made here. Often, people either do not use large initializations or do not use infinitesimally small step sizes. But this is a good thing  because we do not want to be in the NTK regime; being in the NTK regime implies that we are doing no different or better than just using a kernel method.
\end{remark}

We can now prove Theorem~\ref{lec16:thm:interpolatingAlpha}, which is similar to the overparametrized linear regression proof of Theorem~\ref{lec13:thm:linear-main}.

This proof follows in two steps:
\begin{enumerate}
\item We find an invariance maintained by the optimizer. In the overparametrized linear regression proof of Theorem~\ref{lec13:thm:linear-main}, we required $\theta \in \text{span}\{x\sp{i}\}$.  For this proof, we will use a slightly more complicated invariance.
\item We characterize the solution using this invariance.  The invariance, which depends on $\alpha$, will tell us which zero error solution the optimization converges to.
\end{enumerate}
Note also that all of these conditions only depend upon the empirically observed samples. The invariance and minimum is not defined with respect to any population quantities.
\begin{proof}  
Let
\al{
\tilde{X} = \begin{bmatrix}x & -x\end{bmatrix} \in \R^{n \times 2d} \quad \text{ and } \quad w(t) = \begin{bmatrix} w_+(t) \\ w_-(t) \end{bmatrix} \in \mathbb{R}^{2d}.
}
\tnote{litter x to capital X in the above equations; could you help check if other occurrences}
Then, the model output on $n$ data points can be described in matrix notation as follows:
\al{
\tilde{X} w(t)^{\odot 2} = \begin{bmatrix}x & -x\end{bmatrix} \begin{bmatrix} w_+(t)^{\odot 2} \\ w_-(t)^{\odot 2} \end{bmatrix} = \begin{bmatrix} f_{w(t)} (x\sp{1}) \\ \vdots \\ f_{w(t)}(x\sp{n})\end{bmatrix} \in \R^n.
}
Given the loss function,
\al{
L(w(t)) = \frac{1}{2} \Norm{\tilde{X} w(t)^{\odot 2} - \vec{y}}_2^2,
}
the gradient of $w(t)$ can be computed as
\al{
\dot{w}(t) &= -\nabla L(w(t)) \\
&= - \nabla \left( \Norm{\tilde{X} w(t)^{\odot 2} - \vec{y}}_2^2 \right) \\
&= \left(\tilde{X}^\top r(t)\right) \odot w(t) \quad \quad \quad \text{(chain rule)}\label{lec16:eqn:Xtrtwt}
}
where $r(t) = \tilde{X} w(t)^{\odot 2} - \vec{y}$ denotes the residual vector.  We see that the $\tilde{X}^\top r(t)$ term in \eqref{lec16:eqn:Xtrtwt} is reminiscent of linear regression for which it would correspond to the gradient, although the $\odot w(t)$ reminds us that this problem is indeed quadratic.

We cannot directly solve this differential equation, but we claim that
\al{ \label{lec16:eqn:w_claim}
w(t) = w(0) \odot \text{exp}\left(-2\tilde{X}^\top \int_0^\top r(s) ds \right) \quad \text{(exp is applied entry-wise)}
}
which is not quite a closed form solution of equation \ref{lec16:eqn:Xtrtwt} since $r(s)$ is still a function of $w(t)$.  To understand how we obtained this ``solution,'' we consider a more abstract setting. Suppose that
\al{
\dot{u}(t) &= v(t) \dot u(t)
}
We can then ``solve'' this differential equation as follows. Rearranging, we observe that
\al{
\frac{\dot{u}(t)}{u(t)} &= v(t) \\
\frac{d \log u(t)}{dt} &= v(t) \quad \text{(chain rule)} \\
\log u(t) - \log u(0) &= \int_0^t v(s) ds \quad \text{(integration)} \\
\frac{u(t)}{u(0)} &= \text{exp} \left( \int_0^t v(s) ds\right)
}
In our problem, $u \leftrightarrow w_i$ and $v \leftrightarrow (\tilde{X}^\top r(t))_i$.

We have characterized $w$, but we want to transform this to a characterization that involves $\theta$.
Recall that \(w_+(0) = \alpha \vec{\mathbf{1}}\) and \(w_-(0) = \alpha \vec{\mathbf{1}}\) so that \(w(0) = \alpha \vec{\mathbf{1}} \in \R^{2d}\). Additionally, we have that \(\theta(t) = w_+(t)^{\odot 2} - w_-(t)^{\odot 2} \).
We can now apply \eqref{lec16:eqn:w_claim} to expand \(w(t)\) and simplify. 
\tnote{I think from here until 9.126, all the little $x$ should be capital $X$}Note that if we have \(\tilde{x}^\top = \begin{bmatrix} x^\top \\ -x^\top \end{bmatrix} \in \R^{2n\times d}\), then for some vector \(v\),
\al{
    \left(\exp(-2\tilde{x}^\top v) \right)^{\odot 2} &=
    \begin{bmatrix}
    \exp(-2x^\top v) \\
    \exp(2x^\top v)
    \end{bmatrix}^{\odot 2} \\
    &= \begin{bmatrix}
    \exp(-4x^\top v) \\
    \exp(4x^\top v)
    \end{bmatrix}.
}
Applying this result for $v = \int_0^T r(s) ds$, we obtain that:
\al{
    \theta(t) &= w_+(t)^{\odot 2} - w_-(t)^{\odot 2} \\
    &= \alpha^2 \left[ \exp \left( -4 x^\top \int_0^t r(s) ds \right) - \exp \left( 4 x^\top \int_0^t r(s) ds \right)\right] \\
    &= 2 \alpha^2 \sinh \left(-4 x^\top \int_0^t r(s) ds \right).
}
Letting $t \to \infty$, we have that
\al{\label{lec16:eqn:theta_infty}
    \theta_\alpha = 2 \alpha^2 \sinh \left(-4x^\top \int_0^\infty r(s) ds \right).
}
Lastly, we also know 
\al{
    X \theta_\alpha = y \label{lec16:eqn:theta_constraint}
 } 
 since this is the assumption by the theorem (which should can be proven because the optimization should converge to a zero-error solution). We next show that \eqref{lec16:eqn:theta_infty} and \eqref{lec16:eqn:theta_constraint} are also sufficient conditions for a solution to the constrained optimization problem given by \eqref{lec16:eqn:constrained_complexity}. In particular, \eqref{lec16:eqn:theta_infty} and \eqref{lec16:eqn:theta_constraint} correspond to the Karush-Kuhn-Tucker (or KKT) conditions of \eqref{lec16:eqn:constrained_complexity}.

A KKT condition is an optimality condition for constrained optimization problems. While these conditions can have a variety of formulations and typically one can invoke some off-the-shelf theorems to use them, we can motivate the conditions we encountered by considering the following general optimization program:
\al{
    \argmin \quad &Q(\theta) \\
    \text{s.t.} \quad &X\theta = y.
}
We say that \(\theta\) satisfies the (first order) KKT conditions if
\begin{align}
    \nabla Q(\theta) &= X^\top \nu \text{ for some } \nu \in \R^n \\
    X\theta &=y
\end{align}
More intuitively, we know that optimality implies that there are no first order local improvements that satisfy the constraint (up to first order). Then, consider a perturbation \(\Delta \theta\). In order to satisfy the constraint, we must enforce the following:
\begin{align}
\Delta \theta \perp \text{row-span}\{X\}  \quad \text{ so } \quad X \Delta \theta = 0
\end{align}
So, if we look at \(\theta + \Delta \theta \) satisfying the constraint, we can use a Taylor expansion to show that
\al{
Q(\theta + \Delta \theta) = Q(\theta) + \langle \Delta \theta, \nabla Q(\theta) \rangle \leq Q(\theta)
}
because if \( \langle \Delta \theta, \nabla Q(\theta) \rangle\) is positive it violates the optimality assumption.
In fact, it is very easy to make the sign flip for \( \langle \Delta \theta, \nabla Q(\theta) \rangle\) because you can flip \(\Delta \theta\) to be the opposite direction. This means that
\al{
    \forall \, \Delta \theta \perp \text{row-span}\{X\}, \quad \langle \Delta \theta, \nabla Q(\theta) \rangle = 0
}
because if it is negative, you can equivalently flip it to be positive which violates optimality.
This means that \(Q(\theta) \subseteq \text{row-span}\{X\}\), or \(Q(\theta) = X^\top \nu\) for some $\nu$.

Returning to our problem, the KKT condition gives
\al{
    \nabla Q(\theta) = X^\top \nu
}
\tnote{little x to big $X$ again}
and the invariance gives us
\al{
    \theta_\alpha &= 2 \alpha^2 \sinh\left(-4x^\top \int_0^\infty r(s) ds \right) \\
    &= 2\alpha^2 \sinh \left( -4x^\top v'\right)
}
where we let \(v' = \int_0^\infty r(s) ds\) for simplicity.
Taking the gradient of \(Q\) gives
\al{
    \nabla Q_\alpha (\theta) = \operatorname{arcsinh}\left(\frac{1}{2\alpha^2} \theta \right)
}
Plugging in \(\theta_\alpha\), we get
\al{
    \nabla Q(\theta_\alpha) = \operatorname{arcsinh}\left (\frac{1}{2\alpha^2} \theta_\alpha \right ) = -4 x^\top v'
}
Thus, \(\theta_\alpha\) satisfies both KKT conditions. Even further, since our optimization problem~\eqref{lec16:eqn:constrained_complexity} is convex (we do not formally argue this), we conclude that \(\theta_\alpha\) is a global minimum.
\end{proof}

\sec{Implicit Regularization Towards Max-margin Solutions In Classification Problems}
We now switch our focus to classification problems. We consider linear models (though these results also apply to nonlinear models with a weaker version of the conclusion). We assume that our data is separable and will prove that gradient descent converges to the max-margin solution. This result holds for any initialization and does not require any additional regularization; we only require the use of gradient descent and the standard logistic loss function.

\tnote{is this backslash parenthesis a latex standard way to replace dollar sign? If not, could you replace them back to dollar sign? It will create overheads for future editing so let's avoid using it now.}
Assume we have data \(\{(x\sp{i}, y\sp{i}) \}_{i=1}^n \), where \(x\sp{i} \in \R^d\) and \(y\sp{i} \in \{\pm 1 \}\). We consider the linear model \( h_w(x) = w^\top x\) and the cross entropy loss function \(\hatL (w) = \sum_{i=1}^n \ell\left(y\sp{i}, h_w\l (x\sp{i} \r )\right)\), where \( \ell(t) = \log(1 + \exp(-t))\) is the logistic loss.

As we have separable data, there can be multiple global minima, as you can trivially take an infinite number of separators. More formally, there are an infinite number of unit vectors \(\bar{w}\) such that $\bar{w}^\top x\sp{i} y\sp{i} > 0$ for all $i$ as one can perturb any strict separator  while still maintaining a separation of classes. Then, we can scale the separator to make the loss arbitrarily small---we have that \( \hatL(\alpha \bar{w}) \to 0\) as \( \alpha \to \infty\). Thus, informally, for any unit vector $\bar{w}$ that separate the data, $\infty \cdot \bar{w}$ is a global minimum.\tnote{this paragraph didn't make much sense before..} %Thus, even if we arbitrarily scale the unit vector, you still have that the loss goes to zero as \(\ell(t)\) approaches zero as \(t\) gets large. Thus, all choices of $w$ correspond to global minima, as the loss function goes to zero for infinite scalings.

We would like to understand which global minimum gradient descent converges to. We will now show that it finds the max-margin solution. Before we can do so, we recall/introduce the following definitions.

\begin{definition}[Margin]
Let \(\{(x\sp{i}, y\sp{i}) \}_{i=1}^n \) be given data. Assuming \(w\) is linearly separable, a \textit{margin} is defined as
\al{
    \min_{i \in [1,...,n]} y\sp{i} w^\top x\sp{i}
}
\end{definition}
\tnote{change $[1,...,n]$ to $[n]$ (it also applies to other occurrences)}

\begin{definition}[Normalized Margin]\label{lec16:def:norm_margin}
Let \(\{(x\sp{i}, y\sp{i}) \}_{i=1}^n \) be given data. Assuming \( w\) is linearly separable\tnote{a classifier cannot be linearly separable}, a \textit{normalized margin} is defined as
\al{
    \gamma(w) = \frac{\min_{i \in [1,...,n]} y\sp{i} w^\top x\sp{i}}{\norm{w}_{2}}
}
\end{definition}

\begin{definition}[Max-Margin Solution]
Using the normalized margin \(\gamma\) defined in Definition~\ref{lec16:def:norm_margin}, we define a \textit{max-margin solution} as
\al{
    \bar{\gamma} = \max_{w} \gamma(w)
}
and let \(w^*\) be the unit-norm maximizer. \tnote{add a footnote on the scale invariance of this notion}
\end{definition}

Using these definitions, we claim the following result.
\begin{theorem} \label{lec16:thm:maxmargin_gd}
Gradient flow converges to the direction of max-margin solution in the sense that
\al{
    \gamma(w(t)) \to \bar{\gamma} \text{  as  } t \to \infty
}
where \(w(t)\) is the iterate at time \(t\).
\end{theorem}

The following observations provide some intuition for Theorem~\ref{lec16:thm:maxmargin_gd}.
\begin{enumerate}
    \item \(\hatL(w(t)) \to 0\) by a standard optimization argument. This is because if our optimization iteration is working, \(w(t)\) at large \(t\) will cause the loss function to go to zero.\tnote{to reword a bit}
    \item Using a Taylor expansion, we can show that \( \ell(z) = \log(1 + \exp(-z)) \approx \exp(-z)\) for large \(z\). Thus, logistic loss is close to exponential loss when \(z\) is very large.
    \item Using observation 1, we see that \(\norm{w(t)}_{2} \to \infty\) because if \(\norm{w(t)}_{2}\) were instead bounded, then the loss \(\hatL (w(t))\) will be bounded below by a constant that is strictly greater than zero, contradicting observation 1. Formally, if
    \(\norm{w(t)}_{2} \leq B,\)
    then
    \al{
        |y\sp{i} w^t x\sp{i}| \leq B \norm{x\sp{i}},
    }
    and therefore we get
    \al{
        \hatL(w(t)) \geq \sum_{i=1}^n \exp\left(-B\norm{x\sp{i}}_{2} \right)> 0.
    }
    \item Suppose we have \(w\) such that \(\norm{w}_{2} = q \) is very big. Then, using observation 2, we see that
    \al{
        \hatL(w) &= \sum_{i=1}^n \ell(y\sp{i} w^\top x\sp{i}) \\
        &\approx \sum_{i=1}^n \exp\left(-y\sp{i} w^\top x\sp{i} \right) \\
        \log \hatL(w) &\approx \log \sum_{i=1}^n \exp\left(-y\sp{i} w^\top x\sp{i} \right) \\
        &= \log \sum_{i=1}^n \exp \left(-q y\sp{i} \bar{w}^\top x\sp{i} \right) \\
        &\approx \max_{i \in [1,2,...,n]} -q y\sp{i} \bar{w}^\top x\sp{i}
    }
    where \( \bar{w} = \frac{w}{\norm{w}_{2}}\) and the last step holds because the log of a sum of exponentials (\textit{log-sum-exp}) is a smooth approximation to the maximum function. To motivate this claim, observe that:  
    \al{
         \log \sum_{i=1}^n \exp(a u_i) &\geq q \max_i u_i  \\
        \log \sum_{i=1}^n \exp(a u_i) &\leq \log \left(n \exp(q \max_i u_i)\right) \\
        &= \log n + q \max_i u_i \\
        &\approx q \max_{i \in [1,2,...,n]} u_i + o(q) \text{ as } q \to \infty
    }
    Thus, minimizing the loss is the same as
    \al{
    \min_w \max_{i \in [1,2,...,n]} -qy\sp{i} \bar{w}^\top x\sp{i}
    }
    which can be reformulated as
    \al{
    \max_w \min_{i \in [1,2,...,n]} qy\sp{i} \bar{w}^\top x\sp{i}
    }

\end{enumerate}

The above observations demonstrate that minimizing the logistic loss with gradient descent is equivalent (in the limit) to maximizing the margins. This constitutes an intuitive proof of how gradient flow converges to the direction of the max-margin solution.

\sec{Implicit Regularization Effect of Noise in SGD}

In the previous section, we discussed implicit regularization via initialization and the implicit regularization of gradient descent for logistic loss-minimizing classifiers. 
%These methods were based on a specific model setup and limited to gradient flow. 
In the sequel, we will move forward to the implicit regularization effect of SGD noise. Starting from the quadratic case, we analyze how the SGD noise will affect the optimization solution, and present (heuristically) a result for non-quadratic loss functions. In particular, the main (heuristic) results are:
\begin{enumerate}
\item On the 1-dimensional quadratic function, the iterate can be disentangled into a contraction part and a stochastic part, the latter of which is characterized by the Ornstein–Uhlenbeck (OU) process. The noise makes the iterate bounce around the global minimum.
\item On the multi-dimensional quadratic function, the iterate can be disentangled into multiple separate 1-D OU processes. The noise makes the iterate bounce around the global minimum, while the fluctuation is closely related to the shape of the noise.
\item On non-quadratic functions, SGD with \textit{label noise} on empirical loss $\hat{L}(\theta)$ converges to a stationary point of the regularized loss $\hat{L}(\theta) + \lambda \textup{tr}(\nabla^2\hat{L}(\theta))$, which is mainly due to the accumulation of a third order effect.
\end{enumerate}
 

For the remainder of this section, let $g(x)$ denote the general loss function. Then, the formulation of SGD is: for $t$ in $[0,T]$,
\begin{align}
\theta_{t+1} = x_{t} - \eta(\nabla g(x_{t}) + \xi_t),
\end{align} 
where $\eta > 0$ is the learning rate, $\xi_t$ denotes the SGD noise, and $\Exp[\xi_t] = 0$. Note that in the most general case, $\xi_t$ can depend on $x_t$.
	
\subsec{Warmup: SGD on the 1-dimensional quadratic function}
\tnote{change 1-dimensional to one dimensional globally}
In this section, we consider the 1-dimensional function $g(x) = \frac{1}{2} x^2$. Suppose the noise $\xi_t$ are independent Gaussians, i.e. $\xi_t \sim \mathcal{N}(0,1)$,
\begin{align}
x_{t+1} &= x_t - \eta(\nabla g(x_{t}) + \sigma\xi_t)\\
&= x_t - \eta(x_{t} + \sigma\xi_t)\\
&= \underbrace{(1 - \eta)x_t}_{\text{contraction}} - \underbrace{\eta\sigma\xi_t}_{\text{stochastic}}\label{lec17:eqn:ou}.
\end{align}
$(1 - \eta)x_t$ is called the contraction because $\eta > 0$, which means that this term will shrink after each iteration. The random noise term $\eta\sigma\xi_t$ will accumulate over time, and the scale of $\eta\sigma\xi_t$ remains unchanged. When $x_t$ is large, the contraction term will dominate. When $x_t$ is small, the noise term will dominate. Without the noise term, as $x_t$ continues its contraction, we approach the global minimum $x = 0$. However, with the presence of the noise $\sigma\xi_t$, $x_t$ will not stay at $0$, but instead bounce around it. 

To characterize this intuition more precisely, we have 
\begin{align}
x_{t+1} &= (1 - \eta)x_t - \eta\sigma\xi_t\\
&= (1 - \eta) ((1 - \eta) x_{t - 1}  - \eta \sigma \xi_{t - 1}) - \eta \sigma \xi_t \\
&= (1 - \eta)^2 x_{t - 1} - (1 - \eta) \eta \sigma \xi_{t - 1} - \eta \sigma \xi_{t} \\
&= (1 - \eta)^3 x_{t - 2} - (1 - \eta)^2 \eta \sigma \xi_{t - 2} - (1 - \eta) \eta \sigma \xi_{t - 1} - \eta \sigma \xi_t \\
&\quad \vdots \\
&= (1 - \eta)^{t+1} x_0 - \eta\sigma\sum_{k=0}^{t} \xi_{t-k} (1 - \eta)^{k}. \label{lec17:eqn:warmup_expansion}
\end{align}
The first term in \eqref{lec17:eqn:warmup_expansion} becomes negligible when $\eta t \gg 1$ (since $(1 - \eta)^{t} \approx e^{-\eta t}$). The second term in \eqref{lec17:eqn:warmup_expansion} is the accumulation of noise, which is the sum of Gaussians. Leveraging the properties of Gaussian distributions, we know that its variance equals $\eta^2\sigma^2\sum_{k=0}^{t} (1 - \eta)^{2k}$.

From the analysis above, we know that as $t \rightarrow \infty$, $\Var(x_t) \approx \eta^2\sigma^2\sum_{k=0}^{\infty} (1 - \eta)^{2k} = \frac{\eta^2\sigma^2}{2\eta - \eta^2} = {\Theta}(\eta\sigma^2)$. Therefore, as $t \rightarrow \infty$, $x_t \sim \mathcal{N}(0, {\Theta}(\eta\sigma^2))$.

\paragraph{Interpretation.} In the 1-dimensional case, the noise only makes it harder to converge to the global minimum. Classical convex optimization tells us: (1) noisy GD leads to a less accurate solution and (2) noisy GD is faster than GD. What we do in practice is achieve a balance between (1) and (2). This does \textit{not} lead to implicit regularization since $\Exp[x_t] \rightarrow 0$ as $t \rightarrow \infty$. However, this case is important for further analysis because \eqref{lec17:eqn:ou} corresponds to the Ornstein–Uhlenbeck (OU) process which we use more extensively in the multi-dimensional cases.

\subsec{SGD on multi-dimensional quadratic functions}
Consider a PSD matrix $A \in \R^{d\times d}$. In this section, $g(x) = \frac{1}{2}x^\top A x$. Suppose $\xi_t \sim \mathcal{N}(0, \Sigma)$. For ease of presentation, assume that $A$ and $\Sigma$ are simultaneously diagonizable (they have the same set of eigenvectors). We use $K$ to denote the span of the eigenvectors of $A$/$\Sigma$. Then, consider the following SGD iterate:
\begin{align}
x_{t+1} &= x_t - \eta(\nabla g(x_{t}) + \xi_t)\\
&= x_t - \eta(Ax_t + \xi_t)\\
&= (I- \eta A)x_t - \eta\xi_t\\
&= \underbrace{(I- \eta A)^{t+1} x_0}_{\text{contraction}} - \underbrace{\eta\sum_{k=0}^{t} (I- \eta A)^{k}\xi_{t-k}}_{\text{noise accumulation}}.
\end{align}
Similar to the analysis in the 1-D case above, we have $x_t \sim \mathcal{N}(0, \eta^2\sum_{k=0}^{\infty} (I- \eta A)^{k}\Sigma (I- \eta A)^{k})$ as $t \rightarrow \infty$. \footnote{For random variable $\xi\in \R^d$, $\Exp[(W\xi)(W\xi)^\top] = W\Exp[\xi\xi^\top]W^\top$}

Since $A$ and $\Sigma$ are simultaneously diagonizable, we can easily disentangle the iterates into d separate OU process in the eigencoordinate system. Concretely, by eigendecomposition, suppose that $A = U^\top \text{diag}(d_i) U$ and $\Sigma = U^\top \text{diag}(\sigma_i^2) U$, where $U$ is the orthogonal matrix consisting of the eigenvectors of $A$ and $\Sigma$. We can express the covariance of the stationary distribution as
\begin{align}
\eta^2\sum_{k=0}^{\infty} (I- \eta A)^{k}\Sigma (I- \eta A)^{k} &= \eta^2 U\text{diag}\left(\sum_{k=0}^{\infty}\sigma_i^2(1-\eta d_i)^{2k}\right)U^\top\\
&= \eta U\text{diag}\left(\frac{\sigma_i^2}{d_i}\right)U^\top.
\end{align}
\paragraph{Interpretation.} Intuitively, $\frac{\sigma_i^2}{d_i}$ here is the iterate fluctuation in the direction of the $i$-th eigenvector. This results tell us that the fluctuation of the iterates depends on the shape of $\Sigma$ and $A$. If $\Sigma$ is not full rank, the fluctuations will be limited to the subspace $K$. Also note that $\Exp[\|x_t\|_2] = \Theta(\sqrt{\eta})$. This reflects the noise accumulation since the scale of noise in each step is $\Theta({\eta})$. However, we still do not have any implicit regularization effect. 

% \begin{figure}[ht]
% \includegraphics[width=0.90\textwidth]{figures/noise_regularization_sgd.pdf}
% \centering
% \caption{Comparison of SGD on quadratic functions (Left) and non-quadratic functions (Right). $K$ is the span of the noise covariance $\Sigma$. In the quadratic case, the iterates will fluctuate in $K$, but remains unchanged in $K^\perp$. When the function is non-quadratic, the third order effect slowly accumulates in $K^\perp$, resulting in implicit regularization. } 
% \label{lec17:fig:OU}
% \end{figure}

In the sequel, we separately analyze the second order and third order effects of SGD on a general non-quadratic function. The second order effect exactly corresponds to this section's analysis when $A$ equals the Hessian of the general non-quadratic function.

\tnote{need to add some figures}

\subsec{SGD on non-quadratic functions}
In this section, we analyze SGD on non-quadratic functions based on \cite{damian2021label}. Due to the complexity of the analysis, we provide heuristic derivations to convey the main insights. 

Without loss of generality, suppose a global minimum of $g(x)$ is $x=0$. Therefore, $\nabla_x g(0) = 0$ and $\nabla_x^2 g(0)$ is PSD. We also  assume the iterates $x_t$ are close to $0$, so we can Taylor expand around $0$.
\begin{align}
x_{t+1} &= x_t - \eta(\nabla g(x_t) + \xi_t)\\
&= x_t - \eta(\nabla g(0) + \nabla^2g(0)(x_t - 0) + \nabla^3g(0)[x_t,x_t] + \text{higher order terms} + \xi_t). \label{lec17:eqn:full_gradient_update}
\end{align}

Let $H = \nabla^2_x g(0)$ and $T = \nabla^3_x g(0)$. Since $T$ is a tensor, we first clarify our notation. First, for $T \in \R^{d\times d\times d}$, $x,y \in \R^{d}$, $T[x,y]\in \R^d$, and 
\begin{align}
    T[x,y]_i \defeq \sum_{j,k\in[d]}T_{ijk}x_jy_k.
\end{align} 
For $S\in \R^{d\times d}$, $T(S)\in \R^d$, and 
\begin{align} 
    T(S)_i \defeq \sum_{j,k\in[d]}T_{ijk}S_{jk}
\end{align} 

Now returning to \eqref{lec17:eqn:full_gradient_update}, after dropping the higher order terms, we obtain the following third-order Taylor expansion:
\begin{align}
x_{t+1} &\approx x_t - \eta Hx_t - \eta\xi_t - \eta T[x_t,x_t]\\
&= (I-\eta H)x_t - \eta \xi_t - \eta T [x_t,x_t].\label{lec17:eqn:iterate}
\end{align}

If we don't consider the third order term $\eta T [x_t,x_t]$, the update reduces to the one we studied in the previous subsection. Next, recall that $\|x_t\|_2 \approx \sqrt{\eta}$. Therefore, $\eta T[x_t,x_t] \approx \eta^2$. This quantity is dominated by both $\eta \xi_t$ and $\eta Hx_t \approx {\eta}^{1.5}$. 

So, when $H$ is positive definite, the third order term can be negligible. However, in overparametrized models, $H$ is typically low-dimensional. For instance, if the NTK matrix is full rank, then the manifold of interpolators has dimension $d-n$. Then, in the direction orthogonal to the span of $H$, the contraction term disappears. Letting $\Pi_{A}$ denote projections onto the subspace $A$, we see that $\eta H \Pi_{K^\perp}(x_t) = 0$ and $T[x_t,x_t] \approx \eta^2$ will dominate the update in that direction.

Consider the case in which both $H$ and $\Sigma$ are not full rank. When the loss is quadratic as in the previous section, we know that the iterate $x_t$ bounces in the subspace $K$ and remains stable in the subspace $K^\perp$. What happens when the loss is not quadratic, i.e. $T[x_t,x_t]$ affects the gradient update? 

To answer this question, we decompose the effect of the update in \eqref{lec17:eqn:iterate} between the two subspaces of interest, $K$ and $K^\perp$. First, observe that $(I-\eta H)x_t - \eta \xi_t$ is working in $K$, and $- \eta T [x_t,x_t]$ is only working in $K^\perp$ because in $K$ the effect of $\eta T [x_t,x_t]$ is dominated by $(I-\eta H)x_t - \eta \xi_t$. In previous section, we already well-characterized the effect of optimization without a third order effect. To refine our analysis of the gradient update, we define an iterate $u_{t+1} = (I - \eta H)y_t - \eta \xi_t$ in which we do not have the third order effect.\footnote{Note that $\xi_t$ is the same for each $u_t$ and $x_t$.} Then, to analyze what the implicit regularization effect is, we study $r_t = x_t - u_t$.
\begin{align*}
r_{t + 1} &= x_{t+1} - u_{t+1}\\
&= (I-\eta H)(x_t - u_t) - \eta T[x_t,x_t]\\
&= (I-\eta H)r_t - \eta T[x_t,x_t]\\
&\approx (I-\eta H)r_t - \eta T[u_t,u_t].
\end{align*}
Note that we only have the contraction and the bias terms for the $r_t$ iterate. The stochasticity term $\eta \xi_t$ is canceled out. 

In the subspace $K = \text{span}(H)$, the effect of $\eta T [x_t,x_t]$ is again dominated by $(I-\eta H)x_t - \eta \xi_t$, so no meaningful regularization occurs. But letting $\Pi_{A}$ denote the projection onto the subspace $A$, we have that in $K^\perp$,
\begin{align}
\Pi_{K^\perp}r_{t+1} &= \Pi_{K^\perp}r_t - \eta \Pi_{K^\perp} T[u_t,u_t]\\
&=\Pi_{K^\perp}r_0 - \eta \sum_{k=0}^{t}\Pi_{K^\perp}T[u_k,u_k].
\end{align}
Namely, the effect of $T[u_k,u_k]$ is slowly accumulating in ${K^\perp}$.

Note that the OU process is a Markov chain and a Gaussian process. Here we assume that $H$ is constructed such that $u_t$ converges to its stationary distribution. Suppose the Markov chain $u_t$ mixes as $t\rightarrow \infty$. Then, $\sum_{k=0}^{t}\Pi_{K^\perp}T[u_k,u_k] \approx tT(S)$, where $S:=\Exp[u_{\infty}u_{\infty}^\top]$ is the covariance of the stationary distribution.

\paragraph{Interpretation.} Intuitively, the direction of the implicit regularization is $T(S) = \nabla_x \left(\langle\nabla_x^2g(0), S\rangle\right)$. In other words, the implicit bias $-T(S)$ is trying to make $\langle\nabla^2g(0), S\rangle$ small. \cite{damian2021label} further prove that SGD with label noise on loss $\hat{L}(\theta)$ converges to a stationary point of the regularized loss $\hat{L}(\theta) + \lambda \textup{tr}(\nabla^2\hat{L}(\theta))$.

\paragraph{Relationship to generalization.} Why is $\textup{tr}(\nabla^2\hat{L}(\theta))$ a good regularizer? \cite{wei2019improved} show that the complexity of neural networks can be controlled by its Lipschitzness. $\textup{tr}(\nabla^2\hat{L}(\theta))$ is intimately related to the Lipschitzness of the networks. \cite{foret2020sharpness} also discover empirically that regularizing the sharpness of the local curvature leads to better generalization performance on a wide range of tasks.
	
	\chapter{Unsupervised Learning}
	\metadata{18}{Haoran Xu and Lewis Liu}{Nov 17th, 2021}

We venture into unsupervised learning by first studying classical (and analytically tractable) approaches to unsupervised learning. Classical unsupervised learning usually consists of specifying a latent variable model and fitting using the expectation-maximization (EM) algorithm. However, so far we do not have a comprehensive theoretical analysis for the convergence of EM algorithms because fundamentally analyzing EM algorithms involves understanding non-convex optimization. Most analysis of EM only applies to special cases (e.g., see ~\citet{xu2016global,daskalakis2016ten}) and it is not clear whether any of the results can be extended to more realistic, complex setups, without a fundamentally new technique for understanding nonconvex optimization. 

Instead, we will analyze a family of algorithms which are broadly referred to as spectral methods or tensor methods, which are a particular application of the method of moments~\citep{pearson1894} with the algorithmic technique of tensor decomposition~\citep{anandkumar2015learning}. While the spectral method appears to be not as empirically sample-efficient as EM, it has provable guarantees and arguably is more reliable than EM given the provable guarantees.

After discussing the basics of classical unsupervised learning, we will move on to modern applications of deep learning such as self-training and contrastive learning. Here we note that so-called ``semi-supervised'' and domain adaptation approaches in deep learning can often be reduced to unsupervised learning problems, so in some sense, our analysis here is related to many fields in deep learning that are not always referred to as unsupervised. 

\sec{Method of Moments}

We begin by formally describing the unsupervised learning problem. First, assume that we are studying a family of distributions $P_{\theta}$ parameterized by $\theta \in \Theta$, where $P_{\theta}$ can be described by a latent variable model. Then, given data $x^{(i)},...,x^{(n)}$ that is sampled i.i.d. from some distribution in $\{P_\theta\}_{\theta \in \Theta}$, our goal is to recover the true $\theta$. 

Perhaps the most well-studied latent variable model in machine learning is the mixture of Gaussians. We consider the following model for the mixture of $k$ $d$-dimensional Gaussians. Let 
\begin{align}
\theta = \l ( (\mu_1, \cdots, \mu_k), (p_1, \cdots, p_k)\r ),
\end{align}
where $\mu_i\in \R^d$ is the mean of the $i$-th component and $p$ is a vector of probabilities belonging to the $k$-simplex, which represents the mixture coefficient for clusters. Formally, for $\Delta(k) \defeq \{ p: \|p\|_1 = 1, p\geq 0, p\in\R^k\}$, 
\begin{align}
    p = (p_1, \cdots, p_k) \in \Delta(k).
\end{align}
We then sample $x \sim P_\theta$ in a two-step approach: 
\begin{align}
    i &\sim \text{categorical}(p), \notag\\
    x &\sim \cN(\mu_i, I).
\end{align}
Here $i$ is called the latent variable since we only observe $x$. Here we assume the covariances of the Gaussians to be identity, but they can also be parameters that are to be learned.

There are many other latent variables that could be defined via a similar generative process, such as Hidden Markov Models, Independent Component Analysis, which we will discuss later. %, and Expectation-Maximization, but here we focus on the so-called Moment Method.

\subsec{Warm-up: mixture of two Gaussians}
We first study a simple case: the mixture of two Gaussians.
In this case, $k=2$, and we assume $p_1=p_2=\frac{1}{2}$. For simplicity, we also assume $\mu_1=-\mu_2$, that is, the means of the two Gaussians are symmetric around the origin. To simplify our notation, let $\mu_1=\mu$ and $\mu_2=-\mu$. These assumptions yield the following model for $x$:
\begin{equation}
    x \sim \frac{1}{2}\mathcal{N}(\mu,I) + \frac{1}{2}\mathcal{N}(-\mu,I).
\end{equation}
To implement the moment method, we need to complete the following two tasks:
\begin{enumerate}
    \item Estimate the moment(s) of $x$ using empirical samples.
    \item Recover parameters from the moment(s) of $x$.
\end{enumerate}

The first moment of $x$ is
\begin{align}
    M_1 &\defeq \Exp [x] \\
    &= \frac{1}{2}\Exp [x|i=1]+\frac{1}{2}\Exp[x|i=2] \\
    &= \frac{1}{2}\mu + \frac{1}{2}(-\mu) \\
    &= 0.
\end{align}
Therefore, the first moment provides no information about $\mu$. We compute the second moment as
\begin{align}
    M_2 &\defeq \Exp[xx^\top] \\
    &= \frac{1}{2}\Exp[xx^\top|i=1]+\frac{1}{2}\Exp[xx^\top|i=2]
\end{align}
To compute these expectations, consider an arbitrary $Z \sim \cN(\mu, I)$. Then,
\begin{align}
    \Exp [ZZ^\top] &= \Exp[Z] \Exp[Z]^\top + \Cov(Z) \\
    &= \mu \mu^\top + I
\end{align}
Recognizing that this second moment calculation is the same for both Gaussians in our mixture, we obtain:
\begin{align}
    M_2 &= \frac{1}{2}(\mu\mu^\top+I)+\frac{1}{2}(\mu\mu^\top+I) \\
    &=\mu\mu^\top+I
\end{align}
Since the second moment provides information about $\mu$, we can complete the two tasks required for the moment method using the second moment.

If we had access to infinite data, then we can compute the exact second moment $M_2=\mu\mu^\top+I$. Then, we can recover $\mu$ by evaluating the top eigenvector and eigenvalue of $M_2$.\footnote{This approach is known as the spectral method.} The top eigenvector and eigenvalue of $M_2$ is $\bar{\mu} \defeq \frac{\mu}{\norm{\mu}_2}$ and $\norm{\mu}_2^2+1$, respectively. 

In practice, however, we do not have infinite data. In that case, we need to estimate the second moment by an empirical average.
\begin{align}
    \widehat{M}_2=\frac{1}{n}\sum_{i=1}^nx\sp{i} {x\sp{i}}^\top
\end{align}
We can then recover $\mu$ by evaluating the top egivenvector and eigenvalue of $\widehat{M}_2$. However, we need this algorithm to be robust to errors, i.e., similar estimates, $\widehat{M}_2$, of the second moment should yield similar estimates of $\mu$. Fortunately, most algorithms we might use for obtaining the top eigenvector and eigenvalue are robust, so we can limit our attention to the infinite data case. Having outlined the moment method approach to the mixture of two Gaussians problem, we study a generalization of this problem in the sequel.

\subsec{Mixture of Gaussians with more components}

The general moment method for solving latent variable models is summarized by the following steps.
\begin{enumerate}
    \item Compute $M_1=\Exp[x]$, $M_2=\Exp[xx^\top]$, $M_3=\Exp[x\otimes x\otimes x],$ $M_4 = \cdots$. Note that $x\otimes x\otimes x$ is in $\mathbb{R}^{d\times d\times d}$ and $(x\otimes x\otimes x)_{ijk}=x_i\cdot x_j\cdot x_k$. For example, $M_{3,ijk}=\Exp[x_ix_jx_k]$.
    \item Design as algorithm $A(M_1, M_2, M_3,\dots)$ that outputs $\theta$.
    \item Show that $A$ is robust to errors in our moment estimates, i.e., we apply $A$ to $(\widehat{M}_1,\widehat{M}_2,\widehat{M}_3,...)$ in reality.
\end{enumerate}
In the sequel, we instantiate this paradigm for mixtures of $k$ Gaussians ($k\geq 3$). 

For the simplicity of demonstrating the idea, we assume $p_1 = \cdots = p_k =\frac{1}{k}$, i.e. $i \stackrel{\text{unif}} \sim[k]$, and $x\sim\mathcal{N}(\mu_i,I)$. Equivalently, 
\begin{equation}
    x\sim\frac{1}{k}\sum_{i=1}^k\mathcal{N}(\mu_i,I).
\end{equation}
In this example, we only describe steps (1) and (2) in the general algorithm described above.  

We first evaluate the first and second moments. The first moment follows from
\begin{align}
    M_1 &=\Exp[x] \\
    &=\sum_{i=1}^k\frac{1}{k}\Exp[x|i] \\
    &=\frac{1}{k}\sum_{i=1}^k\mu_i,
\end{align}
and the second moment is computed as
\begin{align}
    M_2 &= \Exp[xx^\top] \\
    &=\sum_{i=1}^k\frac{1}{k}\Exp[xx^\top|i] \\
    &=\sum_{i=1}^k\frac{1}{k}(\mu_i\mu_i^\top+I) \\
    &=\frac{1}{k}\sum_{i=1}^k\mu_i\mu_i^\top + I.
\end{align}

\subsubsection{Second moments do not suffice}
Can we recover $\mu=(\mu_1,...,\mu_k)$ from $\frac{1}{k}\sum_{i=1}^k\mu_i$ and $\frac{1}{k}\sum_{i=1}^k\mu_i\mu_i^\top$? Unfortunately, in most of the cases when $k\geq 3$, the first and second moments are not sufficent to recover $\mu$. 

One reason is the so-called ``missing rotation information'' problem. Let 
\begin{equation}
    U =\begin{bmatrix} \vrule & & \vrule \\ \mu_1 & \cdots & \mu_k \\ \vrule & & \vrule \end{bmatrix} \in\mathbb{R}^{d\times k}
\end{equation}
denote the matrix we aim to recover. Then, consider some rotation matrix $R\in\mathbb{R}^{k\times k}$. We consider $U$ versus $U R$:
\begin{align}
    \frac{1}{k}\sum_{i=1}^k\mu_i\mu_i^\top &= \frac{1}{k}U U^\top \\
    &=\frac{1}{k}(U R)(U R)^\top &\text{($RR^\top=I$)}
\end{align}
This result proves that the second moment is invariant to rotations. To prove a similar claim for the first moment, we also constrain our choice of $R$ such that
\begin{align}
    R\cdot\Vec{1}=\Vec{1}
\end{align}
Then,
\begin{align}
    \frac{1}{k}\sum_{i=1}^k\mu_i&=\frac{1}{k} U \cdot\Vec{1} \\
    &=\frac{1}{k} U R\cdot\Vec{1}
\end{align}
Therefore, the first and second moments of $U$ and $U R$ are indistinguishable, and we must consider the third moment in order to identify $U$.

\subsubsection{Computing the third moment}

The third moment is the tensor $\Exp[x \otimes x \otimes x] \in \mathbb{R}^{d \times d \times d}$. To express this expectation in terms of tractable quantities, we can condition on the Gaussian observed and average:
\begin{align}
	\Exp[x \otimes x \otimes x] = \frac{1}{k} \sum_{i=1}^k \Exp[x \otimes x \otimes x \mid i]
\end{align}

Each term in the sum now corresponds to the third moment for some multivariate Gaussian. Fortunately, Lemma~\ref{lec19:lma:gaussian_third_moment} suggests a formula for estimating its value.
\begin{lemma} \label{lec19:lma:gaussian_third_moment}
Suppose $z \in \cN(v, I)$. Then, 
\begin{align}
	\Exp[z \otimes z \otimes z] = v \otimes v \otimes v +  \sum_{l=1}^d \Exp[z] \otimes e_l \otimes e_l + \sum_{l=1}^d  e_l \otimes \Exp[z] \otimes e_l + \sum_{l=1}^d  e_l \otimes e_l \otimes \Exp[z] 
\end{align}
where $e_1,\dots,e_d$ denote the canonical basis vectors.
\end{lemma}
Note that $\Exp[z] = v$, which means that we can compute $v \otimes v \otimes v$ from a linear combination of $\Exp[z \otimes z \otimes z]$  and $\Exp[z]$.

\begin{proof}
We compute the third moment element-wise. That is,
\begin{align}
	(\Exp[z \otimes z \otimes z])_{ijk} &= \Exp[z_i  z_j z_k] \\
	&= \Exp[(v_i + \xi_i)\cdot (v_j + \xi_j) \cdot (v_k + \xi_k)]  &\text{$(z = v + \xi, \xi \sim \cN(0, I))$}\\
	&= v_i v_j v_k + \underbrace{\Exp [v_i v_j \xi_k] +  \Exp [v_i \xi_j v_k] +  \Exp [\xi_i v_j v_k]}_{=0} \nonumber \\ 
	&\quad + \Exp[v_i \xi_j \xi_k] + \Exp[v_j \xi_i \xi_k] + \Exp[v_k \xi_i \xi_j] + \Exp[\xi_i \xi_j \xi_k] \label{lec19:eqn:higher_moments} 
\end{align}
To explicitly compute the last four terms in \eqref{lec19:eqn:higher_moments}, we note that:
\begin{align}
    \Exp[\xi_i \xi_k] = \begin{cases} 0 &\text{if $i \neq k$} \\ \Exp[\xi_i^2] = 1 &\text{if $i = k$} \end{cases}
\end{align} 
and that for any choice of $i, j,$ and $k$,
\begin{align}
    \Exp[\xi_i \xi_j \xi_k] = 0.
\end{align}
Therefore, 
\begin{equation}
	(\Exp[z \otimes z \otimes z])_{ijk} = v_i v_j v_k + v_i \ind{j=k} +  v_j \ind{i=k}  +  v_k \ind{i=j} 
\end{equation}

Since $(\sum_{l=1}^d v \otimes e_l \otimes e_l)_{ijk} = \sum_{l=1}^d v_i  e_{lj}  e_{lk} = v_i \bbI(j=k)$, we have proven that:
\begin{equation}
	\Exp[z \otimes z \otimes z] = v \otimes v \otimes v +  \sum_{l=1}^d v \otimes e_l \otimes e_l + \sum_{l=1}^d  e_l \otimes v \otimes e_l + \sum_{l=1}^d  e_l \otimes e_l \otimes v.
\end{equation} 
\end{proof}

We can now apply Lemma~\ref{lec19:lma:gaussian_third_moment} to compute the third moment of the mixture of $k$ Gaussians. In particular,
\begin{align}
	\Exp[x \otimes x \otimes x] & =  \frac{1}{k} \sum_{i=1}^k \Exp[x \otimes x \otimes x \mid i] \\
	&=  \frac{1}{k} \sum_{i=1}^k \l (\mu_i \otimes \mu_i \otimes \mu_i +  \sum_{l=1}^d \mu_i \otimes e_l \otimes e_l + \sum_{l=1}^d  e_l \otimes \mu_i \otimes e_l + \sum_{l=1}^d  e_l \otimes e_l \otimes \mu_i \r ) \\
	&=  \frac{1}{k} \sum_{i=1}^k \mu_i \otimes \mu_i \otimes \mu_i +  \sum_{l=1}^d \frac{1}{k} \l (\sum_{i=1}^k \mu_i \r ) \otimes e_l \otimes e_l + \sum_{l=1}^d  e_l \otimes \frac{1}{k} \l (\sum_{i=1}^k \mu_i \r ) \otimes e_l \nonumber \\
    &\quad + \sum_{l=1}^d  e_l \otimes e_l \otimes \frac{1}{k} \l (\sum_{i=1}^k \mu_i \r) \\
	& =  \frac{1}{k} \sum_{i=1}^k \mu_i \otimes \mu_i \otimes \mu_i +  \sum_{l=1}^d \Exp[x] \otimes e_l \otimes e_l + \sum_{l=1}^d  e_l \otimes \Exp[x] \otimes e_l + \sum_{l=1}^d  e_l \otimes e_l \otimes \Exp[x]\\
\end{align} 

For notational convenience, let
\begin{equation}
    a^{\otimes 3} \defeq a \otimes a \otimes a.
\end{equation} 
So far, we have shown how to compute $\frac{1}{k} \sum_{i=1}^k \mu_i^{\otimes 3}$ from $\Exp[x^{\otimes 3}]$ and $\Exp[x]$. In the sequel, we show how to estimate the latent variable by describing several approaches one might take to recover $\{\mu_i\}_{i = 1}^k$ from $\frac{1}{k} \sum_{i = 1}^k \mu_i^{\otimes 3}$. 

\paragraph{Tensor decomposition}
Recovering the Gaussian means, $\{\mu_i\}_{i = 1}^k$,  from the third mixture moment, $\frac{1}{k} \sum_{i = 1}^k \mu_i^{\otimes 3}$, is a special case of the general tensor decomposition problem. That problem is set up as follows: Assume that $a_1, \cdots a_k \in \bbR^d$ are unknown. Then, given $\sum_{i=1}^k a_i^{\otimes 3}$, our goal is to reconstruct the $a_i$ vectors. 

Before we present some standard results on tensor decomposition, we first describe some basic facts about tensors. Much like matrices, tensors have an associated rank. For example, $a \otimes b \otimes c$ is a rank-1 tensor. In general, the rank of a tensor $T$ is the minimum $k$ such that $T$ can be decomposed as 
\begin{equation}
    T = \sum_{i=1}^k a_i \otimes b_i \otimes c_i.
\end{equation} 
for some $\{a_i\}_{i =1}^k, \{b_i\}_{i =1}^k, \{c_i\}_{i =1}^k$.
Another difference between tensors and matrices is that the former objects do not have the typical rotational invariance. In particular, there is no tensor analogue to the standard matrix result that 
\begin{equation}
    AA^\top = A RR^\top A^\top.
\end{equation}
However, tensors do maintain an interesting (and useful) permutation invariance; that is, $T = \sum_{i = 1}^k a_i^{\otimes 3}$ is invariant to permutations of the indices of $a_i$.

We now summarize some standard results regarding tensor decomposition for $T = \sum_{i = 1}^k a_i^{\times 3}$. We will not prove these results here.
\begin{enumerate}
\setcounter{enumi}{-1}
\item  In the most general case, recovering the $a_i$'s from $T$ is computationally hard. Any procedure will either fail to find a unique $a_i$ or it fails to find $a_i$ \emph{efficiently}. 
\item In the orthogonal case, i.e. $a_1,\dots,a_k$ are orthogonal vectors, each $a_i$ is a global maximizer of 
\begin{equation}
    \max_{\|x\|_2 = 1} T(x,x,x) = \sum_{i,j,k} T_{ijk} x_i x_j x_k
\end{equation}
We can heuristically think of $a_i$ as eigenvectors of $T$ and there exists an algorithm to compute $a_i$ in poly-time.
\item In the independent case, i.e. $a_1,\dots,a_k$ are linearly independent. Jennrich's algorithm can be used to efficiently recover $\{a_i\}_{i = 1}^k$.
\end{enumerate} 
Results 1 and 2 above both involve the so-called ``under-complete'' case ($k \leq d$), e.g., when the number of Gaussians in the mixture is smaller than the dimension of the data. Next, we describe certain overcomplete cases for which efficient tensor decomposition is possible.
\begin{enumerate}
\setcounter{enumi}{2}
\item Suppose $a_1^{\otimes2},\dots,a_k^{\otimes2}$ are independent for $k \leq d^2$. Then, applying Result 2, we can recover $a_i$ from $\sum_{i=1}^k (a_i^{\otimes2})^{\otimes3} = \sum_{i=1}^k (a_i^{\otimes6}) \in \bbR^{d^6}$.
\item Excluding an algebraic set of measure $0$, we can use the FOOBI algorithm to recover $a_i$ from the fourth-order tensor $\sum_{i = 1}^k a_i^{\otimes 4}$ when $k \leq d^2$. A robust version of the FOOBI algorithm is described in \citet{ma2016poly}.
\item Assume $a_i$ are \emph{randomly} generated unit vectors. Then, for the third order tensor, $k$ can be large as $d^{1.5}$ \cite{ma2016poly, schrammsteurer17}. 
\end{enumerate}

In summary, the moment method is a recipe in which we first compute high order moments (i.e. tensors), assume that these estimates are noiseless, and decompose these tensors to recover the latent variables. Though we do not discuss these results here, there is an extensive literature analyzing the robustness of the moment method to error in the moment estimates. Last, we note that even though we only explicitly analyze the mixture of Gaussians model here, latent variable models amenable to analysis by the moment method include ICA, Hidden Markov Models, topic models, etc.

\sec{Spectral Clustering}
Introduced by \citet{shi2000normalized} and \citet{ng2002spectral}, spectral clustering learns a model for the data points using a \emph{pairwise} notion of similarity. Formally, assume that we are given $n$ data points $x\sp{1}, \dots, x\sp{n}$ as well as a similarity matrix $G \in \bbR^{n\times n}$ such that 
\begin{equation}
    G_{ij} = \rho (x\sp{i}, x\sp{j})
\end{equation}
where $\rho$ is some measure of similarity that assigns larger values to more similar pairs of points. 

For example, $x\sp{i}$ could denote images for which $\rho (x\sp{i}, x\sp{j})$ measures the semantic similarity. Alternatively, $x\sp{i}$ might be users of a social network and $\rho (x\sp{i}, x\sp{j}) = 1$ if $x\sp{i}$ and $x\sp{j}$ are friends. 

Our goal is to cluster the data points by viewing $G$ as a graph. For instance, in the social network example, we can naturally view $G$ as an \emph{unweighted} graph, where $G_{ij} \in \{0, 1\}$ defines the edges. Then, the clustering problem is to partition the graph into distinct neighborhoods. As we will see repeatedly in the sequel, the eigendecomposition of $G$ is closely related to this graph paritioning problem.
\subsec{Stochastic block model} 
In the stochastic block model (SBM), $G$ is assumed to be generated randomly from two hidden communities. Formally, 
\begin{equation}
    \{ 1, \cdots n \} = S \cup \bar{S},
\end{equation}
where $S$ and $\bar{S}$ partition $[n]$. Assume $|S| = \frac{n}{2}$. We then assume the following generative model for $G$. 
If $i,j \in S$ or $i,j \in \bar{S}$, then 
\begin{align}
    G_{ij} = \begin{cases}
        1 &\text{w.p. $p$} \\
        0 &\text{w.p. $1-p$} \end{cases}.
\end{align} 
Otherwise, for $i$ and $j$ in distinct components, 
\begin{align}
    G_{ij} = \begin{cases}
        1 &\text{w.p. $q$} \\
        0 &\text{w.p. $1-q$} \end{cases}
\end{align} 
for $p < q$.

Our goal is then to recover $S$ and $\bar{S}$ from $G$; the primary tool we use towards this goal is the eigendecomposition of $G$.

In some trivial cases, it is not necessary to eigendecompose $G$ to recover the two hidden communities. Suppose, for instance, that $p = 0.5$ and $q = 0$. Then, the graph represented by $G$ will contain two connected components that correspond to $S$ and $\bar{S}$.

As a warm-up to motivate our approach, we eigendecompose $\bar{G} = \Exp[G]$. Observe that
\begin{align}
    \bar{G}_{ij} = \begin{cases}
        p &\text{if $i,j$ from the same community} \\
        q &\text{o.w.} \end{cases}.
\end{align}
It is then easy to see that $\bar{G}$ is a matrix of rank $2$:
\begin{align}
    \bar{G} = \left[
        \begin{array}{c|c}
        p \cdots p & q \cdots q \\
        \vdots & \vdots \\
        p \cdots p & q \cdots q\\
        \hline
        q \cdots q & p \cdots p \\
        \vdots & \vdots \\
        q \cdots q & p \cdots p 
        \end{array}
        \right].
\end{align}
\begin{lemma} \label{lec19:lma:sbm_eigen}
Let $\bar{G} = \Exp[G]$ for the stochastic block model. Then, letting $v_i(A)$ denote the $i$-th eigenvector of the matrix $A$,
\begin{align}
    v_1(\bar{G}) &= \vec{1} \label{lec19:eqn:top_eig_G}\\
    v_2(\bar{G}) &= [\underbrace{1, \dots, 1}_{|S|}, \underbrace{-1, \dots, -1}_{|\bar{S}|}]^\top \label{lec19:eqn:second_eig_G}
\end{align}
where $v_2(\bar{G})$ has $|S|$ entries of $1$ and $|\bar{S}|$ entries of $-1$.
\end{lemma}

\begin{proof}
We begin by directly proving \eqref{lec19:eqn:top_eig_G}.
\begin{align}	
	\bar{G} \cdot \vec{1}  &= \begin{bmatrix}
           \frac{pn}{2} + \frac{qn}{2} \\
           \vdots \\
           \frac{pn}{2} + \frac{qn}{2}
         \end{bmatrix} \\
         &= \frac{p+q}{2} \cdot n \cdot \vec{1}.
\end{align}
More generally, $\vec{1}$ is the top eigenvector for any matrix with fixed row sum or any graph with uniform degree. 

Next, we prove \eqref{lec19:eqn:second_eig_G}. Let 
\begin{align}
    G' = \left[
        \begin{array}{c|c}
        r \cdots r \\
        \vdots & \makebox{\text{\huge 0}} \\
        r \cdots r \\
        \hline
        & r \cdots r \\
        \makebox{\text{\huge 0}} & \vdots \\
        & r \cdots r
        \end{array}
        \right]
\end{align}
for $r = p - q$. To precisely define $G'$, we note that $G'$ is block diagonal with two blocks of size $|S|$ and $|\bar{S}|$, respectively. Then, 
\begin{align}	
	\bar{G} &= \vec{1} \vec{1}^\top q + G'. \label{lec19:eqn:barG}
\end{align}
Thus,
\begin{align}
 G' \cdot u &= \left[
    \begin{array}{c|c}
    r \cdots r \\
    \vdots & \makebox{\text{\huge 0}} \\
    r \cdots r \\
    \hline
    & r \cdots r \\
    \makebox{\text{\huge 0}} & \vdots \\
    & r \cdots r
    \end{array}
    \right] \cdot \begin{bmatrix}
           1 \\ \vdots \\ 1\\ -1 \\
           \vdots \\
           -1
         \end{bmatrix} = r \cdot \frac{n}{2} \cdot u. \label{lec19:eqn:gprimeu}
\end{align}
Then, because $u \perp \vec{1}$, we can combine \eqref{lec19:eqn:barG} and \eqref{lec19:eqn:gprimeu} to obtain
\begin{align}
\bar{G} \cdot u =  G' \cdot u =  r \cdot \frac{n}{2} \cdot u  = \frac{p-q}{2} \cdot n \cdot u
\end{align}
Thus, $u$ has eigenvalue $\frac{p-q}{2} \cdot n$. 
\end{proof}

\begin{remark}
    Lemma~\ref{lec19:lma:sbm_eigen} shows that
    \begin{equation}
        \bar{G} = \frac{p + q}{2} \cdot \vec{1} \vec{1}^\top + \frac{p - q}{2} \cdot u u^\top.
    \end{equation}
\end{remark}
More generally, when we have more than two clusters in the graph, $G'$ is block diagonal with more than two blocks. In this setting, the eigenvectors will still align with the blocks. We illustrate this below for a generic block diagonal matrix. Let 
\begin{align}
    A = \left[
        \begin{array}{c|c|c}
        1 \cdots 1 &&\\
        \vdots & \makebox{\text{\huge 0}} & \makebox{\text{\huge 0}} \\
        1 \cdots 1 &&\\
        \hline
        & 1 \cdots 1 \\
        \makebox{\text{\huge 0}} & \vdots & \makebox{\text{\huge 0}} \\
        & 1 \cdots 1 \\
        \hline
        & & 1 \cdots 1 \\
        \makebox{\text{\huge 0}}& \makebox{\text{\huge 0}} & \vdots \\
        & & 1 \cdots 1
        \end{array}
        \right]
\end{align}

Then, the three eigenvectors of $A$ are 
\begin{align}
    \begin{bmatrix}
        1 \\ \vdots \\ 1\\ 0 \\
        \vdots \\
        0 \\ 0 \\ \vdots \\ 0
      \end{bmatrix}, \begin{bmatrix}
        0 \\ \vdots \\ 0\\ 1 \\
        \vdots \\
        1 \\ 0 \\ \vdots \\ 0
      \end{bmatrix}, \begin{bmatrix}
        0 \\ \vdots \\ 0\\ 0 \\
        \vdots \\
        0 \\ 1 \\ \vdots \\ 1
      \end{bmatrix} \label{lec19:eqn:eigenmatrix}
\end{align}
Furthermore, the rows of the matrix formed by the three eigenvectors given by \eqref{lec19:eqn:eigenmatrix} clearly give the cluster IDs of the vertices in $G$. Note also that permutations of $A$ will result in equivalent permutations in the coordinates of each of the three eigenvectors.

Next, we relate this observation to the result in Lemma~\ref{lec19:lma:sbm_eigen}. While there are no negative values in the eigenvectors given in \eqref{lec19:eqn:eigenmatrix}, we observe that any linear combination of eigenvectors is also an eigenvector, so recovering a solution that look more like \eqref{lec19:eqn:second_eig_G} is straightforward. Indeed, taking linear combinations of the eigenvectors defined above shows that there is an alternative eigenbasis that includes the all-ones vector, $\vec{1}$. How for this choice of $A$, the all-ones vector is not the unique top eigenvector. For that to be the case, we require background noise in $\bar{G}$.

In reality, we only observe $G$. In the sequel, we will show that in terms of the spectrum, $G \approx \Exp[G]$. Formally, we will leverage earlier concentration results to prove that $\norm{G - \Exp[G]}_{\text{op}}$ is small. Concretely, then,
\begin{align}
    G &= (G - \Exp[G]) + \Exp[G] \\
    &= (G - \Exp[G]) + \frac{p + q}{2} \cdot \vec{1} \vec{1}^\top + \frac{p - q}{2} \cdot u u^\top
\end{align}
Rearranging, we obtain that:
\begin{align}
    G - \frac{p + q}{2} \cdot \vec{1} \vec{1}^\top &= (G - \Exp[G]) + \frac{p - q}{2} \cdot u u^\top
\end{align}
We then hope that $G - \Exp[G]$ is a small perturbation, so that the top eigenvector of $G - \frac{p + q}{2} \cdot \vec{1} \vec{1}^\top$ is close to $u$. Namely, it suffices to show that 
\begin{equation}
    \norm{G - \Exp[G]}_{\text{op}} \ll \l \|\frac{p - q}{2} \cdot uu^\top\r \|_{\text{op}}.
\end{equation}
%	\input{collection/09-01-data-dependent.tex}
	
	\chapter{Online learning}\label{chap:OL}
	% reset section counter
\setcounter{section}{0}

\metadata{15}{Tianyu Du, Xin Lu and Soham Sinha}{Mar 8th, 2021}

In this chapter, we switch gears and talk about \textit{online learning} and \textit{online convex optimization}. The main idea driving online learning is that we move away from the assumption that the training and test data are both drawn i.i.d from some fixed distribution. In the online setting, training data and test data come to the user in an interwoven manner, and data can be generated \textit{adversarially}. We will describe how online learning can be reduced to online convex optimization, some important algorithms, as well as applications of these algorithms to some illustrative examples.

\sec{Online learning setup}

In classical supervised learning, we train the model with the assumption that $(x^{(i)}, y^{(i)}) \overset{i.i.d.}{\sim} P_{\text{train}}$, where $P_{\text{train}}$ is the underlying distribution of the training data. In most cases, we assume the test data, i.e., the data we want our model to predict well, comes from the same distribution (or at least one that is close to $P_{\text{train}}$). Reality is often more complicated: data could indeed be generated in sequence, or even in an adversarial manner, so it is often the case that $P_\text{test}$ differs from $P_\text{train}$. The situation where $P_\text{test}$ and $P_{\text{train}}$ are different is known as \textit{domain shift}. There are some theories that tackle the issue of domain shift and generalization properties of transfer learning. However, the field is still largely being developed. (See \cite{ben2007analysis}, for example.)

Online learning is an attempt to deal with domain shift in a way that is agnostic to the relationship between the training and test data distributions (i.e. deal with ``worst-case'' domain shift). As an example, many recommendation systems today collect users' historical trace of shopping behavior, which are not i.i.d. samples, and makes adaptive recommendations based on users' changing shopping behavior. Hence, one can see that online learning attempts to adapt to the constantly evolving reality on time. Notice that unlike the ``offline model'' (i.e., classical supervised learning), online learning learns while testing, and hence there is no rigid division in time to differentiate training and testing phase.

Online learning has several distinctive features \cite{percynotes}:
\begin{enumerate}
\item The data may be \textit{adversarial}. We cannot assume that sample is drawn independently from some distribution.    
\item The data and predictions are \textit{sequential}. At each step, the algorithm makes a prediction after given a single piece of data.
\item The feedback is \textit{limited}. For example, in bandit problems, the algorithm only knows if its right or wrong, but no other feedback is given. 
\end{enumerate}

Online learning can be viewed as a game between two parties: (i) the learner/agent/algorithm/player, and (ii) the environment/nature. For simplicity, we will refer to the two parties as ``learner'' and ``environment'' in the remainder of this chapter.

The game takes place over $T$ rounds or time steps. At each step $t = 1, \dots, T$, the learner receives an input $x_t \in \cX$ from the environment and makes a prediction $\yhat \in \cY$ in response. The learner then receives the label $y_t$ from the environment and suffers some loss. This procedure is outlined in Algorithm \ref{lec15:alg:gen-ol} and is illustrated in Figure \ref{lec15:fig:OLgame}.

    \begin{algorithm}[h]
        \caption{General online learning problem}
        \label{lec15:alg:gen-ol}
        \For {$t = 1, ... T$}{
            Learner receives $x_t \in \mathcal{X}$ from environment, which may be chosen adversarially\;
            Learner predicts $\yhat \in \mathcal{Y}$\;
            Learner receives the label $y_t$, from environment, which may be chosen adversarially;
            Learner suffers some loss $\ell(y_t, \yhat_t)$.
        }
    \end{algorithm}

\begin{figure}[ht]
    \centering
    \includegraphics[width=2in]{figures/OLupdated.png}
    \caption{A representation of the online learning problem.}
    \label{lec15:fig:OLgame}
\end{figure}

Later, we will see that the manner in which nature generates  $(x_t, y_t)$ leads to different types of online learning. In the most adversarial setting of online learning, it is possible that the ``true label'' $y_t$ is not generated at the same time as $x_t$. The environment could generate the label $y_t$ depending on the prediction $\hat{y}_t$ made by the learner.  We can also see that Algorithm \ref{lec15:alg:gen-ol} is a very general framework as there are very few constraints on how $x_t$ and $y_t$ are generated.
    
\subsec{Evaluation of the learner}
Given this setup, a natural question to ask is how one can evaluate the performance of the learner. Intuitively, one could simply evaluate the learner's performance by computing the loss between the predicted label and the ``true'' label sent by the environment $\ell(y_t, \hat{y}_t)$. For the entire sequence of tasks, one can then evaluate in terms of the cumulative loss:
    \begin{align}
        \sum_{t=1}^T \ell(y_t, \yhat_t).
    \end{align}
    
However, as the environment can be adversarial, the task itself might be inherently hard and even the best possible learner fails to achieve a small loss. Hence, instead instead of using the cumulative loss for a learner by itself, we compare its performance against a suitable baseline, the ``best model in hindsight''. Assume that our learner comes from a set of hypotheses $\mathcal{H}$. Let us choose the hypothesis $h \in \mathcal{H}$ that minimizes the cumulative loss, i.e.
\begin{equation}
    h^\star = \argmin_{h \in \mathcal{H}} \sum_{t=1}^T \ell(y_t, h(x_t)).
\end{equation}

Note here that in minimizing the cumulative loss, the learner gets to see all the data points $(x_t, y_t)$ at once. The cumulative loss of $h^\star$ is the best we can ever hope to do, and so it would be better to compare the cumulative loss of the learner against it. (This approach is analogous to ``excess risk'', which tells how far the current model is away from the best we could hope for.) This measurement is denoted as \emph{regret}, and is formally defined as:
    \begin{align}
        \text{Regret} \overset{\Delta}{=} 
        \left[\sum_{t=1}^T \ell(y_t, \yhat_t)\right]
        - \underbrace{
        \left[\min_{h \in \mathcal{H}} \sum_{t=1}^T \ell(y_t, h(x_t))\right]
        }_{\text{best loss in hindsight}}
    \end{align}

Using this definition, if the best model in hindsight performs well, then the learner has more responsibility to learn to predict well in order to match up the performance of the baseline.
    
\subsec{The realizable case}
In general, if the environment is too powerful, leading the learner to a large loss, it will also hinder the best model in hindsight from doing well. On the other hand, there are settings where some members of the hypothesis class can actually do well. Such settings/problems are usually referred to as \textit{realizable}:

\begin{definition}[Realizable problem]
An online learning problem is \textit{realizable} (for a family of predictors $\mathcal{H}$) if there exists $h \in \mathcal{H}$ such that for any $T$, $\sum_{t = 1}^T \ell(y_t, h(x_t)) = 0$.
\end{definition}

Note that even though zero error is possible, this is still an interesting problem to consider because the $x_t$'s are not i.i.d. as they are in classical supervised learning. Hence, standard statistical learning theory does not apply, and there is still research to be done here.

\begin{example}
Consider a classification problem on $(x_t, y_t)$, and for simplicity assume $y_t \in \{0, 1\}$. Suppose there exists $h^\star \in \mathcal{H}$ such that we always have $y_t = \yhat^\star_t = h^\star(x_t)$. In this case, the problem is realizable. 
    
In this case, the learner can adopt a ``majority algorithm''. At each time, the learner maintains a set $V_t \subset \mathcal{H}$ so that $\sum_{t=1}^T \ell (y_t, h(x_t)) = 0$ for all $h \in V_t$, and $\hat{y}_t$ is simply the prediction made by the majority of $h \in V_t$. Based on the loss received, learners $h \in V_t$ that fail for time $t + 1$ will be eliminated from future $V_t$'s.
    
With this setup, we can see that for each wrong prediction made by the learner, at least half of the hypotheses $h \in V_t$ will be eliminated. Hence, $1 \leq |V_{t+1}| \leq |\mathcal{H}|2^{-M}$ where $M$ is the number of mistakes made so far. Thus, one has $M \leq \log |\mathcal{H}|$ by taking log on both sides of inequalities and rearrange.
    
Now, if one puts $\ell$ as the zero-one loss, the regret for this example will be
\begin{equation}
\text{Regret} = \sum_{t=1}^T \ell(y_t, h(x_t)) = M,
\end{equation}
so in this example, one has $\text{regret} \leq \log |\mathcal{H}|$, which is a non-trivial bound when $\mathcal{H}$ is finite.
\end{example}
    
As one can see in the example, the realizable case usually indicates that the problem is not too far out of reach. Indeed, for finite hypothesis classes and linear models, the realizable case is considered to be straightforward to solve. This is perhaps why most of the past literature has focused on non-realizable cases. However, the realizable case is still an interesting problem and perhaps a very good starting point when the model class is beyond linear models and when the loss function is no longer convex, because the $x_t$'s are not i.i.d. as they are in classical supervised learning. Hence, standard statistical learning theory does not apply, and there is still research to be done here.
 
In the rest of the chapter, we will only focus on the convex loss case, where we reduce online learning to online convex optimization. 
    
\sec{Online (convex) optimization (OCO)}

\textit{Online convex optimization (OCO)} is a particularly useful tool to get results for online learning. Many online learning problems (and many other types of problems!) can be reduced to OCO problems, which allow them to be solved and analyzed algorithmically. Algorithm \ref{lec15:alg:oco} describes the OCO problem, which is more general than the online learning problem. (Note: \textit{Online optimization (OO)} refers to Algorithm \ref{lec15:alg:oco} except that the $f_t$'s need not be convex. However, due to the difficulty in non-convex function optimization, most research has focused on OCO.)

    \begin{algorithm}
    \caption{Online (convex) optimization problem}
    \label{lec15:alg:oco}
    \For{$t = 1, ..., T$} {
        The learner picks some action $w_t \in \Omega$ from the action space $\Omega$\;
        The environment picks a (convex) function $f_t: \Omega \to [0, 1]$\;
        The learner suffers the loss $f_t(w_t)$ and observes the \emph{entire} loss function $f_t(\cdot)$.
        }
    \end{algorithm}
    
Essentially the learner is trying to minimize the function $f_t$ at each step. As with online learning, one evaluates the performance of learner in online optimization setting using the regret:
\begin{align}
\text{Regret} = \sum_{t=1}^T f_t(w_t) - 
\underbrace{\min_{w \in \Omega} \sum_{t=1}^T f_t(w)}_\text{best action in hindsight}.
\end{align}

At some level, OCO seems like an impossible task, since we are trying to minimize a function $f_t$ that we only get to see \textit{after} we have made our prediction! This is certainly the case for $t = 1$. However, as time goes on, we see more and more functions and, if future functions are somewhat related to past functions, we have more information to make better predictions. (And if the future functions are completely unrelated or contradictory to past functions, then the best action in hindsight would also be bad and therefore our algorithm does not have to do much.)

\subsec{Settings and variants of OCO}
There are multiple settings of the OCO network, which can vary the power of the environment and observations.

\begin{itemize}
    \item \underline{Stochastic setting:} $f_1,...,f_T$ are i.i.d samples from some distribution $P$. This corresponds to $(x_t, y_t)$ being i.i.d. in online learning. Under this setting, the environment is not adversarial.
    \item \underline{Oblivious setting:} $f_1,...,f_T$ are chosen arbitrarily but before the game starts. This corresponds to $(x_t, y_t$ being chosen before the game starts. In this setting, the environment can be adversarial but cannot be adaptive. The environment can choose these functions based on the learner's algorithm, but not the actual action if the learner's algorithm contains randomness. (This is the setting that we focus on in this course.)
    \item \underline{Non-oblivious/adaptive setting:} For all $t$, $f_t$ can depend on the learner's actions $w_1,...w_t$. Under this setting, the environment can be adversarial and adaptive. This is the most challenging setting because the environment is powerful enough to know not only the strategy of the learner, but also the exact choice the learner finally made. (Note however that If the learner is deterministic, the environment does not have more power here than in the oblivious setting. The oblivious adversary can simulate the game before the game starts, and chose the most adversarial input accordingly.)
\end{itemize}
 
\sec{Reducing online learning to online optimization}
There is a natural way to reduce the online learning problem to online optimization, with respect to a specific type of model $h_{w}$ parametrized by $w \in \Omega$. Recall that in online learning problem, the learner predicts $y_t$ upon receiving $x_t$. If the learner possesses oracle to solve online optimization problem, the learner can consult the oracle to obtain $w_t$, the parameter of the model as in online optimization problem, and then predict $\hat{y}_t = h_{w_t}(x_t)$.

In the next two subsections, we give two examples of how an online learning problem can be reduced to an OCO problem.
    
\subsec{Example: Online learning regression problem}

Consider the regression model $h_w(x) = w^\top x$ parameterized by $w$ in parameter space $\Omega$ with squared error loss $\ell$. Here is the online learning formulation of the regression problem:

\begin{algorithm}
\caption{Online learning regression problem}
\For{$t = 1, ..., T$} {
The learner receives $x_t \in \R^d$ from the environment\;
The learner predicts $\yhat_t$\;
The environment selects $y_t$ and sends it to the learner\;
The learner suffers loss $\ell(y_t, \yhat_t) = (y_t-\yhat_t)^2$.
}
\end{algorithm}

This can be reduced to the OCO problem in the following way:

\begin{algorithm}
\caption{OCO formulation of regression problem}
\For{$t = 1, ..., T$} {
The learner receives $x_t \in \R^d$ from the environment\;
The learner gives $x_t$ to the OCO solver and obtains $w_t \in \R^d$\;
The learner predicts $\hat{y}_t = h_{w_t}(x_t) = w_t^\top x_t$\;
The environment selects $y_t$ and sends it to the learner\;
The learner suffers loss $(y_t - h_{w_t}(x_t))^2$\;
With $(x_t, y_t)$ observed, the learner can reconstruct the loss function $f_t(w) = (y_t -h_{w}(x_t))^2$ and give it to the OCO solver.
}
\end{algorithm}

In this example, we have the following correspondence:
\begin{itemize}
\item $f_t$ in online optimization $\leftrightarrow$ squared error loss functions for $(x_t, y_t)$.
\item $w_t$ in online optimization $\leftrightarrow$ parameters of the linear model $h_{w_t}$.
\end{itemize}
    
Since $h_w(\cdot)$ is linear, the corresponding squared error loss function $f_t$ are convex, and so we have effectively reduced the online linear regression problem to an online \emph{convex} optimization problem.
    
Notice that in the previous example, the loss function $f_t$ actually depends on the label $y_t$, which demonstrates that the key challenge in online optimization is that the function $f_t$ is unknown to the learner when the prediction $\hat{y}_t$ is made.
    
\subsec{Example: The expert problem}
Suppose we wish to predict tomorrow's weather and 10 different TV channels provide different forecasts. Which one should we follow? Formally, consider a finite hypothesis class $\mathcal{H}$, where each $h \in \mathcal{H}$ represents an expert, and we wish to choose a $h_t$ wisely at each time step. For simplicity, we assume the prediction is binary, i.e. $\hat{y} \in \{0, 1\}$, and suppose the loss function is 0-1 loss. (The problem can easily be generalized to more general predictions and losses.) The problem is outlined in Algorithm \ref{lec15:alg:expert_discrete}.

\begin{algorithm}[h]
\caption{The expert problem}
\label{lec15:alg:expert_discrete}
\For{$t = 1, ..., T$}{
The learner obtains predictions from $N$ experts\;
The learner chooses to follow prediction of one of the experts $i_t \in [N]$\;
The environment gives the learner the true value. The learner is thus able to learn the loss of each of the experts: $\ell_t \in \{0, 1\}^N$\;
The learner suffers the loss of the expert which was chosen: $\ell_t(i_t)$.
}
\end{algorithm}

We want to design a method that chooses $i_t$ for each step (line 3 in Algorithm \ref{lec15:alg:expert_discrete}) to minimize the regret:
\begin{equation}
\text{Regret} \overset{\Delta}{=} \mathbb{E}\left[
\sum_{t=1}^T \ell_t(i_t)
- \underbrace{\min_{i \in [N]} \sum_{t=1}^T \ell_t(i)}_\text{the best expert in hindsight}
\right],
\end{equation}
where the expected value is over $i_t$, thus covering the case where the $i_t$'s could be random.
    
To make the expert problem amenable to reduction to OCO, we introduce idea of a \textit{continuous action space}. Instead of choosing $i_t$ from $\Omega = [N]$, the learner chooses a distribution $p_t$ from the $N$-dimensional simplex $\Delta(N) = \left\{p \in \R^N : \norm{p}_1 = 1, p \geq 0 \right\}$. The learner then samples $i_t \sim p_t$. With this formulation, instead of selecting particular expert $i_t$ to follow, the learner adjusts the belief $p_t$, and samples from the distribution to choose which expert to follow. Algorithm \ref{lec15:alg:expert_randomized} outlines this procedure. Note that the loss is the expected loss $\mathbb{E}_{i \sim p_t}[\ell_t(i)]$ instead of the sampled $\ell_t(i_t)$.

\begin{algorithm}
\caption{The expert problem with continuous action}
\label{lec15:alg:expert_randomized}
\For{$t = 1, ..., T$}{
The learner obtains predictions from $N$ experts\;
The learner chooses a distribution $p_t \in \Delta(N)$\;
The learner samples one expert $i_t \sim p_t$\;
The environment gives the learner the true value and the loss/error of all experts: $\ell_t \in \{0, 1\}^N$\;
The learner suffers expected loss $\sum_{i\in[N]} p_t(i) \ell_t(i) = \langle p_t, \ell_t \rangle$\;
}
\end{algorithm}
    
With the continuous action space, it is easy to reduce the expert problem to an OCO: see Algorithm \ref{lec15:alg:expert_discrete_oco}. (The problem is convex since the loss function is convex and the parameter space $\Delta(N)$ is convex.)

\begin{algorithm}[h]
\caption{The expert problem}
\label{lec15:alg:expert_discrete_oco}
\For{$t = 1, ..., T$}{
The learner obtains predictions from $N$ experts\;
The learner invokes the OCO oracle to obtain $p_t \in \Delta(N)$\;
The learner chooses to follow prediction of one of the experts $i_t \in [N]$\;
The environment gives the learner the true value. The learner is thus able to learn the loss of each of the experts: $\ell_t \in \{0, 1\}^N$\;
The learner suffers the loss of the expert which was chosen: $\ell_t(i_t)$.
The learner can reconstruct the loss function $f_t (p) = \langle p, \ell_t \rangle$ and give it to the OCO oracle.
}
\end{algorithm}

In this setting, one can rewrite the regret as:
\begin{align}
\text{Regret} &= \sum_{t=1}^T \langle p_t, \ell_t \rangle - \min_{i\in[N]}\sum_{t=1}^T \ell_t(i)  \\
&= \sum_{t=1}^T \langle p_t, \ell_t \rangle - \min_{p \in \Delta(N)}\sum_{t=1}^T \langle p, \ell_t \rangle \label{lec15:eqn:changearg} \\
&= \sum_{t=1}^T f_t(p_t) - \min_{p \in \Delta(N)}\sum_{t=1}^T f_t(p). \label{lec15:eqn:regret}
\end{align}

We obtain \eqref{lec15:eqn:changearg} because
\begin{align}
\sum_{t=1}^T \langle p, \ell_t \rangle &=  \left\langle p,  \sum_{t=1}^T\ell_t \right\rangle \geq \min_{i \in [N]} \left[ \sum_{t=1}^T \ell_t (i) \right],
\end{align}
with equality for the probability distribution $p(i) =1$ when $i = \text{argmin}_i \left[ \sum_{t=1}^T \ell_t (i) \right]$ and $p(i) = 0$ otherwise, and \eqref{lec15:eqn:regret} is by definition of $f_t$.


\sec{Reducing online learning to batch learning}    
In this section, we present a reduction from online learning to standard supervised learning problem, also known as the ``batch problem'' in this literature.

As in the standard supervised learning setting, consider an i.i.d dataset $\{(x_t, y_t)\}_{t=1}^T$ and some parameter $w$. Let $L(w)$ and $\hatL(w)$ be the population loss and empirical loss respectively. For simplicity, assume $|\ell((x_i, y_i), w)| \leq 1$. The theorem below establishes a link between the regret obtained in online learning and the excess risk obtained in the batch setting.
    
\begin{theorem}[Relationship between excess risk and regret]
Assume $\ell((x, y), w)$ is convex. Suppose we run an online learning algorithm on the dataset $\{(x_i, y_i)\}_{i=1}^T$ and obtain a sequence of models $w_1, \dots, w_T$, and regret $R_T$. Let $\overline{w} = \frac{1}{T} \sum_{i=1}^T w_i$, then the excess risk of $\overline{w}$ can be bounded above:
\begin{align}
L(\overline{w}) - L(w^\star) \leq \frac{R_T}{T} + \tilO\left(\frac{1}{\sqrt{T}}\right), \label{lec15:eqn:lec15_ol_gen_bound}
\end{align}
where $w^\star = \argmin_{w \in \Omega} L(w)$.
\end{theorem}

Here are some intuitive interpretations of the theorem:

    \begin{itemize}
        \item If $R_T = O(T)$, then we have some non-trivial result. Otherwise, the bound in \eqref{lec15:eqn:lec15_ol_gen_bound} is increasing $T$ and does not provide any useful information.
        \item If the batch problem has a $1 / \sqrt{T}$ generalization bound, then the best you can hope for in online learning is $R_T = O(\sqrt{T})$.
        \item If the batch problem has a $1 / T$ generalization bound, you can hope for $O(1)$ regret (or $\tilO(1)$ regret in some cases).
        \item We often have $O(\sqrt{T})$ excess risk supervised learning problems; hence it is reasonable to expect $O(\sqrt{T})$ regret in online learning problems.
    \end{itemize}
    
\sec{Follow-the-Leader (FTL) algorithm} \label{lec15:sec:FTL}
In this section, we analyze an algorithm called ``Follow-the-Leader'' (FTL) for OCO, which is intuitive but fails to perform well in many cases.

The FTL algorithm behaves as its name suggests: it always selects the action $w_t$ such that it minimizes the historical loss the learner has seen so far, i.e.
\begin{equation}
w_t = \argmin_{w \in \Omega} \sum_{i=1}^{t-1} f_i(w).
\end{equation}

We now demonstrate how the FTL algorithm can fail for the expert problem. In the expert problem, $f_t(p) = \langle p, \ell_t \rangle$, so 
    \begin{align}
        p_t &= \argmin_{p \in \Delta(N)} \sum_{i=1}^{t-1} f_i(p) \\
        &= \argmin_{p \in \Delta(N)} \sum_{i=1}^{t-1} \langle\ell_i, p\rangle \\
        &= \argmin_{p \in \Delta(N)} \left\langle\sum_{i=1}^{t-1}\ell_i, p\right\rangle.
    \end{align}

The minimizer $p \in \Delta(N)$ is a point-mass probability, with the point mass at the smallest coordinate of $\sum_{i=1}^{t-1} \ell_i$. This gives regret
\begin{equation}
\text{Regret} = \sum_{i=1}^{t-1} \ell_i(i_t),
\quad \text{ where } i_t = \argmin_{j \in [N]} \sum_{i=1}^{t-1}\ell_i(j).
\end{equation}
    
Now, consider the following example: suppose we have only two experts. Suppose expert 1 makes perfect predictions on even days while expert 2 makes perfect predictions on odd days. Assume also that the FTL algorithm chooses expert 1 to break ties (this is not an important point but makes the exposition simpler.) In this setting, the FTL algorithm always selects the \textit{wrong} expert to follow. A few rounds of simulation of this example is shown in Table \ref{lec15:tab:counter example}.

    \begin{table}[h]
        \caption{An example where FTL fails}
        \label{lec15:tab:counter example}
        \medbreak
        \centering
        \small
        \begin{tabular}{l|c c c c c c}
        \toprule
        Day & 1 & 2 & 3 & 4 & $\dots$ & $\dots$ \\
        \midrule 
        Expert 1's loss & 1 & 0 & 1 & 0 & $\dots$ & $\dots$ \\
        Expert 2's loss & 0 & 1 & 0 & 1 & $\dots$ & $\dots$ \\
        \midrule 
        \midrule 
        FTL choice $i_t$ & 1 & 2 & 1 & 2 & 1 & $\dots$ \\
        \bottomrule
        \end{tabular}
    \end{table}

The best expert in hindsight has a loss of $T/2$ (choosing either expert all the time incurs this loss, and so the regret of the FTL algorithm is $T - T/2 = T/2 = \Theta(T)$. The main reason for FTL's failure is that is a deterministic algorithm driven by an extreme update, with no consideration on potential domain shift (it always selects the best expert based on the past with no consideration of the potential next $f_t$). Knowing its deterministic strategy, the environment can easily play in an adversarial manner. To perform better in a problem like this, we need some randomness to hedge risk.
	% reset section counter
%\setcounter{section}{0}

%\metadata{lecture ID}{Your names}{date}
\metadata{16}{Kevin Guo}{Mar 10th, 2021}

% ===============================================
\sec{Be-the-leader (BTL) algorithm}

A better strategy is called \textit{``Be the Leader'' (BTL)}.  At time $t$, the BTL strategy chooses the action that would have performed best on $f_1, \cdots, f_{t-1}$ \textit{and} $f_t$.  In other words, the BTL action at time $t$ is $w_{t+1}$, as defined for the FTL algorithm. Note that this is an ``illegal'' choice for the action because $w_{t+1}$ depends on $f_t$: in online convex optimization, the action at time $t$ is required to be chosen \textit{before} seeing the function $f_t$.  Nevertheless, we can still gain some useful insights by analyzing this procedure. In particular, the following lemma shows that the BTL strategy is worth emulating because it achieves very good regret.

\begin{lemma}\label{lec16:lem:btl_regret}
The BTL strategy has non-positive regret. That, is, if $w_t$ is defined as in the FTL algorithm, then
\begin{align}
\text{BTL regret} = \sum_{t = 1}^T f_t(w_{t + 1}) - \min_{w \in \Omega} \sum_{t = 1}^T f_t(w) \leq 0, \label{lec16:eqn:btl_regret}
\end{align}
for any $T$ and any sequence of functions $f_1, \cdots, f_T$.
\end{lemma}

\begin{proof}
We prove the lemma by induction on $T$. \eqref{lec16:eqn:btl_regret} holds trivially for $T = 1$. Suppose that \eqref{lec16:eqn:btl_regret} holds for all $t \leq T - 1$ and any $f_1, \cdots, f_{T-1}$.  Now we wish to extend \eqref{lec16:eqn:btl_regret} to time $t = T$.  Let $f_T$ be any function.  Since $w_{T+1} = \argmin_w \sum_{t = 1}^T f_t(w)$, we can write:
\begin{align}
\sum_{t = 1}^{T} f_t(w_{t+1}) - \min_{w \in \Omega} \sum_{t = 1}^{T} f_t(w) &= \sum_{t = 1}^T f_t(w_{t+1}) - \sum_{t = 1}^T f_t(w_{T+1})\\
&= \sum_{t = 1}^{T - 1} f_t(w_{t+1}) - \sum_{t = 1}^{T - 1} f_t(w_{T+1}) &\text{(final summands cancel)}\\
&\leq \sum_{t = 1}^{T - 1} f_t(w_{t+1}) - \min_{w \in \Omega} \sum_{t = 1}^{T - 1} f_t(w)\\
&\leq 0. &\text{(induction hypothesis)}
\end{align}
\end{proof}

A useful consequence of this lemma is a regret bound for the FTL strategy.

\begin{lemma}
\label{lec16:lem:ftl_regret}
\textup{(FTL regret bound)} Again, let $w_t$ be as in the FTL algorithm. The FTL strategy has the regret guarantee
\begin{align}
\text{FTL regret} = \sum_{t = 1}^T f_t(w_t) - \min_{w \in \Omega} \sum_{t = 1}^T f_t(w) \leq \sum_{t = 1}^T [f_t(w_t) - f_t(w_{t+1})].
\end{align}
\end{lemma}

\begin{proof}
\begin{align}
\text{FTL regret} &= \sum_{t = 1}^T f_t(w_t) - \min_{w \in \Omega} \sum_{t = 1}^T f_t(w) \\
&= \sum_{t = 1}^T f_t(w_{t+1}) - \min_{w \in \Omega} \sum_{t = 1}^T f_t(w) + \sum_{t = 1}^T [f_t(w_t) - f_t(w_{t+1})] \\
&\leq 0 + \sum_{t = 1}^T [f_t(w_t) - f_t(w_{t+1})],
\end{align}
where the last inequality is due to \eqref{lec16:eqn:btl_regret}.

\end{proof}

Lemma \ref{lec16:lem:ftl_regret} tells us that if terms $f_t(w_t) - f_t(w_{t+1})$ are small (e.g. $w_t$ does not change much from round to round), then the FTL strategy can have small regret. It suggests that the player should adopt a \textit{stable} policy, i.e. one where the terms $f_t(w_t) - f_t(w_{t+1})$ are small.  It turns out that following this intuition will lead to a strategy that improves the regret all the way to $O(\sqrt{T})$ in certain cases.

% ===============================================
\sec{Follow-the-regularized-leader (FTRL) strategy}

Now, we discuss a OCO strategy aims to improve the stability of FTL by controlling the differences $f_t(w_t) - f_t(w_{t+1})$. To describe the method, we will first need a preliminary definition.

\begin{definition}
We say that a differentiable function $\phi : \Omega \mapsto \R$ is \textit{$\alpha$-strongly-convex} with respect to the norm $|| \cdot ||$ on $\Omega$ if we have 
\begin{equation}\label{lec16:eqn:strongly-convex}
\phi(x) \geq \phi(y) + \langle \nabla f(y), x - y \rangle + \frac{\alpha}{2} \norm{x - y}^2
\end{equation}
for any $x, y \in \Omega$.
\end{definition}

\begin{remark}
If $\phi$ is convex, then we know that $f(x)$ has a linear lower bound $\phi(y) + \langle \nabla f(y), x - y \rangle$. Being $\alpha$-strong-convex means that $f(x)$ has a quadratic lower bound, the RHS of \eqref{lec16:eqn:strongly-convex}. This quadratic lower bound is very useful in proving theorems in optimization.
\end{remark}

\begin{remark}
If $\nabla^2 f(y) \succeq \alpha I$ for all $y$, then $f$ is $\alpha$-strongly-convex. This follows directly from writing the second-order Taylor expansion of $f$ around $y$.
\end{remark}

Given a $1$-strongly-convex function $\phi(\cdot)$, which we call a \textit{regularizer}, we can implement the \textit{``Follow the Regularized Leader'' (FTRL)} strategy.  At time $t$, this strategy chooses the action
\begin{align}
w_t = \argmin_{w \in \Omega} \left[ \sum_{i = 1}^{t -1} f_i(w) + \frac{1}{\eta} \phi(w) \right], \label{lec16:eqn:ftrl}
\end{align}
where $\eta > 0$ is a tuning parameter that we will tune later.

\subsec{Regularization and stability}

To understand why we might use the FTRL policy, we first establish that it achieves the intended goal of controlling the differences $f_t(w_t) - f_t(w_{t+1})$. Actually, we will show a more general result that adding a regularizer induces stability for any convex objective.

\begin{lemma}
\label{lec16:lem:regularizers_stability}
\textup{(Regularizers induce stability)} Let $F$ and $f$ be functions taking $\Omega$ into $\R$, and assume that $F$ is $\alpha$-strongly-convex with respect to the norm $\norm{\cdot}$ and that $f$ is convex.  Let $w = \argmin_{z \in \Omega} F(z)$ and $w' = \argmin_{z \in \Omega} [f(z) + F(z)]$.  Then
\begin{equation}\label{lec16:eqn:regularizers_stability}
0 \leq f(w) - f(w') \leq \frac{1}{\alpha} \norm{\nabla f(w)}_*^2,
\end{equation}
where $\norm{\cdot}_*$ is the dual norm of $\norm{\cdot}$.
\end{lemma}

\begin{proof}
By strong convexity,
\begin{align}
F(w') - F(w) &\geq \langle \nabla F(w), w' - w \rangle + \frac{\alpha}{2} \norm{w - w'}^2 \\
&\geq \frac{\alpha}{2} \norm{w - w'}^2,
\end{align}
where in the second step we used the fact that the KKT optimality conditions for $w$ imply $\langle \nabla F(w), w' - w \rangle \geq 0$. (Informally, if $\Omega = \R^d$, then $\nabla F(w) = 0$ as $w$ minimizes $F$. If $\Omega$ is a convex subset of $\R^d$, then the gradient $\nabla F(w)$ must be perpendicular to the tangent to $\Omega$ at $w$; otherwise, we could move in the direction of the negative gradient and project back to the set $\Omega$ to lower the value of $F$.) Since $F + f$ is also $\alpha$-strongly convex, exactly the same argument implies:
\begin{align}
[F(w) + f(w)] - [F(w') + f(w')] \geq \frac{\alpha}{2} \norm{w - w'}^2.
\end{align}
Adding these two inequalities gives
\begin{align}
f(w) - f(w') \geq \alpha \norm{w - w'}^2. \label{lec16:eqn:lower_bound}
\end{align}
Since this lower bound is clearly positive, this shows $0 \leq f(w) - f(w')$.

Next, we prove the upper bound on $f(w) - f(w')$. Rearranging the inequality \eqref{lec16:eqn:lower_bound}, we obtain
\begin{align}
\norm{w - w'} \leq \sqrt{\frac{1}{\alpha} [f(w) - f(w')]}. \label{lec16:eqn:upper_bound}
\end{align}
Since $f$ is convex, we have $f(w') \geq f(w) + \langle \nabla f(w), w' - w \rangle$.  Rearranging this gives
\begin{align*}
f(w) - f(w') &\leq \langle \nabla f(w), w - w' \rangle\\
&\leq \norm{\nabla f(w)}_* \cdot \norm{w - w'} &\text{(by Cauchy-Schwarz)} \\
&\leq \norm{\nabla f(w)}_* \sqrt{ \frac{1}{\alpha} [f(w) - f(w')]}. &\text{(by \eqref{lec16:eqn:upper_bound})}
\end{align*}
Since $f(w) - f(w') \geq 0$, we can square both sides of this inequality to conclude that
\begin{equation}
[f(w) - f(w')]^2 \leq || \nabla f(w) ||_*^2 \frac{1}{\alpha} [f(w) - f(w')].
\end{equation}
Dividing both sides of this expression by $f(w) - f(w')$ gives the desired upper bound.
\end{proof}

\begin{remark}
Consider the special case where $\nabla f(w) = 0$. In this situation, $w$ is the minimizer of both $F$ and $f$, and hence is the minimizer of $F + f$. This implies that $w = w'$, and the inequalities in \eqref{lec16:eqn:regularizers_stability} become equalities.
\end{remark}

\subsec{Regret of FTRL}
We are now ready to prove a regret bound for the FTRL procedure, based on the idea that strongly convex regularizers induce stability.

\begin{theorem}\label{lec16:thm:ftrl_regret}
\textup{(Regret of FTRL)} Let $\phi$ be a 1-strongly-convex regularizer with respect to the norm $\norm{\cdot}$ on $\Omega$.  Then the FTRL algorithm (\ref{lec16:eqn:ftrl}) satisfies the regret guarantee
\begin{align}
\text{FTRL regret} = \sum_{t = 1}^T f_t(w_t) - \argmin_{w \in \Omega} \sum_{t = 1}^T f_t(w)  \leq \frac{D}{\eta} + \eta \sum_{t = 1}^T \norm{\nabla f_t(w_t)}_*^2,
\end{align}
where $D = \max_{w \in \Omega} \phi(w) - \min_{w \in \Omega} \phi(w)$.
\end{theorem}

\begin{remark}
Suppose that for all $t$ and $w$, we have the uniform bound $|| \nabla f_t(w) ||_* \leq G$.  Then Theorem \ref{lec16:thm:ftrl_regret} implies that the regret is upper bounded by $D / \eta + \eta G T$.  Optimizing this upper bound over $\eta$ by taking $\eta = \sqrt{\dfrac{D}{TG^2}}$ gives the guarantee
\begin{equation}\label{lec17:eqn:ftrl-regret-ub}
\text{FTRL regret} \leq 2 \sqrt{D G} \times \sqrt{T}.
\end{equation}
In other words, optimally-tuned FTRL can achieve $O(\sqrt{T})$ regret in many cases.
\end{remark}

\begin{proof}
For convenience, define $f_0(w) = \phi(w) / \eta$.  Then the FTRL policy can be written as
\begin{equation}
w_t = \argmin_{w \in \Omega} \sum_{i = 0}^{t - 1} f_i(w),
\end{equation}
i.e. FTRL is just FTL with an additional ``round'' of play at time zero. Thus, by Lemma \ref{lec16:lem:ftl_regret} with time starting from $t = 0$, we have
\begin{align}
\sum_{t = 0}^T f_t(w_t) - \argmin_{w \in \Omega} \sum_{t = 0}^T f_t(w) &\leq \sum_{t = 0}^T [f_t(w_t) - f_t(w_{t+1})].
\end{align}
For any $t \geq 1$, applying Lemma \ref{lec16:lem:regularizers_stability} with $F(w) = \sum_{i = 0}^{t-1} f_i(w)$ (which is $1/\eta$-strongly-convex) and $f(w) = f_t(w)$ gives the bound $f_t(w_t) - f_t(w_{t+1}) \leq \eta || \nabla f_t(w_t) ||_*^2$.  Plugging this into the preceding display gives the upper bound:
\begin{align}
\sum_{t = 0}^T f_t(w_t) - \argmin_{w \in \Omega} \sum_{t = 0}^T f_t(w) &\leq f_0(w_0) - f_0(w_1) + \eta \sum_{t = 1}^T \norm{\nabla f_t(w_t)}_*^2. \label{lec16:eqn:ftrl_ub}
\end{align}

Next, we need to relate the LHS of the above display (which starts at time $t = 0$) to the actual regret of FTRL (which starts at time $t = 1$). To do this, define $w^* = \argmin_{w \in \Omega} \sum_{t = 1}^T f_t(w)$. Then,
\begin{align}
\sum_{t = 0}^T f_t(w_t) - \argmin_{w \in \Omega} \sum_{t = 0}^T f_t(w) &\geq \sum_{t = 0}^T f_t(w_t) - \sum_{t = 0}^T f_t(w^*)\\
&= f_0(w_0) - f_0(w^*) + \underbrace{\left( \sum_{t = 1}^T f_t(w_t) - \argmin_{w \in \Omega} \sum_{t = 1}^T f_t(w)  \right)}_{\text{Regret of FTRL}}.
\end{align}
Combining this inequality with (\ref{lec16:eqn:ftrl_ub}) gives
\begin{align}
\text{Regret of FTRL} &\leq f_0(w_0) - f_0(w_1) + f_0(w^*) - f_0(w_0) + \eta \sum_{t = 1}^T \norm{\nabla f_t(w_t)}_*^2\\
&= \frac{\phi(w^*) - \phi(w_1)}{\eta} + \eta \sum_{t = 1}^T \norm{\nabla f_t(w_t)}_*^2\\
&\leq \frac{D}{\eta} + \eta \sum_{t = 1}^T \norm{\nabla f_t(w_t)}_*^2.
\end{align}
This concludes the proof of the theorem.
\end{proof}

\subsec{Applying FTRL to online linear regression}

We apply the FTRL algorithm to a concrete machine learning problem. Let $\Omega = \{ \omega \, : \, \norm{w}_2 \leq 1 \}$, and let $f_t(\omega) = \tfrac{1}{2}(y_t - \omega^{\top} x_t)^2$ for some observation pair $(x_t, y_t)$ satisfying $\norm{x_t}_2 \leq 1$ and $|y_t| \leq 1$.  This corresponds to a problem where we are trying to make accurate predictions using a linear model, but we do not assume any structure on the observation sequence $(x_t, y_t)$ beyond boundedness.

Consider using FTRL in this problem with a ridge regularizer, $\phi(\omega) = \tfrac{1}{2} \norm{w}_2^2$.  One can check that $\phi$ is 1-strongly-convex with respect to the $\ell_2$-norm, and also that $D = \max_{\omega \in \Omega} \phi(\omega) - \min_{\omega \in \Omega} \phi(\omega) = \tfrac{1}{2}$.  Moreover, for all $t$ and $w$ we have 
\begin{align}
\nabla f_t(w) &= - (y_t - w^\top x_t) x_t, \\
\norm{\nabla f_t(w)}_2 &\leq |y_t - w^\top x_t| \cdot \norm{x_t}_2 \\
&\leq 2 \cdot 1 = 2.
\end{align}
Therefore, by choosing $\eta = \sqrt{1/(8T)}$ and applying the FTRL regret theorem (Theorem \ref{lec16:thm:ftrl_regret}), we can obtain the regret guarantee
\begin{align}
\sum_{t = 1}^T (y_t - w_t^{\top} x_t)^2 - \min_{|| w ||_2 \leq 1} \sum_{t = 1}^T  (y_t - w^{\top} x_t)^2 \leq 4 \sqrt{T}.
\end{align}

\subsec{Applying FTRL to the expert problem}

For the expert problem, recall that the action space is $\Delta (N)$ and $f_t = \langle \ell_t , p \rangle$, where $\ell_t \in [0,1]^N$. As a first attempt at applying FTRL to this problem, we set $\phi (p) = \frac{1}{2}\norm{p}_2^2$. With this choice,
\begin{align}
D &= \max_{p \in \Delta(N)} \phi (p) - \min_{p \in \Delta(N)} \phi (p) \\
&\leq \max_{p \in \Delta(N)} \frac{1}{2}\norm{p}_2^2 \\
&\leq \max_{p \in \Delta(N)} \frac{1}{2}\norm{p}_1^2 \\
&= \frac{1}{2}.
\end{align}

Also,
\begin{align}
\norm{\nabla f_t}_2 &= \norm{\ell_t}_2 \leq \sqrt{N}.
\end{align}

Thus, the regret bound is $O(G\sqrt{DT}) = O(\sqrt{NT})$. This is optimal dependency on $T$, but not good dependency on $N$.

Next, we show that if we change our regularization, we can get a better regret guarantee which is logarithmic in $N$, i.e., the regret is $O(\sqrt{(log N) \cdot T})$. The new regularizer we choose is the \textit{(negative) entropy regularizer}:
\begin{equation}
\phi(p) = -H(p) = \sum_{j=1}^N p(j)\log p(j),
\end{equation}
where $p \in \Delta(N)$ is in the set of distributions over $[N]$. We first introduce the following nice property of this regularizer:
\begin{lemma}
	$\phi(p)$ defined above is 1-strongly convex with respective to the $\ell_1$ norm $\|\cdot\|_1$. 
\end{lemma}

\begin{proof}
By definition of strong convexity, we need to show that for all $p, q \in \Delta(N)$,
\begin{equation}\label{lec17:eqn:entropy-sc}
\phi(p) - \phi(q) - \langle \nabla \phi(q), p-q\rangle \geq \frac{1}{2} \|p-q\|_1^2.
\end{equation}
	
From direct computation, we know the gradient of $\phi(q)$ is 
\begin{equation}
\nabla\phi(q) = \begin{bmatrix} 1+\log q(1)\\\cdots \\ 1+\log q(N) \end{bmatrix}.
\end{equation}
	
Plugging this into the LHS of \eqref{lec17:eqn:entropy-sc}, we get
\begin{align}
&\phi(p) - \phi(q) - \langle \nabla \phi(q), p-q\rangle  \\
=& \sum_{j=1}^N p(j)\log p(j) - \sum_{j=1}^N q(j)\log q(j) - \sum_{j=1}^N \left(1 + \log q(j)\right)\left(p(j) - q(j)\right) \\
=& \sum_{j=1}^N p(j)\log p(j) - \sum_{j=1}^N p(j)\log q(j) - \sum_{j=1}^N \left(p(j) - q(j)\right)\\
=& \sum_{j=1}^N p(j) \log \frac{p(j)}{q(j)} \label{lec17:eqn:entropy-sc-proof} \\
=& KL(p||q),
\end{align}
where $KL(p || q)$ is the KL-divergence between $p$ and $q$. (We used the fact that $\sum_{j=1}^N p(j) = \sum_{j=1}^N q(j) = 1$ to get \eqref{lec17:eqn:entropy-sc-proof}.) Finally, we finish the proof by applying Pinsker's inequality: $KL(p||q) \geq \frac{1}{2} \norm{p-q}_1^2$. 
	
\end{proof}

Hence, $\phi$ is a satisfies the condition on the regularizer for our FTRL regret guarantee. To obtain the regret bound \eqref{lec17:eqn:ftrl-regret-ub}, we also need to bound $D = \sup \phi(p) - \inf \phi(p)$ and $G = \sup \|\nabla f_t(w)\|_\infty$ (since $\|\cdot\|_\infty$ is the dual norm of $\|\cdot \|_1$ ). Since negative entropy is always non-positive and (positive) entropy is always bounded above by $\log N$, we bound $D$ with
\begin{equation} 
D = \sup \phi(p) - \inf \phi(p) \leq -\inf \phi (p) = -\inf (-H(p)) = \sup (H(p)) \leq \log N,
\end{equation}
and we bound $G$ with
\begin{equation}
G = \|\nabla f_t(w)\|_\infty = \|l_t\|_\infty \leq 1.
\end{equation}

Plugging these two into the regret bound \eqref{lec17:eqn:ftrl-regret-ub} we get bound $O(\sqrt{(\log N) \cdot T})$. 

Thus far, we have looked at FTRL and the expert problem abstractly: at each time $t$ we choose action $p_t$ based on the update
\begin{equation}
p_t = \argmin_{p \in \Delta(N)} \sum_{i=1}^{t+1} f_t(p) - \frac{1}{\eta} H(p).
\end{equation}

\textbf{Can we get an exact analytical solution for $p_t$?} Since we are minimizing a convex function, we can call some off-the-shelf convex optimization algorithm to solve this at each step. Another way is to write down the KKT conditions and solve that set of equations.  Instead, we will show that there exists much simpler ways to solve this update. In particular, we will be using the \textit{Gibbs variational principle}, which is essentially the KKT conditions under the hood.

\begin{lemma}[Gibbs variational principle] \label{lec17:lem:gibbs}
Let $\nu, \mu$ be probability distributions on $[N]$. Then 
\al{\sup_\nu \left(\Exp_\nu[f] - KL(\nu||\mu)\right) = \log \Exp_\mu \left[e^f\right],} where $\Exp_\nu[f] = \Exp_{x \sim \nu} [f(x)] = \langle v, f\rangle$ and $\Exp_\mu \left[e^f\right] = \Exp_{x \sim \mu}  \left[e^{f(x)}\right]$. Moreover, the optimal solution is attained at \al{\nu(x) \propto \mu(x) \cdot e^{f(x)}.}
\end{lemma}

Intuitively, Lemma~\ref{lec17:lem:gibbs} says that taking the supremum over distributions $\mu$ of a linear function plus the KL divergence as the regularizer will give us the same distribution as exponentiating $f$. 

If we take $\mu$ to be the uniform distribution on $[N]$ and replace $f$ with $-f$ in Lemma~\ref{lec17:lem:gibbs}, we get the following corollary:

\begin{corollary}\label{lec17:cor:gibbs}
	Let $\nu$ be a probability distribution. Then, 
	$\Exp_\nu[f] - H(\nu)$ is minimized at $\nu(x) \propto e^{-f(x)}$.
\end{corollary} 

\begin{proof}
When $\mu$ is uniform distribution, we have
\begin{align}
KL(\nu||\mu) &= \sum_x \nu(x) \log \frac{\nu(x)}{\mu(x)} \\
&= \log N - \sum_x \nu(x) \log \frac{1}{\nu(x)} \\
&= \log N - H(\nu).
\end{align}

So $\sup_\nu \left(\Exp_\nu[-f] - KL(\nu||\mu)\right) = -\inf_\nu \left(\Exp_\nu[f] - H(\nu) + \log N\right)$. This means that the value of $\nu$ that attains the infimum of $\Exp_\nu[f] - H(\nu)$ is the same $\nu$ attaining the supremum of $\Exp_\nu[-f] - KL(\nu||\mu)$, which by Lemma~\ref{lec17:lem:gibbs} is proportional to $e^{-f(x)}$.
\end{proof}

We now apply the Gibbs variational principle to the expert problem. Notice that our FTRL update for the expert problem at time $t$ can be written as
\begin{equation}
\argmin_{p_t \in \Delta(N)} \l\langle \sum_{i=1}^{t-1}l_i, p_t \r\rangle - \frac{1}{\eta}H(p_t) = \argmin_{p_t \in \Delta(N)} \l\langle\eta \sum_{i=1}^{t-1}l_i, p_t \r\rangle - H(p_t),
\end{equation}
where $l_i$ is the vector of expert losses at time $i$. Letting $f = \eta \sum_{i=1}^{t-1} l_i$, we know from Corollary~\ref{lec17:cor:gibbs} that the minimizer is attained at $p_t \propto \exp \l(-\eta \sum_{i=1}^{t-1}l_i \r)$, or equivalently,  
\begin{equation}
p_t(j) = \frac{\exp(-\eta L_t(j))}{\sum_{k=1}^N \exp(-\eta L_t(k))},
\end{equation}
where $L_t = \sum_{i=1}^{t-1}l_i$ is the cumulative loss vector. Basically, solving the expert problem is to look a the historical loss of each expert and take softmax to find the probability distribution of how much to trust each expert. 

This algorithm is also called the ``Multiplicative Weights Update'', which has been studied before online learning framework became popular~\cite{arora2005fast, freund1997decision, littlestone1994weighted}. One way of doing multiplicative weights update is the following: Let $\tilde{p}_t$ be the unnormalized distribution that we keep track of. At each time step $t$, for each expert $j$, we look at $l_{t-1}(j)$. if $l_{t-1}(j)=1$, i.e. the expert made a mistake at the previous time step, we update $\tilde{p}_t(j) = \tilde{p}_{t-1}(j) \cdot \exp(-\eta)$; otherwise we make no change. We then get a distribution by normalizing $\tilde{p}_t$:
\begin{equation}
p_t = \frac{\tilde{p}_t}{\|\tilde{p}_t\|_1}.
\end{equation}

\sec{Convex to linear reduction}

In the previous section we considered the expert problem, where the loss function is a \textit{linear} function of the parameters. At first glance we may think this is a very restrictive constraint for online convex optimization. However, as we will see in this section, we can always assume $f_t$ to be linear in online convex optimization without loss of generality. That means that for online learning, the linear case is the hardest one. 

More concretely, assume we have an algorithm $\cA$ that solves the linear case. Given any online convex optimization, we will build an algorithm $\cA'$ which invokes algorithm $\cA$ in the following fashion: for $t = 1, \dots, T$,
\begin{enumerate}
	\item The learner invoke $\cA$ to get output action $w_t \in \Omega$. 
	\item The environment gives the learner the loss function $f_t(\cdot)$. 
	\item The learner construct a linear function $g_t(w) = \langle\nabla f_t(w_t), w \rangle$, which is the local linear approximation of $f$ at $w$. (Technically the local linear approximation of $f$ and $w$ is $\langle \nabla f_t(w_t), w - w_t\rangle$, but we drop the $w_t$ shift for convenience.)
	\item The learner feeds $g_t(\cdot)$ to algorithm $\cA$ as the loss function. 
\end{enumerate}

We have the following informal claim\footnote{For rigorous proof, we need additional assumptions and restrictions on $f_t, g_t$.}:
\begin{proposition}[Informal]
	If a deterministic algorithm $\cA$ has regret no more than $\gamma (T)$ for linear cases for some function $\gamma (\cdot)$, then $\cA'$ stated above has regret no more than $\gamma(T)$ for convex functions. 
\end{proposition}

\begin{proof}
	For all $w \in \Omega$, the regret guarantee on $\cA$ tells us that
	\begin{align}
		\sum_{t=1}^T g_t(w_t) - \sum_{t=1}^T g_t(w) \leq \gamma(T).
	\end{align}
	Since $f_t$ is convex, we also know that
	\begin{align}
		g_t(w_t) - g_t(w) = \langle \nabla f_t(w_t), w_t- w \rangle \geq f_t(w_t) - f_t(w).
	\end{align}
	Therefore, for all $w \in \Omega$,
	\begin{align}
		\sum_{t=1}^T f_t(w_t) - \sum_{t=1}^Tf_t(w) &\leq \sum_{t=1}^T g_t(w_t) - \sum_{t=1}^T g_t(w) \\
		&\leq \gamma(T).
	\end{align}
	
	Hence, the regret for $\cA'$ is upper bounded by $\gamma(T)$ as well.
\end{proof}

\subsec{Online gradient descent}
In this section we combine the FTRL framework with $\ell_2$-regularization and the online-to-linear reduction. The resulting algorithm is \textit{online gradient descent}.

Concretely, given any convex online optimization problem, we first do the online-to-linear reduction, then we use FTRL with $\ell_2$ regularization ($\phi (w) = \frac{1}{2}\norm{w}_2^2$) to solve the resulting linear case. This gives us the following update:
\begin{align}
w_t &= \argmin \sum_{i=1}^{t-1} g_i(w) + \frac{1}{\eta} \|w\|_2^2 \\
&= \argmin_{w \in \Omega} \sum_{i=1}^{t-1} \langle \nabla f_i(w_i), w \rangle + \frac{1}{\eta} \|w\|_2^2 \\
&= \Pi_\Omega \left( -\eta \cdot \sum_{i=1}^{t-1} \nabla f_i(w_i) \right),
\end{align}
where $\Pi_\Omega (\cdot)$ is the projection operator onto the set $\Omega$.The last equality is because for any vector $a$, we have 
\begin{align}
\argmin_{w \in \Omega} \langle a, w\rangle + \frac{1}{\eta} \|w\|_2^2 & = \argmin_{w \in \Omega} \frac{1}{2\eta} \|w + \eta a\|_2^2 - \eta \|a\|_2^2 \\
& = \argmin_{w \in \Omega} \|w + \eta a\|_2^2 \\
& = \argmin_{w \in \Omega} \|w - (-\eta a)\|_2^2 \\
& = \Pi_\Omega(-\eta a).
\end{align}

Intuitively, we can think of this algorithm as gradient descent with ``lazy'' projection:
\begin{align}
u_t &= u_{t-1} - \eta \nabla f_{t-1}(w_{t-1}), \\
w_t &= \Pi_\Omega(u_t).
\end{align}

Similarly, we can define gradient descent with ``eager'' projection (which can get similar regret bounds):
\begin{align}
u_t &= w_{t-1} - \eta \nabla f_{t-1}(w_{t-1}), \\
w_t &= \Pi_\Omega(u_t).
\end{align}

This concludes our discussion of online learning in this course.
	
	
	\appendix
	%    Include appendix "chapters" here.
	
	
	\backmatter
	%    Bibliography styles amsplain or harvard are also acceptable.
	\bibliographystyle{plainnat}
	\bibliography{all, bibliography}
	%    See note above about multiple indexes.
	%\printindex
	
\end{document}

%-----------------------------------------------------------------------
% End of amsbook-template.tex
%-----------------------------------------------------------------------
