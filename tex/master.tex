%% filename: amsbook-template.tex
%% version: 1.1
%% date: 2014/07/24
%%
%% American Mathematical Society
%% Technical Support
%% Publications Technical Group
%% 201 Charles Street
%% Providence, RI 02904
%% USA
%% tel: (401) 455-4080
%%      (800) 321-4267 (USA and Canada only)
%% fax: (401) 331-3842
%% email: tech-support@ams.org
%% 
%% Copyright 2006, 2008-2010, 2014 American Mathematical Society.
%% 
%% This work may be distributed and/or modified under the
%% conditions of the LaTeX Project Public License, either version 1.3c
%% of this license or (at your option) any later version.
%% The latest version of this license is in
%%   http://www.latex-project.org/lppl.txt
%% and version 1.3c or later is part of all distributions of LaTeX
%% version 2005/12/01 or later.
%% 
%% This work has the LPPL maintenance status `maintained'.
%% 
%% The Current Maintainer of this work is the American Mathematical
%% Society.
%%
%% ====================================================================

%    AMS-LaTeX v.2 driver file template for use with amsbook
%
%    Remove any commented or uncommented macros you do not use.

\documentclass[oneside, openany]{book}
\usepackage{amsfonts,bm,amsthm,amsmath,bbm,amssymb,mathtools}
\usepackage{fullpage}
\usepackage{tikz, pgfplots} % added for Lecture 2	
\usepackage{caption, subcaption}
\usepackage{float}  % added for Lecture 8
\usepackage[ruled,vlined,linesnumbered]{algorithm2e}  % added for Lecture 15
\usepackage{esdiff}
\usepackage{booktabs}  % added for Lecture 15
\usepackage{hyperref}
\hypersetup{linktocpage}
\usepackage{natbib}
\usepackage{cleveref}
\renewcommand{\cite}[1]{\citep{#1}}



\newtheorem{theorem}{Theorem}[chapter]
\newtheorem{lemma}[theorem]{Lemma}

\theoremstyle{definition}
\newtheorem{definition}[theorem]{Definition}
\newtheorem{example}[theorem]{Example}
\newtheorem{xca}[theorem]{Exercise}
\newtheorem{corollary}[theorem]{Corollary}  % added for Lecture 5
\newtheorem{proposition}{Proposition}[section]  % added for Lecture 6

\theoremstyle{remark}
\newtheorem{remark}[theorem]{Remark}

\numberwithin{section}{chapter}
\numberwithin{equation}{chapter}

%    For a single index; for multiple indexes, see the manual
%    "Instructions for preparation of papers and monographs:
%    AMS-LaTeX" (instr-l.pdf in the AMS-LaTeX distribution).
\makeindex
\def\lectureformat{0}
\usepackage{color}
\usepackage{lipsum}
\usepackage{enumitem}

% for potential improvements
\def\shownotes{1}  %set 1 to show author notes
\ifnum\shownotes=1
\newcommand{\authnoteimp}[2]{[#1: #2]}
\else
\newcommand{\authnoteimp}[2]{}
\fi
\newcommand{\tnoteimp}[1]{{\color{blue}\authnoteimp{todo}{#1}}}
\newcommand{\tnote}[1]{{\color{blue}\authnoteimp{TM}{#1}}}
\newcommand{\todo}[1]{\tnoteimp{#1}}


% for course staff to edit or comment
\def\shownotes{1}  %set 1 to show author notes
\ifnum\shownotes=1
\newcommand{\authnote}[2]{[#1: #2]}
\else
\newcommand{\authnote}[2]{}
\fi
\newcommand{\ttodo}[1]{{\color{blue}\authnote{todo for TM}{#1}}}


% for long term commnets 
\def\shownotes{1}  %set 1 to show author notes
\ifnum\shownotes=1
\newcommand{\authnotelong}[2]{[#1: #2]}
\else
\newcommand{\authnotelong}[2]{}
\fi
\newcommand{\tnotelong}[1]{{\color{blue}\authnotelong{TM}{#1}}}




\ifnum\lectureformat=1
\newcommand{\metadata}[3]
{
	\newpage
	
	\def\lectureID{#1}
	
	\setcounter{chapter}{\lectureID}

%	\draftnotice
	
	\begin{center}
		\bf\large CS229M/STATS214: Machine Learning Theory
	\end{center}
	
	\noindent
	Lecturer: Tengyu Ma   %%% FILL IN LECTURER (if not RS)
	\hfill
	Lecture \# \lectureID              %%% FILL IN LECTURE NUMBER HERE
	\\
	Scribe: #2                  %%% FILL IN YOUR NAME HERE
	\hfill
	#3           %%% FILL IN LECTURE DATE HERE
	
	\noindent
	\rule{\textwidth}{1pt}
	
	\medskip
}
\else 
\newcommand{\metadata}[3]{}
\fi

\newcommand*\circled[1]{\tikz[baseline=(char.base)]{
	\node[shape=circle,draw,inner sep=2pt] (char) {#1};}}

\DeclareMathOperator*{\Exp}{\mathbb{E}}
\DeclareMathOperator*{\argmin}{\textup{argmin}}
\DeclareMathOperator*{\argmax}{\textup{argmax}}

\newcommand{\Cov}{\operatorname{Cov}}
\newcommand{\KL}{\operatorname{KL}}
\newcommand{\margin}{\text{margin}}
\newcommand{\poly}{\operatorname{poly}}
\newcommand{\sd}{\operatorname{sd}}
\newcommand{\sgn}{\text{sgn}}
\newcommand{\tr}{\operatorname{tr}}
\newcommand{\Var}{\operatorname{Var}}
\newcommand{\ind}[1]{\mathbf{1}[#1]}

\newcommand{\err}{\ell_{\textup{0-1}}}
\newcommand{\Err}{L_{\textup{0-1}}}
\newcommand{\thetaerm}{\theta_{\textup{ERM}}}
\newcommand{\hatL}{\widehat{L}}
\newcommand{\tilO}{\widetilde{O}}
\newcommand{\iid}{\overset{\textup{iid}}{\sim}}
\newcommand\defeq{\stackrel{\mathclap{\text{\tiny\mbox{$\Delta$}}}}{=}}

\newcommand{\gammamin}{\gamma_{\mathrm{min}}}
\newcommand{\phirelu}{\phi_{\textup{relu}}}
\newcommand{\supunitball}{\sup_{\overline{u}:\norm{\overline{u}}_2 \le 1}}
\newcommand{\ubar}{\overline{u}}
\newcommand{\thetazero}{\theta^{0}}
\newcommand{\popL}{L(\beta)}
\newcommand{\empL}{\hatL(\beta)}
\newcommand{\popLt}{L(\beta^t)}
\newcommand{\empLt}{\hatL(\beta^t)}
\newcommand{\yhat}[0]{\hat{y}}

\newcommand{\norm}[1]{\|#1\|}
\newcommand{\Norm}[1]{\left\|#1\right\|}
\renewcommand{\l}{\left}
\renewcommand{\r}{\right}
\newcommand{\rbr}[1]{\left(#1\right)}
\newcommand{\sbr}[1]{\left[#1\right]}
\newcommand{\cbr}[1]{\left\{#1\right\}}
\newcommand{\abs}[1]{\left\lvert#1\right\rvert}
\newcommand{\inprod}[1]{\left\langle#1\right\rangle}
\renewcommand{\because}{{\textup{because}~}}

\newcommand{\al}[1]{
	\begin{align}
	#1
	\end{align}
}

\renewcommand{\sp}[1]{^{(#1)}}

\newcommand{\cA}{\mathcal A}	
\newcommand{\cB}{\mathcal B}
\newcommand{\cC}{\mathcal C}
\newcommand{\cD}{\mathcal D}
\newcommand{\cE}{\mathcal E}
\newcommand{\cF}{\mathcal F}
\newcommand{\cG}{\mathcal G}
\newcommand{\cH}{\mathcal H}
\newcommand{\cI}{\mathcal I}
\newcommand{\cJ}{\mathcal J}
\newcommand{\cK}{\mathcal K}
\newcommand{\cL}{\mathcal L}
\newcommand{\cM}{\mathcal M}
\newcommand{\cN}{\mathcal N}
\newcommand{\cO}{\mathcal O}
\newcommand{\cP}{\mathcal P}
\newcommand{\cQ}{\mathcal Q}
\newcommand{\cR}{\mathcal R}
\newcommand{\cS}{\mathcal S}
\newcommand{\cT}{\mathcal T}
\newcommand{\cU}{\mathcal U}
\newcommand{\cV}{\mathcal V}
\newcommand{\cW}{\mathcal W}
\newcommand{\cX}{\mathcal X}
\newcommand{\cY}{\mathcal Y}
\newcommand{\cZ}{\mathcal Z}

\newcommand{\bbB}{\mathbb B}
\newcommand{\bbS}{\mathbb S}
\newcommand{\bbR}{\mathbb R}
\newcommand{\bbZ}{\mathbb Z}
\newcommand{\bbI}{\mathbb I}
\newcommand{\bbQ}{\mathbb Q}
\newcommand{\bbP}{\mathbb P}
\newcommand{\bbE}{\mathbb E}
\newcommand{\bbN}{\mathbb N}

\newcommand{\N}{\mathbb{N}}
\newcommand{\R}{\bbR}
\newcommand{\Rnonneg}{\bbR_{\ge 0}}
\newcommand{\Z}{\mathbb{Z}}



\pgfplotsset{compat=1.17}
\begin{document}
	
	\frontmatter
	
	\title{Lecture Notes for Machine Learning Theory}
	
	%    Remove any unused author tags.
	
	%    author one information
	\author{Instructor: Tengyu Ma
	}
	%\address{}
	%\curraddr{}
	%\email{}
	%\thanks{}
	
	%    author two information
	%\author{}
	%\address{}
	%\curraddr{}
	%\email{}
	%\thanks{}
	
	%\subjclass[2010]{Primary }
	
	%\keywords{}
	
	%\date{}
	
	%\begin{abstract}
	%\end{abstract}
	
	
	\maketitle
	
	%    Dedication.  If the dedication is longer than a line or two,
	%    remove the centering instructions and the line break.
	%\cleardoublepage
	%\thispagestyle{empty}
	%\vspace*{13.5pc}
	%\begin{center}
	%  Dedication text (use \\[2pt] for line break if necessary)
	%\end{center}
	%\cleardoublepage
	
	%    Change page number to 6 if a dedication is present.
	%\setcounter{page}{1}
	
	\tableofcontents
	
	%    Include unnumbered chapters (preface, acknowledgments, etc.) here.
	%\include{}
	\mainmatter
	\let\sec\section
	\let\subsec\subsection
	\let\subsubsec\subsubsection
	
	
	
	
	\chapter*{Introduction} \label{chap:iintro}
	\setcounter{page}{5}

This monograph aims to provide an overview of modern machine learning theory, including both classical statistical learning techniques that remain highly relevant and fundamental today and newer developments in deep learning. The first six chapters on generalization bounds, along with sections on the method of moments, spectral clustering, and online learning, cover foundational concepts in statistical learning that are essential for understanding the current state of the field. The remaining chapters delve into more recent advances in deep learning theory, covering a selection of topics developed since 2013, such as non-convex optimization, implicit regularization effect and generalization theory under over-parameterization, and self-supervised learning / foundation models.

{\bf Organization.} The first chapter 

\tnote{todo}


%The overarching goal of this monograph is to offer a self-contained exposition of existing theory for modern machine learning. I attempt to select a minimal set of results and techniques from classical statistical learning theory that are still heavily relevant and necessary in the era of deep learning (the first six Chapters, Section~\ref{sec:method-of-moments} and Section~\ref{section:spectral_clustering}, and Chapter~\ref{chap:OL}), and then cover a selected subset of new topics on deep learning theory developed since 2013 (the rest of chapters). 



\noindent{\bf Notes and Acknowledgments.}


\sloppy I am deeply grateful to the many individuals who have contributed to the development of this monograph. This monograph is based on a collection of scribe notes for the courses CS229T/STATS231 and CS229M/STATS214 at Stanford University. The materials in Chapter \ref{chap:supervised}--\ref{chap:gen-bounds} are mostly based on Percy Liang's lecture notes~\citep{percynotes}, and Chapter~\ref{chap:OL} is largely based on Haipeng Luo's lectures~\cite{haipengnotes}. Kenneth Tay and John Cherian contributed significantly to the revision of these notes when they serve as the head teaching assistant for the course in winter 2020-2021 and fall 2021-2022 quarter, respectively. The original contributor to the scribe notes are Stanford students including but not limited to Anusri Pampari, Gabriel Poesia, Alexander Ke, Trenton Chang, Brad Ross, Robbie Jones, Yizhou Qian, Will Song, Daniel Do, Spencer M. Richards, Thomas Lew, David Lin, Jinhui Wang, Rafael Rafailov, Aidan Perreault, Kevin Han, Han Wu, Andrew Wang, Rohan Taori, Jonathan Lee, Rohith Kuditipudi, Kefan Dong, Roshni Sahoo, Sarah Wu, Tianyu Du, Xin Lu, Soham Sinha, Kevin Guo, Jeff Z. HaoChen, Carrie Wu, Kaidi Cao, and Ruocheng Wang.  The notes will be updated annually with new materials.  Please note that the reference list is far from complete.


	\part{Classical Statistical Learning Theory}\label{part:supervised}
	
	\chapter{Basic Formulations}\label{chap:supervised}
	% reset section counter

\setcounter{section}{0}

%\metadata{1}{Anusri Pampari and Gabriel Poesia}{Jan 11th, 2021}


In this chapter, we will set up the standard theoretical formulation of supervised learning and introduce the \textit{empirical risk minimization} (ERM) paradigm. 
We will also establish the basic notations that will be used throughout the monograph.
The ERM paradigm will be the main focus of Chapter~\ref{chap:asymp}, \ref{chap:uc}, and \ref{chap:gen-bounds}. 

\sec{Supervised learning}\label{lec1:sec:sup-learn}
In the supervised learning setting, we are given a training dataset consisting of multiple pairs of datapoints and the associated labels, and our goal is to learn a function that maps datapoints to their corresponding labels. The learned function should be able to predict the labels of test data that were not seen in the training dataset.

%The learned  function is supposed to infer the labels of test datapoints unseen in the training dataset. 

More formally, suppose the datapoints, also called inputs,  belong to the \textbf{input space} $\cX$ (e.g. natural images of birds), and labels belong to the \textbf{output space} $\cY$ (e.g. the discrete set of bird species). Suppose we are interested in a specific joint probability distribution $P$ over $\cX \times \cY$ (e.g. images of birds in North America), from which we draw a \textbf{training dataset}, i.e., we draw a a set of $n$ independent and identically distributed (i.i.d.) examples $\{(x^{(i)}, y^{(i)})\}_{i=1}^n$ from $P$. The goal is to learn a mapping (i.e., a function) from $\cX$ to $\cY$ using this training dataset. Any such function $h : \cX \rightarrow \cY$ is called a \textbf{predictor}, also known as a \textbf{hypothesis} or \textbf{model}.

What is the evaluation criterion for which predictors are better? We define a \textbf{loss function} that measures the quality of the predictions made by each predictor. 
A loss function is a function $\ell : \cY \times \cY \rightarrow \Rnonneg$ that takes as input the prediction made by a model $\hat{y}$ and the true label $y$, and produces a \emph{non-negative} value $\ell(\hat{y}, y)$ that indicates the difference between the two labels. 
%There are several ways to define loss functions: for now, define a loss function $\ell$ as a function $\ell : \cY \times \cY \rightarrow \R$. 
%Intuitively, the loss function takes two labels, the prediction made by a model $\hat{y}$ and the true label $y$, and gives a number that captures how different the two labels are. 
%We assume $\ell$ is non-negative, i.e $\ell(\hat{y}, y) \geq 0$. 
A small value of $\ell(\hat{y}, y)$ indicates a good prediction, while a large value indicates a poor prediction. 
The loss of a model $h$ on an example $(x, y)$ is give by $\ell(h(x), y)$, i.e. which is the difference between the prediction made by $h$ and the true label, as measured by the loss $\ell$.


With these definitions, we can now formalize the problem of supervised learning as finding a model $h$ that minimizes the \textbf{expected loss}, also known as the \textbf{population loss}, \textbf{expected risk},  or \textbf{population risk}:
\al{
    L(h) \defeq{} \Exp_{(x, y)\sim P} [\ell(h(x), y)]\,.
}


Note that the expected loss $L$ is nonnegative because the loss function $\ell$ is nonnegative. 
Typically, the loss function is designed such that the optimal loss is zero when the prediction $\hat{y}$ exactly matches the true label $y$. 
Therefore, the goal is to find a model $h$ such that $L(h)$ is as close to zero as possible. 
This indicates that the model is making accurate predictions on average.
% The best possible $h$ would have , we would find an $h$ with expected loss $0$, since that's the best possible we can do.

\paragraph{Regression and classification problems.}

Supervised learning problems can be broadly classified into two categories based on the properties of the output space:

\begin{itemize}
    \item In \emph{regression} problems, predictions and true labels are real numbers ($\cY = \R$). A common loss function that captures the difference between predictions and true labels is the squared error, $\ell(\hat{y}, y) = (\hat{y} - y)^2$.
    \item In \emph{classification} problems, predictions are in a discrete set of $k$ unordered classes $\cY = [k] = \{1, \cdots, k \}$. A common classification loss is the 0/1 loss: $\ell_{0/1}(\hat{y}, y) = \mathbbm{1}(\hat{y} \neq y)$, which is $0$ if the prediction is equal to the true label, and $1$ otherwise. There are other common loss functions for classification problems for the purpose of efficient training and/or theoretical analysis. 
\end{itemize}

\paragraph{Hypothesis class.}

Thus far, we have stated that our goal is to find a function that minimizes population loss. However, in practice, we cannot numerically optimize over arbitrary functions. Instead, we restrict ourselves to a set of functions, known as the \textbf{hypothesis class}, \textbf{model class}, or \textbf{model family}, denoted by $\cH$. Each element of $\cH$ is a function $h : \cX \rightarrow \cY$. Usually, we choose a set $\cH$ that we are able to optimize within, such as linear functions or neural networks.

For a given predictor/hypothesis/model $h \in \cH$, we define the \textbf{excess risk} of $h$ with respect to $\cH$ as the difference between the population loss of $h$ and the lowest possible population loss within $\cH$:
\begin{align}
E(h) \defeq{} L(h) - \inf_{g\in\cH} L(g)\,.\text{\footnotemark}
\end{align}
In other words, the population loss $L(h)$is equal to lthe sum of lowest possible loss within $\cH$ and the excess risk.  The former quantity reflects the fundamental expressivity or representational power of the hypothesis class $\cH$, and is independent of the number of training examples. In this monograph, we will generally assume that the hypothesis class is sufficiently expressive so that $\inf_{g\in\cH} L(g)$ is zero or negligible, and we focus on bounding from above the excess risk based on the properties of $\cH, P$ and the number of examples $n$.\footnotetext{The symbol $\inf$ represents the infimum~\citep{enwiki:inf}. For the purposes of this monograph, it can be thought of as equivalent to $\min$, the minimum value. In order to simplify the presentation, we will use these terms interchangeably without strictly adhering to mathematical rigor. }  
%Generally we need more assumptions about a specific problem and hypothesis class to bound absolute population risk, hence we focus on bounding the excess risk.

Usually, the hypothesis class is parameterized by a vector $\theta \in \Theta \subset \mathbb{R}^p$. In this case, we can refer to an element of $\cH$ as $h_\theta$, making the dependence on $\theta$ explicit. For example, the hypothesis class of linear models is $\cH = \{ h: h_\theta(x) = \theta^\top x, \theta \in \mathbb{R}^d \}$, where $h_\theta(x)$ is a linear function of $x$ with parameter $\theta$.

For notational simplicity, for parameterized hypothesis class, we often rewrite $\ell(h_\theta(x), y)$, the loss of model $h$ on an example $(x,y)$, as $\ell((x,y), \theta)$. Here $\ell$ is interpreted as a loss function that maps $\cX\times \cY \times \Theta$ to nonnegative real numbers. Also we view $L(\cdot)$ as a function of $\theta$ and use $L(\theta)$ to denoted the population loss (instead of $L(h_\theta)$). The same applies $\hatL$. 

%We typically parameterize the hypothesis class by a parameter vector $\theta \in \Theta\subset \R^p$. In that case, we can refer to an element of $\cH$ by $h_\theta$, making that explicit. An example of such a parametrization of the hypothesis class is $\cH = \{ h: h_\theta(x) = \theta^\top x, \theta \in \mathbb{R}^d \}$.

\sec{Empirical risk minimization}

Though our ultimate goal is to minimize population loss, we can only observe the population distribution $P$ through a \emph{training set} of $n$ datapoints that are sampled from $P$.  While we cannot compute and minimize the population loss, we can approximate the population loss by the so-called \textbf{empirical loss/risk}, which is the loss over the training set, and minimize the empirical loss. This paradigm is known as \textbf{empirical risk minimization} (ERM): we optimize the empirical loss on the training dataset with the hope that it will lead us to a model with low
population loss. From now on, with some abuse of notation, we often write $\ell(h_\theta(x),y)$ as $\ell((x,y),\theta)$ and use the two notations interchangeably.  Formally, given a training dataset $\{(x^{(i)}, y^{(i)})\}_{i=1}^n$, we define the empirical loss/risk of a model $h$ as:
\al{
	\hatL(h) \defeq{} \frac{1}{n} \sum_{i=1}^n \ell(h(x^{(i)}), y^{(i)})\,.%= \frac{1}{n} \sum_{i=1}^n \ell((x^{(i)}, y^{(i)}), \theta).
}
\emph{Empirical risk minimization} is the method of finding the minimizer of $\hatL$ as the learned predictor/model:%, which we call $\hat{\theta}$:
\al{
	\label{lec1:eqn:erm}
	\hat{h} \defeq{} \argmin_{h\in\cH} \hatL(h)\,.
}
Therefore, when the hypothesis class is parameterized by $\theta\in \Theta$, ERM refers to finding the minimizer $\hat{\theta}$ of the empirical risk:
\al{
	\label{lec1:eqn:erm:theta}
	\hat{\theta} \defeq{} \argmin_{\theta\in\Theta} \hatL(\theta).
}

%\al{
%\hatL(h_\theta) \defeq{} \frac{1}{n} \sum_{i=1}^n \ell(h_\theta(x^{(i)}), y^{(i)}) = \frac{1}{n} \sum_{i=1}^n \ell((x^{(i)}, y^{(i)}), \theta).
%}
%\emph{Empirical risk minimization} is the method of finding the minimizer of $\hatL$, which we call $\hat{\theta}$:
%\al{
%    \label{lec1:eqn:erm}
%    \hat{\theta} \defeq{} \argmin_{\theta\in\Theta} \hatL(h_\theta).
%}
Because here we assumed that our training examples are drawn from the population distribution $P$, the empirical risk equals to the population risk 
\emph{in expectation} (over the randomness of the training dataset):
\al{
    \Exp_{\{(x^{(i)}, y^{(i)})\}_{i=1}^n \iid P}\ \hatL(\theta) &= \Exp_{\{(x^{(i)}, y^{(i)})\}_{i=1}^n \iid P} \frac{1}{n} \sum_{i=1}^n \ell((x^{(i)}, y^{(i)}), \theta) \\
    &= \frac{1}{n} \sum_{i=1}^n \Exp_{(x^{(i)}, y^{(i)}) \sim P} \ell((x^{(i)}, y^{(i)}), \theta) \label{eqn:2}\\
%    &= \frac{1}{n} \cdot{} n \cdot{} \Exp_{(x^{(i)}, y^{(i)}) \sim P} \ell(h_\theta(x^{(i)}), y^{(i)}) \\
    &= \frac{1}{n} \sum_{i=1}^n  L(\theta) =  L(\theta)\,. 
}


In other words, the empirical loss is  an unbiased estimator of the population loss, and this is the primary motivation of ERM. However, obviously we need a certain number of examples for the empirical loss to be close enough to the population loss that they have similar minimizers. The central question in statistical learning theory is to provide guarantees or upper bounds on the excess risk for the model learned by ERM, which is the main focus in Part~\ref{part:supervised} of this monograph. 
%The hope with ERM is that minimizing the training error will lead to small testing error. One way to make this rigorous is by showing that the ERM minimizer's excess risk is bounded.

%	
%	\chapter{Asymptotic Analysis}\label{chap:asymp}
%	% reset section counter
\setcounter{section}{0}

%\metadata{lecture ID}{Your names}{date}
\metadata{2}{Alexander Ke and Trenton Chang}{Jan 13th, 2021}

This chapter briefly reviews the classical asymptotic results and techniques, which assume the number of training examples $n$ goes to infinity while all other (hyper-)parameters are treated as constants. It is not clear whether these results and techniques are still applicable to modern machine learning, where the number of parameters can often be larger than the number of examples. None of the later chapters depend on this chapter, so readers who are only interested in deep learning can feel free to skip it.  

Nevertheless, I still include these results in this text because the key ideas of using Taylor expansion and law of large numbers are so elegant and foundational that most machine learning theorists should probably be aware of them. 
The rest of the monograph will work in the non-asymptotic setting and one of the key technical difference will be that Taylor expansion cannot be applied and that the concentration inequalities need to be strengthened. 
Due to limited space, we will focus on the core ideas at the expense of some mathematical rigor. The conceptual ideas and techniques of this chapter is self-contained, whereas a fully rigorous treatment of these concepts require additional mathematical language that is beyond the scope of this monograph. We refer the readers to~\citep{vaart_1998} for more in-depth presentation of this line of work. 




%In this chapter, we use an asymptotic approach (i.e. assuming number of training samples $n \to \infty$) to achieve a bound on the ERM. 
%We then instantiate these results to the case where the loss function is the maximum likelihood  and discuss the limitations of asymptotics. 
%(In future chapters we will assume finite $n$ and provide a non-asymptotic analysis.)

\sec{Asymptotics of empirical risk minimization}

The goal of the asymptotic analysis of ERM is to bound from above the excess risk in the following form:
\al{
    L(\hat{\theta}) - \inf_{\theta \in \Theta} L(\theta) \leq \frac{c}{n} + o\left(\frac{1}{n}\right)\,. 
    \label{lec1:eqn:erm-bound}
}
Here $c$ is a problem-dependent constant that is independent of $n$, and the $o(\cdot)$ notation hides all dependencies except for $n$. The equation above shows that as the number of training examples $n$ increases, the excess risk of ERM decreases at the rate of $\frac{1}{n}$.

Let $\{(x^{(1)},y^{(1)}), \cdots, (x^{(n)},y^{(n)})\}$ be the training data and let $\cH = \{ h_\theta: \theta \in \R^p \}$ be the parameterized family of hypothesis functions. Let $\hat{\theta}$ be the empirical risk minimizer as defined in~\Cref{lec1:eqn:erm:theta}. Let $\theta^{*}$ be the minimizer of the population risk $L$ (i.e., $\theta^{*} = \argmin_\theta L(\theta)$, )which is assumed to be unique. The theorem below characterize the excess risk $L(\hat{\theta}) - L(\theta^{*})$, which is a random variable (where the randomness comes from the training dataset). 

\begin{theorem}[Informally stated] \label{lec1:thm:asymp}
Assume that (a) the consistency of the ERM estimator $\hat{\theta}$ in the sense that $\hat{\theta}  \overset{p}{\to} \theta^{*}$ as $n \to \infty$, (b) the Hessian of the population loss at $\theta^*$, $\nabla^{2}L(\theta^{*})$, is full rank, and  (c) other appropriate regularity conditions hold.\footnote{$X_n \overset{p}{\to} X$ implies that for all $\epsilon > 0$, $\bbP \left (\norm{X_n - X} > \epsilon \right ) \to 0$, while $X_n \overset{d}{\to} X$ implies that $\bbP(X_n \leq t) \to \bbP(X \leq t)$ at all points $t$ for which $\bbP(X \leq t)$ is continuous. These two notions of convergence are known as convergence in probability and convergence in distribution, respectively. These concepts are not essential to this course, but additional information can be found by reading the Wikipedia \href{https://en.wikipedia.org/wiki/Convergence_of_random_variables}{article} on convergence of random variables.} 
Then,
\begin{enumerate}
    \item The estimated model $\hat{\theta}$ converges to the $\theta^*$ with a $1/\sqrt{n}$ rate in the sense that $\sqrt{n} (\hat{\theta} - \theta^{*}) = O_P(1)$, that is, for every $\epsilon > 0$, there is an $M$ such that $\forall n, \bbP (\| \sqrt{n} (\hat{\theta} - \theta^{*}) \|_2 > M) \le \epsilon$.\footnote{In other words, the sequence of random variables $\{ \sqrt{n} (\hat{\theta} - \theta^{*}) \}$ indexed by $n$ is ``bounded in probability" . } 
    \item  The scaled parameter error $\sqrt{n}(\hat{\theta}-\theta^{*})$ has asymptotic normality distribution in the sense that as $n\rightarrow \infty$, 
    \al{\sqrt{n}(\hat{\theta}-\theta^{*}) \overset{d}{\to} \mathcal{N} \left(0, (\nabla^{2}L(\theta^{*}))^{-1}\Cov(\nabla \ell((x,y), \theta^*)) (\nabla^{2}L(\theta^{*}))^{-1} \right) \,.}
     \item The excess loss decays with an $1/n$ rate
    \al{n (L(\hat{\theta}) - L(\theta^{*})) = O_P(1)\,.
    	\label{eqn:3}
    }
     Moreover, 
     \begin{align}
     \lim_{n \to \infty} \Exp \left[ n (L(\hat \theta) - L(\theta^*)) \right] = \frac12 \tr\left( \nabla^2 L(\theta^*)^{-1} \Cov(\nabla \ell ((x, y), \theta^*) \right)\,.\label{eqn:4}
     \end{align}
    \item The scaled excess loss has a $\chi^2$ distribution: 
    \begin{align}
    n (L(\hat{\theta}) - L(\theta^{*})) \overset{d}{\to} \frac{1}{2} ||S||_{2}^{2} \,.
    \end{align}
    where $S \sim \mathcal{N} \left(0, (\nabla^{2}L(\theta^{*}))^{-1/2}\Cov(\nabla \ell((x,y), \theta^*)) (\nabla^{2}L(\theta^{*}))^{-1/2}\right)$.
\end{enumerate}
\end{theorem}
%\textbf{Remark:} In the theorem above, Parts 1 and 3 only show the rate or order of convergence, while Parts 2 and 4 define the limiting distribution for the random variables.
%is a powerful conclusion because once we know that $\sqrt{n}(\hat \theta  - \theta^*)$ is (asymptotically) Gaussian, we can easily work out the distribution of the excess risk. 
%If we believe in our assumptions and $n$ is large enough such that we can assume $n \to \infty$, this allows us to analytically determine quantities of interest in almost any scenario (for example, if our test distribution changes). 

The key takeaway is that our parameter error $\hat{\theta} - \theta^*$ decreases on the order of $1/\sqrt{n}$ and the excess risk decreases on the order of $1/n$. We note that the twice-differentiable loss function $L$ plays a crucial role in the rate here---e.g., if the loss for a regression problem is $|\hat{y}-y|$ instead of the squared loss, the excess rate will not have a $1/n$ rate. 

Theorem \ref{lec1:thm:asymp} is powerful because the characterization of the distribution of  $\sqrt{n}(\hat \theta  - \theta^*)$ can lead us to almost any property of $\hat{\theta}$ of interest. For example, in principle, one can characterize the expected loss of the estimator $\hat{\theta}$ on a test distribution different from the distribution $P$. On the other hand, the theorem requires taking the limit as $n$ goes to infinity while ignoring the dependencies on other problem parameters (such as dimension). This significantly limits the applicability of the theorem in modern machine learning settings (see Section~\ref{sec:limit-asymp} for more discussion). 
%While we will not discuss the regularity assumptions in Theorem~\ref{lec1:thm:asymp} in great detail, we note that the assumption that $L$ is twice differentiable is crucial. 

\subsec{Key proof ideas} 

We will sketch a proof Theorem \ref{lec1:thm:asymp} by applying the following two key ideas. 
\begin{enumerate}
    \item[1] \textbf{Taylor expansion.} We will derive a formula for $\hat{\theta}$ and the excess risk by Taylor-expanding the empirical loss $\nabla \hatL(\theta)$ (and its gradient )around the reference point $\theta^{*}$.
    \item[2] \textbf{Law of large number and central limit theorem.} We will simplify and characterize various empirical quantities, e.g.,  $\hatL(\theta)$ and $\hatL^2(\theta)$ by the law of large numbers and central limit theorems. For exmaple, we will use the facts that $\hatL(\theta) \overset{p}{\to} L(\theta)$, $\nabla\hatL(\theta) \overset{p}{\to} \nabla L(\theta)$   and  $\nabla^{2}\hatL(\theta) \overset{p}{\to} \nabla^{2} L(\theta)$ as $n \to \infty$, and that $\nabla \hatL(\theta)-\nabla L(\theta)$ is asymptotically normal. 
\end{enumerate}
 
To prepare us with more detail, we first state the the central limit theorem for a sum of i.i.d. random variables.

\begin{theorem}[Central Limit Theorem] \label{lec1:thm:CLT}
Let $X_1, \cdots, X_n$ be i.i.d. $d$-dimensional random variables and $\widehat{X}=\frac{1}{n} \sum_{i=1}^{n} X_i$ and the covariance matrix $\Sigma = \Exp[X_iX_i^\top]\in \R^{d\times d}$ is finite. Then, as $n \to \infty$, we have
\begin{enumerate}
    \item $\widehat{X} \overset{p}{\to} \Exp[X]$, and
    \item $\sqrt{n} (\widehat{X}-\Exp[X]) \overset{d}{\to} \mathcal{N}(0,\Sigma)$. %In particular, $\sqrt{n} (\widehat{X}-\Exp[X]) = O_P(1)$.
\end{enumerate}
\end{theorem}

We also state a lemma that asserts the linear transformation of a Gaussian random variable is still Gaussian and computes its covariance. %The second part claims that quadratic form of Gaussian random variable has a $\chi^2$ distribution.  
\begin{lemma}\label{lec1:lem:dist}
%\quad\quad
%    \begin{enumerate}
If $Z \sim N(0, \Sigma)$ and $A$ is a deterministic matrix, then $AZ \sim N(0, A \Sigma A^\top)$. \tnote{todo second part of the old lemma}
%\subsec{Main proof}        
%        \item If $Z \sim N(0, \Sigma^{-1})$ and $Z \in \bbR^p$, then $Z^\top \Sigma Z \sim \chi^2(p)$, where $\sim \chi^2(p)$ is the chi-squared distribution with $p$ degrees of freedom.
%    \end{enumerate}
\end{lemma}
With these preparations, we will sketch a proof for parts 1 and 2 of Theorem~\ref{lec1:thm:asymp}. 

By definition, the gradient of the empirical risk evaluated at the empirical risk minimizer, that is, $\nabla \hatL(\hat{\theta})$, is equal to $0$. From the Taylor expansion of $\nabla \hatL$ around $\theta^*$, we have that 
\begin{align}
    0 = \nabla \hatL(\hat{\theta}) = \nabla \hatL(\theta^*) + \nabla^2 \hatL(\theta^*)(\hat{\theta} - \theta^*) + O(\|\hat{\theta} - \theta^*\|^2_2)\perm\text{\footnotemark}
\end{align}

Rearranging the equation above,\footnotetext{Technically, the big-O notation here and in the sequel should be $O_P(\cdot)$ because $\hat{\theta}$ is a random variable. However, we omit this distinction in this proof sketch.} gives a solution for $\hat{\theta}-\theta^*$, 
\al{
 \hat{\theta}-\theta^{*} = -(\nabla^{2}\hatL(\theta^{*}))^{-1} \nabla \hatL(\theta^{*}) + O(||\hat{\theta}-\theta^{*}||_{2}^{2}). \label{lec1:eqn:branch}} 

Multiplying by $\sqrt{n}$ on both sides (so that we deal with quantities on the constant order), we have
 \al{
\sqrt{n} (\hat{\theta}-\theta^{*}) &= -(\nabla^{2}\hatL(\theta^{*}))^{-1} \sqrt{n} (\nabla \hatL(\theta^{*})) + O(\sqrt{n} ||\hat{\theta}-\theta^{*}||_{2}^{2}) \\
&\approx -(\nabla^{2}\hatL(\theta^{*}))^{-1} \sqrt{n} (\nabla \hatL(\theta^{*})). \label{lec1:eqn:interm}}

 
Applying the Central Limit Theorem (Theorem~\ref{lec1:thm:CLT}) using $X_i = \nabla \ell ((x^{(i)}, y^{(i)}), \theta^*)$ and $\widehat{X} = \nabla \hatL(\theta^*)$, and noticing that $\Exp[\nabla \hatL(\theta^{*})] = \nabla L(\theta^{*})$, we have
$\sqrt{n} (\nabla \hatL(\theta^{*}) - \nabla L(\theta^{*})) \overset{d}{\to} \mathcal{N}(0,\Cov(\nabla \ell((x, y), \theta^{*}))).$ 
 
Note that $\nabla L(\theta^{*}) = 0$ because $\theta^{*}$ is the minimizer of  $L$. Thus, we have \al{\sqrt{n} \cdot\nabla \hatL(\theta^{*}) \overset{d}{\to} \mathcal{N}(0,\Cov(\nabla \ell((x, y), \theta^{*})))\perm}
By the law of large numbers, we have $\nabla^2 \hatL(\theta^*) \stackrel{p}{\rightarrow} \nabla^2 L(\theta^*)$. We use these results to replace the empirical quantity in \Cref{lec1:eqn:interm} by the population quantity (technically this is an application of Slutsky's theorem), we have
\al{
\sqrt{n} (\hat{\theta}-\theta^{*}) &\overset{d}{\to} \nabla^{2}L(\theta^{*})^{-1} \mathcal{N}(0,\Cov(\nabla \ell((x,y),\theta^{*}))) \\
&\stackrel{d}{=} \mathcal{N} \left( 0,\nabla^{2}L(\theta^{*})^{-1}\Cov(\nabla \ell((x,y), \theta^{*})) \nabla^{2}L(\theta^{*})^{-1} \right),
}
where the second step is due to Lemma~\ref{lec1:lem:dist}. This proves Part 2 of Theorem~\ref{lec1:thm:asymp}.

Part 1 follows directly from Part 2 by the fact a sequence of random variables that converges to a fixed Gaussian distribution is also bounded in probability.

We now turn to proving Parts 3 and 4 of Theorem~\ref{lec1:thm:asymp} which uses Part 2. A Taylor expansion of $L$ at the reference point $\theta^*$ gives
\begin{equation}
L(\hat \theta) = L(\theta^*) 
+ \langle \nabla L(\theta^*), \hat \theta - \theta^* \rangle 
+ \frac12 \langle \hat \theta - \theta^*, \nabla^2 L(\theta^*) (\hat \theta - \theta^*) \rangle + o(\|\hat \theta - \theta^*\|_2^2).
\end{equation}
Since $\theta^*$ is the minimizer of the population loss $L$, we have $\nabla L(\theta^*) = 0$ and the linear term $\langle \nabla L(\theta^*), \hat \theta - \theta^* \rangle$ is equal to 0. Rearranging and multiplying by $n$, we can write
\begin{align}
n (L(\hat \theta) - L(\theta^*)) &= \frac{n}{2} \langle \hat \theta - \theta^*, \nabla^2 L(\theta^*) (\hat \theta - \theta^*) \rangle + o(\|\hat \theta - \theta^*\|_2^2) \\
&\approx \frac12 \langle \sqrt n(\hat \theta - \theta^*), \nabla^2 L(\theta^*) \sqrt n (\hat \theta - \theta^*) \rangle \\
&= \frac12 \left\|\nabla^2 L(\theta^*)^{1/2} \sqrt n(\hat \theta - \theta^*) \right\|_2^2,
\end{align}
where the last equality follows from the fact that for any vector $v$ and positive semi-definite matrix $A$ of appropriate dimensions, the inner product $\langle v, Av\rangle = v^\top Av = \lVert A^{1/2}v \rVert_2^2$. Let $S = \nabla^2 L(\theta^*)^{1/2} \sqrt n(\hat \theta - \theta^*)$, i.e., the random vector inside the norm. By Part 2, we know the asymptotic distribution of $\sqrt n(\hat \theta - \theta^*)$ is Gaussian, and by Lemma~\ref{lec1:lem:dist} we know $S$ also has a Gaussian distribution: 
\begin{align}
    S %&\sim \nabla^2 L(\theta^*)^{1/2} \cdot \cN \left(0, \nabla^2 L(\theta^*)^{-1} \Cov(\nabla \ell ((x, y), \theta^*)) \nabla^2 L(\theta^*)^{-1} \right) \\
    &\stackrel{d}{=} \cN \left(0, \nabla^2 L(\theta^*)^{-1/2} \Cov(\nabla \ell ((x, y), \theta^*)) \nabla^2 L(\theta^*)^{-1/2} \right).
\end{align}
This proves Part 4, and Part 3, \Cref{eqn:3} follows directly from the definition of the $O_P$ notation. 

Finally, we derive \Cref{eqn:4} by using the fact that the trace operator is invariant under cyclic permutations,  that $\Exp [S] = 0$, and some regularity conditions,
\begin{align}
    \lim_{n \to \infty} \Exp \left[ n (L(\hat \theta) - L(\theta^*)) \right] &= \frac12 \Exp\left[ \|S\|_2^2 \right] = \frac12 \Exp \left[ \tr(S^\top S) \right] \\
    &= \frac12 \Exp \left[ \tr(S S^\top) \right]  = \frac12 \tr \left(\Exp[S S^\top] \right) \\
    &= \frac12 \tr \left( \Cov(S) \right) \\
    &= \frac12 \tr\left( \nabla^2 L(\theta^*)^{-1} \Cov(\nabla \ell ((x, y), \theta^*)) \right).
\end{align}

\subsec{Well-specified case}

Theorem \ref{lec1:thm:asymp} is quite general without any assumptions of a probabilistic model of the data distribution $P$. In many applications, we assume a probabilistic model of our data and use the log-likelihood as the loss function. This is often referred to as the well-specified case in this context, and it allows some simplification of the conclusions of Theorem~\ref{lec1:thm:asymp} with nice interpretations. 


Formally, suppose that we have a family of probability distributions $P_\theta$, parameterized by $\theta \in \Theta$. We assume the data distribution $P$ belongs to this family, that is, there exists $\theta_*$ such that $P = P_{\theta_*}$. This is known as the well-specified case. Theorem \ref{lec1:thm:asymp} implies the following Theorem \ref{lec2:thm:applied} as a direct corollary.

\begin{theorem}
\label{lec2:thm:applied}
    In addition to the assumptions of Theorem~\ref{lec1:thm:asymp}, suppose there exists a parametric model $P(y \mid x; \theta)$, $\theta \in \Theta$, such that $\{ y\sp{i} \mid x\sp{i} \}_{i=1}^n \sim P( y\sp{i} \mid x\sp{i} ; \theta_*)$ for some $\theta_* \in \Theta$. Assume that we perform maximum likelihood estimation (MLE), i.e., our loss function is the negative log-likelihood $\ell((x\sp{i}, y\sp{i}), \theta) = - \log P( y\sp{i} \mid x\sp{i} ; \theta)$. As before, let $\hat\theta$ and $\theta^*$ denote the minimizers of empirical loss and population loss, respectively. Then, we have
    \al{
    \label{lec2:eqn:applied1}
        \theta^* = \theta_*,
    }
    \al{
    \label{lec2:eqn:applied2}
        \Exp \left[ \nabla \ell ((x, y), \theta^*) \right] = 0,
    }
    \al{
    \label{lec2:eqn:applied3}
        \Cov \left( \nabla \ell ((x, y), \theta^*) \right) = \nabla^2 L(\theta^*), \text{ and}
    }
    \al{
    \label{lec2:eqn:applied4}
        \sqrt n (\hat \theta - \theta^*) \overset d \to \cN (0, \nabla^2 L(\theta^*)^{-1}).
    }
\end{theorem}

\textbf{Remark 1:} You may also have seen \eqref{lec2:eqn:applied4} in the following form: under the maximum likelihood estimation (MLE) paradigm, the MLE is asymptotically efficient as it achieves the Cramer-Rao lower bound. That is, the parameter error of the MLE estimate converges in distribution to $\mathcal{N}(0, \mathcal{I}(\theta)^{-1})$, where $\mathcal{I}(\theta)$ is the Fisher information matrix (in this case, equivalent to the risk Hessian $\nabla^2 L(\theta^*)$)~\cite{rice2006mathematical}.

\textbf{Remark 2:} \eqref{lec2:eqn:applied3} is also known as Bartlett's identity~\cite{percynotes}.

Although the proofs were not presented in live lecture, we include them here.

\begin{proof}
From the definition of the population loss,
\begin{align}
    L(\theta) &= \Exp \left[ \ell((x\sp{i}, y\sp{i}), \theta) \right]\\
    &= \Exp \left[ - \log P(y \mid x; \theta) \right] \\
    &= \Exp \left[ - \log P(y \mid x; \theta) + \log P(y \mid x; \theta_*) \right] + \Exp \left[ - \log P(y \mid x; \theta_*) \right] \\
    &= \Exp \left[ \log \frac{P(y \mid x; \theta_*)}{P(y \mid x; \theta)} \right] + \Exp \left[ - \log P(y \mid x; \theta_*) \right].
\end{align}
Notice that the second term is a constant which we will express as $\cH(y \mid x; \theta_*)$. We expand the first term using the tower rule (or law of total expectation):
\begin{align}
    L(\theta) &= \Exp \left[ \Exp \left[ \log \frac{P(y \mid x; \theta_*)}{P(y \mid x; \theta)} \biggr\vert x \right] \right] + \cH(y \mid x; \theta_*).
\end{align}
The term in the expectation is just the KL divergence between the two probabilities, so 
\begin{align}
    L(\theta) &= \Exp \left[ \KL \left( y \mid x; \theta_* \| y \mid x; \theta \right) \right] + \cH(y \mid x; \theta_*) \\
    &\geq \cH(y \mid x; \theta_*),
\end{align}
since KL divergence is always non-negative. Since $\theta_*$ makes the KL divergence term 0, it minimizes $L(\theta)$ and so $\theta_* \in \argmin_\theta L(\theta)$. However, the minimizer of $L(\theta)$ is unique because of consistency, so  we must have $\argmin_\theta L(\theta) = \theta^*$ which proves (\ref{lec2:eqn:applied1}).

For \eqref{lec2:eqn:applied2}, recall $\nabla L(\theta^*) = 0$, so we have
\begin{equation}
0 = \nabla L(\theta^*) = \nabla \Exp \left[ \ell((x\sp{i}, y\sp{i}), \theta^*) \right] = \Exp \left[ \nabla \ell((x\sp{i}, y\sp{i}), \theta^*) \right],
\end{equation}
where we can switch the gradient and expectation under some regularity conditions.

To prove \eqref{lec2:eqn:applied3}, we first expand the RHS using the definition of covariance and express the marginal distributions as integrals:
\begin{align}
    \Cov \left( \nabla \ell ((x, y), \theta^*) \right) &= \Exp \left[ \nabla \ell ((x, y), \theta^*) \nabla \ell ((x, y), \theta^*)^\top \right] \\
    &= \int P(x) \left( \int P(y \mid x; \theta^*) \nabla \log P( y\sp{i} \mid x\sp{i} ; \theta^*) \nabla \log P( y\sp{i} \mid x\sp{i} ; \theta^*)^\top dy \right) dx \\
    &= \int P(x) \left( \int \frac{\nabla P(y \mid x; \theta^*) \nabla P(y \mid x; \theta^*)^\top}{P(y \mid x; \theta^*)}dy \right) dx.
\end{align}
Now we expand the LHS using the definition of the population loss and differentiate repeatedly:
\begin{align}
    \nabla^2 L(\theta^*) &= \Exp \left[- \nabla^2 \log P(y \mid x; \theta^*) \right] \\
    &= \int P(x) \left( \int - \nabla^2 P(y \mid x; \theta^*) + \frac{\nabla P(y \mid x; \theta^*) \nabla P(y \mid x; \theta^*)^\top}{P(y \mid x; \theta^*)}dy  \right) dx.
\end{align}
Note that we can express 
\begin{equation} \int \nabla^2 P(y \mid x; \theta^*) dy = \nabla^2 \int P(y \mid x; \theta^*) dy = \nabla 1  = 0 \end{equation}
so we find
\begin{equation} \nabla^2 L(\theta^*) = \int P(x) \left( \int \frac{\nabla P(y \mid x; \theta^*) \nabla P(y \mid x; \theta^*)^\top}{P(y \mid x; \theta^*)}dy \right) dx = \Cov \left( \nabla \ell ((x, y), \theta^*) \right). \end{equation}

Finally, \eqref{lec2:eqn:applied4} follows directly from Part 2 of Theorem~\ref{lec1:thm:asymp} and \eqref{lec2:eqn:applied3}.
\end{proof}

Using similar logic to our proof of Part 4 and 5 of Theorem~\ref{lec1:thm:asymp}, we can see that $n (L(\hat \theta) - L(\theta^*)) \overset d \to \frac12 \|S\|_2^2$ where $S \sim N(0, I)$. Since a chi-squared distribution with $p$ degrees of freedom is defined as a sum of the squares of $p$ independent standard normals, it quickly follows that $2n (L(\hat \theta) - L(\theta^*)) \sim  \chi^2(p)$, where $\theta \in \R^p$ and $n \to \infty$. We can thus characterize the excess risk in this case using the propertes of a chi-squared distribution:

\al{
    \lim_{n \to \infty} \Exp \left[ L(\hat \theta) - L(\theta^*) \right] = \frac{p}{2n}.
}

\sec{Limitations of asymptotic analysis}\label{sec:limit-asymp}

One limitation of asymptotic analysis is that our bounds often obscure dependencies on higher order terms. As an example, suppose we have a bound of the form
	\al{
		\frac{p}{2n} + o\left(\frac{1}{n}\right).
		\label{lec2:eqn:spicy_bound}
	}
(Here $o(\cdot)$ treats the parameter $p$ as a constant as $n$ goes to infinity.) 
We have no idea how large $n$ needs to be for asymptotic bounds to be ``reasonable." Compare two possible versions of \eqref{lec2:eqn:spicy_bound}: 
\begin{align}
    \frac{p}{2n} + \frac{1}{n^2} \quad \text{vs.} \quad \frac{p}{2n} + \frac{p^{100}}{n^2}.
\end{align}
Asymptotic analysis treats both of these bounds as the same, hiding the polynomial dependence on $p$ in the second bound. Clearly, the second bound is significantly more data-intensive than the first: we would need $n > p^{50}$ for $\frac{p^{100}}{n^2}$ to be less than one. Since $p$ represents the dimensionality of the data, this may be an unreasonable assumption.

This is where non-asymptotic analysis can be helpful. Whereas asymptotic analysis uses large-sample theorems such as the central limit theorem and the law of large numbers to provide convergence guarantees, non-asymptotic analysis relies on concentration inequalities to develop alternative techniques for reasoning about the performance of learning algorithms.


%
%	\chapter{Concentration Inequalities}\label{chap:conc}
%	% reset section counter
\setcounter{section}{0}

%\metadata{lecture ID}{Your names}{date}
\metadata{3}{Brad Ross and Robbie Jones}{Jan 20, 2021}

In this chapter, we take a little diversion and develop the notion of \emph{concentration inequalities}. Assume that we have independent random variables $X_1, \ldots, X_n$. We will develop tools to show results that formalize the intuition for these statements:
\begin{enumerate}
    \item $X_1 + \ldots + X_n$ concentrates around $\Exp[X_1 + \ldots + X_n]$.
    \item More generally, $f(X_1, \ldots, X_n)$ concentrates around $\Exp[f(X_1, \ldots, X_n)]$.
\end{enumerate}

These inequalities will be used in subsequent chapters to bound several key quantities of interest.

As it turns out, the material from this chapter constitutes arguably the important mathematical tools in the entire course. No matter what area of machine learning one wants to study, if it involves sample complexity, some kind of concentration result will typically be required. Hence, concentration inequalities are some of the most important tools in modern statistical learning theory.

\sec{The big-O notation}

Throughout the rest of this course, we will use ``big-O" notation in the following sense: every occurrence of $O(g(x))$ is a placeholder for some function $f(x)$ such that for every $x$, $|f(x)| \leq Cg(x)$ for some absolute/universal constant $C$. In other words, suppose $O(g_1(n)),\dots, O(g_k(n))$ occur in a statement, it means that \textbf{there exists} absolute constants $C_1,\dots, C_k > 0$ and functions $f_1,\dots, f_k$ satisfying $|f_i(n)|\le C_ig_i(n)$ for all $n$, such that after replacing each occurrence $O(g_i(n))$ by $f_i(n)$,  the statement is true.  (The difference with traditional ``big-O" notation is that we do not need to send $n \to \infty$ in order to define ``big-O".)

Also, for any $a, b \geq 0$, we will let $a \lesssim b$ mean that there is some absolute constant $c > 0$ such that $a \leq cb$.


\sec{Hoeffding's inequality}\label{lec2:subsec:hoeffding}

We provide a brief overview of Hoeffding's inequality, an important concentration inequality for bounded random variables:

\begin{theorem}[Hoeffding's inequality]
    Let $X_1, X_2, \dots, X_n$ be i.i.d. real-valued random variables drawn from some distribution, such that $a_i \leq X_i \leq b_i$ almost surely. Define $\bar{X} = \frac{1}{n}\sum_{i=1}^n X_i$, and let $\mu = \E [\bar{X}]$. Then for any $\varepsilon > 0$,
    \al {
    \Pr \left[ |\bar{X} - \mu | \leq \varepsilon \right] \geq 1 - 2  \exp\left(\frac{-2n^2\varepsilon^2}{\sum_{i=1}^n (b_i - a_i)^2}\right). \label{lec2:eqn:hoeffding}
    }
\end{theorem}

Note that the demoninator within the exponential term, $\sum_{i=1}^n (b_i - a_i)^2$, can be thought of as an upper bound or proxy for the variance $\Var(X_i)$. In fact, under the i.i.d. assumption, we can show
\begin{align}
    \Var\left(\bar{X} \right) &= \frac{1}{n^2}\sum_{i=1}^n \Var(x_i) \leq \frac{1}{n^2}\sum_{i=1}^n (b_i - a_i)^2.
\end{align}

Let $\sigma^2 = \frac{1}{n^2}\sum_{i=1}^n (b_i - a_i)^2$. If we take $\varepsilon = O(\sigma \sqrt{\log{n}}) = \sigma \sqrt{c \log n}$; i.e. $\varepsilon$ bounded above by some large (i.e., $c \geq 10$) multiple of the standard deviation of the $x_i$'s times $\sqrt{\log{n}}$, we can substitute this value of $\varepsilon$ into \eqref{lec2:eqn:hoeffding} to reach the following conclusion: 
\begin{align}
    \Pr \left[ |\bar{X} - \mu| \leq \varepsilon \right] &\geq 1 - 2\exp\left(\frac{-2 \varepsilon^2}{\sigma^2}\right)\\
    &= 1 - 2 \exp(-2 c \log n)\\
    &= 1 - 2 n^{-2c}
\end{align}

We can see that as $n$ grows, the right-most term tends to zero such that $\Pr[|\bar{X} - \mu| \leq \varepsilon]$ very quickly approaches 1. Intuitively, this result tells us that, with high probability, the sample mean $\bar{X}$ will not be ``much farther" from the population mean $\mu$ by some sublogarithmic ($\sqrt{c \log n}$) factor of the standard deviation.\footnote{This is with the caveat, of course, that $\sigma$ is not exactly standard deviation but a loose upper bound on standard deviation.} Thus, we can restate the above claim we reached as follows:

\begin{remark}
    For sufficiently large $n$, $|\bar{X} - \mu | \leq O(\sigma \sqrt{\log{n}})$ with high probability.
\end{remark}

\begin{remark}\label{lec2:rem:hoeffding}
    If, in addition, we have $a_i = -O(1)$ and $b_i = O(1)$, then $\sigma^2 = O \left( \frac{1}{n}\right)$, and $|\bar{X} - \mu | \leq O\left(\sqrt{\frac{\log n}{n}}\right) = \tilO\left(\frac{1}{\sqrt{n}}\right)$.\footnote{$\tilO$ is analogous to Big-$O$ notation, in that $\tilO$ hides logarithmic factors. That is; if $f(n) = O(\log n)$, then $f(n) = \tilO(1)$.}
\end{remark}

Remark~\ref{lec2:rem:hoeffding} provides a compact form of the Hoeffding bound that we can use when the $X_i$ are bounded almost surely.

\sec{Chebyshev's inequality}

\tnoteimp{This subsection should go before the subsection Hoelfding inequality; the transition sentences and the discussion around the end of this subsection need to be rephrased a bit after reorganizing the subsections}

Consider an arbitrary random variable $Z$ with finite variance. One of the most famous results characterizing its tail behavior is the following theorem:

\begin{theorem}[Chebyshev's inequality]
    Let $Z$ be a random variable with finite expectation and variance. Then
    \al{
        \Pr[|Z - \Exp[Z]| \geq t] \leq \frac{\Var(Z)}{t^2}, \quad \forall t > 0.
        \label{lec3:eqn:chebyshev}
    }
\end{theorem}

Intuitively, this means that as we approach the tails of the distribution of $Z$, the density decreases at a rate of at least $1 / t^2$. Moreover, for any $\delta \in (0, 1]$, by plugging in $t = \sd(Z) / \sqrt{\delta}$ to \eqref{lec3:eqn:chebyshev} we see that 
    \al{
        \Pr\left[|Z - \Exp[Z]| \leq \frac{\sd(Z)}{\sqrt{\delta}}\right] \geq 1 - \delta.
        \label{lec3:eqn:chebyshevdelta}
    }
    
Unfortunately, it turns out that Chebyshev's inequality is a rather weak concentration inequality. To illustrate this, assume $Z \sim \cN(0, 1)$. We can show (using the Gaussian tail bound derived in Problem 3(c) in Homework 0) that
\al{
    \Pr\left[|Z - \Exp[Z]| \leq \sd(Z)\sqrt{2 \log (2 / \delta)}\right] \geq 1 - \delta.
    \label{lec3:eqn:normaltailbound}
}
for any $\delta \in (0, 1]$. In other words, the density at the tails of the normal distribution is decreasing at an exponential rate, while Chebyshev's inequality only gives a quadratic rate. The discrepancy between \eqref{lec3:eqn:chebyshevdelta} and \eqref{lec3:eqn:normaltailbound} is made more apparent when we consider inverse-polynomial $\delta = \frac{1}{n^c}$ for some parameter $n$ and degree $c$ (we will see concrete instances of this setup in future chapters). Then the tail bound for the normal distribution \eqref{lec3:eqn:normaltailbound} implies that
\al{
    |Z - \Exp[Z]| \leq \sd(Z) \cdot \sqrt{\log{O\left(n^c\right)}} = \sd(Z) \cdot O\left(\sqrt{\log{n}}\right) \quad w.p. \; 1 - \delta,
}
while Chebyshev's inequality gives us the weaker result
\al{
    |Z - \Exp[Z]| \leq \sd(Z) \cdot \sqrt{O(n^c)} = \sd(Z) \cdot O(n^{c / 2})  \quad w.p. \; 1 - \delta.
}

Despite the previous example, Chebyshev's inequality is actually optimal without further assumptions, in the sense that there exist distributions with finite variance for which the bound is tight. With that in mind, we will need to assume more about our random variables if we want to improve upon the Chebyshev inequality's $1/t^2$ rate of tail decay. As an example, recall that when $0 \leq X_i \leq 1$ for $i = 1, \ldots, n$, Hoeffding's inequality is applicable:
\al{
    \Pr[|Z - \Exp[Z]| \geq t] \leq 2\exp(-2t^2 / n).
}
This tail probability is exponentially decaying in $t$ instead of polynomially decaying as in Chebyshev's inequality! Certainly requiring boundedness in $[0, 1]$ (or $[a, b]$ more generally) is limiting, so it is worth asking what types of distributions permit such an exponential tail bound. The following section will explore such a class of random variables: \emph{sub-Gaussian} random variables.

\sec{Sub-Gaussian random variables}

\tnoteimp{add motivations---we will extend Hoelfding inequality to unbounded random variables} 
We begin by defining the class of sub-Gaussian random variables by way of a bound on their moment generating functions, after which we will see how this bound guarantees the exponential tail decay we are after.

\begin{definition}[Sub-Gaussian Random Variables]
    A random variable $X$ with finite mean $\mu$ is \textit{sub-Gaussian} with parameter $\sigma$ if
    \al{
        \Exp \left[ e^{\lambda(X - \mu)} \right] \leq e^{\sigma^2\lambda^2 / 2}, \quad \forall\lambda\in\R.
        \label{lec3:eqn:subgassdefn}
    }
    We say that $X$ is $\sigma$-sub-Gaussian and say it has \emph{variance proxy} $\sigma^2$.
\end{definition}

\begin{remark}\label{lec3:rem:mgf_strong}
    As it turns out, \eqref{lec3:eqn:subgassdefn} is quite a strong condition, requiring that infinitely many moments of $X$ exist and do not grow too quickly. To see why, assume without loss of generality that $\mu = 0$ and take a power series expansion of the moment generating function:
    \al{
        \Exp[\exp(\lambda X)] = \Exp\left[\sum_{k = 0}^\infty \frac{(\lambda X)^k}{k!}\right] = \sum_{k = 0}^\infty\frac{\lambda^k}{k!}\Exp[X^k].
    }
    A bound on the moment generating function then is a bound on infinitely many moments of $X$, i.e. a requirement that the moments of $X$ are all finite and grow slowly enough to allow the power series to converge.
\end{remark}

\noindent Although \eqref{lec3:eqn:subgassdefn} is not a particularly intuitive definition, it turns out to imply exactly the type of exponential tail bound we want:

\begin{theorem}[Tail bound for sub-Gaussian random variables]\label{lec3:thm:subgausstail}
    If a random variable $X$ with finite mean $\mu$ is $\sigma$-sub-Gaussian, then
    \al{ 
        \Pr[|X - \mu| \geq t] \leq 2 \exp \left( -\frac{t^2}{2\sigma^2} \right), \quad \forall t \in \R.
        \label{lec3:eqn:subgausstail}
    }
\end{theorem}

\begin{proof}
Fix $t > 0$. For any $\lambda > 0$,
\al{
    \Pr[X - \mu \geq t] &= \Pr[\exp(\lambda (X - \mu)) \geq \exp(\lambda t)]  \\
    &\leq \exp(-\lambda t)\Exp[\exp(\lambda (X - \mu))] && \text{(by Markov's inequality)}  \\
    &\leq \exp(-\lambda t)\exp(\sigma^2\lambda^2/2) && \text{(by \eqref{lec3:eqn:subgassdefn})} \\
    &= \exp(-\lambda t + \sigma^2\lambda^2/2). \label{lec3:eqn:non_opt_tail_bound}
}
Because the bound \eqref{lec3:eqn:non_opt_tail_bound} holds for any choice of $\lambda$ and $\exp(\cdot)$ is monotonically increasing, we can optimize the bound \eqref{lec3:eqn:non_opt_tail_bound} by finding $\lambda$ which minimizes the exponent $-\lambda t + \sigma^2 \lambda^2/2$. Differentiating and setting the derivative equal to zero, we find that the optimal choice is $\lambda = t/\sigma^2$, yielding the one-sided tail bound
\al{\label{lec3:eqn:opt_tail_bound_right}
    \Pr[X - \mu \geq t] \leq \exp\left(-\frac{t^2}{2\sigma^2}\right).
}
Going through the same line of reasoning but for $-X$ and $-t$, we can also show that for any $t > 0$,
\al{\label{lec3:eqn:opt_tail_bound_left}
    \Pr[X - \mu \leq -t] \leq \exp\left(-\frac{t^2}{2\sigma^2}\right).
}

We can then obtain \eqref{lec3:eqn:subgausstail} by applying the union bound:
\al{
    \Pr[|X - \mu| \geq t] = \Pr[X - \mu \geq t] + \Pr[X - \mu \leq -t] \leq 2\exp\left(-\frac{t^2}{2\sigma^2}\right).
}
\end{proof}

\begin{remark}[Tail bound implies sub-Gaussianity]\label{lec3:rem:tail_bound_remark}
    In addition to being a necessary condition for sub-Gaussianity (Theorem \ref{lec3:thm:subgausstail}), the tail bound \eqref{lec3:eqn:subgausstail} for sub-Gaussian random variables is also a sufficient condition up to a constant factor. In particular, if a random variable $X$ with finite mean $\mu$ satisfies \eqref{lec3:eqn:subgausstail} for some $\sigma > 0$, then $X$ is $O(\sigma)$-sub-Gaussian. Unfortunately, the proof of this reverse direction is somewhat more involved, so we refer the interested reader to Theorem 2.6 and its proof in Section 2.4 of \cite{wainwright2019high} and Proposition 2.5.2 in \cite{vershynin2018high} for details. While the tail bound is the property we ultimately care about most when studying sub-Gaussian random variables, the definition in \eqref{lec3:eqn:subgassdefn} is a more technically convenient characterization, as we will see in the proof of Theorem \ref{lec3:thm:sum_sub_gaussian}.
\end{remark}

\begin{remark}
    Note that in light of Remark \ref{lec3:rem:mgf_strong}, the tail bound \eqref{lec3:eqn:normaltailbound} requires all central moments of $X$ to exist and not grow too quickly. In contrast, Chebyshev's inequality (and more generally any polynomial variant of Markov's inequality $\Pr[|X-\mu| \geq t] = \Pr[|X-\mu|^k \geq t^k] \leq t^{-k}\E[|X-\mu|^k]$) only requires that the second central moment $\E[(X-\mu)^2]$ (more generally, the $k$th central moment $\E[|X - \mu|^k]$) is finite to yield a tail bound. If infinite moments exist however, it turns out that $\inf_{k \in \mathbb{N}} t^{-k}\E[|X-\mu|^k] \leq \inf_{\lambda > 0} \exp(-\lambda t) \Exp[\exp(\lambda (X-\lambda))]$, i.e. the optimal polynomial tail bound is tighter than the optimal exponential tail bound (see Exercise 2.3 in \cite{wainwright2019high}). As we will see shortly though, using exponential functions of random variables allows us to prove results about sums of random variables more conveniently, which is why most researchers use exponential tail bounds in practice.
\end{remark}

Having defined and derived exponential tail bounds for sub-Gaussian random variables, we can now accomplish the first of the goals we set out at the beginning of the chapter: to show that under certain conditions, namely independence and sub-Gaussianity of $X_1, \dotsc, X_n$, the sum $Z = \sum_{i = 1}^n X_i$ concentrates around $\Exp[Z] = \Exp[\sum_{i = 1}^n X_i]$.

\begin{theorem}[Sum of sub-Gaussian random variables is sub-Gaussian]\label{lec3:thm:sum_sub_gaussian}
    If $X_1, \ldots, X_n$ are independent sub-Gaussian random variables with variance proxies $\sigma_1^2, \ldots, \sigma_n^2$, then $Z = \sum_{i = 1}^n X_i$ is sub-Gaussian with variance proxy $\sum_{i = 1}^n \sigma_i^2$. As a consequence, we have the tail bound
    \al{
        \Pr[|Z - \Exp[Z]| \geq t] \leq 2\exp\left(-\frac{t^2}{2\sum_{i = 1}^n \sigma_i^2}\right),
    }
    for all $t \in \R$.
\end{theorem}

\begin{proof}
Using the independence of $X_1, \ldots, X_n$, we have that for any $\lambda \in \R$:
 \al{
    \Exp \left[ \exp \left\{\lambda(Z - \Exp[Z]) \right\} \right] &= \Exp\left[\prod_{i = 1}^n \exp \left\{\lambda(X_i - \Exp[X_i]) \right\}\right] \\
    &= \prod_{i = 1}^n \Exp \left[ \exp \left\{\lambda(X_i - \Exp[X_i]) \right\} \right] \\
    &\leq \prod_{i = 1}^n \exp \left( \frac{\lambda^2\sigma_i^2}{2} \right) \\
    &= \exp \left( \frac{\lambda^2 \sum_{i = 1}^n\sigma_i^2}{2} \right),
 }
 so $Z$ is sub-Gaussian with variance proxy $\sum_{i = 1}^n \sigma_i^2$. The tail bound then follows immediately from \eqref{lec3:eqn:subgausstail}.
\end{proof}

The proof above demonstrates the value of the moment generating functions of sub-Gaussian random variables: they factorize conveniently when dealing with sums of independent random variables.

\subsec{Examples of sub-Gaussian random variables}

We now provide several examples of classes of random variables that are sub-Gaussian, some of which will appear repeatedly throughout the remainder of the course.

\begin{example}[Rademacher random variables]
    A \textit{Rademacher random variable} $\epsilon$ takes a value of 1 with probability $1/2$ and a value of $-1$ with probability $1/2$. To see that $\epsilon$ is $1$-sub-Gaussian, we follow Example 2.3 in \cite{wainwright2019high} and upper bound the moment generating function of $\epsilon$ by way of a power series expansion of $\exp(\cdot)$:
    \al{
        \Exp[\exp(\lambda \epsilon)] &= \frac{1}{2}\left\{\exp(-\lambda) + \exp(\lambda)\right\} \\
        &= \frac{1}{2}\left\{\sum_{k = 0}^\infty \frac{(-\lambda)^k}{k!} + \sum_{k = 0}^\infty \frac{\lambda^k}{k!}\right\} \\
        &= \sum_{k = 0}^\infty \frac{\lambda^{2k}}{(2k)!} && \text{(for odd $k$, $(-\lambda)^k + \lambda^k = 0$)} \\
        &\leq 1 + \sum_{k = 1}^\infty \frac{\left(\lambda^2\right)^{k}}{2^k k!} && \text{($2^k k!$ is every other term of $(2k)!$)} \\
        & = \exp(\lambda^2/2),
    }
    which is exactly the moment generating function bound \eqref{lec3:eqn:subgassdefn} required for $1$-sub-Gaussianity.
\end{example}

\begin{example}[Random variables with bounded distance to mean]\label{lec3:ex:rand_var_bound_dist_to_mean}
    Suppose a random variable $X$ satisfies $|X - \Exp[X]| \leq M$ almost surely for some constant $M$. Then $X$ is $O(M)$-sub-Gaussian.
\end{example}
We now provide an even more general class of sub-Gaussian random variables that subsume the random variables in Example \ref{lec3:ex:rand_var_bound_dist_to_mean}:
\begin{example}[Bounded random variables]
    If $X$ is a random variable such that $a \leq X \leq b$ almost surely for some constants $a, b \in \R$, then
    \begin{equation*}
        \Exp\left[e^{\lambda(X - \Exp[X])}\right] \leq \exp \left[ \frac{\lambda^2(b - a)^2}{8} \right],
    \end{equation*}
    i.e., $X$ is sub-Gaussian with variance proxy $(b - a)^2/4$. (We will prove this in Question 2(a) of Homework 1.) Note that combining the $(b - a)/2$-sub-Gaussianity of i.i.d. bounded random variables $X_1, \dotsc, X_n$ and Theorem \ref{lec3:thm:sum_sub_gaussian} yields a proof of Hoeffding's inequality.
\end{example}

\begin{example}[Gaussian random variables]
If $X$ is Gaussian with variance $\sigma^2$, then $X$ satisfies \eqref{lec3:eqn:subgassdefn} and \eqref{lec3:eqn:subgausstail} with equality. In this special case, the variance and the variance proxy are the same.
\end{example}

\sec{Concentrations of functions of random variables}
We now introduce some important inequalities related to the second of our two goals, namely showing that for independent $X_1, \dotsc, X_n$ and certain functions $f$, $f(X_1, \dotsc, X_n)$ concentrates around $\Exp[f(X_1, \dotsc, X_n)]$.

\begin{theorem}[McDiarmid's inequality]
    Suppose $f : \R^n \to \R$ satisfies the \emph{bounded difference condition}: there exist constants $c_1, \ldots, c_n \in \R$ such that for all real numbers $x_1, \ldots, x_n$ and $x_i'$,
    \al{\label{lec3:eqn:mcdiarmid_fn_cond}
        |f(x_1, \ldots, x_n) - f(x_1, \ldots, x_{i - 1}, x_i', x_{i + 1}, \ldots, x_n)| \leq c_i.
    }
    (Intuitively, \eqref{lec3:eqn:mcdiarmid_fn_cond} states that $f$ is not overly sensitive to arbitrary changes in a single coordinate.) Then, for any independent random variables $X_1, \ldots, X_n$,
    \al{
        \Pr \left[ f(X_1, \ldots, X_n) - \Exp[f(X_1, \ldots, X_n)] \geq t \right] \leq \exp\left(-\frac{2t^2}{\sum_{i = 1}^n c_i^2}\right).
    }
    Moreover, $f(X_1, \ldots, X_n)$ is $O\left(\sqrt{\sum_{i = 1}^n c_i^2}\right)$-sub-Gaussian.
\end{theorem}

\begin{remark}
    Note that McDiarmid's inequality is a generalization of Hoeffding's inequality with $f(x_1, \dotsc, x_n) = \sum_{i = 1}^n \min\{\max\{x_i, b\}, a\}$.
\end{remark}

\begin{proof}
    See the proof of Corollary 2.21 in \cite{wainwright2019high}, which relies on the Azuma-Hoeffding inequality for martingale difference sequences.
\end{proof}

A more general version of McDiarmid's inequality comes from Theorem 3.18 in~\cite{vanhandel2016high}. The setup for this theorem requires defining the \emph{one-sided differences} of a function $f : \R^n \to \R$:
\al{
    D_i^-{f(x)} &= f(x_1, \ldots, x_n) - \inf_z f(x_1, \ldots, x_{i - 1}, z, x_{i + 1}, \ldots, x_n) \\
    D_i^+{f(x)} &= \sup_z f(x_1, \ldots, x_{i - 1}, z, x_{i + 1}, \ldots, x_n) - f(x_1, \ldots, x_n).
}
These two quantities are functions of $x \in \R^n$, and hence can be interpreted as describing the sensitivity of $f$ \emph{at a particular point}. (Contrast this with the bounded difference condition \eqref{lec3:eqn:mcdiarmid_fn_cond}, which bounds the sensitivity of $f$ universally over all points.) For convenience, define
\al{
    d^+ &= \Norm{\sum_{i = 1}^n |D_i^+{f}|^2}_\infty = \sup_{x_1, \ldots, x_n}\sum_{i = 1}^n[|D_i^+{f(x_1, \ldots, x_n)}]^2 \\
    d^- &= \Norm{\sum_{i = 1}^n |D_i^-{f}|^2}_\infty = \sup_{x_1, \ldots, x_n}\sum_{i = 1}^n [D_i^-{f(x_1, \ldots, x_n)}]^2.
}
\begin{theorem}[Bounded difference inequality, Theorem 3.18 in~\cite{vanhandel2016high}]
    Let $f : \R^n \to \R$, and let $X_1, \ldots, X_n$ be independent random variables. Then, for all $t \geq 0$,
    \al{
        \Pr[f(X_1, \ldots, X_n) \geq \Exp[f(X_1, \ldots, X_n)] + t] &\leq \exp\left(-\frac{t^2}{4d^-}\right) \\
        \Pr[f(X_1, \ldots, X_n) \leq \Exp[f(X_1, \ldots, X_n)] - t] &\leq \exp\left(-\frac{t^2}{4d^+}\right).
    }
\end{theorem}

\subsec{Bounds for Gaussian random variables}
Unfortunately, the bounded difference condition (\ref{lec3:eqn:mcdiarmid_fn_cond}) is often only satisfied by bounded random variables or a bounded function. To get similar concentration inequalities for unbounded random variables, we need some other special conditions. The following inequalities assume that the random variables have the standard normal distribution.

\begin{theorem}[Gaussian Poincar\'{e} inequality, Corollary 2.27 in~\cite{vanhandel2016high}]
    Let $f : \R^n \to \R$ be smooth. If $X_1, \ldots, X_n$ are independently sampled from $\cN(0, 1)$, then
    \al{
        \Var(f(X_1, \ldots, X_n)) \leq \Exp \left[ \norm{\nabla{f}(X_1, \ldots, X_n)}_2^2 \right].
    }
\end{theorem}

Before introducing the next theorem, we recall that a function $f : \R^n \to \R$ is \emph{$L$-Lipschitz} with respect to the $\ell_2$-norm if there exists a non-negative constant $L \in \R$ such that for all $x, y \in \R^n$,
\al{
    |f(x) - f(y)| \leq L\norm{x - y}_2.
}
We emphasize that $L$ is universal for all points in $\R^n$.

\begin{theorem}[Theorem 2.26 in~\cite{wainwright2019high}]
    Suppose $f : \R^n \to \R$ is $L$-Lipschitz with respect to Euclidean distance, and let $X = (X_1, \ldots, X_n)$, where $X_1, \ldots, X_n \iid \cN(0, 1)$. Then for all $t \in \R$,
    \al{
        \Pr[|f(X) - \Exp[f(X)]| \geq t] \leq 2\exp\left(-\frac{t^2}{2L^2}\right).
    }
In particular, $f(X)$ is sub-Gaussian.
\end{theorem}


	\chapter{Generalization Bounds via Uniform Convergence}\label{chap:uc}
	% reset section counter
\setcounter{section}{0}

%\metadata{lecture ID}{Your names}{date}
\metadata{4}{Yizhou Qian}{Jan 25th, 2021}

In Chapter \ref{chap:asymp}, we pointed out some limitations of asymptotic analysis. In this chapter, we will turn our focus to \textit{non-asymptotic analysis}, where we provide convergence guarantees without having the number of observations $n$ go off to infinity. A key tool for proving such guarantees is \textit{uniform convergence}, where we have bounds of the following form:
\al{
 \Pr \left[ \sup_{h \in \cH} \vert \hat L(h) - L(h) \vert \leq \epsilon \right] \geq 1 - \delta.
\label{lec4:eqn:uniformconvergence}
}
In other words, the probability that the difference between our empirical loss and population loss is larger than $\epsilon$ is at most $\delta$. We give motivation for uniform convergence and show how it can give us non-asymptotic guarantees on excess risk.

\sec{Basic concepts}

A central goal of learning theory is to bound the \emph{excess risk} $L(\hat{\theta}) - L(\theta^*)$. This is important as we don't want the expected risk of our ERM to be much larger than the expected risk of the best possible model. As we will see in the remainder of this section, uniform convergence is a technique that helps us achieve such bounds.

Uniform convergence is a property of a parameter set $\Theta$, which gives us bounds of the form
\al{
    \Pr \left[|\hat{L}(\theta) - L(\theta)| \geq \varepsilon \right] \leq \delta; \; \forall \theta \in \Theta.\label{lec2:eqn:uc}
}
In other words, uniform convergence tells us that for any choice of $\theta$, our empirical risk is always close to our population risk with high probability. Let's look at a motivating example for why this type of bound is useful.

\subsec{Motivation: Uniform convergence implies generalization}\label{sec:uc-gen}

Consider the standard supervised learning setup where we have some i.i.d. $(x\sp{i}, y\sp{i})$. Furthermore, assume that we have a bounded loss function; specifically, suppose that $0 \leq \ell((x, y); \theta) \leq 1$, as in the case of the zero-one loss function. We show that uniform convergence implies generalization.

First, via telescoping sums, we can decompose the excess risk into three terms:
\al{
    L(\hat{\theta}) - L(\theta^*) = \underbrace{L(\hat{\theta}) - \hat{L}(\hat{\theta})}_\text{\circled{1}} + \underbrace{\hat{L}(\hat{\theta}) - \hat{L}(\theta^*)}_\text{\circled{2}} + \underbrace{\hat{L}(\theta^*) - L(\theta^*)}_\text{\circled{3}}.
}
We know that $\hat{L}(\hat{\theta}) - \hat{L}(\theta^*) \leq 0$ since $\hat{\theta}$ is a minimizer of $\hat{L}$. This allows us to write
\begin{align}
L(\hat{\theta}) - L(\theta^*) &\leq |L(\hat{\theta}) - \hat{L}(\hat{\theta})| + \hat{L}(\hat{\theta}) - \hat{L}(\theta^*) + |\hat{L}(\theta^*) - L(\theta^*)|\\
&\leq |L(\hat{\theta}) - \hat{L}(\hat{\theta})| + 0 + |\hat{L}(\theta^*) - L(\theta^*)|\\
&\leq 2\;\sup_{\theta \in \Theta } |L(\theta) - \hat{L}(\theta)|. \label{lec2:eqn:1}
\end{align}
This result tells us that if $\sup_{\theta \in \Theta } |L(\theta) - \hat{L}(\theta)|$ is small (say, less than $\varepsilon/2$), then excess risk $L(\hat{\theta}) - L(\theta^*)$ is less than $\varepsilon$. But this is exactly in the form of the bound in \eqref{lec2:eqn:uc}. Hence, if we can show that a parameter family exhibits uniform convergence, we can get a bound on excess risk as well.

For future references, Equation~\eqref{lec2:eqn:1} can be strengthened straightforwardly into the following with slightly more careful treatment of the signs of each term:
\begin{align}
L(\hat{\theta}) - L(\theta^*) \le |\hat{L}(\theta^*) - L(\theta^*)|+  L(\hat{\theta}) - \hat{L}(\hat{\theta})  \le |\hat{L}(\theta^*) - L(\theta^*)|+ \sup_{\theta \in \Theta} \left(L(\hat{\theta}) - \hat{L}(\hat{\theta})\right)\label{lec2:eqn:2}
\end{align}
This will make some of our future derivations technically slightly more convenient, but the nuanced difference between Equations~\eqref{lec2:eqn:1} and~\eqref{lec2:eqn:2} does not change the fundamental idea and the discussions in this chapter. 

Let us try to apply our knowledge of concentration inequalities to this problem. Earlier we assumed that $\ell((x, y); \theta)$ is bounded, so we can bound $\circled{3}$ \todo{replace $\circled{3}$ etc by something that looks more aesthetic} 
by $\tilO\left(\frac{1}{\sqrt{n}}\right)$ via Hoeffding's inequality (Remark \ref{lec2:rem:hoeffding}). However, we cannot apply the same concentration inequality to $\circled{1}$: since $\hat{\theta}$ is data-dependent by definition, the i.i.d. assumption no longer holds. (To see this, note that $\hat\theta$ depends on the training dataset $(x\sp{i}, y\sp{i})$, so the terms in $\hatL (\theta)$, $\ell ((x\sp{i}, y\sp{i}); \hat\theta)$, all depend on the training dataset too.) This is concerning: it is certainly possible that $L(\hat{\theta}) - \hat{L}(\hat{\theta})$ is large. You've probably encountered this yourself when a model exhibits low training loss, but high validation/testing loss. 

\subsec{Deriving uniform convergence bounds}

Uniform convergence is one way we can fix this issue. The high-level idea is as follows: 
\begin{itemize}
    \item Suppose we have a bound of the form $\Pr[|\hat{L}(\theta) - L(\theta)| \geq \varepsilon'] \leq \delta'$ for some single, fixed choice of $\theta$.
    \item If we know \emph{all possible values of $\theta$} in advance, we can use the above bound to create a more general bound over all values of $\theta$.
\end{itemize}
In particular, we can use the union-bound inequality to create the general bound described in the second bullet point, using the bound in the first bullet point:
\al{
    \Pr \left[\forall \theta \in \Theta, |\hat{L}(\theta) - L(\theta)| \geq \varepsilon' \right] \leq \sum_{\theta \in \Theta} \Pr \left[|\hat{L}(\theta) - L(\theta)| \geq \varepsilon' \right].
}
We can then use Hoeffding's inequality to deal with the summands as $\theta$ there is no longer data-dependent. We will talk more later about proving statements of this form.

\subsec{Intuitive interpretation of uniform convergence}

Since uniform convergence implies generalization, if we know that population risk and empirical risk are always ``close," then excess risk is ``small" as well (Figure \ref{lec2:fig:uc}). In fact, it is possible to show that not only is $L(\theta)$ ``close" to $\hat{L}(\theta)$ for sufficiently large data, but that the ``shape" of $\hat{L}$ is ``close" to the shape of $L$ as well (Figure \ref{lec2:fig:shape}). This holds for the convex case; furthermore, there are conditions under which this holds in the non-convex case, for which a rigorous treatment can be found in~\cite{mei2017landscape}. (\emph{Figure design and some wording in this section was inspired by~\cite{percynotes, thomasliu2018}.})

\begin{figure}[t]
    \centering
    \begin{tikzpicture}
    \draw[help lines, color=gray!30, dashed] (-4.9,-4.9) grid (4.9,4.9);
    \draw[->,ultra thick] (-5,0)--(5,0) node[right]{$\theta$};
    \draw[->,ultra thick] (0,-5)--(0,5) node[above]{$L$};
    \draw[blue, thick]   plot[smooth,domain=-5:5] (\x, {0.1* (\x*\x)});
    \draw[red, dashed]   plot[smooth,domain=-5:5] (\x, {0.1* (\x*\x) + 1});
    \draw[red, dashed]   plot[smooth,domain=-5:5] (\x, {0.1* (\x*\x) - 1});
    \draw[green, thick]   plot[smooth,domain=-5:5] (\x, {0.1* (\x*\x) + 0.8 * sin(3000 * \x)});
\end{tikzpicture}
    \caption{Empirical risk landscape under uniform convergence: \textcolor{green}{Green:} empirical risk, \textcolor{blue}{blue:} population risk, \textcolor{red}{red, dashed:} $\varepsilon$ additive error bounds for excess risk.
    \todo{make the figure slightly smaller, and have a more self-contained caption}
    }
    \label{lec2:fig:uc}
\end{figure}

\begin{figure}[t]
    \centering
    \begin{tikzpicture}
    \draw[help lines, color=gray!30, dashed] (-4.9,-4.9) grid (4.9,4.9);
    \draw[->,ultra thick] (-5,0)--(5,0) node[right]{$\theta$};
    \draw[->,ultra thick] (0,-5)--(0,5) node[above]{$L$};
    \draw[black, thick]   plot[smooth,domain=-5:5] (\x, {0.01*\x*\x*\x*\x - 0.3*\x*\x + 2});
    \draw[blue, thick]   plot[smooth,domain=-5:5] (\x, {0.011*\x*\x*\x*\x -0.003*\x*\x*\x - 0.32*\x*\x + 2.15});
\end{tikzpicture}
    \caption{Empirical risk landscape under uniform convergence: \textcolor{blue}{Blue:} empirical risk, \textcolor{black}{black:} population risk.\todo{a bit more self-contained caption, and perhaps merge with the previous figure (making them both subfigures)}
    }
    \label{lec2:fig:shape}
\end{figure}

\sec{Finite hypothesis class}

In this section, assume that $\cH$ is finite. The following theorem gives a bound for the excess risk $L(\hat{h}) - L(h^{*})$, where $\hat{h}$ and $h^*$ are the minimizers of the empirical loss and population loss respectively.

\begin{theorem}\label{lec4:thm:finite}
Suppose that our hypothesis class $\cH$ is finite and that our loss function $\ell$ is bounded in $[0,1]$, i.e. $0 \leq \ell((x, y), h) \leq 1$. Then $\forall \delta \  s.t. \  0 < \delta < \frac{1}{2}$ , with probability at least $1 - \delta$, we have 
\al {
\vert L(h) - \hat{L}(h) \vert \leq \sqrt{\frac{\ln{\vert \cH \vert} + \ln{(2 / \delta)}}{2n}} \qquad \forall h \in \cH.
\label{lec4:eqn:finiteuniformbound}
}
As a corollary, we also have 
\al {
L(\hat{h}) - L(h^{*}) \leq \sqrt{\frac{ 2(\ln{\vert \cH \vert} + \ln{(2 / \delta)}) }{n}}.
\label{lec4:eqn:finiteexcessriskbound}
}
\end{theorem}

\begin{proof}
We will prove this in two steps:
\begin{enumerate}
\item Use concentration inequalities to prove the bound for a fixed $h \in \cH$, then
\item Use a union bound across the $h$'s. (Recall that if $E_1, \dots, E_k$ are a finite set of events, then the union bound states that $\Pr ( E_1 \cup \dots \cup E_k) \leq \sum_{i = 1}^k \Pr(E_i)$.)
\end{enumerate}

Fix some $\epsilon > 0$. By applying Hoeffding's inequality on the $\ell( (x\sp{i}, y\sp{i}), h)$, we know that 

\al{
\Pr \left( \vert \hat{L}(h) - L(h) \vert \geq \epsilon \right) &\leq 2\exp\left(-\frac{2n^2\epsilon^2}{\sum_{i = 1}^n(b_i - a_i)^2}\right) \\
&= 2\exp\left(-\frac{2n^2\epsilon^2}{n}\right) \\
&= 2\exp(-2n\epsilon^2),
\label{lec4:eqn:boundedconcentration}
}
since we can set $a_i = 0, b_i = 1$. The bound above holds for a single fixed $h$. To prove a similar inequality that holds for all $h \in \cH$, we apply the union bound with $E_h = \{\vert \hat{L}(h) - L(h) \vert \geq \epsilon \}$:

\al {
\Pr \left( \exists h \text{ s.t. } \vert \hat{L}(h) - L(h) \vert \geq \epsilon \right) &\leq \sum_{h \in \cH} \Pr \left(\vert \hat{L}(h) - L(h) \vert \geq \epsilon \right) \\
&\leq \sum_{h \in \cH} 2\exp(-2n\epsilon^2) \\
&= 2\vert \cH \vert \exp(-2n\epsilon^2). 
\label{lec4:eqn:unionboundforh}
}
If we take $\delta$ such that $2\vert \cH \vert \exp(-2n\epsilon^2) = \delta$, then it follows that 
\al {
\epsilon = \sqrt{\frac{\ln{\vert \cH \vert} + \ln{(2 / \delta)}}{2n}},
\label{lec4:eqn:probabilitytoerror}
}
which proves \eqref{lec4:eqn:finiteuniformbound}. \eqref{lec4:eqn:finiteexcessriskbound} follows by the inequality we stated in Section \ref{sec:uc-gen} and taking $\epsilon = \sqrt{\frac{ 2(\ln{\vert \cH \vert} + \ln{(2 / \delta)}) }{n}}$:
\begin{align}
\Pr \left( | L(\hat{h}) - L(h^{*}) | \geq \epsilon \right) &\leq \Pr \left( 2 \sup_{h \in \cH} | L(h) - \hat{L}(h) | \geq \epsilon \right) \\
&\leq 2 |\cH| \exp \left( -\frac{n\epsilon^2}{2} \right).
\end{align}
\end{proof}

\subsec{Comparing Theorem \ref{lec4:thm:finite} with standard concentration inequalities}
With standard concentration inequalities, we have the following bound that depends on empirical risk:
\al{
\forall h \in \cH, \quad w.h.p. \quad \vert \hat{L}(h) - L(h) \vert \leq \tilde{O} \left( \frac{1}{\sqrt{n}} \right).
\label{lec4:eqn:centrallimitconvergence}
}
The bound here depends on each $h$. In contrast, the uniform convergence bound we obtain from \eqref{lec4:eqn:probabilitytoerror} is uniform over all $h \in \cH$:
\al{
w.h.p., \quad \forall h \in \cH, \quad \vert \hat{L}(h) - L(h) \vert \leq \tilde{O} \left( \frac{ \ln |\cH|}{\sqrt{n}} \right),
}
if we omit the $\ln{(1/\delta)}$ factor (we can do this since $\ln{(1/\delta)}$ is small in general and we take $\delta = \frac{1}{poly(n)}$). Hence, the extra $\ln{\vert \cH \vert}$ term that depends on the size of our finite hypothesis family $\cH$ can be viewed as a trade-off in order to make the bound uniform.

\begin{remark}
There is no standard definition for the term \textit{with high probability} (\textit{w.h.p}). For this class, the term is equivalent to the condition that the probability is higher than $1 - n^{-c}$ for some constant $c$.
\end{remark}

\subsec{Comparing Theorem \ref{lec4:thm:finite} with asymptotic bounds}
We can also compare the bound in Theorem \ref{lec4:thm:finite} with our original asymptotic bound, namely,
\al{
L(\hat{h}) - L(h^*) \leq \frac{c}{n} + o \left(n^{-1} \right).
\label{lec4:eqn:asymptotics}
}
The $o(n^{-1})$ term can vary significantly depending on the problem. For instance, both $n^{-2}$ and $p^{100}n^{-2}$ are $o(n^{-1})$ but the second one converges much more slowly. With the new bound, there are no longer any constants hidden in an $o(n^{-1})$ term (in fact that term is no longer there). However, we now have a slower convergence rate of $O(n^{-1/2})$.

\begin{remark}
$O(n^{-1/2})$ convergence is sometimes known as the \textit{slow rate} while $O(n^{-1})$ convergence is known as the \textit{fast rate}. We were only able to get the slow rate from uniform convergence: we needed asymptotics to get the fast rate. (It is possible to get the fast rate from uniform convergence under certain conditions, e.g. when the population risk on the true $h^*$ is very low.)
\end{remark}

\sec{Bounds for infinite hypothesis class via discretization}
Unfortunately, we cannot generalize the results from the previous section directly to the case where the hypothesis class $\cH$ is infinite, since we cannot apply the union bound to an infinite number of hypothesis functions $h \in \cH$. However, if we consider a \emph{bounded} and \emph{continuous} parameterized space of $\cH$, then we can obtain a similar uniform bound by applying a technique called \emph{brute-force discretization}.

For this section, assume that our infinite hypothesis class $\cH$ can be parameterized by $\theta \in \mathbb{R}^p$ with $\Vert \theta \Vert_2 \leq B$ for some fixed $B > 0$. That is, we have 
\al{
\cH = \{h_{\theta} : \theta \in \mathbb{R}, \Vert \theta \Vert_2 \leq B \}.
\label{lec4:eqn:infiniteclass}
}

The intuition behind brute-force discretization is as follows: Let $E_\theta = \{ |\hatL(\theta) - L(\theta)| \geq \epsilon \}$ be the ``bad" event. We want the bound the probability of any one of these bad events happening (i.e. $\bigcup_\theta E_\theta$). The union bound does not work as we end up with an infinite sum. However, the union bound is very loose: these events can overlap with each other significantly. Instead, we can try to find ``prototypical" bad events $E_{\theta_1}, \dots, E_{\theta_N}$ that are somewhat disjoint so that $\bigcup_\theta E_\theta \approx \bigcup_{i=1}^N E_{\theta_i}$. We can then use the union bound on $\bigcup_{i=1}^N E_{\theta_i}$ to get a non-vacuous upper bound.

We make these ideas precise in the following section.

\subsec{Discretization of the parameter space by $\epsilon$-covers}

We start by defining the notion of an \emph{$\epsilon$-cover} (also \textit{$\epsilon$-net}):

\begin{definition}[$\epsilon$-cover]
Let $\epsilon>0$. An \emph{$\epsilon$-cover} of a set $S$ with respect to distance metric $\rho$ is a subset $C \subseteq S$ such that $\forall x \in S$, $\exists x' \in C$ such that $\rho(x,x') \le \epsilon$, or equivalently,
\begin{align}
S &\subseteq \bigcup_{x \in C} \mathrm{Ball}(x, \epsilon, \rho), \quad \text{where} \\
\mathrm{Ball}(x, \epsilon, \rho) &\triangleq \{ x': \rho(x, x') \leq \epsilon \}.
\end{align}
\end{definition}

(We note that in some definitions it is possible for points in $C$ to lie outside of $S$; we do not worry about this technicality the class.) The following lemma tells us that our parameter space $S = \{\theta \in \R^p: \|\theta\|_2 \le B\}$ has an $\epsilon$-cover with not too many elements:

\begin{lemma}[$\epsilon$-cover of $\ell_2$ ball]\label{lec4:lem:ECSize}
Let $B,\epsilon>0$ with $\epsilon \le B \sqrt{p}$, and let $S = \{x \in \R^p: \|x\|_2 \le B\}.$ Then there exists an $\epsilon$-cover of $S$ with respect to the $\ell_2$-norm with at most $\left(\frac{3B\sqrt{p}}{\epsilon}\right)^p$ elements.
\end{lemma}

\begin{proof}
	\tnoteimp{I think we should remove the condition ``$\epsilon \le B \sqrt{p}$'' in the lemma, and remark here that if $\epsilon > B \sqrt{p}$, even though the set $C$ contains only 1 point (the origin), the proof still works}
Set
\begin{equation}
C = \left\{ x \in S: x_i = k_i \frac{\epsilon}{\sqrt{p}}, k_i \in \mathbb{Z}, |k_i| \leq  \frac{B\sqrt{p}}{\epsilon}  \right\},
\end{equation}
i.e. $C$ is the set of grid points in $\R^p$ of width $\tfrac{\epsilon}{\sqrt{p}}$ that are contained in $S$. See Figure \ref{lec5:fig:ecover} for an illustration. 
\begin{figure}[ht]
\centerline{\includegraphics[width=3in]{figures/ECover.png}}
\caption[lec5:fig:ecover]{Our chosen $\epsilon$-cover (shown in red) of $S$. For $x \in S$, we choose the grid point $x'$ such that $\norm{x-x'}_2 \le \epsilon$. \tnoteimp{mention which lemma this figure is for}}
\label{lec5:fig:ecover}
\end{figure}

We claim that $C$ is an $\epsilon$-cover of $S$ with respect to the $\ell_2$-norm: $\forall x \in S$, there exists a grid point $x' \in C$ such that $|x_i-x_i'| \le \tfrac{\epsilon}{\sqrt{p}}$ for each $i$. Therefore,
$$\norm{x-x'}_2 = \sqrt{\sum_{i = 1}^p |x_i - x_i'|^2} \leq \sqrt{p\cdot \frac{\epsilon^2}{p}} = \epsilon.$$

We now bound the size of $C$. Since each $k_i$ in the definition of $C$ has at most $2\tfrac{B\sqrt{p}}{\epsilon}+1$ choices, we have 
\begin{equation}
|C| \le \left( \frac{2B\sqrt{p}}{\epsilon} +1\right)^p \le \left(\frac{3B\sqrt{p}}{\epsilon}\right)^p.
\end{equation}
\end{proof}

\begin{remark}
If $\epsilon > B\sqrt{p}$, then $S$ is contained in the ball centered at the origin with radius $\epsilon$ and the $\epsilon$-cover has size 1.
\end{remark}

\begin{remark}\label{lec4:rem:enet}
We can actually prove a stronger version of Lemma \ref{lec4:lem:ECSize}: there exists an $\epsilon$-cover of $S$ with at most $\left(\frac{3B}{\epsilon}\right)^p$ elements. We will be using this version of the lemma in the proof below. (We will leave the proof of this stronger version as a homework exercise.)
\end{remark}

\subsec{Uniform convergence bound for infinite $\cH$}

\begin{definition}[$\kappa$-Lipschitz functions]
Let $\kappa \ge 0$ and $\norm{\cdot}$ be a norm on the domain $D$. A function $L:D \to \R$ is said to be \emph{$\kappa$-Lipschitz} with respect to $\norm{\cdot}$ if for all $\theta, \theta' \in D$, we have
$$
    |L(\theta)-L(\theta')| \le \kappa \norm{\theta-\theta'}.
$$
\end{definition}

Assume that our infinite hypothesis class $\cH$ can be parameterized by $\cH = \{h_{\theta} : \theta \in \mathbb{R}, \Vert \theta \Vert_2 \leq B\}$. We have the following uniform convergence theorem for our infinite hypothesis class $\cH$:

\begin{theorem}\label{lec4:thm:main}
Suppose $\ell((x,y), \theta) \in [0,1]$, and $\ell((x,y), \theta)$ is $\kappa$-Lipschitz in $\theta$ with respect to the $\ell_2$-norm for all $(x, y)$. Then, with probability  at least $1-O(\exp(-\Omega(p)))$, we have
\begin{equation}
    \forall \theta, \quad |\hat L(\theta)- L(\theta)| \leq  O\left(\sqrt{\frac{p \max(\ln{(\kappa Bn), 1)}}{n}}\right).
\end{equation}
\end{theorem}

\begin{proof}[Proof of Theorem \ref{lec4:thm:main}]
Fix parameters $\delta, \epsilon>0$ (we will specify their values later). Let $C$ be the $\epsilon$-cover of our parameter space $S$ with respect to the $\ell_2$-norm constructed in Lemma \ref{lec4:lem:ECSize}. Define event $E = \left\{ \forall \theta \in C, \; |\hat L(\theta) - L(\theta)| \le \delta \right\}$. By Theorem 4.1, we have $\Pr (E) \ge 1 - 2|C|\exp(-2n\delta^2)$.

Now for any $\theta \in S$, we can pick some $\theta_0 \in C$ such that $\norm{\theta-\theta_0}_2 \le \epsilon$. Since $L$ and $\hatL$ are $\kappa$-Lipschitz functions (this follows from the Lipschitzness of $\ell$), we have
\begin{align}
|L(\theta) - L(\theta_0)| &\le \kappa \norm{\theta-\theta_0}_2 \le \kappa \epsilon, \text{ and} \\
|\hat L(\theta) - \hat L(\theta_0)| &\le \kappa \norm{\theta-\theta_0}_2 \le \kappa \epsilon.
\end{align}

Therefore, conditional on $E$, we have
\begin{equation}
    |\hat L(\theta) -  L(\theta)| \le |\hat L(\theta)-\hat L(\theta_0)| + |\hat L(\theta_0) -  L(\theta_0)| + | L(\theta_0) - L(\theta)| \le 2 \kappa\epsilon+\delta.
\end{equation}

It remains to choose suitable parameters $\delta$ and $\epsilon$ to get the desired bound in Theorem \ref{lec4:thm:main} while making the failure probability small. First, set $\epsilon = \delta / (2 \kappa)$ so that conditional on $E$,
\begin{equation} \label{lec4:eqn:triangle}
    |\hat L(\theta) -  L(\theta)| \le 2\delta.
\end{equation}

If we set $\delta = \sqrt{\frac{c_0 p \max(1, \ln{(\kappa Bn)})}{n}}$ with $c_0 = 36$ (see Remark \ref{lec4:rem:delta} for some intuition), then by Remark \ref{lec4:rem:enet},
\begin{align}
\ln{\vert C \vert} - 2n\delta^2 &\leq p \ln\left(\frac{6B \kappa}{\delta}\right) - 2n\delta^2 \\
&\leq p \ln\left(\frac{6B\kappa \sqrt{n}}{ \sqrt{c_0 p \max(1, \ln{(\kappa Bn)})} }\right) - 2n \frac{c_0p}{n} \ln(\kappa Bn) &(\text{dfn of } \delta)  \\
&\leq p\ln\left(\frac{B\kappa \sqrt{n}}{\sqrt{p}}\right) - 72 p \ln(\kappa Bn) &(\max(1, \ln{(\kappa Bn)}) \geq 1, c_0 = 36) \\
&\leq p \ln(B\kappa n) - 72 p \ln(B\kappa n) &(\sqrt{n/p} \leq n) \\
&\leq -p,
\end{align}
since $\ln (B\kappa n) \geq 1$ for large enough $n$. Therefore, with probability greater than $1 - 2|C| \exp(-2n\delta^2) = 1 - 2 \exp(\ln{|C|} - 2n\delta^2) \geq 1 - O(e^{-p})$, we have
\al{
\vert \hat L(\theta) - L(\theta) \vert \leq 2\delta = O\left(\sqrt{\frac{p}{n}\max(1,\ln(\kappa Bn))}\right).
}
\end{proof}

\begin{remark}\label{lec4:rem:delta}
	\tnoteimp{I suggest we move this remark to the proofs (so that it should up before the actual proof which is harder to understand.)}
\sloppy Here is the intuition for the choice of $\delta$: The event $E$ happens with probability $1 - 2|C|\exp(-2n\delta^2) = 1 - 2 \exp(\ln{|C|} - 2n\delta^2)$. From Remark \ref{lec4:rem:enet}, we know that $\ln{|C|} \leq p \ln{ (3B / (\delta / 2)) }$. If we ignore the log term and assume $\ln{|c|} \leq p$, then this would give us the high probability bound we want:
\al{
   2|C| \exp(-2n\delta^2)  = 2\exp(\ln{\vert C \vert} - 2n\delta^2) \leq 2\exp(p - 2p) = 2\exp(-p).
}

(At the same time, we see from \eqref{lec4:eqn:triangle} that this choice of $\delta$ gives $|\hat L(\theta)- L(\theta)| \le 2 \sqrt{\frac{p}{n}}$, which is roughly the bound we want.)

Since we cannot drop the log term in the inequality, we need to make $\delta$ a little big bigger. $\delta$ in the proof was chosen with this intuition in mind to make the subsequent chain of logic work.
\end{remark}

\begin{remark}
We bounded the generalization error $\vert \hat L(\theta) - L(\theta) \vert$ by $\delta + 2\epsilon \kappa \leq \sqrt{\frac{\ln{\vert C \vert}}{n}} + 2\epsilon \kappa$. The term $2\epsilon \kappa$ represents the error from our brute-force discretization. It is not a problem is because we can always choose $\epsilon$ small enough without worrying about the growth of the first term $\sqrt{\frac{\ln{\vert C \vert}}{n}}$. This in turn is because $\ln{\vert C \vert} \approx p\ln{\epsilon^{-1}}$, which is very insensitive to $\epsilon$, even if we let $\epsilon = \frac{1}{poly(n)}$. We also observe that both $\sqrt{\frac{\ln{\vert C \vert}}{n}}$ and $\sqrt{\frac{p}{n}}$ are bounds that depend on the ``size" of our hypothesis class, in terms of either its total size or dimensionality. This possibly explains why one may need more training samples when the hypothesis class is larger.
\end{remark}

	% reset section counter
%\setcounter{section}{0}

%\metadata{lecture ID}{Your names}{date}
\metadata{5}{Will Song}{Jan 27th, 2021}

\sec{Rademacher complexity}

\subsec{Motivation for a new complexity measure}

Recall that our goal is to bound the \textit{excess risk} $L(\hat{h}) - L(h^*)$, where $L$ is the expected loss (or population loss), $\hat{h}$ is our estimated hypothesis and $h^*$ is the hypothesis in the hypothesis class $\cH$ which minimizes the expected loss. We previously showed that to do so, it suffices to upper bound $\sup_{h\in \cH} (L (h) - \hatL(h))$. (Note: we often call $L(\hat{h}) - \hatL(\hat{h})$ the \textit{generalization gap} or \textit{generalization error}.)

In the previous sections, we derived bounds for the generalization gap in two cases:
\begin{enumerate}
	\item If the hypothesis class $\cH$ is finite,
	\begin{equation}\label{lec5:eqn:bound-finite}
	L(\hat h) - \hat L(\hat h) \leq \tilde O \l( \sqrt{\frac{\log |\cH|}{n}} \r).
	\end{equation}
	\item If the hypothesis class $\cH$ is $p$-dimensional,
	\begin{equation}\label{lec5:eqn:bound-p}
	L(\hat h) - \hat L(\hat h) \leq \tilde O \l( \sqrt{\frac{p}{n}} \r).
	\end{equation}
\end{enumerate} 
Both of these bounds have a $\frac{1}{\sqrt{n}}$-dependency on $n$, which is known as the ``slow rate". The terms in the numerator ($\log |\cH|$ and $p$ resp.) can be thought of as complexity measures of $\cH$.

The bound \eqref{lec5:eqn:bound-p} is not precise enough: it depends solely on $p$ and is not always optimal. For example, this would be a poor bound if the hypothesis class $\cH$ has very high dimension but small norm. One specific example is for the following two hypothesis classes:
$$ \{\theta : \|\theta\|_1 \leq B\} \qquad \text{vs.} \qquad \{\theta : \|\theta\|_2 \leq B\},$$
\eqref{lec5:eqn:bound-p} would give both hypothesis classes the same bound of $\tilde O \l( \sqrt{\frac{p}{n}} \r)$. Intuitively, we should take into account the norms to prove a better bound.

With the complexity measure to be introduced, we will prove a bound of the form
\begin{align}
    L(\hat h) - \hat L(\hat h) \leq \tilde O\l(\sqrt{\frac{\text{Complexity}(\Theta)}{n}}\r).
\end{align}

This complexity measure will depend on the distribution $P$ over $\cX \times \cY$ (the input and output spaces), and hence takes into account how easy it is to learn $P$. If $P$ is easy to learn, then this complexity measure will be small even if the hypothesis space is big.

One of the practical implications of having such a complexity measure is that we can restrict the hypothesis space by regularizing the complexity measure (assuming it is something we can evaluate and train with). If we successfully find a low complexity model, then this generalization bound guarantees that we have not overfit.

\subsec{Definitions}

In uniform convergence, we sought a high probability bound for $\sup_{h \in H}(L(h) - \hat L (h))$. Here we have a weaker goal: we try to obtain an upper bound for its expectation instead, i.e.
\begin{equation}
\Exp\l[ \sup_{h \in H}(L(h) - \hat L (h)) \r] \leq \text{ upper bound}. \label{lec5:eq:generror}
\end{equation}
The expectation is over the randomness in the training data $\{(x^{(i)}, y^{(i)})\}_{i=1}^n$.\footnote{Though we might like to pull the $\sup$ outside of the $\Exp$ operator, and bound the expectation of the excess risk (a far simpler quantity to deal with!), in general, the $\sup$ and $\Exp$ operators do not commute. In particular, $\Exp\left [\sup_{h \in \cH} (L(h) - \hat{L}(h)) \right ] \geq \sup_{h \in \cH} \Exp \left[ L(h) - \hat{L} (h) \right]$.}

To do so, we first define \textit{Rademacher complexity}.

\begin{definition}[Rademacher complexity] \label{lec5:dfn:rc}
Let $\cF$ be a family of functions mapping $Z \mapsto \bbR$, and let $P$ be a distribution over $Z$. The \textit{(average) Rademacher complexity} of $\cF$ is defined as 
\begin{align}
    R_n(F) \triangleq \Exp_{z_1, \dots, z_n \iid P} \l[ 
    \Exp_{\sigma_1, \dots, \sigma_n \iid\{ \pm 1 \}} \l[ \sup_{f\in F} \frac{1}{n} \sum^n_{i=1} \sigma_i f(z_i) \r] \r], \label{lec5:eqn:Rn}
\end{align}
where $\sigma_1, \dots, \sigma_n$ are independent \textit{Rademacher random variables}, i.e. each taking on the value of $1$ or $-1$ with probability $1/2$.
\end{definition}

\begin{remark}
For applications to empirical risk minimization, we will take $\cZ = \cX \times \cY$. However, Definition \ref{lec5:dfn:rc} holds for abstract input spaces $\cZ$ as well.
\end{remark}

\begin{remark}
Note that $R_n(\cF)$ is also dependent on the measure $P$ of the space, so technically it should be $R_{n,P}(\cF)$, but for brevity, we refer to it as $R_n(\cF)$.
\end{remark}

An interpretation is that $R_n(\cF)$ is the maximal possible correlation between outputs of some $f \in \cF$ (on points $f(z_1), \dots, f(z_n)$) and random Rademacher variables $ (\sigma_1, \dots, \sigma_n).$ Essentially, functions with more random sign outputs will better match random patterns of Rademacher variables and have higher complexity (greater ability to mimic or express randomness).

The following theorem is the main theorem involving Rademacher complexity:

\begin{theorem} \label{lec5:thm:thm1}
    \begin{align}
       \Exp_{z_1, \dots, z_n \iid P} \l[ \sup_{f\in F} \l[ \frac{1}{n} \sum^n_{i=1} f(z_i) -  \Exp_{z\sim P} [f(z)] \r]\r] \leq 2 R_n(\cF). \label{lec5:eqn:thm1}
    \end{align}
\end{theorem}

\begin{remark}
We can think of $\frac{1}{n} \sum^n_{i=1} f(z_i)$ as an empirical average and $\Exp_{z\sim P} [f(z)]$ as a population average.
\end{remark}
\noindent\textit{Why is Theorem \ref{lec5:thm:thm1} useful to us?} We can set $\cF$ to be the family of loss functions, i.e.
\begin{equation}
\cF = \l\{ z = (x,y) \in \cZ \mapsto \ell((x,y),h) \in \bbR : h \in \cH \r\}.
\end{equation} 
This is the family of losses induced by the hypothesis functions in $\cH$. We also define the function class $-\cF$ as $\{-f : f \in \cF\}$. It should be obvious from this definition that $R_n(\cF) = R_n(-\cF)$ since $\sigma_i \stackrel{d}{=} -\sigma_i$ for all $i$. Then, letting $z_i = (x^{(i)}, y^{(i)})$,
\begin{align}
    \Exp\l[ \sup_{h \in \cH}\l( L(h) - \hat L (h) \r) \r] &= \Exp_{\{(x^{(i)}, y^{(i)})\}} \l[ \sup_{h \in \cH} \l[L(h) - \frac{1}{n} \sum^n_{i=1} \ell((x^{(i)}, y^{(i)}), h) \r] \r] \\
    &= \Exp_{\{z_i\}} \l[\sup_{f \in \cF} \l(\Exp[f(z)] - \frac{1}{n} \sum^n_{i=1} f(z_i) \r)\r] \\
    &= \Exp_{\{z_i\}} \l[\sup_{f \in -\cF} \l(\frac{1}{n} \sum^n_{i=1} f(z_i) - \Exp[f(z)] \r)\r] \\
    &\leq 2 R_n(-\cF) = 2R_n(\cF)
\end{align}
where the last step follows by Theorem \ref{lec5:thm:thm1}. 

Thus, $2R_n(\cF)$ is an upper bound for the generalization error. In this context, $R_n(\cF)$ can be interpreted as how well the loss sequence $\ell((x^{(1)}, y^{(1)}), h), \dots \ell((x^{(n)}, y^{(n)}), h)$ correlates with $\sigma_1, \dots, \sigma_n$.
\begin{example}[Binary classification]
Consider the binary classification setting where $y \in \{\pm 1\}$ and the model family $\cH$ consists of functions that maps $\cX$ to $\{\pm 1\}$.\footnote{We note that this is not necessarily the best abstraction of the classification problems. See Section~\ref{chap:generalization:sec:margin} for a more margin-based approach that is more practically relevant.} 
Let $\ell_{0-1}$ denote the zero-one loss function. Note that
\begin{equation}\label{lec5:eqn:01}
    \ell_{0-1}((x,y), h) = \mathbf{1}\{h(x) \neq y\} = \frac{1-yh(x)}{2}.
\end{equation}

Hence,
\begin{align}
    R_n(\cF) &= \Exp_{\{(x^{(i)}, y^{(i)})\}, \sigma_i} \l[ \sup_{h \in \cH} \frac{1}{n}\sum^n_{i=1} \ell_{0-1}((x^{(i)}, y^{(i)}),h)\sigma_i \r] &(\text{by definition}) \\
    &= \Exp_{\{(x^{(i)}, y^{(i)})\}, \sigma_i} \l[ \sup_{h \in \cH} \frac{1}{n}\sum^n_{i=1} \l(\frac{-h(x^{(i)})y^{(i)}+1}{2}\r)\sigma_i \r] &(\text{by } \eqref{lec5:eqn:01}) \\
    &= \frac{1}{2} \Exp_{\{(x^{(i)}, y^{(i)})\}, \sigma_i} \l[ \frac{1}{n}\sum^n_{i=1}\sigma_i + \sup_{h \in \cH} \frac{1}{n}\sum^n_{i=1} -h(x^{(i)})y^{(i)}\sigma_i \r] &(\sup \text{only over } \cH) \\
    &= \frac{1}{2} \Exp_{\{(x^{(i)}, y^{(i)})\}, \sigma_i} \l[\sup_{h \in \cH} \frac{1}{n}\sum^n_{i=1} -h(x^{(i)})y^{(i)}\sigma_i \r] &(\Exp [\sigma_i] = 0) \\
    &=\frac{1}{2} \Exp_{\{(x^{(i)}, y^{(i)})\}, \sigma_i} \l[\sup_{h \in \cH} \frac{1}{n}\sum^n_{i=1} h(x^{(i)})\sigma_i \r] &(-y_i \sigma_i \stackrel{d}{=} \sigma_i) \\
    &= \frac{1}{2}R_n(\cH). &(\text{by definition})
\end{align}

In this setting, $R_n(\cF)$ and $R_n(\cH)$ are the same (except for the factor of 2). $R_n(\cH)$ has a slightly more intuitive interpretation: it represents how well $h \in \cH$ can fit random patterns.

\textbf{Warning!} $R_n(\cF)$ is not always the same as $R_n(\cH)$ in other problems.
\end{example}

\begin{remark}
Rademacher complexity is invariant to translation. This property manifests in the previous example when the $+1$ in the $\l(\frac{-h(x^{(i)})y^{(i)}+1}{2}\r)$ term essentially vanishes in the computation.
\end{remark}

Let us now prove Theorem \ref{lec5:thm:thm1}.

\begin{proof}[Proof of Theorem \ref{lec5:thm:thm1}]
We use a technique called \textit{symmetrization}, which is a very important technique in probability theory. We first fix $z_1, \dots, z_n$and draw $ z_1', \dots z_n' \iid P$. Then we can rewrite the term in the expectation on the LHS of \eqref{lec5:eqn:thm1}:
\begin{align}
    \sup_{f \in \cF} \l( \frac{1}{n} \sum^n_{i=1} f(z_i) - \Exp[f] \r) &= \sup_{f \in \cF} \l( \frac{1}{n} \sum^n_{i=1} f(z_i) - \Exp_{z_1',\dots, z_n'} \l[ \frac{1}{n} \sum^n_{i=1} f(z_i')\r] \r) \\
    &= \sup_{f \in \cF} \l( \Exp_{z_1',\dots, z_n'} \l[\frac{1}{n} \sum^n_{i=1} f(z_i) -  \frac{1}{n} \sum^n_{i=1} f(z_i')\r] \r)\\
    &\leq \Exp_{z_1',\dots, z_n'} \l[\sup_{f \in \cF} \l( \frac{1}{n} \sum^n_{i=1} f(z_i) -  \frac{1}{n} \sum^n_{i=1} f(z_i')\r)\r]. \label{lec5:eqn:thm1-pf1}
\end{align}

The last inequality is because in general,
\begin{align}
    \sup_u \l(\Exp_v[g(u,v)]\r) \leq \sup_u \l( \Exp_v\l[\sup_{u'}(g(u',v))\r]\r) = \Exp_v \l[\sup_u (g(u,v))\r]
\end{align}
since the $\sup$ over $u$ becomes vacuous after we replace $u$ with $u'$.

Now, if we take the expectation over $z_1, \dots, z_n$ for both sides of \eqref{lec5:eqn:thm1-pf1},
\begin{align}
    \Exp_{z_1, \dots, z_n} \l[\sup_{f \in \cF} \l( \frac{1}{n} \sum^n_{i=1} f(z_i) - \Exp[f] \r) \r] 
    &\leq \Exp_{z_i} \l[ \Exp_{z_i'} \l[\sup_{f \in \cF} \l( \frac{1}{n} \sum^n_{i=1} \l(f(z_i) -  f(z_i')\r)\r)\r]\r]\\
    &= \Exp_{z_i,z_i'} \l[ \Exp_{\sigma_i} \l[\sup_{f \in \cF} \l( \frac{1}{n} \sum^n_{i=1} \sigma_i\l(f(z_i) -  f(z_i')\r)\r)\r]\r] \label{lec5:eqn:thm1-pf2} \\
 &\leq \Exp_{z_i,z_i', \sigma_i} \l[\sup_{f \in \cF} \l( \frac{1}{n} \sum^n_{i=1} \sigma_i f(z_i)\r)+\sup_{f \in \cF} \l( \frac{1}{n} \sum^n_{i=1} -\sigma_i f(z_i')\r)\r] \\
    &= 2R_n(\cF),
\end{align}
where \eqref{lec5:eqn:thm1-pf2} is because $\sigma_i(f(z_i) - f(z_i')) \stackrel{d}{=} f(z_i) - f(z_i')$ since $f(z_i) - f(z_i')$ has a symmetric distribution. The last equality holds since $-\sigma_i \overset{d}{=} \sigma_i$ and $z_i, z_i'$ are drawn iid from the same distribution. 
\end{proof}

Here is an intuitive understanding of what Theorem \ref{lec5:thm:thm1} achieves. Consider the quantities on the LHS and RHS of \eqref{lec5:eqn:thm1}:
\begin{align*}
    \sup_{f\in \cF} \l(\frac{1}{n} \sum_{i=1}^n f(z_i) - \Exp[f(z)]\r) \qquad \text{vs.} \qquad \sup_{f\in \cF} \l(\frac{1}{n} \sum_{i=1}^n \sigma_i f(z_i)\r).
\end{align*}
First, we removed $\Exp[f(z)]$, which is hard to control quantitatively since it is deterministic. Second, we added more randomness in the form of Rademacher variables. This will allow us to shift our focus from the randomness in the $z_i$'s to the randomness in the $\sigma_i$'s. In the future, our bounds on the Rademacher complexity will typically only depend on the randomness from the $\sigma_i$'s.

\subsec{Dependence of Rademacher complexity on \texorpdfstring{$P$}{P}}
For intuition on how Rademacher complexity depends on the distribution $P$, consider the extreme example where $P$ is a point mass, i.e. $z = z_0$ almost surely. Assume that $-1 \leq f(z_0) \leq 1$ for all $f \in \cF$. Then
\begin{align}
    \Exp_{z_1, \dots, z_n \sim P} \l[ \sup_{f \in \cF} \frac{1}{n} \sum^n_{i=1} \sigma_i f(z_i)\r]
    &= \Exp_{\sigma_1, \dots, \sigma_n} \l[ \sup_{f \in \cF} \frac{1}{n}f(z_0) \sum^n_{i=1} \sigma_i \r] \\
    &\leq \Exp_{\sigma_1, \dots, \sigma_n} \l[ \l| \frac{1}{n} \sum^n_{i=1} \sigma_i \r|\r] &(\text{since } f(z_0) \in [-1,1]) \\
    &\leq \Exp_{\sigma_i} \l[ \l( \frac{1}{n} \sum^n_{i=1} \sigma_i \r)^2\r]^\frac{1}{2} &(\text{Jensen's Inequality}) \\
    &= \frac{1}{n}\l( \Exp_{\sigma_i, \sigma_j} \l[ \sum^n_{i, j=1} \sigma_i\sigma_j \r] \r)^\frac{1}{2}\\
    &= \frac{1}{n}\l( \Exp_{\sigma_i} \l[ \sum^n_{i=1} \sigma_i^2 \r] \r)^\frac{1}{2} \\
    &= \frac{1}{n} \cdot \sqrt{n} = \frac{1}{\sqrt{n}}.
\end{align}
This bound does not depend on $\cF$ (except on the fact that $f \in \cF$ is bounded). This example illustrates that a bound on the Rademacher complexity can sometimes depend only on the (known) distribution of the Rademacher random variables.

\sec{Empirical Rademacher complexity}

In the previous section, we bounded the expectation of $\sup_{f\in F} \l[ \frac{1}{n} \sum^n_{i=1} f(z_i) -  \Exp_{z\sim P} [f(z)] \r]$. This expectation is taken over the training examples $z_1, \dots, z_n$. In many instances we only have one training set, and do not have access to many training sets. Thus, the bound on the expectation does not give a guarantee for the one training set that we have. In this section, we seek to bound the quantity itself with high probability.

\begin{definition}[Empirical Rademacher complexity]
Given a dataset $S = \{z_1, \dots, z_n\}$, the \textit{empirical Rademacher complexity} is defined as
\begin{equation}
R_S(\cF) \overset{\Delta}{=} \Exp_{\sigma_1,\dots, \sigma_n} \l[ \sup_{f\in \cF} \frac{1}{n} \sum^n_{i=1} \sigma_i f(z_i) \r].
\end{equation}
$R_S(\cF)$ is a function of both the function class $\cF$ and the dataset $S$.
\end{definition}

As the name suggests, the expectation of the empirical Rademacher complexity is the Rademacher complexity:
\begin{align}
    R_n(\cF) = \underset{S=\{z_1,\dots, z_n\}}{\underset{z_1, \dots, z_n \iid P}\Exp}\l[ R_S(\cF) \r].
\end{align}

Here is the theorem involving empirical Rademacher complexity:
\begin{theorem}\label{lec5:thm:thm2}
    Suppose for all $f \in \cF$, $0 \leq f(z) \leq 1$. Then, with probability at least $1-\delta$,
    \begin{align}
        \sup_{f\in \cF} \l[ \frac{1}{n} \sum^n_{i=1} f(z_i) - \Exp[f(z)] \r] \leq 2 R_S(\cF) + 3\sqrt{\frac{\log{(2/\delta)}}{2n}}.
    \end{align}
\end{theorem}

\begin{proof}
For conciseness, define
\begin{equation} g(z_1, \dots, z_n) \triangleq \sup_{f\in F} \l[ \frac{1}{n} \sum^{n}_{i=1} f(z_i) - \Exp[f(z)]\r]. \end{equation}

We prove the theorem in 4 steps.

\textbf{Step 1:} We bound $g$ using McDiarmid's Inequality. To use McDiarmid's Inequality, we check that the bounded difference condition holds:
\begin{align}
    g(z_1, \dots, z_n) - g(z_1, \dots, z_i', \dots, z_n)
    &\leq \sup_{f\in \cF} \l[ \frac{1}{n} \sum^{n}_{j=1} f(z_j) \r] - \sup_{f\in \cF} \l[ \l(\frac{1}{n} \sum^{n}_{j=1, j \neq i} f(z_j)\r) + \frac{f(z_i')}{n} \r]  \\
    &\leq \sup_{f\in \cF} \l[ \frac{1}{n} \sum^{n}_{j=1} f(z_j) - \l(\frac{1}{n} \sum^{n}_{j=1, j \neq i} f(z_j)\r) - \frac{f(z_i')}{n} \r] \label{lec5:eqn:thm2-pf1} \\
    &= \sup_{f\in \cF}\l[ \frac{1}{n} \l( f(z_i) - f(z_i') \r) \r] \\
    &\leq \frac{1}{n}. \label{lec5:eqn:thm2-pf2}
\end{align}
\eqref{lec5:eqn:thm2-pf1} holds because in general, $\sup_f A(f) - \sup_f B(f) \leq \sup_f [A(f) - B(f)]$, and \eqref{lec5:eqn:thm2-pf2} holds since $f$ is bounded by $[0,1]$. We can thus apply McDiarmid's Inequality with parameters $c_1 = \dots = c_n = 1/n$:
\begin{align}
    \Pr\l[ g(z_1, \dots, z_n) \geq \Exp_{z_1,\dots, z_n \iid P}[g] + \epsilon \r] \leq \exp{\l( \frac{-2\epsilon^2}{\sum^n_{i=1} c_i^2 }\r)} = \exp(-2n\epsilon^2).
\end{align}

\textbf{Step 2:} We apply Theorem \ref{lec5:thm:thm1} to get 
\begin{align}
 \Exp_{z_1,\dots, z_n \iid P}[g] \leq 2R_n(\cF).
\end{align}

\textbf{Step 3:} Define
\begin{equation} \tilde g (z_1, \dots, z_n) = R_S(\cF) \triangleq \Exp_{\sigma_i}\l[\sup_{f\in \cF} \frac{1}{n} \sum^n_{i=1} \sigma_i f(z_i)\r]. \end{equation}

Using a similar argument to that of Step 1, we show that $\tilde g$ satisfies the bounded difference condition:
\begin{align}
    &\tilde g(z_1, \dots, z_n) - \tilde g(z_1, \dots, z_i', \dots, z_n) \nonumber \\
    &\leq \Exp_{\sigma_i} \l[\sup_{f\in F} \l[ \frac{1}{n} \sum^{n}_{j=1} \sigma_j f(z_j) \r] - \sup_{f\in F} \l[ \l(\frac{1}{n} \sum^{n}_{j=1, j \neq i} \sigma_j f(z_j)\r) + \frac{1}{n} \sigma_if(z_i')\r]\r]\\
    &\leq \Exp_{\sigma_i}\l[\sup_{f\in F} \l(\frac{1}{n} \sigma_i(f(z_i) - f(z_i'))\r)\r] \\
    &\leq \frac{1}{n},
\end{align}
since the term inside the $\sup$ is always upper bounded by 1. We can thus apply McDiarmid's Inequality with parameters $c_1 = \dots = c_n = 1/n$:
\begin{align}
    \Pr\l[ \tilde g - \Exp[\tilde g] \geq \epsilon \r] \leq \exp(-2n \epsilon^2), \quad\text{and}\quad
    \Pr\l[ \tilde g - \Exp[\tilde g] \leq -\epsilon \r] \leq \exp(-2n \epsilon^2).
\end{align}

\textbf{Step 4:} We set $\delta$ such that $\exp(-2n \epsilon^2) = \delta/2$. (This implies that $\epsilon = \sqrt{\frac{\log(2/\delta)}{2n}}$.) Then, with probability $\geq 1 - \delta$,
\begin{align}
    \sup_{f\in \cF} \l[ \frac{1}{n} \sum^n_{i=1} f(z_i) - \Exp[f]\r] = g &\leq \Exp[g] + \epsilon &\text{(Step 1)} \\
    &\leq 2R_n(\cF) + \epsilon &\text{(Step 2)} \\
    &\leq 2(R_S(\cF) + \epsilon) + \epsilon &\text{(Step 3)} \\
    &= 2R_S(\cF) + 3\epsilon,
\end{align}
as required.
\end{proof}

Setting $\cF$ to be a family of loss functions bounded by $[0,1]$ in Theorem \ref{lec5:thm:thm2} gives the following corollary:
\begin{corollary}\label{lec6:cor:ggap-rsbound}
Let $\cF$ be a family of loss functions $\cF = \l\{ (x,y) \mapsto \ell((x,y),h): h \in \cH \r\}$ with $\ell((x,y), h) \in [0,1]$ for all $\ell$, $(x,y)$ and $h$. Then, with probability $1-\delta,$ the generalization gap is
    \begin{equation}\label{lec6:eqn:ggap-rsbound}
        \hat{L}(h) - L(h) \leq 2R_S(\cF) + 3\sqrt{\frac{\log(2/\delta)}{2n}} \quad \text{for all } h\in \cH.
    \end{equation}
\end{corollary}

\begin{remark}
If we want to bound the generalization gap by the average Rademacher complexity instead, we can replace the RHS of \eqref{lec6:eqn:ggap-rsbound} with $2R_n(\cF) + \sqrt{\frac{\log(2/\delta)}{2n}}$.
\end{remark}

\paragraph{Interpretation of  Corollary \ref{lec6:cor:ggap-rsbound}.}
\sloppy It is typically the case that $O\l(\sqrt{\frac{\log (2/\delta)}{n}}\r) \ll R_S(\cF)$ and $O\l(\sqrt{\frac{\log (2/\delta)}{n}}\r) \ll R_n(\cF)$. This is the case because $R_S(\cF)$ and $R_n(\cF)$ often take the form $\frac{c}{\sqrt{n}}$ where $c$ is a big constant depending on the complexity of $\cF$, whereas we only have a logarithmic term in the numerator of $O\l(\sqrt{\frac{\log (2/\delta)}{n}}\r)$. As a result, we can view the $3\sqrt{\frac{\log (2/\delta)}{n}}$ term in the RHS of Corollary \ref{lec6:cor:ggap-rsbound} as negligible. Another way of seeing this is noting that a $\tilO \left( \frac{1}{\sqrt{n}} \right)$ term is necessary even for the concentration bound of a single function $h\in\cH$. Previously, we bounded $L(h)-\hat{L}(h)$ using a union bound over $h\in\cH$, which necessarily needs to be larger than $\tilO \left(\frac{1}{\sqrt{n}} \right)$. As a result, the $O\l(\sqrt{\frac{\log (2/\delta)}{n}}\r)$ term is not significant.

%\subsec{Empirical Rademacher complexity viewed in the output/function space}
%Assume we have a fixed dataset $S = \{z_1, \dots, z_n\}$. Since $z_1,\dots, z_n$ is fixed, each function $f\in\cF$ corresponds to a single output $(f(z_1),\dots,f(z_n))\in \R^n$. Hence, we can express the set of outputs for every function $f\in\cF$ as
%\begin{align}
%    Q_\cF = \left\{ \begin{pmatrix}f(z_1), \dots, f(z_n) \end{pmatrix} \mid f\in\cF \right\}.
%\end{align}
%
%Now we can mathematically re-express the empirical Rademacher complexity as an inner product:
%\begin{align}
%R_S(\cF) &= \Exp_{\sigma_1,\dots, \sigma_n} \l[ \sup_{f\in \cF} \frac{1}{n} \sum^n_{i=1} \sigma_i f(z_i) \r] \\
%&= \Exp_{\sigma_1,\dots, \sigma_n} \l[ \sup_{v\in Q} \frac{1}{n}\langle\sigma, v\rangle \r],
%\end{align}
%where $\sigma=(\sigma_1,\dots,\sigma_n)$. (See Figure \ref{lec6:fig:rs-innerprod} for an illustration of this idea.) This perspective will be helpful later on when proving bounds on the empirical Rademacher complexity.




\subsec{Rademacher complexity is translation invariant}
A useful fact is that both empirical Rademacher complexity and average Rademacher complexity are translation invariant. (This is not obvious when thinking of how translation affects the picture in Figure \ref{lec6:fig:rs-innerprod}.)

\begin{proposition}
Let $\cF$ be a family of functions mapping $Z \mapsto \R$ and define $\cF' = \{f'(z) = f(z) + c_0\mid f\in \cF\}$ for some $c_0\in\R$. Then $R_S(\cF) = R_S(\cF')$ and $R_n(\cF) = R_n(\cF')$.
\end{proposition}

\begin{proof}
We will prove here that empirical Rademacher complexity is translation invariant.
\begin{align}
R_S(\cF') &= \Exp_{\sigma_1,\dots, \sigma_n} \l[ \sup_{f'\in \cF'} \frac{1}{n} \sum^n_{i=1} \sigma_i f(z_i) \r] \\
&= \Exp_{\sigma_1,\dots, \sigma_n} \l[ \sup_{f\in \cF} \frac{1}{n} \sum^n_{i=1} \sigma_i (f(z_i)+c_0) \r] \\
&= \Exp_{\sigma_1,\dots, \sigma_n} \l[ \frac{1}{n} \sum^n_{i=1} \sigma_i c_0 + \sup_{f\in \cF} \frac{1}{n} \sum^n_{i=1} \sigma_i f(z_i) \r] \\
&= \Exp_{\sigma_1,\dots, \sigma_n} \l[\sup_{f\in \cF} \frac{1}{n} \sum^n_{i=1} \sigma_i f(z_i) \r] = R_S(\cF), \label{lec6:eqn:rs-translation}
\end{align}
where \eqref{lec6:eqn:rs-translation} follows because $\Exp_{\sigma_1,\dots,\sigma_n} \frac{1}{n}\sum_{i=1}^n \sigma_i c_0 = 0$, since the $\sigma_i$'s are Rademacher random variables.
\end{proof}


 \input{collection/04-03-uniform.tex}
%	
	\chapter{Rademacher Complexity Bounds for Linear Classifiers}\label{chap:gen-bounds}
	% reset section counter
%\setcounter{section}{0}

%\metadata{lecture ID}{Your names}{date}
\metadata{6}{Daniel Do}{February 1st, 2021}

In this chapter, we will instantiate Rademacher complexity for two important hypothesis classes: linear models and two-layer neural networks. In the process, we will develop margin theory and use it to bound the generalization gap for binary classifiers.

\sec{Margin theory for classification problems} \label{chap:generalization:sec:margin}

\subsec{Intuition}
Assume that we are in the same setting as in the previous section. A fundamental problem we face in this setting is that we do not have a continuous loss: everything is discrete in the output space. We need to find a way to reason about the scale of the output. An example of this is logistic regression: the logistic regression model outputs a probability, and when we compare it to the outcome (0 or 1), its closeness to the true output gives us a measure of how confident we are in the prediction.

Figure \ref{lec6:fig:margin} gives similar intuition for linear classifiers. Intuitively, the black line is a ``better'' decision boundary than the red line because the minimum distance from any point to the black boundary is greater than the minimum distance from any point to the red boundary. In the next section, we will formalize this intuition by proving that the larger this margin is, the smaller the bound on the generalization gap is.

\begin{figure}[ht!]
    \begin{center}
  \includegraphics[width=0.5\textwidth]{figures/margin.png}
  \end{center}
  \caption{The red and black lines are two decision boundaries. The X's are positive examples and the O's are negative examples. The black line has a larger margin than the red line, and is intuitively a better classifier.}
  \label{lec6:fig:margin}
\end{figure}

\subsec{Formalizing margin theory} \label{sec:formal_margin}
First, assume that the dataset $\cD = ((x\sp{1}, y\sp{1}), \dots, (x\sp{n}, y\sp{n}))$ is \textit{completely separable}. In other words, there exists some $h_\theta\in\cH$ such that $y^{(i)} = \sgn(h_\theta(x^{(i)}))$ holds for all $( x^{(i)},y^{(i)})\in \cD$. This is not a necessary condition for our final bound but will make the derivation cleaner.

\begin{definition}[(Unnormalized) Margin]
Fix the hypothesis $h_\theta$. The \textit{(unnormalized) margin} for example $(x, y)$ is defined as $\margin(x) = yh_\theta(x)$. Margin is only defined on examples where $\sgn(h_\theta(x)) = y$. (Note that $\margin(x)\geq 0$ because of our assumption of complete separability.)
\end{definition}

\begin{definition}[Minimum margin] Given a dataset $\cD = ((x\sp{1}, y\sp{1}), \dots, (x\sp{n}, y\sp{n}))$, the \textit{minimum margin} over the dataset is defined as $\gamma_{\min} \triangleq \min_{i\in\{1,\dots,|\cD|\}} y^{(i)}h_\theta(x^{(i)})$.
\end{definition}

Our final bound will have the form (generalization gap) $\leq f(\text{margin},\text{parameter norm})$. This is very generic since there are many different bounds we could derive based on what margin we use. For this current setting we are using $\gamma_{\min}$, which is the minimum margin, but in other settings could use $\gamma_{\text{average}}$, which is the average margin of each point in the dataset.

We will begin by introducing the idea of a \textit{surrogate loss}, a loss function which approximates zero-one loss but takes the scale of the margin into account. The \textit{margin loss} (also known as \textit{ramp loss}) is defined as 
\begin{equation}
    \ell_\gamma(t) = \begin{cases} 
      0, & t\geq \gamma \\
      1, & t\leq 0 \\
      1-t/\gamma, & 0\leq t\leq \gamma
   \end{cases} \label{lec6:eqn:ramp_loss}
\end{equation}

\begin{figure}[ht!]
    \begin{center}
  \includegraphics[width=0.5\textwidth]{figures/margin_loss.png}
  \end{center}
  \caption{Plotted margin loss.}
  \label{lec6:fig:marginloss}
\end{figure}

It is plotted in Figure \ref{lec6:fig:marginloss}. For convenience, define $\ell_\gamma((x,y), h) \triangleq \ell_\gamma(yh(x))$. We can view $\ell_\gamma$ as a continuous version of $\err$ that is more sensitive to the scale of the margin on $[0,\gamma]$. Notice that $\err$ is always less than or equal to the $\ell_\gamma$ when $\gamma\geq 0$, i.e.
\begin{equation}
    \err((x,y), h) = \ind{yh(x) < 0}\leq \ell_\gamma(yh(x)) =\ell_\gamma ((x,y), h)
\end{equation}
holds for all $(x,y)\sim P$. Taking the expectation over $(x,y)$ on both sides of this inequality, we see that
\begin{equation}
    L(h) = \Exp_{(x,y)\sim P} \left[ \err((x,y), h) \right] \leq \Exp_{(x,y)\sim P} \left[ \ell_\gamma ((x,y), h) \right].
\end{equation}

Therefore, the population loss is bounded by the expectation of the margin loss, and so it is sufficient to bound the expectation of the margin loss in order to bound the population loss.

Define the population and empirical versions of the margin loss:
\begin{equation}
L_\gamma(h) = \Exp_{(x,y)\sim P}\l[ \ell_\gamma((x,y), h)\r], \quad \hat{L}_\gamma(h) = \sum_{i=1}^n\l [\ell_\gamma((x^{(i)},y^{(i)}), h)\r].
\end{equation}

By Corollary \ref{lec6:cor:ggap-rsbound}, we see that with probability at least $1-\delta$,
\begin{equation}
L_\gamma(h) - \hat{L}_\gamma(h)\leq 2R_S(\cF) + 3\sqrt{\frac{\log (2/\delta)}{2n}},
\end{equation}
where $\cF = \{(x,y)\mapsto \ell_\gamma((x,y), h)\mid h\in\cH\}$. Note that if we set $\gamma\leq \gamma_{\min}$, then $\hat{L}_{\gamma}(h) = 0$. This follows because by definition of $\gamma_{\min}$, $y^{(i)}h(x^{(i)})\geq \gamma_{\min}$ for any $(x^{(i)}, y^{(i)})\in \cD$. As a result, $\ell_\gamma((x^{(i)}, y^{(i)}), h) = \ell_\gamma(y^{(i)}h(x^{(i)})) = 0$ holds. Therefore, it suffices to bound $R_S(\cF)$.

We will now use \textit{Talagrand's lemma} to bound $R_S(\cF)$ in terms of $R_S(\cH)$ to remove any dependence on the loss function from the upper bound. 
 
\begin{lemma}[Talagrand's lemma] \label{lec6:lem:talagrand_lemma}
Let $\phi:\R\to\R$ be a $\kappa$-Lipschitz function. Then \begin{equation}
    R_S(\phi\circ \cH)\leq \kappa R_S(\cH),
\end{equation} 
where $\phi\circ\cH = \{z\mapsto \phi(h(z))\mid h\in\cH\}$.
\end{lemma}

We can use Talagrand's lemma directly with $\phi(t) = \ell_\gamma(t)$, which is $\frac{1}{\gamma}$-Lipschitz. We can express $\cF$ as $\cF=\ell_\gamma\circ\cH'$ where $\cH' = \{(x,y)\to yh(x)\mid h\in\cH\}$. Applying Talagrand's lemma, we see that

\begin{align}
R_S(\cF) &\leq \frac{1}{\gamma}R_S(\cH') \\
&= \frac{1}{\gamma}\Exp_{\sigma_1,\dots, \sigma_n} \l[ \sup_{h\in \cH} \frac{1}{n} \sum^n_{i=1} \sigma_i y^{(i)}h(x^{(i)}) \r] \\
&= \frac{1}{\gamma}\Exp_{\sigma_1,\dots, \sigma_n} \l[ \sup_{h\in \cH} \frac{1}{n} \sum^n_{i=1} \sigma_i h(x^{(i)})  \r] \\
&= \frac{1}{\gamma}R_S(\cH).
\end{align}

Putting this all together, we have shown that for $\gamma = \gamma_{\min}$,
\begin{align}
\Err(h) \leq L_\gamma(h) &\leq 0 + O \left( \frac{R_S(\cH)}{\gamma} \right) + \tilO \left( \sqrt{\frac{\log (2 / \delta)}{2n}} \right) \\
&= O \left( \frac{R_S(\cH)}{\min_i y\sp{i} h(x\sp{i}) } \right) + \tilO \left( \sqrt{\frac{\log (2 / \delta)}{2n}} \right).
\end{align}

In other words, for training data of the form $S = \{(x\sp{i},y\sp{i})\}_{i=1}^n \subset \mathbb{R}^d \times \{-1,1\}$, a hypothesis class~$\mathcal{H}$ and 0-1 loss, we can derive a bound of the form
\begin{equation}\label{lec7:eqn:generalization_loss}
    \text{generalization loss} \leq \frac{2R_S(\mathcal{H})}{\gamma_{\mathrm{min}}} + \text{low-order term},
\end{equation}
where $\gamma_\mathrm{min}$ is the minimum margin achievable on~$S$ over those hypotheses in $\cH$ that separate the data, and $R_S(\cH)$ is the empirical Rademacher complexity of $\cH$. Such bounds state that simpler models will generalize better beyond the training data, particularly for data that is strongly separable.

\begin{remark} \label{lec7:rmk:union_bound_margin}
Note there is a subtlety here. If we think of the dataset as random, it follows that $\gamma_{\min}$ is a random variable. Consequently, the $\gamma$ we choose to define the hypothesis class is random, which is not a valid choice when thinking about Rademacher complexity! Technically we cannot apply Talagrand's lemma with a random $\kappa$ (which we took to be $1/\gamma$). Also, when we use concentration inequalities, we implicitly assume that the $\ell_\gamma((x\sp{i}, y\sp{i}), h)$ are independent of each other. That is not the case if $\gamma$ is dependent on the data.

We sketch out how one might address this issue below. The main idea is to do another union bound over $\gamma$. Choose a family $\Gamma = \left\{ 2^k: k \in [-B, B] \right\}$ for some $B$. Then, for every fixed $\gamma \in \Gamma$, with probability greater than $1 - \delta$,
\begin{align}
\Err(h) \leq \hatL_\gamma (h) + O \left( \frac{R_S(\cH)}{\gamma} \right) + \tilO \left( \sqrt{\frac{\log \frac{1}{\delta}}{n}} \right).
\end{align}
Taking a union bound over all $\gamma \in \Gamma$, it further holds that for all $\gamma \in (0, B)$, 
\begin{align}
    \Err(h) \leq \hatL_\gamma (h) + O \left( \frac{R_S(\cH)}{\gamma} \right) + \tilO \left( \sqrt{\frac{\log \frac{1}{\delta}}{n}}\right) + \tilO \left ( \sqrt{\frac{\log B}{n}} \right ). \label{lec7:eqn:unionboundmargin}
\end{align}
Last, choose the largest $\gamma \in \Gamma$ such that $\gamma \leq \gamma_{\min}$. Then, for this value of $\gamma$, our desired bound directly follows from the bound in \eqref{lec7:eqn:unionboundmargin}. Namely, we have that $\hatL_{\gamma} (h) = 0$ and $O \left( \frac{R_S(\cH)}{\gamma} \right) = O \left( \frac{R_S(\cH)}{\gamma_{\min}} \right)$. The additional term, $\tilO\left ( \sqrt{\frac{\log B}{n} }\right )$, is the price exacted by the uniform convergence argument required to correct the heuristic bound given in \eqref{lec7:eqn:generalization_loss}.

\end{remark}

	% reset section counter
%\setcounter{section}{0}

%\metadata{lecture ID}{Your names}{date}
\metadata{7}{Spencer M.~Richards and Thomas Lew}{Feb.~3rd, 2021}

\sec{Linear models}\label{lec7:sec:lin_models}

\subsec{Linear models with weights bounded in \texorpdfstring{$\ell_2$}{L2} norm}
We begin with the Rademacher complexity of linear models using weights with bounded $\ell_2$ norm.

\begin{theorem}\label{lec7:thm:l2-thm}
    Let $\mathcal{H} = \{x \mapsto \inprod{w,x} \mid w \in \R^d, \Norm{w}_2 \le B\}$ for some constant $B > 0$. Moreover, assume $\Exp_{x \sim P}\sbr{\Norm{x}_2^2} \leq C^2$, where $P$ is some distribution and $C > 0$ is a constant. Then
    \begin{align}
        R_S(\mathcal{H}) &\le \frac{B}{n} \sqrt{\sum_{i=1}^n \Norm{x\sp{i}}_2^2},  \label{lec7:eqn:linear-sample}
        \intertext{and}
        R_n(\mathcal{H}) &\le \frac{BC}{\sqrt{n}}.  \label{lec7:eqn:linear}
    \end{align}
\end{theorem}

Generally speaking, there are two methods with which we can bound the Rademacher complexity of a model. The first method, which we used in Chapter \ref{chap:uc}, consists of discretizing the space of possible outputs from our hypothesis class, then using a union bound or covering number argument to bound the Rademacher complexity of the model. While this method is powerful and generally applicable, it yields bounds that depend on the logarithm of the cardinality of this discretized output space, which in turn depends on the number of data points~$n$. In the proof below, we will instead use a more elegant, albeit limited technique which does not rely on discretization of the output space.

\begin{proof}
We start with the proof of \eqref{lec7:eqn:linear-sample}. By definition,
\begin{align}
    R_S(\mathcal{H}) 
    &= \Exp_\sigma\sbr{ \sup_{\Norm{w}_2 \le B} \frac{1}{n} \sum_{i=1}^n\sigma_i \inprod{w,x\sp{i}} }
    \\&= \frac{1}{n} \Exp_\sigma\sbr{ \sup_{\Norm{w}_2 \le B} \inprod{w,\sum_{i=1}^n\sigma_i x\sp{i}} }
    \\&= \frac{B}{n} \Exp_\sigma\sbr{ \Norm{\sum_{i=1}^n \sigma_i  x\sp{i}}_2 }
        &&\text{($\textstyle\sup_{\Norm{w}_2 \le B} \langle w,v\rangle =B\Norm{v}_2$)}
    \\&\leq \frac{B}{n} \sqrt{ \Exp_\sigma\sbr{\Norm{ \sum_{i=1}^n \sigma_i x\sp{i} }_2^2} }
        &&\text{(Jensen's ineq. for $\alpha \mapsto \alpha^2$)} 
    \\&= \frac{B}{n} \sqrt{ \Exp_\sigma \sbr{\sum_{i=1}^n \rbr{\sigma_i^2 \Norm{x\sp{i}}_2^2 + \inprod{\sigma_ix\sp{i},\sum_{j \ne i}^n \sigma_j x\sp{j}} }} }
    \\&= \frac{B}{n} \sqrt{\sum_{i=1}^n \Norm{x\sp{i}}_2^2}.
        &&\text{($\sigma_i$ indep. and $\Exp[\sigma_i]=0$)}
\end{align}
This completes the proof of \eqref{lec7:eqn:linear-sample} for the empirical Rademacher complexity. The bound on the average Rademacher complexity in \eqref{lec7:eqn:linear} follows from taking the expectation of both sides to get
\begin{equation}
    R_n(\mathcal{H}) = \Exp\sbr{ R_S(\mathcal{H}) }
    = \frac{B}{n} \Exp\sbr{ \sqrt{\sum_{i=1}^n \Norm{x\sp{i}}_2^2} }
    \le \frac{B}{n} \sqrt{ \sum_{i=1}^n \Exp\sbr{\Norm{x\sp{i}}_2^2} }
    \le \frac{BC}{\sqrt{n}},
\end{equation}
where the first inequality is another application of Jensen's inequality, and the second follows from the assumption $\Exp_{x \sim P}\sbr{\Norm{x}_2^2} \leq C^2$.

\end{proof}

We observe that both the empirical and average Rademacher complexities scale with the upper $\ell_2$-norm bound $\Norm{w}_2 \le B$ on the parameters~$w$, which motivates regularizing the model. However, smaller weights in the model may reduce the margin $\gamma_\mathrm{min}$, which in turn hurts generalization according to \eqref{lec7:eqn:generalization_loss}.

\begin{remark}
Note that if we scale the data by some multiplicative factor, the bound on empirical Rademacher complexity $R_S(\cH)$ will scale accordingly. However, at the same time, we expect the margin to scale by the same multiplicative factor, so the bound on the generalization gap in \eqref{lec7:eqn:generalization_loss} does not change. This lines up with our intuition that the bound should not depend on the scaling of the data.
\end{remark}

\subsec{Linear models with weights bounded in \texorpdfstring{$\ell_1$}{L1} norm}
Now, we consider linear models again, except we restrict the $\ell_1$-norm of the parameters and assume an $\ell_\infty$-norm bound on the data.

\begin{theorem}\label{lec7:thm:l1-thm}
    Let $\mathcal{H} = \cbr{x \mapsto \inprod{w,x} \mid w \in \R^d, \Norm{w}_1 \le B}$ for some constant $B > 0$. Moreover, assume $\Norm{x\sp{i}}_\infty \leq C$ for some constant $C > 0$ and all points in $S = \{x\sp{i}\}_{i=1}^n \subset \R^d$. Then
    \begin{equation}
        R_S(\mathcal{H}) \leq BC\sqrt{\frac{2\log(2d)}{n}}.
    \end{equation}
\end{theorem}

To prove the theorem, we will need Massart's lemma, which provides a bound for the Rademacher complexity of a finite hypothesis class.

    \begin{lemma}[Massart's lemma]
        Suppose $\mathcal{Q} \subset \R^n$ is finite and contained in the $\ell_2$-norm ball of radius $M\sqrt{n}$ for some constant $M > 0$, i.e.,
        \begin{equation}
            \mathcal{Q} \subset \{v \in \R^n \mid \Norm{v}_2 \leq M\sqrt{n} \}.
        \end{equation}
        Then, for Rademacher variables $\sigma = (\sigma_1,\sigma_2,\dots,\sigma_n) \in \R^n$,
        \begin{equation}
            \Exp_\sigma \left[ \sup_{v\in \mathcal{Q}} \frac{1}{n}\inprod{\sigma,v} \right] \leq M\sqrt{\frac{2\log|\mathcal{Q}|}{n}}.
        \end{equation}
        As a corollary, if $\mathcal{F}$ is a set of real-valued functions satisfying
        \begin{equation}
            \sup_{f\in\mathcal{F}} \frac{1}{n}\sum_{i=1}^n f(z\sp{i})^2 \leq M^2,
        \end{equation}
        over some data $S = \{z\sp{i}\}_{i=1}^n$, then
        \begin{align}
            R_S(\mathcal{F}) \leq M\sqrt{\frac{2\log|\mathcal{F}|}{n}}, \quad\text{and}\quad
            R_n(\mathcal{F}) \leq M\sqrt{\frac{2\log|\mathcal{F}|}{n}}.
        \end{align}
    \end{lemma}

We will not prove Massart's lemma in detail. The intuition is to use concentration inequalities to bound $\frac{1}{n}\inprod{\sigma, v}$ for fixed $v$, then to use a union bound over the elements $v \in \mathcal{Q}$.

We will now prove Theorem \ref{lec7:thm:l1-thm}:

\begin{proof}[Proof of Theorem \ref{lec7:thm:l1-thm}]
    By definition,
    \begin{align}
        R_S(\mathcal{H}) &= \Exp_\sigma\sbr{ \sup_{\Norm{w}_1 \le B} \frac{1}{n} \sum_{i=1}^n\sigma_i \inprod{w,x\sp{i}} } \\
        &= \frac{1}{n} \Exp_\sigma\sbr{ \sup_{\Norm{w}_1\le B} \inprod{w,\sum_{i=1}^n\sigma_i x\sp{i}} } \\
        &= \frac{B}{n} \Exp_\sigma\sbr{ \Norm{\sum_{i=1}^n \sigma_i  x\sp{i}}_\infty  },
    \end{align}
    
    where the last equality is because $\sup_{\Norm{w}_1 \leq B}\inprod{w,v} = B\Norm{v}_\infty$, i.e., the $\ell_\infty$-norm is the dual of the $\ell_1$-norm, which is a consequence of H\"older's inequality. However, the $\ell_\infty$-norm is difficult to simplify further. Instead, we use the fact that $\sup_{\Norm{w}_1 \leq 1} \inprod{w,v}$ for any $v \in \R^d$ is always attained at one of the vertices $\mathcal{W} = \bigcup_{i=1}^d \{-e_i,e_i\}$, where $e_i \in \R^d$ is the $i$-th coordinate unit vector. Defining the restricted hypothesis class $\bar{\mathcal{H}} = \{x \mapsto \inprod{w,x} \mid w \in \mathcal{W}\} \subset \mathcal{H}$, this yields
    \begin{align}
        R_S(\mathcal{H}) &= \frac{1}{n} \Exp_\sigma\sbr{ \sup_{\Norm{w}_1 \le B} \inprod{w,\sum_{i=1}^n\sigma_i x\sp{i}} } \\
        &= \frac{B}{n} \Exp_\sigma\sbr{ \max_{w\in\mathcal{W}} \inprod{w,\sum_{i=1}^n\sigma_i x\sp{i}} } \\
        &= BR_S(\bar{\mathcal{H}}).
    \end{align}
    
    In particular, the model class $\bar{\mathcal{H}}$ is bounded and finite with cardinality $|\bar{\mathcal{H}}| = 2d$. This suggests using Massart's lemma to complete the proof. To do so, we need to confirm that $\mathcal{\bar{H}}$ is bounded with respect to the $\ell_2$-metric. Indeed, since the inner product of $x\sp{i}$ with a coordinate vector $e_j$ just selects the $j$-th coordinate of $x\sp{i}$, for any $w \in \mathcal{W}$ we have
    \begin{equation}
        \frac{1}{n}\sum_{i=1}^n \inprod{w,x\sp{i}}^2 \leq \frac{1}{n}\sum_{i=1}^n \Norm{x\sp{i}}^2_\infty \leq \frac{1}{n}\sum_{i=1}^n C^2 = C^2,
    \end{equation}
    where the last inequality uses the assumption $\Norm{x_i}_\infty \leq C$. So $\bar{\mathcal{H}}$ is bounded in the $\ell_2$-metric and finite, thus by Massart's Lemma we have
    \begin{equation}
        R_S(\mathcal{H}) = B R_S(\bar{\mathcal{H}}) \leq BC\sqrt{\frac{2\log|\bar{\mathcal{H}}|}{n}} = BC\sqrt{\frac{2\log(2d)}{n}},
    \end{equation}
    which completes the proof.
\end{proof}

\subsec{Comparing the bounds for different \texorpdfstring{$\cH$}{H}}

First, we note that for this hypothesis class of linear models, it is possible to obtain an upper bound proportional to $\sqrt{d/n}$ using the VC~dimension, which grows quickly with the data dimension~$d$. Our bound is better since it does not have as strong of a dependence on~$d$, and accounts for the norms of our model parameters and the data.

In the two subsections above, we considered two different hypothesis classes of linear models, each restricting different norms. In both cases, the bound on the average Rademacher complexity depended on the product of the norm bound on the parameters $w$ and the norm bound on each data point $x$. To determine which choice of hypothesis class is better, consider the bounds
    \begin{equation*}
        \Norm{w}_2\Norm{x}_2 \quad\text{vs.}\quad \Norm{w}_1\Norm{x}_\infty
    \end{equation*}
    and see how they compare in different settings. We consider 3 settings here:
    
    \begin{itemize}
    \item Suppose $w$ and $x$ are random variables with $w_i$ and $x_i$ close to the set of values $\{-1,1\}$. Then we have
    \begin{equation*}
        \sqrt{d}\cdot \sqrt{d} \quad\text{vs.}\quad d\cdot 1.
    \end{equation*}
    In this case, there is no difference in using either linear hypothesis class.
    
    \item If we additionally suppose $w$ is sparse with at most $k$ non-zero entries, then we have
    \begin{equation*}
        \sqrt{k}\cdot\sqrt{d} \quad\text{vs.}\quad k\cdot 1.
    \end{equation*}
    So for $d \gg k$, we have $\sqrt{kd} \gg k$ and thus $\ell_1$-norm regularization leads to a better complexity bound when $w$ is suspected to be sparse. Indeed, $\sqrt{d}\Norm{x}_\infty \approx \Norm{x}_2$ when the entries of $x$ are somewhat uniformly distributed, and so in the sparse case we have
    \begin{equation}
        \Norm{w}_2\Norm{x}_2 \geq \sqrt{d}\Norm{w}_2\Norm{x}_\infty \geq \Norm{w}_1\Norm{x}_\infty. 
    \end{equation}
    
    \item On the other hand, if $w$ is dense in the sense that $\Norm{w}_2\approx {\sqrt{d}}\Norm{w}_1$ (i.e., if all entries in $w$ are close to each other in magnitude), then
    \begin{equation}
        \Norm{w}_2\Norm{x}_2 \leq \frac{1}{\sqrt{d}}\Norm{w}_1 \cdot \sqrt{d} \Norm{x}_\infty \leq \Norm{w}_1\Norm{x}_\infty.
    \end{equation}
    In this case, it makes sense to regularize the $\ell_2$-norm instead.
    \end{itemize}
    
    In practice, other multiplicative factors enter the generalization bound, so regularizing both the $\ell_1$- and $\ell_2$-norms of the model parameters $w$ is preferable.

    Continuing with this rough style of analysis, for the hypothesis class with restricted $\ell_2$-norm, we can write the bound on the generalization gap in \eqref{lec7:eqn:generalization_loss} as
    \begin{equation}
        \text{generalization loss} \lesssim \frac{\Norm{w}_2\Norm{x}_2}{\sqrt{n}\gamma_{\mathrm{min}}} + \text{low-order term}.
    \end{equation}
    The presence of $\Norm{w}_2/\gamma_{\mathrm{min}}$ motivates both the minimum norm and the maximum margin formulations of the Support Vector Machine (SVM) problem as good methods to improve generalization performance of binary classifiers.

%*****************************************************************************

%	\input{collection/05-03-deep-nets.tex}

\part{Deep Learning Theory}	

\chapter{Generalization Bounds for Neural Networks} \label{chap:gen-bounds-nns}
\tnote{todo: intro for chapter}
\sec{Two-layer neural networks}
We now compute a bound for the Rademacher complexity of two-layer neural networks.  Throughout this section, we use the following notation:
\begin{itemize}
    \item $\theta = (w, U)$ are the parameters of the model with $w \in \R^m$ and $U \in \R^{m \times d}$, where $m$ denotes the number of hidden units. We use $u_i\in\R^d$ to denote the $i$-th row of $U$ (written as a column vector).
    \item $\phi(z) = \max(z, 0)$ is the ReLU activation function applied element-wise.
    \item $f_\theta(x) = \inprod{w,\phi(Ux)} = w^\top \phi(Ux)$ is the model.
    \item $\{ (x\sp{i}, y\sp{i}) \}_{i=1}^n$ is the training set, with $x\sp{i}\in\R^d$ and $y\sp{i}\in\R$.
\end{itemize}
We start with a somewhat weak bound which introduces the technical tools we need to derive tighter bounds subsequently.

\begin{theorem}\label{lec7:thm:thm_3}
    For some constants $B_w > 0$ and $B_u > 0$, let
    \begin{equation}
        \mathcal{H} = \cbr{ f_\theta \mid \Norm{w}_2 \leq B_w,\ \Norm{u_i}_2 \leq B_u,\ \forall i \in \{1,2,\dots,m\} }, \label{lec7:eqn:thm_3}
    \end{equation}
    and suppose $\Exp\sbr{\Norm{x}_2^2} \leq C^2$. Then
    \begin{align}
        R_n(\mathcal{H}) \le 2 B_w B_u C\sqrt{\frac{m}{n}}.
    \end{align}
\end{theorem}

This bound is not ideal as it depends on the number of neurons~$m$. Empirically, it has been found that the generalization error does \emph{not} increase monotonically with~$m$. As more neurons are added to the model, thereby giving it more expressive power, studies have shown that generalization is improved \cite{belkin2019}. This contradicts the bound above, which states that more neurons leads to worse generalization. We also note that the theorem can be generalized straightforwardly to the setting where the $w$ and $U$ are jointly constrained in the sense that we set $\mathcal{H} = \cbr{ f_\theta \mid \Norm{w}_2\cdot \left(\max_i\Norm{u_i}_2\right) \leq B}$ and obtain the generalization bound $        R_n(\mathcal{H}) \le 2 B C\sqrt{\frac{m}{n}}.$ However, the $\sqrt{m}$ dependency still exists under this formulation of $\cH$. 
Nevertheless, we now derive this bound.

\begin{proof}
    By definition,
    \begin{align}
        R_S(\mathcal{H}) 
        &= \Exp_\sigma\sbr{ \sup_\theta \frac{1}{n} \sum_{i=1}^n \sigma_i \inprod{w,\phi(Ux\sp{i})} }
        \\&= \frac{1}{n} \Exp_\sigma\sbr{ \sup_{U : \Norm{u_j}_2 \leq B_u} \sup_{\Norm{w}_2 \leq B_w} \inprod{w,\sum_{i=1}^n \sigma_i \phi(Ux\sp{i})} }
        \\&= \frac{B_w}{n}\Exp_\sigma\sbr{ \sup_{U : \Norm{u_j}_2 \leq B_u} \Norm{ \sum_{i=1}^n \sigma_i \phi(Ux\sp{i})}_2 }
            &&\text{($\textstyle\sup_{\Norm{w}_2\leq B}\inprod{w,v} = B\Norm{v}_2$)}
        \\&\leq \frac{B_w\sqrt{m}}{n}\Exp_\sigma\sbr{ \sup_{U : \Norm{u_j}_2 \leq B_u} \Norm{ \sum_{i=1}^n \sigma_i \phi(Ux\sp{i})}_\infty }
            &&\text{($\Norm{v}_2 \leq \sqrt{m}\Norm{v}_\infty$)}
        \\&= \frac{B_w\sqrt{m}}{n}\Exp_\sigma\sbr{ \sup_{U : \Norm{u_j}_2 \leq B_u} \max_{1\leq j\leq m} \abs{ \sum_{i=1}^n \sigma_i \phi(u_j^\top x\sp{i})} } 
        \\&= \frac{B_w\sqrt{m}}{n}\Exp_\sigma\sbr{ \sup_{\Norm{u}_2 \leq B_u} \abs{ \sum_{i=1}^n \sigma_i \phi(u^\top x\sp{i})} }
        \\&\leq \frac{2B_w\sqrt{m}}{n}\Exp_\sigma\sbr{ \sup_{\Norm{u}_2 \leq B_u} \sum_{i=1}^n \sigma_i \phi(u^\top x\sp{i}) }
            &&\text{(by Lemma \ref{lec8:lemma:absfortwo})} \label{lec7:eqn:nn-proof1}
        \\&\leq \frac{2B_w\sqrt{m}}{n}\Exp_\sigma\sbr{ \sup_{\Norm{u}_2 \leq B_u} \sum_{i=1}^n \sigma_i u^\top x\sp{i} }, \label{lec7:eqn:nn-proof2}
    \end{align}
    where the last inequality follows by applying the contraction lemma (Talagrand's lemma) and observing that the ReLU function is $1$-Lipschitz. (Observe that the expectation in \eqref{lec7:eqn:nn-proof1} is the Rademacher complexity for $\{ x \mapsto \phi(u^\top x) \mid \Norm{u}_2 \leq B_u \}$: this is the family that we are applying the contraction lemma to.)
    
    We now observe that the expectation in \eqref{lec7:eqn:nn-proof2} is the Rademacher complexity of the family of linear models $\{x \mapsto \inprod{u,x} \mid \Norm{u}_2\leq B_u\}$. Thus, applying Theorem~\ref{lec7:thm:l1-thm} yields
    \begin{equation}
        R_S(\mathcal{H}) \leq \frac{2B_w\sqrt{m}}{n}B_u\sqrt{\sum_{i=1}^n \Norm{x\sp{i}}_2^2}.
    \end{equation}
    
    Taking the expectation of both sides and using similar steps to those in the proof of Theorem~\ref{lec7:thm:l1-thm} gives us
    \begin{align}
        R_n(\mathcal{H})  &= \Exp\left[ R_S(\mathcal{H})\right] \\
        &\leq \frac{2B_wB_u\sqrt{m}}{n} \Exp\sbr{\sqrt{\sum_{i=1}^n \Norm{x\sp{i}}_2^2}} \\
        &\leq \frac{2B_wB_u\sqrt{m}}{n} C\sqrt{n} \\
        &= 2 B_w B_u C\sqrt{\frac{m}{n}},
    \end{align}
    which completes the proof.
    
\end{proof}

This upper bound is undesirable since it grows with the number of neurons $m$, contradicting empirical observations of the generalization error decreasing with $m$.

%*****************************************************************************

\subsec{Refined bounds}
\newcommand{\boundsforcomp}{B}
Next, we look at a finer bound that results from defining a new complexity measure. A recurring theme in subsequent proofs will be the functional invariance of two-layer neural networks under a class of rescaling transformations. The key ingredient will be the \textit{positive homogeneity} of the ReLU function, i.e.
\begin{equation}
\alpha \phi(x) = \phi(\alpha x) \qquad \forall \alpha > 0.
\end{equation}
This implies that for any $\lambda_i > 0$ ($i = 1, \dots, m$), the transformation $\theta = \{(w_i, u_i)\}_{1 \leq i \leq m} \mapsto \theta' = \{(\lambda_i w_i,  u_i / \lambda_i )\}_{1 \leq i \leq m}$ has no net effect on the neural network's functionality (i.e. $f_{\theta} = f_{\theta'}$) since 
\begin{equation}
w_i\cdot \phi \left(u_i^\top x\sp i \right) = (\lambda_i w_i) \cdot \phi\l(\l( \frac{u_i}{\lambda_i}\r)^\top x\sp i\r).   
\end{equation}
In light of this, we devise a new complexity measure $C(\theta)$ that is also invariant under such transformations and use it to prove a better bound for the Rademacher complexity. This positive homogeneity property is absent in the complexity measure used in the hypothesis class \eqref{lec7:eqn:thm_3} of Theorem \ref{lec7:thm:thm_3}.

\begin{theorem}\label{lec8:thm:thm-improved-nn-rc}
$\operatorname{Let} C(\theta)=\sum_{j=1}^{m}\left|w_{j}\right|\left\|u_{j}\right\|_{2},$ and for some constant $\boundsforcomp>0$ consider the hypothesis class
\begin{equation}
\mathcal{H}=\left\{f_{\theta} \mid C(\theta) \leq \boundsforcomp\right\}. \label{eqn:H}
\end{equation}
If $\left\|x\sp{i}\right\|_{2} \leq C$ for all $i \in\{1, \ldots, n\},$ then
\begin{equation}
R_{S}(\mathcal{H}) \leq \frac{2 \boundsforcomp C}{\sqrt{n}}.
\end{equation}
\end{theorem}

\begin{remark}
	Compared to Theorem~\ref{lec7:thm:thm_3}, this bound does not explicitly depend on the number of neurons $m$. Thus, it is possible to use more neurons and still maintain a tight bound if the value of the new complexity measure $C(\theta)$ is reasonable. In contrast, the bound of Theorem \ref{lec7:thm:thm_3} explicitly grows with the total number of neurons. In fact, Theorem~\ref{lec8:thm:thm-improved-nn-rc} is strictly stronger than Theorem~\ref{lec7:thm:thm_3} as elaborated below. Note that 
	\begin{align}
		\sum |w_j|\|u_j\|_2 &\le \left(\sum |w_j|^2\right)^{1/2} \left(\sum\|u_j\|_2^2\right)^{1/2} \tag{by Cauchy-Schwarz inequality} \\
		& \le \|w\|_2 \cdot \sqrt{m} \cdot \max_{j}\|u_j\|_2
	\end{align}
	Therefore, if we consider $\cH^1 = \{f_\theta \mid \sum |w_j|\|u_j\|_2\le B'\}$ and $\cH^2 = \{f_\theta \mid \|w\|_2 \cdot \sqrt{m} \cdot \max_{j}\|u_j\|_2 \le B'\}$, then either Theorem~\ref{lec8:thm:thm-improved-nn-rc} on $\cH^1$ or Theorem~\ref{lec7:thm:thm_3} on $\cH^2$ gives the same generalization bound $O(B'/\sqrt{n})$, but $\cH^1 \supset \cH^2$. 
	
	Moreover, Theorem~\ref{lec8:thm:thm-improved-nn-rc} is stronger as we have more neurons---this is because the hypothesis class $\cH$ as defined in~\eqref{eqn:H} is bigger as $m$ increases. Because of this, it's possible to obtain a generalization guarantee that decreases as $m$ increases, as shown in Section~\ref{sec:gen-bounds:decreasing-in-m}. 
	
%	For example, consider solving the constrained problem
%	\begin{equation}
%	\rho_m = \min_\theta C(\theta) 
%	\quad \text{such that}\quad 
%	\text{$f_\theta$ fits the data  $\{(x\sp{i}, y\sp{i})\}_{i=1}^n$.}
%	\end{equation}
%	In this case, $\rho_m$ monotonically decreases as the number of neurons $m$ increases. Indeed, models with more parameters necessarily include models with a lower number of parameters and thus those of lower complexity.  As a result, it is possible to obtain lower complexity models by increasing the number of parameters $m$.
\end{remark}

\begin{proof}[Proof of Theorem~\ref{lec8:thm:thm-improved-nn-rc}]
Due to the positive homogeneity of the ReLU function $\phi$, it will be useful to define the $\ell_2$-normalized weight vector $\bar{u}_j \defeq u_j / \norm{u_j}_2$ so that $\phi\left(u_j^\top x\right) = \norm{u_j}_2 \cdot \phi(\bar{u}_j^\top x)$. The empirical Rademacher complexity satisfies
\allowdisplaybreaks
\al{
R_S(\cH) &= \frac{1}{n}\Exp_{\sigma}\left[ \sup_{\theta} \sum_{i=1}^n \sigma_i f_{\theta}\left(x\sp{i}\right) \right] \\
&= \frac{1}{n}\Exp_{\sigma}\left[ \sup_{\theta} \sum_{i=1}^n \sigma_i \left[\sum_{j=1}^m w_j \phi\left(u_j ^ T x\sp{i}\right) \right] \right] &&\text{(by dfn of $f_\theta$)} \\
&=  \frac{1}{n}\Exp_{\sigma}\left[ \sup_{\theta} \sum_{i=1}^n \sigma_i \left[\sum_{j=1}^m w_j \norm{u_j}_2  \phi\left(\bar{u}_j ^ T x\sp{i}\right) \right] \right]  
    && \text{(by positive homogeneity of $\phi$)}\\
&= \frac{1}{n}\Exp_{\sigma}\left[ \sup_{\theta}  \sum_{j=1}^m w_j \norm{u_j}_2 \left[ \sum_{i=1}^n \sigma_i  \phi\left(\bar{u}_j ^ T x\sp{i}\right) \right] \right] \\ 
&\leq \frac{1}{n}\Exp_{\sigma}\left[ \sup_{\theta}  \sum_{j=1}^m |w_j| \norm{u_j}_2 \max_{k \in [n]}\left| \sum_{i=1}^n \sigma_i  \phi\left(\bar{u}_k ^ T x\sp{i}\right) \right| \right] && \l(\because \sum_j \alpha_j \beta_j \leq \sum_j |\alpha_j| \max_{k} |\beta_k|\r) \\ 
&\leq \frac{\boundsforcomp}{n} \Exp_{\sigma}\sbr{ \sup_{\theta = (w, U)} \max_{k \in [n]} \left| \sum_{i=1}^n \sigma_i  \phi\left(\bar{u}_k ^ T x\sp{i}\right) \right| } && \text{($\because C(\theta) \leq \boundsforcomp$)} \\
&=  \frac{\boundsforcomp}{n} \Exp_{\sigma}\sbr{ \sup_{\bar{u}: \norm{\bar{u}}_2 = 1} \left| \sum_{i=1}^n \sigma_i  \phi\left(\bar{u} ^ T x\sp{i}\right) \right| } \\
&\le \frac{\boundsforcomp}{n} \Exp_{\sigma}\sbr{ \sup_{\bar{u}: \norm{\bar{u}}_2 \le 1} \left| \sum_{i=1}^n \sigma_i  \phi\left(\bar{u} ^ T x\sp{i}\right) \right| } \\
&\le \frac{2\boundsforcomp}{n}  \Exp_{\sigma}\sbr{ \sup_{\bar{u}: \norm{\bar{u}}_2 \le 1} \sum_{i=1}^n \sigma_i  \phi\left(\bar{u} ^ T x\sp{i}\right) } && \text{(see Lemma \ref{lec8:lemma:absfortwo})} \\
&= 2\boundsforcomp R_S(\cH '),
}
where $\cH' = \l\{x \mapsto \phi(\bar{u}^\top x) :  \bar{u} \in \mathbb{R}^d, \norm{\bar{u}}_2 \leq 1 \r\}$. By Talagrand's lemma, since $\phi$ is $1$-Lipschitz, $R_S(\cH') \leq R_S(\cH'')$ where  $\cH'' = \l\{x \mapsto \bar{u}^\top x :  \bar{u} \in \mathbb{R}^d, \norm{\bar{u}}_2 \leq 1 \r\}$ is a linear hypothesis space. Using $R_S(\cH'') \leq \frac{C}{\sqrt{n}}$ by Theorem \ref{lec7:thm:l2-thm} then concludes the proof.

\end{proof}

We complete the proof by deriving the Lemma \ref{lec8:lemma:absfortwo} used in the second-to-last inequality. Notably, the lemma's assumption holds in the current context, since
\al{
\sup_{\theta} \langle \sigma, f_{\theta}(x) \rangle = \sup_{\bar{u}: \norm{\bar{u}}_2 \leq 1} 
\sum_{i=1}^n \sigma_i \phi \l(\bar{u}^\top x\sp i \r)  \geq 0.
}
since one can take $\bar{u} = 0$ for any $\sigma = (\sigma_1, \dots, \sigma_n)$.

\begin{lemma}\label{lec8:lemma:absfortwo}
Let $\sigma = (\sigma_1, ..., \sigma_n)$ and $f_{\theta}(x) = \l(f_{\theta}\l(x\sp{1}\r), ...,  f_{\theta}\l(x\sp{n} \r)\r)$. Suppose that for any $\sigma \in \{\pm 1\}^n$, $\sup_{\theta} \langle \sigma, f_{\theta}(x) \rangle \geq 0$. Then, 
\begin{equation}
\mathbb{E}_{\sigma}\l[ \sup_{\theta}  \l | \langle \sigma, f_{\theta}(x) \rangle \r|  \r] \leq 2 \mathbb{E}_{\sigma}\l[ \sup_{\theta}  \langle \sigma, f_{\theta}(x) \rangle   \r].
\end{equation}
\end{lemma}

\begin{proof}
Letting $\phi$ be the ReLU function, the lemma's assumption implies that $\sup_{\theta} \phi\left(\langle \sigma, f_{\theta}(x) \rangle\right) = \sup_{\theta}\langle \sigma, f_{\theta}(x) \rangle$ for any $\sigma \in \{\pm 1\}^n$. Observing that $|z| = \phi(z) + \phi(-z)$, 
\begin{align}
\sup_{\theta} \abs{\inprod{ \sigma, f_{\theta}(x) }}%
&= \sup_{\theta} \left[ \phi \l(\inprod{ \sigma, f_{\theta}(x) } \r) + \phi \l(\inprod{-\sigma, f_{\theta}(x) } \r)\right] \\
&\le \sup_{\theta}  \phi \l(\inprod{ \sigma, f_{\theta}(x) } \r) +  \sup_{\theta}  \phi \l(\inprod{-\sigma, f_{\theta}(x) } \r)  \\
&= \sup_{\theta} \inprod{ \sigma, f_{\theta}(x) } +  \sup_{\theta}  \inprod{-\sigma, f_{\theta}(x) }. 
\end{align}
Taking the expectation over $\sigma$ (and noting that $\sigma \overset d = -\sigma$), we get the desired conclusion.
\end{proof}



\sec{More implications and discussions on two-layer neural nets}
In this section, we discuss practical implications of the refined neural network bound. 

\subsec{Connection to \texorpdfstring{$\ell_2$}{L2} regularization}\label{sec:gen-bounds:impliciation}

Recall that margin theory yields
\begin{equation}
\text{for all } \theta, \quad \Err(\theta) \leq \frac{2R_S(\cH)}{\gammamin} + \tilO\l(\sqrt{\frac{\log \l( 2 / \delta \r)}{n}}\r), \label{lec8:eqn:margin-bound}
\end{equation}
with probability at least $1 -\delta$. Thus, Theorem \ref{lec8:thm:thm-improved-nn-rc} motivates us to minimize $\frac{R_S(\cH)}{\gammamin}$ by regularizing $C(\theta)$. Concretely, this can be formulated as the optimization problem 
\al{
\text{minimize} & \qquad C(\theta) = \sum_{j=1}^m |w_j|\cdot \norm{u_j}_2 \nonumber \tag{I} \label{lec8:eqn:opt1} \\ 
\text{subject to} & \qquad \gammamin(\theta)\ge 1, \nonumber
}
or equivalently,
\al{
\text{maximize} & \qquad \gammamin(\theta) \nonumber \tag{II} \label{lec8:eqn:opt2} \\ 
\text{subject to} & \qquad C(\theta)\le 1. \nonumber
}

At first glance, the above seems orthogonal to techniques used in practice. However, it turns out that the optimal neural network from \eqref{lec8:eqn:opt1} is functionally equivalent to that of the new problem:
\al{
\text{minimize} & \qquad C_{\ell_2}(\theta) = \frac{1}{2}\sum_{j=1}^m |w_j|^2 + \frac{1}{2}\sum_{j=1}^m \norm{u_j}_2^2 \nonumber \tag{I*} \label{lec8:eqn:opt1star} \\ 
\text{subject to} & \qquad \gammamin(\theta)\ge 1. \nonumber
}
This is a simple consequence of the positive homogeneity of $\phi$. For any scaling factor $\lambda=(\lambda_1, \dots, \lambda_m)\in \R_+^m$, the rescaled neural network $\theta_\lambda \defeq \{(\lambda_i w_i, u_i/\lambda_i)\}$ has the same functionality as the original neural network $\theta = \{w_i, u_i \}$ (i.e. it achieves the same $\gammamin$). Thus, 
\al{
\min_{\theta} C_{\ell_2}(\theta) &= \min_{\theta} \min_{\lambda} \rbr{ \frac{1}{2}\sum_{j=1}^m \lambda_j^2 |w_j|^2 + \frac{1}{2}\sum_{j=1}^m \lambda_j^{-2}\norm{u_j}_2^2 }\\
&= \min_{\theta}  \sum_{j=1}^m |w_j|\cdot \norm{u_j}_2 \\
&= \min_{\theta}  C(\theta)
}
where we have used the equality case of the AM-GM inequality, attainable by $\lambda_j^* = \sqrt{\frac{\norm{u_j}_2}{|w_j|}}$, in the second step. This equality case also shows that $\theta^* = \{(w_i, u_i ) \}$ is the optimal solution of \eqref{lec8:eqn:opt1} if and only if $\hat{\theta}^* = \theta_{\lambda^*}$ is the optimal solution of \eqref{lec8:eqn:opt1star}---proving that $\hat{\theta}^*$ and $\theta^*$ are functionally equivalent since they only differ by a positive scale factor. 

This connects our $C(\theta)$ regularization to $\ell_2$-norm penalties that are more prevalent in practice. In retrospect, we see this equivalence is essentially due to the positive homogeneity of the neural network which ``homogenizes'' any inhomogeneous objective such as $C_{\ell_2}$. Hence, we can just deal with $C(\theta)$ which is transparently homogeneous.

\subsec{Generalization bounds that are decreasing in \texorpdfstring{$m$}{m}} \label{sec:gen-bounds:decreasing-in-m}

Next, we show that the generalization bound given by Theorem \ref{lec8:thm:thm-improved-nn-rc} does not deteriorate with the network width (number of neurons) $m$, which is consistent with experimental results. To this end, the perspective of \eqref{lec8:eqn:opt2} enables us to isolate all dependencies of $m$ in $\gammamin$. Letting $\widehat \theta_m$ denote the minimizer of program \eqref{lec8:eqn:opt2} with width $m$ and defining optimal value $\gamma_m^* = \gammamin\l(\widehat \theta_m\r)$, we can rewrite the margin bound \eqref{lec8:eqn:margin-bound} as 
\begin{equation}
L(\widehat \theta_m) \le \frac{4C}{\sqrt{n}} \cdot \frac{1}{\gamma_m^*} + \text{(lower-order terms)},
\end{equation}
where all dependencies on $m$ are now contained in $\gamma_m^*$. Hence, to show that this bound does not worsen as $m$ grows, we just have to show that $\gamma_m^*$ is non-decreasing in $m$. This is intuitively the case since a neural network of width $m+1$ contains one of width $m$ under the same complexity constraints. The following theorem formalizes this hunch:

\begin{theorem}
Let $\gamma_m^*$ be the minimum margin obtained by solving \eqref{lec8:eqn:opt2} with a two-layer neural network of width $m$. Then $\gamma_m^* \leq \gamma_{m+j}^*$ for all positive integers $j$.
\end{theorem}

\begin{proof}
Suppose $\theta = \{(w_i, u_i)\}_{1 \leq i \leq m}$ is a two-layer neural network of width $m$ satisfying $C(\theta)\le 1$. Then we may construct a neural network $\widetilde \theta = \{(\tilde w_i, \tilde u_i)\}_{1 \leq i \leq m+1}$ of width $m+1$ by simply taking
\al{
(\widetilde w_i, \widetilde u_i) = \begin{cases}
(w_i, u_i) & i\le m, \\
(0,0) & \text{otherwise.}
\end{cases}
}
$\widetilde \theta$ is functionally equivalent to $\theta$ and $C(\widetilde \theta) = C(\theta) \le 1$. This means maximizing $\gammamin$ over $\{C(\widetilde \theta): \widetilde \theta\text{ of width }m+1\}$ should give no lower of a value than the maximum of $\gammamin$ over $\{C(\theta): \theta\text{ of width }m\}$.
\end{proof}

\subsec{Equivalence to an \texorpdfstring{$\ell_1$}{L1}-SVM in \texorpdfstring{$m \to \infty$}{m -> inf} limit}

Since $\gamma_m^*$ is non-decreasing in $m$, the quantity 
\begin{equation}
\gamma_\infty ^* = \lim_{m\to \infty } \gamma_m^*
\end{equation}
is well-defined. The next interesting fact is that in this $m \to \infty$ limit, $\gamma_{\infty}^*$ of the two-layer neural network is equivalent to the minimum margin of an $\ell_1$-SVM. As a brief digression, we recap the formulation of $\ell_p$-SVMs and discuss the importance of $\ell_1$-SVMs in particular.

Since a collection of data points with binary class labels may not be a priori separable, a \textit{kernel model} first transforms an input $x$ to $\varphi(x)$ where $\varphi: \mathbb{R}^d \to \mathcal{G}$ is known as the \textit{feature map}. The model then seeks a separating hyperplane in this new (extremely high-dimensional) feature space $\mathcal{G}$, parameterized by a vector $\mu$ pointing from the origin to the hyperplane. The prediction of the model on an input $x$ is then a decision score that quantifies $\varphi(x)$'s displacement with respect to the hyperplane:
\begin{equation}
g_{\mu, \varphi}(x) \defeq \l\langle \mu, \varphi(x) \r\rangle.
\end{equation}
Motivated by margin theory, it is desirable to seek the maximum-margin hyperplane under a constraint on $\mu$ to guarantee the generalizability of the model. In particular, a kernel model with an $\ell_p$-constraint seeks to solve the following program:
\al{
\text{maximize} & \qquad \gamma_{min} \coloneqq \min_{i \in [n]} y\sp{i}\langle \mu, \varphi(x\sp{i}) \rangle \\ 
\text{subject to} & \qquad \norm{\mu}_p \le 1. \nonumber
}
Observe that both the prediction and optimization of the feature model only rely on inner products in $\mathcal{G}$. The ingenuity of the SVM is to choose maps $\varphi$ such that $K(x, x') = \l\langle \varphi(x), \varphi(x') \r\rangle$ can be directly computed in terms of $x$ and $x'$ in the original space $\mathbb{R}^d$, thereby circumventing the need to perform expensive inner products in the large space $\mathcal{G}$. Remarkably, this ``kernel trick'' enables us to even operate in an implicit, infinite-dimensional $\mathcal{G}$. 

The case of $p=1$ is particularly useful in practice as $\ell_1$-regularization generally produces sparse feature weights (the constrained parameter space is a polyhedron and the optimum tends to lie at one of its vertices). Hence, $\ell_1$-regularization is an important feature selection method when one expects only a few dimensions of $\cG$ to be significant. Unfortunately, the $\ell_1$-SVM is not kernelizable due to the kernel trick relying on $\ell_2$-geometry, and is hence infeasible to implement. However, our next theorem shows that a two-layer neural network can approximate a particular $\ell_1$-SVM in the $m \to \infty$ limit (and in fact, for finite $m$). For the sake of simplicity, we sacrifice rigor in defining the space $\mathcal{G}$ and convey the main ideas.

\begin{theorem}\label{lec8:thm:thm8.5}
Define the feature map $\phirelu: \mathbb{R}^d \to \mathcal{G}$ such that $x$ is mapped to $\phi(u^\top x)$ for all vectors $u$ on the $d-1$-dimensional sphere $\mathcal{S}^{d-1}$. Informally, 
$$\phirelu(x) \defeq \begin{bmatrix} \vdots \\ \phi(u^\top x) \\ \vdots \end{bmatrix}_{u\in S^{d-1}}$$
is an ``infinite-dimensional vector'' that contains an entry $\phi(u^\top x)$ for each vector $u \in \mathcal{S}^{d-1}$, and we let $\phirelu(x)[u]$ denote the ``$u$''-th entry of this vector. Noting that $\mathcal{G}$ is the space of functions which can be indexed by $u \in S^{d-1}$, the inner product structure on $\mathcal{G}$ is defined by $\langle f, g \rangle = \int_{S^{d-1}} f[u]g[u] du$.

Under this set-up, we have
\begin{equation}
\gamma_{\infty}^* = \gamma_{\ell_1}^*,
\end{equation}
where $\gamma_{\ell_1}^*$ is the minimum margin of the optimized $\ell_1$-SVM with $\varphi = \phirelu$.
\end{theorem}

\begin{proof}

We will only prove the $\gamma_{\infty}^* \leq \gamma_{\ell_1}^*$ direction. (The $\gamma_{\infty}^* \geq \gamma_{\ell_1}^*$ direction requires substantial functional analysis.)

Suppose $\gamma_\infty^*$ is obtained by network weights $(w_1,w_2, \cdots), (u_1, u_2, \cdots)$ where $w_i\in \R, u_i\in \R^d$ (there is a slight subtlety here to be rectified later). Define renormalized versions of $\{w_i\}$ and $\{u_i\}$:
\begin{equation}
\widetilde w_i \defeq w_i\cdot \norm{u_i}_2, \qquad \overline u_i \defeq \frac {u_i} {\norm{u_i}_2}.   
\end{equation}
Note that $\{(\widetilde w_i, \overline u_i)\}$ has the same functionality (and also the same complexity measure $C(\theta)$, margin, etc.) as that of $\{(w_i,u_i)\}$, but now $\overline u_i$ has unit $\ell_2$-norm (i.e. $\bar{u}_i \in \mathcal{S}^{d-1}$). Thus, $\phi(\overline u_i ^\top x)$ can be treated as a feature in $\cG$ and we can construct an equivalent $\ell_1$-SVM (denoted by $\mu$) such that $\widetilde w_i$ is the coefficient of $\mu$ associated with that feature. Since $\widetilde w_i$ must only be ``turned on'' at $\overline u_i $, we have 
\al{
\mu[u] = \sum_{i \in \mathcal{S}^{d-1}} \tilde{w}_i \delta(u - \overline u_i),
}
where $ \delta(u)$ is the Dirac-delta function. Given this $\mu$, we can check that the SVM's prediction is
\al{
g_{\mu, \phirelu}(x) &= \int_{S^{d-1}} \mu[u] \phirelu(x)[u] du \\
&= \int_{S^{d-1}}   \sum_{i \in \mathcal{S}^{d-1}} \tilde{w}_i \delta(u - \overline u_i) \phi\left(\overline u ^\top x\right) du \\
&= \sum_{i \in \mathcal{S}^{d-1}}  \tilde{w}_i \phi\left(\overline u_i ^\top x\right) ,
}
which is identical to the output $f_{\{(\widetilde w_i, \overline u_i)\}}(x)$ of the neural network. Furthermore, 
\al{
\norm{\mu}_1 =  \sum_{i=1}^{\infty} |\widetilde w_i| = \sum_{i=1}^{\infty} |w_i|\cdot \norm{u_i}_2 \leq 1,
}
where the last equality holds because $\{(\widetilde w_i, \overline u_i)\}$ satisfies the constraints of \eqref{lec8:eqn:opt2}. This shows that our constructed $\mu$ satisfies the $\ell_1$-SVM constraint. Thus, $\gamma_{\infty}^* \leq \gamma_{\ell_1}^*$ since the functional behavior of the optimal neural network is contained in the search range of the SVM.

\end{proof}

\begin{remark}
How well does a finite-dimensional neural network approximate the infinite-dimensional $\ell_1$ network? Proposition B.11 of \cite{wei2020regularization} shows that you only need $n+1$ neurons. Another way to say this is that $\{\gamma_m\}$ stabilizes once $m=n+1$:
\begin{equation}
\gamma_1^* \le \gamma_2^* \le \dots \le \gamma_{n+1}^* = \gamma_\infty^*.
\end{equation}
The main idea of the proof is that if we have a neural net $\theta$ with $(n+2)$ neurons, then we can always pick a simplification $\theta'$ with $(n+1)$ neurons such that $\theta,\theta'$ agree on all $n$ datapoints.

As an aside, this result also resolves the issue in our partial proof. A priori, $\gamma_{\infty}^*$ may not necessarily be attained by a set of weights $\{(\widetilde w_i, \overline u_i)\}$, but the above shows that it is achievable with just $n+1$ non-zero indices.

\end{remark}


\sec{Deep neural nets (via covering number)}\label{sec:deep_nets}
In Section~\ref{lec9:sec:cover_to_radem}, we discuss how strong our bounds on covering number need to be in order to get a useful result. 
Here we describe some situations in which we know how to obtain these covering number bounds for concrete models such as linear models and neural networks. 

\subsec{Preparation: covering number for linear models}
First, consider the following covering number bound for linear models:

\begin{theorem}[\cite{zhang2002}] \label{lec9:thm:univariate_rad}
	Suppose $x^{(1)}, \cdots, x^{(n)} \in \mathbb{R}^d$ are $n$ data points, and $p, q$ satisfies $1/p + 1/q = 1$ and $2 \le p \le \infty$. Assume that $||x^{(i)}||_p \le C$ for all $i$. Let:
	\begin{align}
	\cF_q = \{x \mapsto w^\top x : ||w||_q \le B\}
	\end{align}
	and let $\rho = L_2(P_n)$. Then, $\log N(\epsilon, \cF_q, \rho) \le \l [\frac{B^2C^2}{\epsilon^2}\r ] \log_2 (2d + 1)$. When $p = 2, q = 2$, we further obtain that:
	\begin{align}
	\log N(\epsilon, \cF_2, \rho) \le \l [\frac{B^2C^2}{\epsilon^2} \r ] \log_2 (2 \min (n, d ) + 1)
	\end{align}
\end{theorem}
\begin{remark}
	Applying \eqref{lec9:eqn:rademacherbound_three} to the covering number bound derived above with $R = B^2C^2$, we conclude that the Rademacher complexity of this class of linear models satisfies
	\begin{align}
	R_S(\cF_q) &\le \tilO{\left( \frac{BC}{\sqrt{n}} \right)}.
	\end{align} 
	We also prove this result without relying on Dudley's theorem in Theorem~\ref{lec7:thm:l2-thm}.
\end{remark}
Next, we consider multivariate linear functions as they are building blocks for multi-layer neural networks. Let $M = (M_1, \cdots, M_n) \in \mathbb{R}^{m \times n}$ and $\norm{M}_{2,1} = \sum_{i = 1}^n \norm{M_i}_2$. Then, $\norm{M^\top}_{2,1}$ denotes the sum of the $\ell_2$ norms of the rows of $M$. 
\begin{theorem}\label{lec9:thm:multivariate_rad}
	Let $\cF = \{x \to Wx : W \in \mathbb{R}^{m \times d}, ||W^\top||_{2, 1} \le B\}$ and let $C = \sqrt{\frac{1}{n} \sum_{i = 1}^n ||x^{(i)}||_2^2}$. Then, 
	\begin{equation}
	\log N(\epsilon, \cF, L_2(P_n)) \le \l [\frac{c^2B^2}{\epsilon^2} \r ] \ln (2dm).
	\end{equation}
\end{theorem}
\begin{remark}
	In some sense, Theorem~\ref{lec9:thm:multivariate_rad} arises from treating each dimension of the multivariate problem independently. We can view the linear layer as applying $m$ different linear functions. Explicitly, if $W = \begin{pmatrix} w_1^\top \\ \vdots \\ w_m^\top \end{pmatrix}$ and $Wx = \begin{pmatrix} w_1^\top x \\ \vdots \\ w_m^\top x \end{pmatrix}$, then as we expect, $\norm{W^\top}_{2,1} = \sum \norm{w_i}_2$.
\end{remark}


\subsec{Deep neural networks}
In this lecture, we discuss a bound on the Rademacher complexity of a dense neural network. We set up notation as follows: $W_i$ denotes the linear weight matrix at the $i$-th layer of the neural network, we have a total of $r$ layers, and $\sigma$ is the activation function which is 1-Lipschitz (for example, ReLU, softmax, or sigmoid). If the input is a vector $x$, the neural network's output can be represented as follows:

\begin{align}
f_\theta(x) = W_r\sigma(W_{r-1}\sigma(\cdots \sigma(W_1x)\ldots)),
\end{align}
Using this notation, we establish an upper bound on the Rademacher complexity of a dense neural network.

\begin{theorem}[\cite{bartlett2017}]
	\label{lec10:thm:dnn_rademacher}
	Suppose that $\forall i, \norm{x^{(i)}}_2 \leq c$ and let
	\begin{align}
	\cF = \{f_\theta : \norm{W_i}_{\textup{op}} \leq \kappa_i, \norm{W_i^\top}_{2,1} \leq b_i\}.
	\end{align}
	Then,
	\begin{equation}
	R_S (\cF) \leq \frac{c}{\sqrt{n}} \cdot \underbrace{\left(\prod_{i=1}^r \kappa_i \right)}_{\textup{(I)}} \cdot \underbrace{\left( \sum_{i=1}^r\frac{b_i^{2/3}}{\kappa_i^{2/3}}\right)^{3/2}}_{\textup{(II)}}. \label{lec10:eqn:bartlett_rad_bound}
	\end{equation}
\end{theorem}
We use $\norm{W}_{\textup{op}}$ to denote the operator norm (or spectral norm) of $W$, and recall that $\norm{W_i^\top}_{2,1}$ denotes the sum of the $\ell_2$ norms of the rows of $W_i$. Examining \eqref{lec10:eqn:bartlett_rad_bound}, we see that (II) is relatively small as it is a sum of matrix norms, and so the bound is dominated by (I), which is a product of matrix norms.

\begin{remark}
	We note that $f(x) = Wx$ is Lipschitz with a Lipschitz constant of $\norm{W}_{\textup{op}}$. This is because 
	\begin{align}
	\norm{f(x)-f(y)}_2 &= \norm{Wx-Wy}_2 \\
	&\leq \norm{W}_{\textup{op}}\norm{x-y}_2 &\text{$(\norm{W}_{\textup{op}} = \max_{x:\norm{x}_2=1}\norm{Wx}_2)$}
	\end{align}. 
\end{remark}

\begin{remark}
	As a corollary of the above theorem, we also get a bound on the generalization error for the margin loss of the following form:
	\begin{equation}
	\textup{generalization error} \leq \tilde{O}\left(\frac{1}{\gamma_{\min}} \cdot \frac{1}{\sqrt{n}} \cdot \left(\prod_{i=1}^r\norm{W_i}_{\textup{op}} \right) \cdot {\left( \sum_{i=1}^r\frac{\norm{W_i^\top}^{2/3}_{2,1}}{\norm{W_i}_{\textup{op}}^{2/3}}\right)^{3/2}}  \right),
	\end{equation}
	where $\gamma_{\min}$ denotes the margin.
\end{remark}

First, we motivate the proof by presenting the main idea, and then work through each part of the proof. The main ideas of the proof can be summarized as follows:

\begin{itemize}
	\item At a high level, we want to show that the covering number $N(\epsilon, \cF, \rho)$ for a dense neural network is $\leq \frac{R}{\epsilon^2}$. Proving this would enable us to apply Theorem~\ref{lec9:thm:better-dudley} to get a bound on the Rademacher Complexity.
	\item To bound the covering number for a dense neural network, we use $\epsilon$-covers to cover each layer of $f_\theta$ separately, and then combine them to prove that there exists an $\epsilon$-cover of the original function $f_\theta$. 
	\item To combine the $\epsilon$-covers of each layer, we use the Lipschitzness of each layer.
	\item We control and approximate the error propagation that is introduced through discretizing each layer using $\epsilon_i$-coverings in order to get a reasonable final $\epsilon$.
\end{itemize}

As a prelude to the proof of Theorem~\ref{lec10:thm:dnn_rademacher}, let us abstractify each layer of $\cF$ as $\cF_i$ where $\cF_i$ corresponds to matrix multiplication by $W_i$ composed with a nonlinear activation function $\sigma$. We then denote $\cF$ as the composition of each of these (single layer) function spaces as follows:
\begin{align}
\cF = \cF_r \circ \cF_{r - 1} \circ \cdots \circ \cF_1 = \{f_r \circ f_{r - 1} \circ \cdots f_{1} : f_i \in \cF_i\}
\end{align}
We will assume throughout that $f_i$ is $\kappa_i$-Lipschitz, i.e.
\begin{align}
\norm{f_i(x) - f_i(y)}_2 \leq \kappa_i \norm{x - y}_2 \label{lec10:eqn:lipschitz-def}
\end{align} 
Let us also assume, for simplicity, that $f_i(0) = 0$ and $\norm{x\sp{j}}_2 \leq c$ for all $j = 1,\dots,n$. Then, by applying the definition of Lipschitz continuity, we obtain that:
\begin{align}
\norm{f_i(f_{i - 1}(\cdots(f_1(x\sp{j}))))}_2 \leq \underbrace{\kappa_{i} \cdot \kappa_{i - 1} \cdots \kappa_1 \cdot c}_{\defeq c_i}
\end{align}

We now derive an $\epsilon$-covering of $\cF$ in two steps:
\begin{enumerate}
	\item Given inputs to the $i^{th}$ layer, we construct an $\epsilon_i$-covering of the output space of the function $f_i$.
	\item Using the $\epsilon_i$-covering as inputs to the $(i + 1)$-th layer, we show that we can use several single layer coverings to construct an $\epsilon$-covering for a multilayer network.
\end{enumerate}

Formally, the following lemma answers the second step in the above outline. Namely, given a covering number for a single layer, we show how to compute a covering number bound for multiple layers.
\begin{lemma}
	Under the setup given above, if every input to $f_i$ satisfies $\norm{z\sp{j}}_2 \leq c_{i - 1}$, we assume that  
	\begin{align}
	\log N(\epsilon_i, \cF_i, L_2(P_n)) \leq g(\epsilon_i, c_{i - 1}).\footnotemark \label{lec10:eqn:single_cover_bound}
	\end{align}
	\footnotetext{If $\cF_i$ defines a collection of linear models, then $\log N(\epsilon_i, \cF_i, L_2(P_n)) \leq \l \lceil \frac{c_{i - 1}^2}{\epsilon_i^2} \r \rceil$.}
	Then, there exists an $\epsilon$-cover $\cC$ of $\cF_r \circ \cdots \circ \cF_1$ for $\epsilon = \epsilon_r + \kappa_r\epsilon_{r-1} + \cdots + \kappa_r\kappa_{r-1}\dots\kappa_2\epsilon_1$ such that
	\begin{align}
	\log \abs{\cC} \leq \sum_{i=1}^{r} g\left(\epsilon_i, c_{i-1}\right)
	\end{align}
	\label{lec10:lma:additive_cover}
\end{lemma}
\begin{figure}[ht!]
	\begin{center}
		\includegraphics[width=\textwidth]{figures/multilayer_covering.png}
	\end{center}
	\caption{We visualize the covering strategy adopted in the proof of Lemma~\ref{lec10:lma:additive_cover}. The two grey sets depict the output spaces of the first and second layers, namely, $\cQ_1$ and $\cQ_2$, respectively. The blue dots in $\cQ_1$ are the outputs of three functions in the $\epsilon_1$-cover $\cC_1$, while the blue subsets of $\cQ_2$ depict $\cF_2 \circ f_1'$ and $\cF_2 \circ f_1''$. The red circles show how we construct a covering, $\cC_2$, of $\cQ_2$. In particular, the two collections of red circles depict the $\cC_{2, f_1'}$ and $\cC_{2, f_1''}$ covers. Taking the union of such covers over all functions in $\cC_1$ yields $\cC_2$.}
	\label{lec10:fig:multilayer-covering}
\end{figure}
\begin{proof}
	Let $\epsilon_1,\dots,\epsilon_r$ be the radius for each layer. Let $\cC_1$ be an $\epsilon_1$-cover of $\cF_1$. Then, for all $f_1' \in \cC_1$, we define $\cC_{2, f_1'}$ as an $\epsilon_2$-covering of  the set 
	\begin{equation}
	\cF_2 \circ f_1' = \left\{f_2\left(f_1'\left(X\right)\right) : f_2 \in \cF_2 \right\}.
	\end{equation}
	Taking a union of this covering over all $f_1' \in \cC_1$ clearly yields an $\epsilon$-covering for $\cF_2 \circ \cF_2$. In paricular, if 
	\begin{align}
	\cC_2 = \bigcup_{f_1'\in \cC_1}\cC_{2,f_1'},
	\end{align} 
	then $\cC_2$ is an $\epsilon$-cover of $\cF_2 \circ \cF_1$ with $\epsilon = \epsilon_1 \cdot \kappa_2 + \epsilon_2$. We depict this covering procedure in Figure~\ref{lec10:fig:multilayer-covering}, and we prove this claim rigorously in the sequel.
	
	Next, we bound the sizes of these covers. Directly applying the assumption given by \eqref{lec10:eqn:single_cover_bound}, we conclude that
	\begin{align}
	\log \abs{\cC_{2, f_1'}} \leq g\left(\epsilon_2, c_1\right).
	\end{align}
	Then, because $\cC_2 = \bigcup_{f_1'\in \cC_1}\cC_{2,f_1'}$, it immediately follows that
	\begin{align}
	\abs{\cC_{2}} &\leq \abs{\cC_{1}} \exp\left(g\left(\epsilon_2, c_1\right)\right) \label{lec10:eqn:iterative_cover_bound-1}\\
	\log\abs{\cC_{2}} &\leq \log\abs{\cC_{1}} + g\left(\epsilon_2, c_1\right) \\
	&\leq g\left(\epsilon_1, c_0\right) + g\left(\epsilon_2, c_1\right).  \label{lec10:eqn:iterative_cover_bound-3}
	\end{align}
	Similarly, given $\cC_k$, for any $f_k' \circ f_{k-1}' \circ \cdots \circ f_1' \in \cC_k$, we construct a $\cC_{k+1, f_k', \dots, f_1'}$ that is an $\epsilon_{k+1}$-covering of $\cF_{k+1} \circ f_k' \circ \cdots \circ f_1'$. We similarly define 
	\begin{equation}
	\cC_{k+1} = \bigcup_{\substack{f_i \in \cC_i \\ i \leq k}} C_{k+1, f_k', \dots, f_1'}.
	\end{equation}
	Then, inducting on the argument given in \eqref{lec10:eqn:iterative_cover_bound-1}-\eqref{lec10:eqn:iterative_cover_bound-3}, we conclude that
	\begin{align}
	\log \abs{\cC_{k+1}} \leq g\left(\epsilon_{k+1}, c_k\right) + \cdots + g\left(\epsilon_1, c_0\right)
	\end{align}
	Next, we show that for the above construction, the radius of the cover for $\cF$ is
	\begin{align}
	\epsilon = \sum_{i=1}^{r} \left(\epsilon_i \prod_{j=i+1}^{r}\kappa_{j}\right).
	\end{align}
	For any choice of $f_r \circ \cdots \circ f_1 \in \cF_r \circ \cF_{r-1} \circ \cdots \circ \cF_1$,  then, by definition of $\cC_1$, there exists $f_1' \in \cC_1$ such that 
	\begin{equation}
	\rho(f_1, f_1') \leq \epsilon_1.
	\end{equation} 
	Similarly, we know there exists $f_2' \circ f_1' \in \cC_{2, f_1'}$ such that 
	\begin{equation} 
	\rho\left(f_2' \circ f_1', f_2\circ f_1' \right) \leq \epsilon_2.
	\end{equation}
	We can leverage these two facts and the triangle inequality to now prove that $f_2' \circ f_1'$ is close to $f_2 \circ f_1$. Namely,
	\begin{align}
	\rho\left(f_2' \circ f_1', f_2 \circ f_1\right) &\leq \rho\left(f_2' \circ f_1', f_2 \circ f_1'\right) + \rho\left(f_2 \circ f_1', f_2 \circ f_1\right) &\text{(triangle ineq.)} \\ 
	&\leq \epsilon_2 + \rho\left(f_2 \circ f_1', f_2 \circ f_1\right) &\text{(def. of $\cC_{2, f'_1}$)}\\ 
	&\leq \epsilon_2 + \kappa_2 \rho\left(f_1', f_1\right) &\text{\eqref{lec10:eqn:lipschitz-def}}\\ 
	&\leq \epsilon_2 + \kappa_2\epsilon_1 &\text{(def. of $\cC_{1}$)}
	\end{align}
	Inducting to prove this argument for arbitrary $k$, we similarly apply the definition of $\cC_{k, f'_{k - 1},\dots,f'_1}$ to conclude that there exists $f'_{k} \circ f'_{k - 1} \circ \cdots \circ f'_1 \in \cC_k$ such that
	\begin{equation}
	\rho(f'_k \circ f'_{k - 1} \circ \cdots f'_1, f_k \circ f'_{k - 1} \circ \cdots f'_1) \leq \epsilon_k
	\end{equation}
	Then, expanding using the triangle inequality and peeling off terms by applying the definition of our $\epsilon_i$-coverings, we again show that
	\begin{align}
	\rho\left(f_k' \circ f_{k-1}' \circ \cdots \circ f_1', f_k \circ \cdots \circ f_1\right) &\leq \rho\left(f_k' \circ f_{k-1}'\circ \cdots \circ f_1', f_k \circ f_{k-1}'\circ \cdots \circ f_1' \right) \\ 
	&\quad + \rho\left(f_k \circ f_{k-1}'\circ f_{k-2}' \circ \cdots \circ f_1', f_k \circ f_{k-1}\circ f_{k-2}' \circ \cdots \circ f_1'\right) \nonumber \\ 
	&\quad + \cdots + \rho\left(f_k \circ f_{k-1}\circ \cdots \circ f_2 \circ f_1', f_k \circ f_{k-1}\circ \cdots \circ f_1\right) \nonumber \\ 
	&\leq \rho\left(f_k' \circ f_{k-1}'\circ \cdots \circ f_1', f_k \circ f_{k-1}'\circ \cdots \circ f_1' \right) \\
	&\quad + \kappa_{k} \cdot \rho(f'_{k - 1} \circ \cdots \circ f'_1, f_{k - 1} \circ f'_{k - 2} \circ \cdots \circ f'_1) \\
	&\quad + \cdots + \left(\prod_{j = 2}^k \kappa_j\right) \rho(f'_1, f_1) \nonumber  \\
	& \leq \sum_{i=1}^{k} \left(\epsilon_i\prod_{j=i+1}^{k}\kappa_{j}\right).
	\end{align}
	Note that the first inequality follows by the triangle inequality, the second by the $\kappa_i$-Lipschitz continuity of $f_i$, and the third by applying the definition of each of our $\epsilon_i$-covers.
\end{proof}

\begin{proof}[Proof of Theorem~\ref{lec10:thm:dnn_rademacher}]
	We now apply Lemma~\ref{lec10:lma:additive_cover} to dense neural networks. Dense neural networks consist of a composition of layers, where each layer is a linear model composed with a 1-Lipschitz activation. Using Theorem~\ref{lec9:thm:multivariate_rad} along with the property that 1-Lipschitz functions will only contribute a factor of at most $1$ (Lemma~\ref{lec9:lma:talagrand}), the covering number of each layer can be bounded by:
	\begin{align}
	g\left(\epsilon_i, c_{i-1}\right) = \tilde{O}\left(\frac{c_{i-1}^2b_i^2}{\epsilon_i^2}\right),
	\end{align}
	where $c_{i-1}^2$ is the norm of the inputs, $b_i^2$ is $\norm{W_i^\top}_{2,1}$, and $\epsilon_i^2$ is the radius. From Lemma~\ref{lec10:lma:additive_cover}, we know that 
	\begin{align}
	\log N(\epsilon, \cF, \rho) &= \tilde{O}\left(\sum_{i=1}^{r}\frac{c_{i-1}^2b_i^2}{\epsilon_i^2}\right) 
	\end{align}
	for
	\begin{align}
	\epsilon &= \sum_{i=1}^{r} \left(\epsilon_i \prod_{j=i+1}^{r}\kappa_j\right)
	\end{align}
	
	We now have a bound on $N(\epsilon, \cF, \rho)$ that relies on $\epsilon_i$'s, but $N(\epsilon, \cF, \rho)$ should only be a function of $\epsilon$. Since we already know that $\epsilon = \sum_{i=1}^{r} \left(\epsilon_i \prod_{j=i+1}^{r}\kappa_j\right)$, we keep $\epsilon$ constant and optimize the upper bound of $N(\epsilon, \cF, \rho)$ over different choices of $\epsilon_i$. To find the optimal $\epsilon_i$, we will first find a lower bound on $N(\epsilon, \cF, \rho)$. We then choose $\epsilon_i$ so that this lower bound is achieved. Ultimately, our optimized $\epsilon_i$ yields a bound on the covering number of the following form: $\log\left(N\left(\epsilon, \cF, \rho\right)\right) \leq \frac{R}{\epsilon^2}$, where $R$ is some constant independent of $\epsilon$. 
	
	We derive this lower bound using Holder's inequality, which states that
	\begin{align}
	\langle a,  b \rangle \leq \|a\|_p \|b\|_q
	\end{align}
	when $\frac{1}{p} + \frac{1}{q} = 1$. Writing out the vectors $a, b$, we get that 
	\begin{align}
	\sum_{i}a_ib_i \leq \left(\sum a_i^p\right)^{\frac{1}{p}}\left(\sum b_i^q\right)^{\frac{1}{q}}
	\end{align}
	
	Let $\alpha_i^2 = c_{i-1}^2b_i^2, \beta_i = \prod_{j=i+1}^{r}\kappa_j$. By Holder's inequality, using $p = 3, q = \frac{3}{2}$, we get
	\begin{align}
	\left(\sum_{i=1}^{r}\frac{\alpha_i^2}{\epsilon_i^2}\right)\left(\sum_{i=1}^{r}\beta_i\epsilon_i\right)^2 &\geq \left(\sum_{i=1}^{r}\left(\alpha_i\beta_i\right)^{\frac{2}{3}}\right)^{\frac{3}{2}}
	\end{align}
	\begin{align}
	\sum_{i=1}^{r}\frac{\alpha_i^2}{\epsilon_i^2} &\geq \frac{R}{\epsilon^2},
	\end{align}
	where $R = \left(\left(\sum_{i=1}^{r}\left(c_{i-1}b_i\prod_{j=i+1}^{r}\kappa_j\right)^{\frac{2}{3}}\right)\right)^{\frac{3}{2}}$. We note that equality holds when 
	\begin{align}
	\epsilon_i = \left(\frac{c_{i-1}^2b_i^2}{\prod_{j=i+1}^{r}\kappa_j}\right)^{\frac{1}{3}} \cdot \underbrace{\frac{\epsilon}{\left(\sum_{i=1}^{r}\frac{b_i^{\frac{2}{3}}}{\kappa_i^{\frac{2}{3}}}\right)\prod_{i=1}^{r}\kappa_i^{\frac{2}{3}} }}_{\epsilon'} \label{eqn:lec10:holder_eps_defn}
	\end{align}
	Using this choice of $\epsilon_i$ and letting $\epsilon'$ equal the second factor in \eqref{eqn:lec10:holder_eps_defn} for notational convenience, we know that the log covering number is (up to a constant factor):
	\al{
		\sum_{i=1}^r \frac{c_{i-1}^2b_i^2}{\epsilon_i^2} &= \sum_{i=1}^r \frac{c_{i-1}^2b_i^2(\kappa_{i+1}\cdots\kappa_r)^\frac{2}{3}}{c_{i-1}^\frac{4}{3}b_i^\frac{4}{3}(\epsilon')^2} \\
		&= \sum_{i=1}^r (c_{i-1}b_i\kappa_{i+1}\cdots\kappa_r)^\frac{2}{3}\frac{1}{(\epsilon')^2} \\
		&= c^\frac{2}{3}\sum_{i=1}^r \left(\frac{b_i}{\kappa_i}\right)^\frac{2}{3} \prod_{i=1}^r \kappa_i^\frac{2}{3} \frac{\left(c^\frac{2}{3}\left(\sum_{i=1}^r (\frac{b_i}{\kappa_i})^\frac{2}{3} \prod_{i=1}^r \kappa_i^\frac{2}{3}\right)\right)^2}{\epsilon^2} \\
		&= \left(c^\frac{2}{3}\sum_{i=1}^r \left(\frac{b_i}{\kappa_i}\right)^\frac{2}{3} \prod_{i=1}^r \kappa_i^\frac{2}{3}\right)^3\frac{1}{\epsilon^2} \\
		&= c^2\prod_{i=1}^r \kappa_i^2\left(\sum_{i=1}^r \left(\frac{b_i}{\kappa_i}\right)^\frac{2}{3}\right)^3\frac{1}{\epsilon^2}.
	}
	Since this log covering number is of the form $R / \epsilon^2$, we can apply \eqref{lec9:eqn:rademacherbound_three} and conclude that
	\al{
		\mathcal{R}_S(\cF) \lesssim \sqrt\frac{R}{n}
	}
	Last, plugging in
	\al{
		R = c^2\prod_{i=1}^r \kappa_i^2\left(\sum_{i=1}^r \left(\frac{b_i}{\kappa_i}\right)^\frac{2}{3}\right)^3
	}
	we obtain the desired result
	\al{
		\mathcal{R}_S(\cF) \lesssim \frac{c}{\sqrt n}\prod_{i=1}^r \kappa_i\left(\sum_{i=1}^r \left(\frac{b_i}{\kappa_i}\right)^\frac{2}{3}\right)^\frac{3}{2}.
	}
	
\end{proof}

\sec{Data-dependent generalization bounds for deep Nets}\label{sec:deep_nets_data_dependent}

In Theorem~\ref{lec10:thm:dnn_rademacher}, we proved the following bound on the Rademacher complexity of deep neural networks:
\begin{align}
R_S(\cF) \leq \prod_{i = 1}^r \norm{W_i}_{\text{op}} \cdot \mathsf{poly}(\norm{W_1}, \dots, \norm{W_r}).
\end{align}
This bound, however, suffers from multiple deficiencies. In particular, it grows exponentially in the depth, $r$, of the network and $\norm{W_i}_{\text{op}}$ measures the worst-case Lipschitz-ness of the network layers over the input space. %As a consequence, the bound fails to accurately predict the good generalization properties of deep nets.

In this section, we obtain a tighter generalization bound that depends upon the realized Lipschitz-ness of the model on the training data. To further motivate this approach, we also note that stochastic gradient descent, i.e. the typical optimization method typically used to fit deep neural networks, prefers models that are more Lipschitz (see Chapter (TBD) for further discussion) \tnotelong{add references later}. This preference must be realized by the model \emph{on empirical data}, however, as no learning algorithm has access to the model's properties over the entire data space.

Ultimately, we aim to prove a tighter bound on the population loss that grows polynomially in the Lipschitz-ness of $f$ on the empirical data. Namely, given that $f$ is parameterized by some $\theta$, we hope to derive a bound on the population loss at $\theta$ that is a \emph{polynomial} function of the Lipschitz-ness of $f$ on $x\sp{1},\dots,x\sp{n}$ as well as the norm of $\theta$.

\paragraph{Uniform convergence with a data-dependent hypothesis class.}
%Classical uniform convergence does not have a single consistent definition. 
So far in this course, given some complexity measure we denote as $\text{comp}(\cdot)$, our uniform convergence results always appear in one of the two following forms (which are essentially equivalent). Namely, with high probability,
\begin{align}
\forall f\in \cF, ~~L(f) &\leq \frac{\text{comp}(\cF)}{\sqrt{n}} &&\text{(I)} \\
\forall f, ~~ L(f) &\leq \frac{\text{comp}(f)}{\sqrt{n}}  &&\text{(II)}
\end{align}

\begin{remark}
	Most of the results we have obtained so far are of type I, e.g. with $\text{comp}(\cF)/\sqrt{n} = R_n(\cF)$. We obtain results of type II by considering a restricted set of functions $\cF_C = \{f : \text{comp}(f) \leq C\}$. We then apply a type I bound to $\cF_C$ and take a union bound over all $C$. Therefore, these two type of bounds are essentially equivalent (up to a small additive factor difference due to the additional union bound over the choices of $C$.)
\end{remark}

Note, however, that neither of these approaches produce bounds that depend upon the data. By contrast, in the sequel, we will derive a new \textit{data-dependent} generalization bound. These bounds state that with high probability over the choice of the empirical data and, for all functions $f\in \cF$,
\begin{align}
L(f) \leq \text{comp}\left (f, \{(x\sp{i}, y\sp{i})\}_{i = 1}^n\right)
\end{align}
Even though the complexity measure depends on the training data, and is thus a random variable by itself, it can be used as a regularizer which can be added to the original training loss.

\begin{remark}
	Although there is no universal consensus on the type of generalization bound we should derive, we can argue that there is no way to leverage more information in a generalization bound beyond the empirical data. For example, one might try to use the input distribution $P$ to define the complexity measure, but if we allowed ourselves access to $P$, we could just define $\text{comp}(f, P) = \Exp_P[f(X)]$. In some sense, defining a generalization bound using the true distribution amounts to cheating, and the dependence on the empirical data seems to be proper because the bound can still be used as a regularizer. %, so it becomes difficult to define a distibution-dependent generalization bound in a principled way.
\end{remark}

In this new paradigm, we can no longer take the previous approach of obtaining type I bounds and then derive a type II bound via a reduction. To see why, suppose that we have the hypothesis class
\begin{align}
\cF_C = \{f: \text{comp}(f, \{(x\sp{i}, y\sp{i})\}_{i=1}^n) \leq C)\}
\end{align}
If our complexity measure depends on the empirical data, then so does our hypothesis class $\cF_C$, which makes $\cF_C$ itself a random variable. However, our theorems regarding Rademacher complexity require that the hypothesis class be fixed before we ever see the empirical data.

We may hope to get around this by changing the way we think about uniform convergence. Consider the simplified case where our new complexity measure is separable, i.e.
\begin{align}
\text{comp}(f, \{(x\sp{i}, y\sp{i})\}_{i=1}^n) = \sum_{i=1}^n h(f, x\sp{i}),
\end{align}
for some function $g$. Then we can consider an \textit{augmented loss}:
\begin{align}
\tilde{\ell}(f) = \ell(f) \ind{h(f, x\sp{i} \leq C)} \label{eqn:5}
\end{align}
\begin{figure}[ht]
	\centering
	\begin{tikzpicture}
	\draw[->] (0, 0) -- (5, 0) node[right] {$\theta$};
	\draw[->] (0, 0) -- (0, 5) node[above] {loss};
	\draw[scale=0.5, domain=0:6, smooth, ultra thick, variable=\x, blue] plot (\x, {7 - \x + sin(\x r)});
	\draw[scale=0.5, domain=0:5.5, smooth, ultra thick, variable=\x, green] plot (\x, {10 - 1.6*\x + sin(0.90*\x r)});
	\draw[scale=0.5, domain=6:10, smooth, ultra thick, variable=\x, blue] plot (\x, {\x - 5 + sin(\x r)}) node[right] {test};
	\draw[scale=0.5, domain=5.5:10, smooth, ultra thick, variable=\x, green] plot (\x, {1.4*\x - 6.67 + sin(\x r)}) node[right] {train};
	\draw[dashed] (2, 0) -- (2, 5);
	\draw[dashed] (3.5, 0) -- (3.5, 5);
	\draw [decorate,decoration={brace,amplitude=5pt,mirror,raise=2ex}]
	(2,0) -- (3.5,0) node[midway,yshift=-2em]{low-complexity params};
	\end{tikzpicture}
	\caption{These curves depict a ``low-complexity'' region in parameter space. The \textcolor{blue}{blue} curve is the unobserved test loss we aim to bound, while the \textcolor{green}{green} curve denotes the empirical training loss we observe. Observe that in the region of $\theta$ that we identify as being ``low-complexity,'' the gap between the train and test losses is smaller than in the high-complexity regions.}
	\label{lec11:fig:low_vs_high_complexity}
\end{figure}
Suppose we have a region of low complexity in our existing loss function as depicted in Figure~\ref{lec11:fig:low_vs_high_complexity}. Because this region is random, so we cannot selectively apply uniform convergence. However, we can use our new surrogate loss function $\tilde{\ell}$ in that region. By modifying the loss function in this way, we can still fix the hypothesis class ahead of time, allowing us to apply existing tools to $\tilde{\ell}(f)$. The surrogate loss was used in~\cite{wei2019data} to obtain a data-dependent generalization bound, though there are possibly various other ways to define surrogate losses and apply existing uniform convergence guarantees. In the sequel, we introduce a particular surrogate ``margin'' that allows us to cleanly apply our previous results to a (implicitly) data-dependent hypothesis class \cite{wei2019data}.

\subsection{All-layer margin} \label{sec:all_layer_margin}
We next introduce a new surrogate loss called the \textit{all-layer margin} that can also be thought of as a surrogate margin. This loss will essentially zero out high-complexity regions so that we may focus on low-complexity regions for which we can expect small generalization gap. Note that the all-layer margin we analyze will not explicitly zero-out high-complexity regions using an indicator function, but instead implicitly takes into account some data-dependent characteristics of the model. Once we adopt this new loss function, we will be able to apply some of our earlier methods.

Let $f: \R^d \to \R$ be a classification model. Recall that the standard margin is defined as $y f(x)$, with $y$ in $\{-1, 1\}$. We will say that $g_f(x, y)$ is a \textit{generalized margin} if it satisfies
\begin{align}
g_f(x, y) = \begin{cases}
0,& \text{ if } f(x)y \leq 0 \text{ (an incorrect classification)}\\
> 0,& \text{ if } f(x)y > 0 \text{ (a correct classification)}
\end{cases}.
\end{align}
%That is, the generalized margin ``zeroes out" incorrect classifications.
To simplify the exposition of the machinery below, we also introduce the \textit{$\infty$-covering number} $N_\infty(\epsilon, \cF)$ as the minimum cover size with respect to the metric $\rho$ defined as the infinity-norm distance on an input domain $\cX$: 
\begin{equation}
\rho(f, f) \triangleq \sup_{x \in \mathcal{X}} |f(x) - f'(x)| \triangleq \|f - f'\|_\infty.\footnote{If $f$ maps $\cX$ to multi-dimensional outputs, we will define $\rho(f, f) \triangleq \sup_{x \in \mathcal{X}} \|f(x) - f'(x)\| \triangleq \|f - f'\|_\infty$ where the norm in $\|f(x) - f'(x)\|$ is a norm in the output space of $f$ (which will be the Euclidean norm in this rest of this section).}
\end{equation}
\begin{remark}
	Notice that $N_\infty(\epsilon, \cF) \geq N(\epsilon, \cF, L_2(P_n))$. This is because the $\rho = L_\infty(\cX)$ is a more demanding measure of error: $f$ and $f'$ must be close on \textit{every} input, not just the empirical data. That is,
	\begin{equation}
	\sqrt{\frac{1}{n} \sum_{i=1}^n (f(x_i) - f'(x_i))^2} \leq \sup_{x \in \mathcal{X}} |f(x) - f'(x)|. \label{lec11:eqn:l_inf_vs_l2pn}
	\end{equation}
\end{remark}

\begin{lemma}
	Suppose $g_f$ is a generalized margin. Let $\cG = \{g_f: f \in \mathcal{F}\}$. Suppose that for some $R$, $\log N_\infty(\epsilon, \cG) \leq \lfloor \frac{R^2}{\epsilon^2} \rfloor$ for all $\epsilon > 0$.\footnote{Recall that this is the worst dependency on $\epsilon$ that we can tolerate when converting covering number bounds to Rademacher complexity.} Then, with high probability over the randomness in the training data, for every $f$ in $\mathcal{F}$ that correctly predicts all the training examples,
	\begin{equation}
	L_{01} \leq \tilO \l (\frac{1}{\sqrt{n}} \cdot \frac{R}{\min_{i \in [n]} g_f(x\sp{i}, y\sp{i})} \r ) + \tilO\l (\frac{1}{\sqrt{n}}\r ).
	\end{equation}
	\label{lec11:genmargin-lemma}
\end{lemma}

\begin{proof}
	The high-level idea of our proof is to replace $\cF$ with $\cG$ before repeating the standard margin theory argument from Section~\ref{sec:formal_margin}.
	
	Let $\ell_\gamma$ be the ramp loss given in \eqref{lec6:eqn:ramp_loss}, which is 1 for negative values, 0 for values greater than $\gamma$, and a linear interpolation between 1 and 0 for values between 0 and $\gamma$. 
	We define the surrogate loss as $\hat{L}_\gamma(\theta) = \frac{1}{n} \sum_{i = 1}^n \ell_\gamma(g_{f_\theta}(x\sp{i}, y\sp{i}))$, and the surrogate population loss as $L_\gamma(\theta) = \Exp[\ell_\gamma(g_{f_\theta}(x, y))]$. Applying Corollary~\ref{lec6:cor:ggap-rsbound}, where we used the Rademacher complexity to control the generalization error, we conclude that
	\begin{equation}
	L_\gamma(\theta) - \hat{L}_\gamma(\theta) \leq R_S(\ell_\gamma \circ \cG) + \tilO\l (\frac{1}{\sqrt{n}}\r ).
	\end{equation}
	Next we observe that 
	\begin{align}
	\log N(\epsilon, \ell_\gamma \circ \cG, L_2(P_n)) &\leq \log N(\epsilon\gamma, \cG, L_2(P_n)) &\text{(Lemma~\ref{lec9:lma:talagrand})} \\
	&\leq \log N_\infty(\epsilon\gamma, \cG) &\text{\eqref{lec11:eqn:l_inf_vs_l2pn}} \\
	&\leq \l \lfloor \frac{R^2}{\epsilon^2 \gamma^2} \r \rfloor &\text{(by assumption)}.
	\end{align}
	Then, using our results relating the log of the covering number to a bound on the Rademacher complexity (recall \eqref{lec9:eqn:rademacherbound_three} and Theorem~\ref{lec9:thm:better-dudley}), we conclude that $R_S(\ell_\gamma \circ \cG) \leq \tilO\l (\frac{R}{\gamma \sqrt{n}}\r )$.
	Take $\gamma = \gamma_{\min} = \min_{i} g_\gamma(x\sp{i}, y\sp{i})$.\footnote{A caveat: because $\gamma$ is a random variable, proving this result rigorously requires taking a union bound over a discretized $\gamma$. We sketched out this argument more thoroughly in Remark~\ref{lec7:rmk:union_bound_margin}.} Using Corollary~\ref{lec6:cor:ggap-rsbound}, we conclude that $\hat{L}_{\gamma_\text{min}} (\theta) \leq 0 + \tilO\l (\frac{R}{\sqrt{n} \cdot \gamma_\text{min}} \r ) + \tilO\l (\frac{1}{\sqrt{n}}\r )$, as desired.
\end{proof}
For which $g_f$ can we bound the covering number? If we take $g_f(x, y) = yf(x)$, then the covering number depends on the product $\prod_i \norm{W_i}_{\text{op}}$, but we originally set out to do better than this. If we have a linear model $w^\top x$, the normalized margin, $\frac{y \cdot w^\top x}{\norm{w}}$, governs the generalization performance. But how do we normalize for more general models? 

For a deep neural net, a potential normalizer is the product of the Lipschitz constants of the layers. However, we do not want to normalize by a constant that depends only on the function class, so we take a different approach. We interpret the normalized margin as the solution to the following optimization problem:
\begin{equation}
\begin{aligned}
\min_\delta \quad & \norm{\delta}_2 \\
\textrm{s.t.} \quad & w^\top(x + \delta) y \leq 0
\end{aligned}
\end{equation}
In plain English, this problem searches for the minimum perturbation that gets our data point across the boundary.

This perturbation view of the standard margin can be extended naturally to multiple layers. For the math to work, it turns out that we need to perturb all the layers. We define the \textit{all-layer margin} as below. We will consider perturbed models $\delta = (\delta_1, \dots, \delta_r)$, where each $\delta_i$ is a perturbation \textit{vector} associated with the $i$-th layer (and it has the same dimensionality as the $i$-th layer activation). We incorporate these perturbations into our model in the following way (so that we can handle the scaling in a clean way):
\begin{align}
h_1(x, \delta) &= W_1 x + \delta_1 \cdot \norm{x}_2 \\
h_2(x, \delta) &= \sigma(W_2 h_1(x, \delta)) + \delta_2 \cdot \norm{h_1(x, \delta)}_2 \\
&\vdots \nonumber \\
f(x, \delta) = h_r(x, \delta) &= \sigma(W_r h_{r - 1}(x, \delta)) + \delta_r \cdot \norm{h_{r - 1}(x, \delta)}_2.
\end{align}
We can then ask: what was the smallest perturbation that changed our decision? That is, let
\begin{align}
m_f(x, y) \defeq \min_\delta \sqrt{\sum_{i=1}^r ||\delta_i||_2^2} \quad \text{s.t.} \quad f(x, \delta) y \leq 0,
\end{align}
i.e. the smallest perturbation that yields incorrect predictions.

Informally, $m_f(x, y)$ is a measure of how hard it is to perturb the model $f$. $f$ can be hard to perturb for two reasons: $f$ is Lipschitz (in its intermediate layers) and/or $yf(x)$ is large. In other words, the all-layer margin is a normalized version of the standard margin, normalized by the Lipschitzness of the model at the particular data point $(x,y)$.  %Even more informally, large margins imply confidence in our predictions, and so it becomes harder to change the model's mind.

We now introduce our main result regarding the all-layer margin.
\begin{theorem} \label{lec11:thm:poly_gen_bound_deep_nets}
	With high probability, for all $f$ with training error $0$,
	\begin{equation}
	L_{01}(f) \leq \tilO\l (\frac{1}{\sqrt{n}} \cdot \frac{\sum_{i=1}^r \norm{W_i}_{1, 1}}{\min_{i \in [n]} m_f(x\sp{i}, y\sp{i})}\r ) + \tilO\l (\frac{r}{\sqrt{n}}\r ),
	\end{equation}
	where
	$\norm{W}_{1, 1}$ is the sum of the absolute values of the entries of W.
\end{theorem}
In summary, robustness to perturbations in intermediate layers implies good generalization. We will interpret the bound, compare the bounds with previous works, and discuss further extensions in the remarks following the proofs of the theorem.  (E.g, in Remark~\ref{remark:1}, we will argue that this bound is strictly better than the one we obtained in Theorem~\ref{lec10:thm:dnn_rademacher}; in the worst case, we still have that $\frac{1}{m_f(x, y)} \leq \frac{\prod \norm{W_i}_{\text{op}}}{f(x)}$.)

To prove this theorem, it suffices to bound $N_\infty(\epsilon, \cG)$ by $O(\frac{\sum{\norm{W_i}_{1, 1}}}{\epsilon^2})$ and apply Lemma~\ref{lec11:genmargin-lemma}. Towards this goal, let $\cF_i = \{ z \mapsto \sigma (W_i z) : \norm{W_i}_{1, 1} \leq \beta_i \}$. Then, $\cF = \cF_r \circ \cF_{r-1} \circ \cdots \circ \cF_1$. 

\begin{lemma}[Decomposition Lemma]\label{lec11:lma:decomp}
	Let $m \circ \cF$ denote $\{m_f : f \in \cF \}$. Then, 
	\begin{equation}
	\log N_\infty\l (\sqrt{\sum_{i=1}^r \epsilon_i^2}, m \circ \cF \r ) \leq \sum_{i=1}^r \log N_\infty(\epsilon_i, \cF_i),
	\end{equation}
	where $N_\infty(\epsilon_i, \cF_i)$ is defined with respect to the input domain $\mathcal{X} = \{x : \norm{x}_2 \leq 1 \}$.
\end{lemma}

That is, we only have to find the covering number for each layer, and then we have the covering number for the (all-layer margin of the) composed function class. Notice that we bounded the covering number of $m \circ \cF$ in the above lemma, not $\cF$.

Then, the desired result follows directly from the preceding decomposition lemma.
\begin{corollary} Assume that $\log N_\infty(\epsilon_i, \cF_i) \leq \l \lfloor \frac{c_i^2}{\epsilon_i^2} \r \rfloor$ for every $\cF_i$, i.e. the function class corresponding to the $i$-th layer of $f$ in Theorem~\ref{lec11:thm:poly_gen_bound_deep_nets}. Then, by taking $\epsilon_i = \epsilon \cdot \frac{c_i}{\sqrt{\sum_i c_i^2}}$, we have that
	\begin{equation}
	\log N_\infty(\epsilon, m \circ \cF) \leq \frac{\sum_i c_i^2}{\epsilon^2}.
	\end{equation}
\end{corollary}
This result gives the complexity of the composed model in terms of the complexity of the layers, with each $c_i$ given by $\norm{W_i}_{1, 1}$. For linear models, we can show $N_\infty(\epsilon_i, \cF_i) \leq \tilO\l (\frac{\beta_i^2}{\epsilon^2} \r )$ (where $\beta_i$ is a bound on $\norm{W_i}_{1, 1}$), and this implies Theorem~\ref{lec11:thm:poly_gen_bound_deep_nets}\footnote{Technically, we also need to union bound over the choices of $\beta_i$, which can also be achieved following Remark~\ref{lec7:rmk:union_bound_margin}.} Finally, we are only left with the proof of Lemma~\ref{lec11:lma:decomp}. 

\begin{proof}[Proof of Lemma~\ref{lec11:lma:decomp}]
	Now we will prove a limited form of the decomposition lemma for affine models: $\cF_i = \{ z \mapsto \sigma(W_i z): \norm{W_i}_{1, 1} \leq \beta_i \}$. There are two crucial steps to this problem. First, we will prove that $m_f(x, y)$ is 1-Lipschitz in $f$. That is, for all $\cF = \cF_r \circ \cF_{r-1} \circ \cdots \circ \cF_1$ and $\cF' = \cF_r' \circ 
	\cF_{r-1}' \circ \cdots \circ \cF_1'$,
	\begin{align}
	\abs{m_f(x, y) - m_{f'}(x, y)} \leq \sqrt{\sum_{i=1}^r \max_{\norm{x}_2 \leq 1} \norm{f_i(x) - f_i'(x)}_2^2}. \label{lec11:eqn:one_lipschitz_claim}
	\end{align}
	Notice that now we are working with a clean sum of differences, with no multipliers! 
	
	Second, we construct a cover: Let $U_1, \dots, U_r$ be $\epsilon_1, \dots, \epsilon_r$-covers of $\cF_1, \dots, \cF_r$, respectively, such that $\abs{U_i} = N_\infty(\epsilon_i, \cF_i)$. By definition, for all $f_i$ in $\cF_i$, there exists a $u_i \in U_i$ such that $\max_{\norm{x} \leq 1} \norm{f_i(x) - u_i(x)}_2 \leq \epsilon_i$. Take $U = U_r \circ U_{r-1} \circ \cdots \circ U_1 = \{u_r \circ u_{r-1} \circ \cdots \circ u_1 \}$ as the cover for $m \circ \cF$. Suppose we were given $f = f_r \circ \cdots \circ f_1 \in \cF$. Let $u_r, \dots, u_1$ be the nearest neighbors of $f_r, \dots, f_1$. Then
	\begin{align}
	|m_f(x, y) - m_u(x, y)| &\leq \sqrt{\sum_{i=1}^r \max_{||x|| \leq 1} ||f_i(x) - u_i(x)||_2^2} \\
	&\leq \sqrt{\sum_{i=1}^r \epsilon_i^2} &&\text{(by construction).}
	\end{align}
	
	Having established the validity of our cover, we now return to our claim of 1-Lipschitz-ness stated in \eqref{lec11:eqn:one_lipschitz_claim}. By symmetry, it is sufficient to prove an upper bound for $m_{f'}(x, y) - m_f(x, y)$.
	
	Let $\delta_1^*, \dots, \delta_r^*$ be the optimal choices of $\delta$ in defining $m_f(x, y)$. Our goal is to turn these into a feasible solution of $m_{f'}(x, y)$, which we denote by $\hat{\delta}_1, \dots, \hat{\delta}_r$. If this solution is feasible, we obtain the bound $m_{f'}(x, y) \leq \sqrt{\sum \norm{\hat{\delta}_i}^2_2}$.
	
	Intuitively, we want to define a perturbation for $f'$ that does the same thing as $\delta_1^*,\dots,\delta_r^*$ for $f$. In plain English, $(f', \hat{\delta}_1, \dots, \hat{\delta}_r)$ should do the same thing as $(f_1, \delta_1^*, \dots, \delta_r^*)$. Recall that $f$ has parameters $W_1, \dots, W_r$ and $f'$ has parameters $W_1', \dots, W_r'$. Then, under the optimal perturbation,
	\begin{align}
	h_1 &= W_1 x + \delta_1^* \norm{x}_2 \\
	h_2 &= \sigma(W_2 h_1) + \delta_2^* \norm{h_1}_2 \\
	&\vdots \nonumber \\
	h_r &= \sigma(W_r h_{r - 1}) + \delta_r^* \norm{h_{r - 1}}_2
	\end{align}
	We want to imitate this by perturbing $f'$ in some way. In particular, let
	\begin{equation}
	h_1 = W_1'x + \underbrace{\delta_1^* \norm{x}_2 + (W_1 - W_1')x}_{\defeq \text{ }\hat{\delta}_1 \norm{x}_2},
	\end{equation}
	where the last term serves to compensate for the difference between $W_1$ and $W_1'$. Thus, $\hat{\delta}_1 \defeq \delta_1^* + \frac{(W_1 - W_1')x}{\norm{x}_2}$.
	We repeat this argument for every layer. Using the second layer as an example, 
	\begin{align}
	h_2 &= \sigma(W_2' h_1) + \underbrace{\delta_2^*\norm{h_1} + \sigma(W_2 h_1) - \sigma(W_2' h_1)}_{\defeq \text{ }\hat{\delta}_2 \norm{h}_2}.
	\end{align}
	So, $\hat{\delta}_2 = \delta_2^* + \frac{\sigma(W_2 h_1) - \sigma(W_2' h_1)}{\norm{h_1}_2}$. In general, 
	\begin{align}
	\hat{\delta}_i \defeq \delta_i^* + \frac{\sigma(W_ih_{i-1}) - \sigma(W_i' h_{i-1})}{\norm{h_{i-1}}_2}
	\end{align} 
	
	Then $\hat{\delta}_1,\dots, \hat{\delta}_r$ on $f'$ are making the same predictions as $\delta_1, \dots, \delta_r$ on $f'$. Last, observe that
	\begin{align}
	m_{f'}(x, y) &\leq \sqrt{\sum ||\hat{\delta}_i||_2^2} \\
	&\leq \sqrt{\sum \norm{\delta_i^*}_2^2} + \sqrt{\sum_{i = 1}^r \left (\frac{\sigma(W_i h_{i-1}) - \sigma(W_i' h_{i-1})}{\norm{h_{i-1}}_2} \right)^2 } &\text{(Minkowski's Ineq.)\footnotemark}\\
	&\leq m_f(x, y) + \sqrt{\sum_{i=1}^r \max_{\norm{x}_2 \leq 1} (\sigma(W_i x)-\sigma(W'_i x))^2} \label{lec11:eqn:l2_constraint} \\
	&= m_f(x, y) + \sqrt{\sum_{i=1}^r \max_{\norm{x}_2 \leq 1} (f_i(x)-f_i'(x))^2}
	\end{align} 
	\footnotetext{Minkowski's inequality, which states that $\sqrt{\sum \norm{a_i + b_i}_2^2} \leq \sqrt{\sum \norm{a_i}_2^2} + \sqrt{\sum \norm{b_i}_2^2}$. In this setting, this inequality can also be proved using Cauchy-Schwarz.}
	Note that in \eqref{lec11:eqn:l2_constraint}, constraining $\norm{x}_2 \leq 1$ is equivalent to dividing by the $\ell_2$-norm of $x$.
\end{proof}

\begin{remark}\label{remark:1}
	We can compare the above with Theorem~\ref{lec10:thm:dnn_rademacher} proven in \cite{bartlett2017}.
	\begin{equation}
	\begin{split}
	f(x, \delta) - f(x) &\leq \norm{\delta_r}_2 \cdot \norm{W_{r-1}}_{\text{op}} \cdots \norm{W_1}_{\text{op}} \\
	&\quad + \norm{W_r}_{\text{op}} \cdot \norm{\delta_{r-1}}_2 \cdot \norm{W_{r - 2}}_{\text{op}} \cdots \norm{W_1}_{\text{op}} \\
	&\quad + \cdots  \\
	&\quad + \norm{W_r}_{\text{op}} \cdots \norm{W_2}_{\text{op}} \cdot \norm{\delta_1}_2.
	\end{split}
	\end{equation}
	Ignoring minor details (e.g. dependency on $r$), we suppose that $y = 1$. Then, if $f(x) > 0$ and $f(x + \delta) \leq 0$, it must be the case that $\norm{\delta}_2 \lesssim \frac{|f(x)|}{\prod_{i = 1}^r \norm{W_i}_{\text{op}}}$. This further implies that 
	\begin{align}
	\frac{m_f(x, y)}{y f(x)} \gtrsim \frac{1}{\prod_{i = 1}^r \norm{W_i}_{\text{op}}}.
	\end{align}
	Rearranging, we conclude that we have obtained a tighter bound since the inverse margin $\frac{1}{m_f(x, y)} \lesssim \frac{1}{yf(x)} \cdot \prod_{i = 1}^r \norm{W_i}_{\text{op}}$.
\end{remark}

\begin{remark}
	Later, we will show that SGD prefers Lipschitz solutions and Lipschitzness on data points.\tnotelong{add a reference later}
	Implicitly, SGD seems to be maximizing the all-layer margin. Since the algorithm is (in a sense) minimizing Lipschitzness on a data point, this likely accounts for the empirically observed gap between the two bounds. 
\end{remark}

\begin{remark}
	The approach we have described here is also similar to other methods in the deep learning literature. Other authors have introduced a method known as SAM (a form of sharpness-aware regularization); this method applies a perturbation to the parameter $\theta$ itself rather than on the intermediate hidden parameters $h_i$. However, these two methods are related! If we consider the (single-example) loss $\frac{\partial \ell}{\partial W_i}$, it equals $\frac{\partial \ell}{\partial h_{i+1}} \cdot h_i^\top$. Note that the norm of the term on the left is bounded by the product of the norms of the two terms of the right; this observation relates the model's Lipschitzness with respect to the parameters to its Lipschitzness with respect to the hidden layer outputs.
	\tnotelong{a reminder for Tengyu to have a stronger argument ehre}
\end{remark}

\begin{remark}
	Finally, we can prove a more general version of this result in which we do not need to study the minimum margin of the entire dataset, and instead consider the average margin. Using this approach, we can show that the test error is bounded above by 
	$\frac{1}{n} \sqrt{\frac{1}{n} \sum_{i=1}^n \frac{1}{m_f(x\sp{i}, y\sp{i})^2}}$ times the sum of complexities of each layer, plus a low-order term.
\end{remark}


\chapter{Conceptual and Theoretical Mysteries in Deep Learning}\label{chap:dl-overview}
% reset section counter
\setcounter{section}{0}

%\metadata{lecture ID}{Your names}{date}
\metadata{9}{Rafael Rafailov and Aidan Perreault}{Feb 10th, 2021}

We now turn to a high-level overview of deep learning theory. To begin, we outline a framework for classical machine learning theory, then discuss how the situation is different from deep learning theory.

\sec{Review: framework for classical machine learning theory}
At the risk of oversimplification, we can divide classical machine learning theory into three parts:

\begin{enumerate}
\item {\bf Approximation theory} attempts to answer whether there is any choice of parameters $\theta$ that achieves low population error. In other words, is the choice of hypothesis class good enough to approximate the ground truth function? Using notation from earlier in this course, the goal is to upper bound $L(\theta^*) = \min_{\theta \in \Theta} L(\theta).$
    
\item {\bf Statistical generalization} focuses on bounding the excess risk $L(\hat{\theta}) - L(\theta^*)$. In Chapter \ref{chap:uc} we obtained the following bound:
    
\begin{equation}
L(\hat{\theta})-L(\theta^*)\leq \underbrace{L(\hat{\theta})-\hat{L}(\hat{\theta})}_{\text{generalization error}} + |L(\theta^*)-\hat{L}(\theta^*)|.
\end{equation}
    
The first term here is the generalization error, which usually has an upper bound of the form $R(\theta)/\sqrt{n}$, where $R(\theta)$ is some complexity measure.\footnote{In earlier chapters, we defined the complexity of a hypothesis class, not of a specific parameter value. To reconcile these two approaches, think of $R$ as a measure of complexity (such as a norm) that we can then use to define a hypothesis class $\Theta$, i.e.~$\Theta = \{\theta' : R(\theta') \le R(\theta)\}$.} This is a demonstration of \href{https://en.wikipedia.org/wiki/Occam%27s_razor}{\textit{Occam's Razor}}: the principle that simple (parsimonious, or low-complexity) explanations tend to generalize better. 
    
This statistical approach allows us to define a regularized loss  $\hat{L}_{\textup{reg}}(\theta)=\hat{L}(\theta)+\lambda R(\theta)$. Minimizing this loss gives us a solution $\hat{\theta}_\lambda$ which simultaneously has low training error and low complexity, which lets us bound both the training error and the generalization error. To summarize, in the classical setting, we can prove statements of the form
    
\begin{equation}\label{lec9:eqn:classical-guarantee}
\text{Any global minimizer }\hat{\theta}_\lambda \text{ of } \hat{L}_{\textup{reg}} \textup{ has small excess risk }  L(\hat{\theta}_\lambda) - L(\theta^*)\,.
\end{equation}

\item {\bf Optimization} considers how to obtain the minimizer $\hat\theta$ or $\hat{\theta}_\lambda$ computationally. This usually involves convex optimization: if $\hat{L}$ or $\hat{L}_{\textup{reg}}$ is convex, then we have a polynomial-time algorithm to find the global minimum.
\end{enumerate}

While there are many tradeoffs to consider between these three components (for example, we may be able to find a loss function for which optimization is easy, but generalization becomes worse), they are conceptually independent, and it is typically possible to study each area individually, then combine all three to get a result.

\sec{Deep learning theory and its differences}
The situation is more complex for deep learning theory. Two prominent differences are (a) the models are non-linear and the objective functions are non-convex, and (b) in deep learning, researchers have observed in many cases that more parameters typically help improve the performance, and many state-of-the-art models have much more parameters than the number of training data. (b) is often referred as to ``over-parameterization".

\begin{figure}[ht]
    \centerline{\includegraphics[width=4in]{figures/overparameterization.png}}
    \caption[lec9:fig:overparam]{The black and red lines denote the training and test error, respectively, of a three layer neural network fit to and evaluated on MNIST \cite{neyshabur2015norm}. While classical generalization theory predicts that beyond some threshold, the test error will increase with complexity (shown by the purple line), the true test error continues to decline with overparameterization. Though not depicted here, Neyshabur et al. observe similar test set error curves for a neural network fit to CIFAR-10.}
    \label{lec9:fig:overparam}
\end{figure}

Let us consider the difference in each of the three components described for classical machine learning theory. 

\begin{enumerate} 
\item {\bf Approximation theory:} Large neural net models are considered to be very expressive. That is, both the population loss $L(\theta)$ and the finite sample loss $\hat{L}(\theta)$ can be made small. In fact, neural networks are \textit{universal approximators}; see for example \cite{hornik1991}. This can be a somewhat misleading statement as the definition of universal approximator allows for the size of the network to be impracticably large, but morally it seems to hold true in practice anyway.
        
This expressivity is possible because neural networks are usually highly \textit{over-parametrized}: they have many more parameters than samples. It is possible to prove that in this regime, the network can ``memorize'' the entire dataset and achieve approximately zero training error \cite{arpit2017memorization}.
    
\item {\bf Statistical generalization:} Relatively weak regularization is used in practice. In many cases only weak $\ell_2$ regularization is used, i.e.
\begin{equation}
\widehat{L}_{\textup{reg}}(\theta)=\hat{L}(\theta)+\lambda\|\theta\|_2^2.
\end{equation}
    
The first interesting fact is that this regularized loss does not have a unique (approximate) global minimizer. This is due to overparametrization: there are so many degrees of freedom that there are many approximate global minimizers with approximately the same $\ell_2$ norm.
    
However, it turns out that these global minimizers are not equally good: many models which achieve zero training error may have very bad test error (Figure~\ref{lec9:fig:bad-global-min}). Take, for example, using stochastic gradient descent (SGD) to learn a model to classify the dataset CIFAR-10. In Figure~\ref{lec9:fig:dl-implicitreg}, we show two instantiations of this: one starting with a large learning rate and slowly decreasing it, and one with a small learning rate throughout. Even though both instantiations result in approximately zero training error, the former leads to much better test performance. 

Therefore, the job of optimizers in deep learning is not just to find an arbitrary global minimum: we need to find the right global minimum. This contrasts sharply with \eqref{lec9:eqn:classical-guarantee} from the classical setting, where achieving a global minimum leads to good guarantees on generalization error. This means that \eqref{lec9:eqn:classical-guarantee} is simply not powerful enough to deal with deep learning, because it cannot distinguish between global minima with good test error and bad test error.

\begin{figure}[t]
    \centering
    \begin{subfigure}[t]{0.49\textwidth}
        \centering
        \includegraphics[width=3in]{figures/bad global min .png}
        \caption{}
        \label{lec9:fig:bad-global-min}
    \end{subfigure}
    \hfill
    \begin{subfigure}[t]{0.49\textwidth}
        \centering
        \hspace*{-1.8em}
        \includegraphics[width=3in]{figures/deep-learning-implicit-reg.png}
        \caption{}
        \label{lec9:fig:dl-implicitreg}
    \end{subfigure}
    \caption{We use dotted and solid lines to depict training and test error, respectively. Figure~\ref{lec9:fig:bad-global-min} demonstrates how global minimizers for the training loss can have differing performance on test data. In Figure~\ref{lec9:fig:dl-implicitreg}, blue and red colors differentiate between the model fit with a decaying learning rate and a small constant learning rate. Though both neural networks shown in this plot achieve 0 training error, the global minimizer obtained by a more sophisticated learning rate schedule appears to generalize better to unseen data.}
    \label{lec9:fig:global_min}
\end{figure}

\item {\bf Optimization:} The discussion above means that optimization plays a significant role in generalization for deep learning. Different training algorithms/optimizers have different ``implicit biases'' or ``implicit regularization effect'', causing them to converge to different global minimizers. Understanding the implicit regularization effect of optimizers is thus a central goal of deep learning theory. The lack of understanding implicit regularization hinders the development of fast optimizers---it is impossible to design a good optimization algorithm without also considering its impact on generalization. In fact, many algorithms for non-convex optimization have been proposed that work well for minimizing training loss, but because their implicit bias is different, they lead to worse test performance and are therefore not too useful.
    
Often these implicit biases or implicit regularization effect can be characterized in the form of showing the optimizers prefer $\hat\theta$ of certain low complexity among all the global minimizers. The deep learning analog of \eqref{lec9:eqn:classical-guarantee} often consists of two statements: (a) the optimizer implicitly prefers low complexity solution according to complexity measure $R(\cdot)$ by converging to a global minimizer $\hat{\theta}$ with low complexity $R(\hat{\theta})$, and (b) low complexity solutions generalize. This means that we end up doing more work on the optimization front---the optimizer needs to ensure both a small training loss and a low complexity solution. On the other hand, proving generalization bounds (statement (b)) works similarly to the classical setting once we understand how our optimizer finds a low-complexity solution.
    
\end{enumerate}

%To explain the success of deep learning, we will cover three tasks in the next two chapters\todo{specify chapter number}:
We summarize some of the results that we will present in the future chapters. \ttodo{add chapter number later}

\begin{enumerate}
    \item \textbf{Optimization.} First, we will prove that under certain data distribution assumption, optimizers such as stochastic gradient decent can converge to an approximate global minimum, even though the objective function is non-convex. Results of this form can be shown on matrix factorization problems and linearized neural networks, even without over-parameterization, but so far are limited to these simple models.  Second, we will discuss a recent approach, called neural tangent kernels (NTK), which proves that for almost any neural networks, with overparameterization, gradient descent can converge to a global minimum, \textit{under specific hyperparameter settings} (e.g, specific learning rate and initialization). However, it turns out that these specific hyperparaemeter settings \textit{does not} provide sufficient implicit regularization effect for the learned models to generalize. (In other words, the optimizer only returns a global minimizer, but not a global minimizer that generalizes well.)
    
    \item \textbf{Implicit regularization effect.} This involves showing that the solution $\hat{\theta}$ obtained by a particular optimizer has low complexity $R(\hat{\theta})\leq C$ according to some complexity measure $R(\cdot)$ (which depends on the choice of optimizers). It's believed and empirically observed that any changes or tricks in the optimizers (e.g., learning rate schedule, batch size, initialization, batchnorm) could introduce additional implicit regularization effects. We will only demonstrate these on some special cases of models (e.g. logistic regression, matrix factorization) and optimizers (e.g. gradient descent, label noise in SGD, dropout, learning rate). Recently, there are also more general results with label noise SGD~\citep{blanc2019implicit,damian2021label}. 
    
    \item \textbf{Generalization bounds.} This part involves showing that for all $\theta$ such that $R(\theta)\leq C$ with $\hat{L}(\theta)\approx 0$, we have $L(\theta)$ is small. That is, we show that low-complexity solutions to the empirical risk problem generalize well. We will be working with more fine-grained complexity measures (e.g., those complexity measures that are similar to the complexity measure in part 2 above that are preferred by the optimizer). Here, many tools we developed in classical machine learning can still apply.
\end{enumerate}

%	\chapter{Nonconvex Optimization}
%	\input{collection/07-01-nonconvex.tex}
%	\input{collection/07-02-nonconvex.tex}
%	% reset section counter
%\setcounter{section}{0}

%\metadata{lecture ID}{Your names}{date}
\metadata{13}{Justin Young and Josh Cho}{November 1, 2021}

\sec{The Neural Tangent Kernel (NTK) Approach} \label{sec:ntk_approach}

In the previous sections, we studied non-convex optimization problems in which all local minima are global. Selecting the parameters of a deep neural network is another commonly encountered non-convex optimization problem, but it is unrealistic to expect that all local minima will also be global minima in this setting. Here we consider a particular objective for which we can identify particular regions of the input space in which all local minima are also global minima. We can show that this objective corresponds to certain types of deep neural networks, but this analysis remains limited. For further reading about this approach to studying neural network optimization, see \cite{liang2018adding} and \cite{du2019width}.

\tnotelong{Tengyu should  double check this later}

To be more formal, we take an appropriate parameter initialization $\theta^0$ such that in a neighborhood around it, which we denote by $B(\theta^0)$, the loss function is convex and its global minimum is attained. Figure \ref{lec13:fig:NTKapproach} depicts a function and region for which this condition holds. 

\begin{figure}[ht]
    \centering
    \includegraphics[scale=0.3]{figures/ntk_initialization.png}
    \caption{Training loss around an initialized $\theta^0$. The dotted lines indicate $B(\theta^0)$, a region where the loss is convex, and where a global minimum exists.}
    \label{lec13:fig:NTKapproach}
\end{figure}


Given a nonlinear $f_\theta(x)$, we examine the Taylor expansion at $\theta^0$: 
\begin{align} 
    f_\theta(x) = \underbrace{f_{\theta^0}(x) + \langle \nabla_\theta f_{\theta^0}(x),\theta-\theta^0 \rangle}_{\defeq g_\theta(x)} + \text{ higher order terms}
\end{align} 

Note that $g_\theta(x)$ is an affine function in $\theta$, as $f_{\theta^0}(x)$ is a constant for fixed $x,\theta^0$. Similarly, defining $\Delta \theta = \theta-\theta^0$, we can say that $g_\theta(x)$ is linear in $\Delta \theta$. For convenience, we will sometimes choose $\theta^0$ such that $f_{\theta^0}(x) = 0$ for all $x$. It is easy to see why such an initialization exists. Consider splitting a two-layer neural network $f_{\theta}(x)$ with width $2m$ into two halves, each with $m$ neurons; the outputs of these two networks are then given by $\sum_{i=1}^m a_i \sigma (w_i^\top x)$ and $\sum_{i=1}^m -a_i \sigma (w_i^\top x)$, respectively. Here, $w_i$ can be randomly chosen so long as $W_i$ is the same in both halves, and $a_i$ can be randomly chosen as long as the other half is initialized with $-a_i$. Summing these two networks together yields $f_{\theta^0}(x) \equiv 0$ for all $x$.

When $f_{\theta^0}(x) \equiv 0$, we have that 
%\begin{align}
  %  y' &= y- f_{\theta^0}(x) \\
 \begin{align}
 g_\theta(x)= \inprod{\nabla_\theta f_{\theta^0}(x), \Delta \theta},
\end{align}
we observe that $\Delta \theta$ depends upon the parameter we evaluate the network at, while $\nabla_\theta f_{\theta^0}(x)$ can be thought of as a feature map since it is a fixed function of $x$ (given the architecture and $\theta^0$) that does not depend on $\theta$ whatsoever. We thus let $\phi(x) \triangleq \nabla_\theta f_{\theta^0}(x)$, which motivates the following definition: 

\begin{definition}[Neural Tangent Kernel]
For simplicity, we assume $f_{\theta^0}(x)=0$ so that $y=y'$. The \textit{neural tangent kernel} $K$ is given by  
\begin{align} 
    K(x,x') &= \inprod{\phi(x), \phi(x')} \\
    &= \inprod{\nabla_\theta f_{\theta^0}(x), \nabla_\theta f_{\theta^0}(x')}.
\end{align} 
\end{definition}
Here, the feature $\nabla_\theta f_{\theta^0}(x)$ is precisely the gradient of the neural network. This is where the ``tangent'' in Neural Tangent Kernel comes from. 

Instead of $f_\theta(x)$, suppose we use the approximation $g_\theta(x)$, which we recall is linear in $\theta$. The kernel method gives a linear model on top of features. When $\theta \approx {\theta^0}$, given a convex loss function $\ell$, we have 
\begin{align} 
    \underbrace{\ell (f_\theta(x),y)}_{\substack{\text{not} \\ \text{necessarily} \\ \text{convex}}} \approx \underbrace{\ell(g_\theta(x),y)}_{\text{convex}}.
\end{align} 
Convexity of the RHS follows from the fact that a convex function, $\ell$, composed with a linear function, $g_\theta$, is still convex. 

A natural question to ask is: how valid is this approximation? We devote the rest of this chapter to answering this question. First, we define the empirical loss: 
\begin{align}
    \hat{L}(f_\theta) & = \frac{1}{n}\sum_{i=1}^n \ell \left( f_\theta\big( x^{(i)} \big) , y^{(i)} \right) \\ 
    \hat{L}(g_\theta) & = \frac{1}{n}\sum_{i=1}^n \ell \left( g_\theta\big( x^{(i)} \big) , y^{(i)} \right).
\end{align} 
The key idea is that the Taylor approximation works for certain cases. We defer a more complete enumeration of these cases to a later section of this monograph. Here we outline the high-level approach we take to validate and use this Taylor expansion. Namely, we will show that there exists a neighborhood around $\theta^0$ called $B(\theta^0)$, such that we have the following:
\begin{enumerate}
    \item Accurate approximation: $f_\theta(x) \approx g_\theta(x)$, and $\hat{L}(f_\theta) \approx \hat{L}(g_\theta)$ for all $\theta \in B(\theta^0)$.
    \item It suffices to optimize in $B(\theta^0)$: There exists an approximate global minimum $\hat{\theta} \in B(\theta^0)$, so $\hat{L}(g_{\hat{\theta}}) \approx 0$. This is the lowest possible loss (because the loss is nonnegative), which implies we are close to the global minimum. Because of 1, this implies that $\hat{L}(f_{\hat{\theta}}) \approx 0$ as well. See Figure~\ref{lec13:fig:ntkglobalmin} for an illustration.
    \item Optimizing $\hat{L} (f_\theta)$ is similar to optimizing $\hat{L}(g_\theta)$ and does not leave $B(\theta^0)$, i.e. everything is confined to this region. Intuitively, this last point to some extent is ``implied" by (1) and (2), but this claim still requires a formal proof. 
\end{enumerate}

\begin{figure}[ht!]
    \centering
    \includegraphics[scale=0.5]{figures/ntk_global_min.png}
    \caption{Here, $\hat{L}(g_{\theta})$ and $\hat{L}(f_{\theta})$ are both plotted. At $\hat{\theta}$, we have reached the approximate global minimum where $\hat{L}(g_{\hat{\theta}}) \approx 0$, in turn implying also that $\hat{L}(f_{\hat{\theta}}) \approx 0$.}
    \label{lec13:fig:ntkglobalmin}
\end{figure}

Note (1), (2), and (3) can all be true in various settings. In particular, to attain all three, we will require: 
\begin{enumerate}[label=\alph*]
    \item[(a)] Overparametrization and/or a particular scaling of the initialized $\theta^0$. 
    \item[(b)] Small (or even zero) stochasticity, so $\theta$ never leaves $B(\theta^0)$. This condition is guaranteed by a small learning rate or full-batch gradient descent. 
\end{enumerate} 
Despite the limitations of the requirements of (a) and (b), the existence of such a region is still surprising. Given the loss landscape which could potentially be highly non-convex, it is striking to find a neighborhood where the loss function is convex (e.g. quadratic) with a global minimum. This suggests there is some flexibility in the loss landscape.  

To begin our formal discussion, we  start by providing tools for proving (1) and (2). Let 
\begin{align}
    \phi^{(i)} = \phi(x\sp{i}) = \nabla_\theta f_{\theta^0}( x\sp{i}) \in \R^p
\end{align}
and 
\begin{align}
    \Phi = \begin{bmatrix} {\phi\sp{1}}^\top \\ \vdots \\ {\phi\sp{n}}^\top \end{bmatrix} \in \R^{n \times p}
\end{align}
where $p$ is the number of parameters. Taking the quadratic loss, we have
\begin{align}
    \hat{L}(g_\theta) = \frac{1}{n}\sum_{i=1}^n \left( y\sp{i} - \phi\l(x\sp{i} \r)^\top \Delta \theta \right)^2 = \frac{1}{n} \norm{\vec{y} - \Phi \cdot \Delta \theta}_2^2
\end{align} 
where $\vec{y} = \l[ y\sp{1}, \cdots, y\sp{n} \r]^\top \in \R^n$. Note that this looks a lot like linear regression, where $\Phi$ and $\Delta \theta$ are the analogues of the design matrix and parameter, respectively. We further assume that $y^{(i)} = O(1)$ and $\norm{y}_2 = O(\sqrt{n})$. Now, we can prove a lemma that addresses the second of the three conditions we described above, i.e. that it is sufficient to optimize in some small ball around $\theta^0$.

\begin{lemma}[for (2)] \label{lec13:lma:nearest_minimum}
    Suppose we are in the setting where $p \geq n$, $\textup{rank}(\Phi) = n$, and $\sigma_{\min}(\Phi) = \sigma > 0$. Then, letting $\Delta \hat{\theta}$ denote the minimum norm solution, i.e. the nearest global minimum, of $\vec{y} = \Phi \Delta \theta$, we have 
    \begin{align} 
        \norm{\Delta \hat{\theta}}_2 \leq O(\sqrt{n} / \sigma)
    \end{align} 
\end{lemma}
\begin{remark} \label{lec13:rmk:intuitiononlemma} 
    The meaning of the bound on $\Delta \hat{\theta}$ becomes clear if we consider the ball given by 
    \begin{align}
        B_{\theta^0} = \{ \theta = \theta^0 + \Delta \theta: \norm{\Delta \theta}_2 \leq O(\sqrt{n}/\sigma )\}.
    \end{align} 
    In particular, notice that $B_{\theta^0}$ contains a global minimum, so this lemma characterizes how large the ball must be to contain a global minimum. 
    \end{remark} 
\begin{remark}
	We also note that the condition $\textup{rank}(\Phi) = n$ and $\sigma > 0$ can be thought of as a ``finite-sample expressivity'' condition, saying that the features $\Phi$ are expressive enough so that there exists a linear model on top of these features that perfectly fit the data. The condition $\textup{rank}(\Phi) = n$ requires $p \ge n$---so we need some amount of over-parameterization to apply these analysis. 
\end{remark}
\begin{proof}
    Letting $\Phi^+$ denote the Moore-Penrose pseudoinverse of $\Phi$, note that $\Delta \hat{\theta} = \Phi^+ \boldsymbol{y}$, and $\norm{\Phi^+} _{\text{op}} = \frac{1}{\sigma_{\min} (\Phi)} = \frac{1}{\sigma}$.  A simple argument shows 
    \begin{align}
        \norm{\Delta \hat{\theta}}_2 &\leq \norm{\Phi^+}_{\text{op}} \cdot \norm{\vec{y}}_2 \\
        &\leq O\left( \frac{1}{\sigma}\cdot \sqrt{n} \right),
    \end{align} 
    where the last inequality follows from the assumption that $\norm{\vec{y}}_2 \leq O(\sqrt{n})$. 
\end{proof}
Next, we prove a lemma that addresses the first of the three steps we described above.
\begin{lemma}[for (1)] 
    \label{lec13:lma:accurate_approximation}
    Suppose $\nabla_\theta f_\theta(x)$ is $\beta$-Lipschitz in $\theta$, i.e. for every $x$, and $\theta, \theta'$, we have 
    \begin{align}
        \norm{\nabla_\theta f_{\theta} (x) - \nabla_{\theta} f_{\theta'}(x)}_2 \leq \beta \cdot \norm{ \theta - \theta'}_2.
    \end{align} 
    Then, 
    \begin{align} 
        \left| f_\theta(x) - g_\theta(x) \right| \leq O \left( \beta \norm{\Delta \theta}_2^2 \right).
    \end{align}  
    If we further restrict our choice of $\theta$ using $B_{\theta^0}$ as defined in Remark~\ref{lec13:rmk:intuitiononlemma}, we obtain that
    \begin{align} 
        | f_\theta(x) - g_\theta(x) | \leq O \left( \frac{\beta n }{\sigma^2 }\right), \quad \forall \theta \in B_{\theta^0}. \label{lec13:eqn:lemma1bound} 
    \end{align} 
\end{lemma}
\begin{proof}
    The proof comes from the following fact:  if $h(\theta)$ is such that $\nabla h(\theta)$ is $\beta$-Lipschitz (which if differentiable is equivalent to $\norm{\nabla^2 h(\theta)}_{\text{op}} \leq \beta$), then
    \begin{align}
        \bigg| \underbrace{h(\theta)}_{f_\theta(x)}  \underbrace{-h(\theta^0) - \inprod{\nabla h(\theta^0), \theta-\theta^0}}_{-g_\theta(x)}\bigg| \leq O\left( \beta \norm{\theta-\theta^0}_2^2 \right).
    \end{align} 
    \tnotelong{add a lemma in the toolbox section about this}
    As shown above, the proof is as simple as plugging in $f_\theta(x) = h(\theta)$ and $g_\theta(x)=h(\theta^0) + \inprod{\nabla h(\theta^0), \Delta \theta}$. 
\end{proof}

\begin{remark}
The lemma above bounds the approximation error. Intuitively, as you move farther away from $\theta^0$, the Taylor approximation gets worse; the approximation error is bounded above by a second order $\Delta \theta$ term.
\end{remark}

\begin{remark}
Note that if $f_\theta$ involves a $\text{relu}$ function, then $\nabla f_\theta$ is not continuous everywhere. This requires a technical fix outside the scope of our discussion.\footnote{A $\text{relu}$ function is continuous almost everywhere, so we can make some minor fixes and still use some modified notion of Lipschitzness to derive an upper bound.} \tnotelong{Tengyu will add a reference here}
\end{remark}

\subsec{Two examples of the NTK regime} \label{sec:ntk:two_examples}
By \eqref{lec13:eqn:lemma1bound}, we have now established a bound on our approximation error, but we have yet to analyze how good it is, as $\beta n /\sigma^2$ is neither obviously either big nor small. An important fact to notice is that $\beta/\sigma^2$ is not scaling invariant, so we can play with the scaling in order to drive this term to $0$. In particular, there are two notable cases (with specific parameterization, initialization, etc) where $\beta/\sigma^2 \to 0$. In the literature, such situation is often referred to as the NTK regime or the lazy training regime~\cite{chizat2018note}. 
\begin{enumerate}
    \item  \textbf{Reparameterize with a scalar} \cite{chizat2018note}. Let $f_\theta(x) = \alpha \cdot \bar{f}_\theta(x)$ where $\bar{f}_\theta(x)$ is an arbitrary neural net with fixed width and depth. We only vary $\alpha$, i.e. the scaling, and we see how the crucial quantity $\beta/\sigma^2$ changes accordingly. Fix an initial $\theta^0$, and let 
    \begin{align}
        \bar{\sigma} = \sigma_{\min}\left( \begin{bmatrix}  \nabla_\theta \bar{f}_{\theta^0} \big( x^{(1)} \big)^\top \\ \vdots \\ \nabla_\theta \bar{f}_{\theta^0} \big(x^{(n)} \big)^\top \end{bmatrix}\right).
    \end{align} 
    Furthermore, let $\bar{\beta}$ be the Lipschitz parameter of $\nabla_\theta \bar{f}_\theta(x)$ in $\theta$. A simple chain-rule gradient argument shows that scaling $\bar{f}_{\theta}$ by $\alpha$ also scales $\sigma$ and $\beta$ accordingly, i.e. $\sigma = \alpha \bar{\sigma}$, and $\beta = \alpha \bar{\beta}$. Some straightforward algebra yields 
    \begin{align} 
        \frac{\beta}{\sigma^2}= \frac{\bar{\beta}}{\bar{\sigma}^2} \cdot \frac{1}{\alpha} \to 0 \quad \text{as} \quad \alpha \to \infty.
    \end{align}
    Once $\alpha$ becomes big enough, then by Lemma~\ref{lec13:lma:accurate_approximation}, the approximation $|f_\theta(x) - g_\theta(x)| \leq O\left( \beta n / \sigma^2 \right)$ becomes very good. 
    
\begin{remark} A priori, such a phenomenon may appear to be too good to be true. To understand it better, we first note that this re-parameterizaton does not change the scale of the loss, but rather change the shape of the loss function. Intuitively, as $\alpha$ becomes larger, the function $f_\theta$ becomes sharper and more non-smooth (leading to higher approximation error). However, on the other hand, we note that we only need to travel a little bit away from $\theta^0$ to find a global minimum given that there is a global minimum within radius $O(\sqrt{n}/\sigma)$. It turns out that the radius needed shrinks faster than the smoothness grows. 
    
    To visualize this effect, we can consider the following example with only 1 data point with 1-dimensional input $(x,y) = (1,1)$ and the quadratic model $\bar{f}_\theta(x) = x(\theta + \beta \theta^2) = \theta + \beta \theta^2$. Using the squared loss, we have 
    \begin{align}
    \hatL(\bar{f}_\theta) = (1- (\theta + \beta \theta^2))^2 
    \end{align}
    Let $\theta^0 = 0$. Taylor expanding at $\theta^0$ gives the linear approximation $\bar{g}_\theta(x) = \theta x$ = $\theta$, and the resulting loss function that is quadratic 
    \begin{align}
    \hatL(\bar{g}_\theta) = (1- \theta)^2 
    \end{align}
	In this case,  $\nabla f_{\theta^0}(x) = 2\beta \theta x = 2\beta \theta$ is $2\beta$-Lipschitz, and $ \sigma = 1$. 
	
	Now we vary $\alpha$ and get 
    \begin{align}
\hatL(\alpha \bar{f}_\theta) = (1- \alpha(\theta + \beta \theta^2))^2 
\end{align}	
and 
\begin{align}
\hatL(\alpha\bar{g}_\theta) = (1- \alpha\theta)^2 
\end{align}
Note that the minimizer of $	\hatL(\alpha\bar{g}_\theta) $ is $1/\alpha$, which is closer to $\theta^0$ as $\alpha\rightarrow \infty$. We zoom into the region $[0, 1/\alpha]$ and find out the difference between $\alpha \bar{f}_\theta$ and $\alpha \bar{g}_\theta$ is $\alpha \beta\theta^2 \le  \beta/\alpha$, which is much smaller than the value of $\alpha \bar{g}_\theta \approx O(1)$. 

We visualize these functions in Figure~\ref{fig:ntk-1d}. We observe that $\hatL(\alpha\bar{g}_\theta) $ becomes a better approximation of $\hatL(\alpha\bar{f}_\theta)$ in the region $[0,1/\alpha]$ as $\alpha \rightarrow \infty$ (though $\hatL(\alpha\bar{g}_\theta)$ is a worse approximation of $\hatL(\alpha\bar{f}_\theta)$ globally.)


\begin{figure}[t]
	\centering
	\includegraphics[width = 0.6\textwidth]{figures/ntk-1d.png}
	\caption{\label{fig:ntk-1d} The approximation $\hatL(\alpha\bar{g}_\theta) $ becomes a better approximation of $\hatL(\alpha\bar{f}_\theta)$ in the region $[0,1/\alpha]$ as $\alpha \rightarrow \infty$ (though $\hatL(\alpha\bar{g}_\theta)$ is a worse approximation of $\hatL(\alpha\bar{f}_\theta)$ globally).}
\end{figure}
	%The minimum norm solution with $\alpha g_\theta$ is
%    \[
%    \argmin{\theta} (1-\alpha  \theta)^2 = 1/\alpha.
%    \]
%    Now, for $\alpha\geq 1$ we compute
%    \[
%    D(\alpha) = \sup_{\theta\in [0,1/\alpha ]} |\alpha f_\theta(1) - \alpha g_\theta(1)|. 
%    \]
%    We will plot $D(\alpha)$ as well as $\hat L(\alpha f_\theta(1))$ and $\hat L(\alpha g_\theta(1))$ for $\alpha = 1,2,4$. In the following plots, we use $\beta = 2$. 
%   
\end{remark}
    \item \textbf{Overparametrization (with specific initialization)}. Early papers on the NTK take this approach (e.g., ~\cite{li2018learning,du2019width}). Consider a  two-layer network with $m$ neurons. 
    \begin{align}
        \hat{y} = \frac{1}{\sqrt{m}} \sum_{i=1}^m a_i \sigma(w_i^\top x )
    \end{align} 
    The scaling $1/\sqrt{m}$ is to ensure that a random initialization with constant scale will have output on the right order, as we see momentarily. We make the following assumptions regarding the network and its inputs.
    \begin{align}
        W &= \begin{bmatrix} w_1^\top \\ \vdots \\ w_m^\top \end{bmatrix} \in \R^{m \times d} \\
        \sigma &\text{ is $1$-Lipschitz and twice-differentiable} \\
        a_i &\sim \{\pm 1\} \quad &\text{(not optimized)} \\
        w_i^0 &\sim \cN(0, I_d) \\
        \norm{x}_2 &= \Theta(1) \\
        \theta &= \text{vec}(W) \in \R^{dm} \quad &\text{(vectorized $W$)}
    \end{align} 
    We will assume $m \to \infty$ polynomially in $n$ and $d$. In particular, for fixed $n,d$, we have $m = \textsf{poly}(n,d)$.
    
    Why do we use the $1/\sqrt{m}$ scaling? Note that $\sigma\big({w_i^0}^\top x\big) \approx 1$ because $\norm{x}_2 = \Theta(1)$ and $w_i^0$ is drawn from a spherical Gaussian. Thus, as some $a_i$ are positive and others are negative, $\left|\sum_{i=1}^m a_i \sigma \big({w_i^0}^\top x\big) \right| = \Theta \left( \sqrt{m} \right)$, and finally $f_{\theta^0} (x) = \Theta(1)$. 
    
    Now we analyze $\sigma$ and $\beta$. We let
    \begin{align}
        \sigma = \sigma_{\min} (\Phi) = \sqrt{\sigma_{\min} \left( \Phi \Phi^\top \right)}
    \end{align}
    where 
    \begin{align}
        \left( \Phi \Phi^\top \right)_{ij} = \inprod{\nabla_\theta f_{\theta^0} \big(x^{(i)} \big) , \nabla_\theta f_{\theta^0} \big(x^{(j)} \big)} \label{lec13:eqn:phifeature} 
    \end{align} 
    Note that the gradient with respect to $w_i$ is given by 
    \begin{align}  
        \frac{\partial f_\theta(x) }{\partial w_i} = \frac{1}{\sqrt{m}} \sigma'(w_i^\top x ) \cdot x 
    \end{align} 
    Now observe that
    \begin{align}
        \norm{\nabla f_\theta(x)}_2^2 & = \frac{1}{m}\sum_{i=1}^m \norm{\sigma'\big({w_i}^\top x \big) \cdot x }_2^2 \\ 
        & = \frac{1}{m}\norm{x}_2^2 \cdot \sum_{i=1}^m \left( \sigma' \big({w_i}^\top x \big) \right)^2 \\ 
        &\to \Exp_{w \sim \cN(0,I_d)} \left[ \sigma' \big( w^\top x \big)^2 \right] \cdot \norm{x}_2^2 \quad \text{as} \quad m\to\infty \\ 
        &= O(1) &\text{(not depending on $m$)}
    \end{align} 
    where the penultimate line follows from the law of large numbers, as $\frac{1}{m} \sum_{i=1}^m \left( \sigma'(w_i^\top x ) \right)^2$ can be interpreted as a mean. 
    
    Note that the scale of $\norm{\nabla_\theta f_{\theta^0} (x)}_2$ does not depend on $m$, so the inner product in \eqref{lec13:eqn:phifeature} also does not depend on $m$ either. As above, we can show 
    \begin{align} 
        \inprod{\nabla_\theta f_{\theta^0} (x), \nabla_\theta f_{\theta^0} (x')} & = \frac{1}{m}\inprod{ x,x'} \sum_{i=1}^m \sigma'(w^\top x) \sigma'(w^\top x')  \\ 
        & \to \Exp_{w \sim \cN(0,I_d)} \left[ \sigma'(w^\top x) \sigma'(w^\top x') \right] \inprod{ x, x'} \label{lec13:eqn:kernelcalc} 
    \end{align}
    
    \eqref{lec13:eqn:kernelcalc} implies that as $m \to \infty$, $\Phi \Phi^\top$ converges to a constant matrix denoted by 
    \begin{align}
        K^\infty = \lim_{m \to \infty} \Phi\Phi^\top 
    \end{align}
    This is precisely the NTK with $m=\infty$.  Though we omit the proof of this claim, it can be shown that $K^\infty$ is full rank. Then, let \begin{align}
        \sigma_{\min} \triangleq \sigma_{\min} (K^\infty) > 0.
    \end{align}
    We can show that 
    \begin{align}
        \sigma = \sigma_{\min} \left( \Phi \Phi^\top \right) > \frac{1}{2}\sigma_{\min} 
    \end{align} 
    Intuitively, $\Phi \Phi^\top \to K^\infty$, so the spectrum of the matrix should also converge. Thus, in some sense, we have shown that $\sigma$ is constant in the limit. 
    
    Now what about $\beta$? If we can show $\beta \to 0$ as $m \to \infty$, we are done. We begin by analyzing this key expression:  
    \begin{align}
        \nabla_\theta f_\theta(x) - \nabla_\theta f_{\theta'} (x) = \left[ \frac{1}{\sqrt{m}} \left( \sigma' \big( w_i^\top x \big) - \sigma' \big({w_i'}^\top x \big) \right) \cdot x \right]_{i=1}^m \label{lec13:eqn:lipschitzmatrix}
    \end{align}
    Note that \eqref{lec13:eqn:lipschitzmatrix} above consists of matrices, as $\theta$ is a vectorized matrix. Then,
    \begin{align}
        \norm{\nabla_\theta f_\theta(x) - \nabla_\theta f_{\theta'}(x)}_2^2 & = \frac{1}{m}\sum_{i=1}^m \norm{x}_2^2 \left( \sigma' \big(w_i^\top x \big) - \sigma' \big({w_i}'^\top x \big) \right)^2  \\ 
        & \leq O \left( \frac{1}{m}\sum_{i=1}^m \norm{ x}_2^2 \big( w_i^\top x - {w_i'}^\top x \big)^2 \right) \\ 
        & =  O \left( \frac{1}{m}\sum_{i=1}^m \norm{ w_i - w_i'}_2^2 \right) \\ 
        & = O \left(\frac{1}{m} \norm{ \theta - \theta' }_2^2 \right)
    \end{align} 
    The first line follows from the fact that $\frac{1}{\sqrt{m}} \left( \sigma' \big( w_i^\top x \big) - \sigma' \big({w_i'}^\top x \big) \right)$ is a scalar. The second line uses the assumption that $\sigma'$ is $O(1)$-Lipschitz. The third line uses Cauchy-Schwarz and the fact that $\norm{x}_2^2 \approx 1$. Taking the square root, we have that
    \begin{align} 
        \norm{\nabla_\theta f_\theta(x) - \nabla_\theta f_{\theta'}(x)}_2 \lesssim \frac{1}{\sqrt{m} } \norm{ \theta -\theta' }_2
    \end{align} 
    Thus, the Lipschitz parameter is $\beta = O(1/\sqrt{m})$. Thus, our key quantity $\beta/\sigma^2$ goes to $0$ as $m$ grows. Namely,
    \begin{align} 
        \frac{\beta}{\sigma^2} \approx \frac{1}{\sqrt{m} }\cdot \frac{1}{\sigma_{\min}^2} \to 0 \quad \text{as} \quad m\to\infty.
    \end{align} 
    Recall here that $\sigma_{\min}$ does not depend on $m$. Concretely, this result tells us that our function becomes more smooth (the gradient has a smaller Lipschitz constant) as we add more neurons. 
\end{enumerate}

\subsec{Optimizing \texorpdfstring{$\hat{L}(g_\theta)$}{L(g)} vs. \texorpdfstring{$\hat{L}(f_\theta)$}{L(f)}}
We now discuss how to establish the last of the three conditions under which we claimed a Taylor approximation is reasonable. We need to show that  optimizing $\hat{L} (f_\theta)$ is similar to optimizing $\hat{L}(g_\theta)$. To do so, we require two steps:
\begin{enumerate}[label=\alph*]
    \item[(A)] Analyze optimization of $\hat{L}(g_\theta)$.
    \item[(B)] Analyze optimization of $\hat{L}(f_\theta)$ by re-using or modifying the proofs in (A).
\end{enumerate}
There are two approaches in the literature for (A), which implies that there exist two approaches for (B) as well. 
\begin{enumerate}
    \item[(i)] We leverage the strong convexity of $\hat{L} (g_\theta)$, and then show an exponential convergence rate.\footnote{Recall that a differentiable function $f$ is $\mu$-strongly convex if 
    \begin{align} 
        f(y) \geq f(x) + \nabla f(x)^\top (y-x) + \frac{\mu}{2} \norm{y-x}_2^2
    \end{align} for some $\mu>0$ and all $x,y$.} 
    \item[(ii)] Instead of strong convexity, we rely on the smoothness of $f_\theta$ (i.e. bounded second derivative). 
\end{enumerate}
We will only discuss the first of these two methods in the sequel.

\begin{remark} In both either approach (i) or (ii), we will implicitly or explicitly use the following simple fact. 
Suppose at any $\theta^t$, we take the Taylor expansion of $f_\theta$ at $\theta^t$:
\begin{align} 
    g_\theta^t(x) = f_{\theta^t} (x) + \inprod{ \nabla f_{\theta^t} (x),\theta-\theta^t } 
\end{align} 
Consider the gradient we are interested in taking: $\nabla \hat{L} ( f_{\theta^t})$. Notice that: \begin{align} 
    \nabla \hat{L} ( f_{\theta^t}) = \nabla \hat{L} ( g_{\theta^t}^t)
\end{align} 
This is really saying that $f_\theta$ and $g_\theta^t$ agree up to first-order at $\theta^t$. This implies that $L(f_\theta)$ and $L(g_\theta^t)$ also agree to first-order at $\theta^t$. This also means that $T$ steps of gradient descent on $\hat{L}(f_\theta)$ is the same as performing online gradient descent\footnote{Online gradient descent is the algorithm that takes one gradient descent step upon receiving a new objective function. See Chapter~\ref{chap:OL} for more discussions about online learning.} on a sequence of changing objectives $L(g_\theta^0), \ldots, L(g_\theta^T)$, and this online learning perspective is useful in the approach (ii). 
\end{remark} 

We will now show that under the strong convexity regime, optimizing a neural network $f_\theta$ is equivalent to optimizing a linear model $g_\theta$. We will also observe that this regime is not particularly practically relevant, but this analysis is nevertheless of interest to us for two reasons. First, the approach used in the subsequent exposition is of technical interest and second, it remains quite interesting that optimizing $f_\theta$ and optimization $g_\theta$ yields the same results under \emph{any} regime. 

\subsubsec{Optimizing $g_\theta$}
We relate the optimization of $g_\theta$ to performing linear regression. Recall that we can think of $\nabla f_{\theta^0}(x)$ as a feature map. Then, the problem of choosing $\Delta \theta$ to get $g_\theta(x)$ to be close to $\vec{y}$ is a linear regression. In particular, we use gradient descent to minimize
\al{
\norm{\vec{y} - \Phi\Delta \theta}_2^2,
}
where 
\al{
\Phi =
\begin{bmatrix}
\nabla f_{\theta^0}(x^{(1)})^\top \\
\vdots \\
\nabla f_{\theta^0}(x^{(n)})^\top
\end{bmatrix}
\in \R^{n \times p}. 
\quad \quad \vec{y} = \begin{bmatrix} y\sp{1} \\ \vdots\\ y\sp{n} \end{bmatrix} \in \R^n
}
For learning rate $\eta$, the gradient descent update rule is 
\al{
{\Delta \theta}^{t+1} = \Delta \theta^{t} - \eta \Phi^\top (\Phi \Delta \theta^t - \vec{y}). \label{lec14:eqn:update-rule}
}
This analysis considers changes in the output space. Define $\hat{y}^t = \Phi \Delta \theta^t$. Then, we're interested in changes in 
\al{
\hat{y}^{t+1} - \vec{y} &= \Phi \Delta \theta^{t+1} - \vec{y}\\
&= \Phi \left( \Delta \theta^{t} - \eta \Phi^\top (\Phi \Delta \theta^t - \vec{y})\right) - \vec{y} &\text{(by \eqref{lec14:eqn:update-rule})}\\
&= \left( \Phi - \eta \Phi \Phi^\top \Phi\right)\Delta \theta^t - (I - \eta \Phi \Phi^\top)\vec{y}\\
&= (I - \eta \Phi \Phi^\top )\Phi \Delta \theta^t - (I - \eta \Phi \Phi^\top )\vec{y}\\
&= (I - \eta \Phi \Phi^\top) (\Phi \Delta \theta^t - \vec{y})\\
&= (I - \eta \Phi \Phi^\top)(\hat{y}^t - \vec{y}). \label{lec14:eqn:g_decomp}
}
From this decomposition, we see that the residuals, $\hat{y}^t - \vec{y}$, are monotonically shrinking since $\eta \Phi \Phi^\top$, i.e. the term we are subtracting from $I$ in \eqref{lec14:eqn:g_decomp}, is positive semidefinite. Next, we quantify how quickly we are shrinking the residuals. Define 
\begin{align}
    \tau^2 &= \sigma_{\text{max}}(\Phi \Phi^\top) \\
    \sigma &= \sigma_\text{min}(\Phi) = \sqrt{\sigma_\text{min}(\Phi\Phi^\top)}. \label{lec14:eqn:sigma_def}
\end{align}
Then, we claim that when $\eta \leq \frac{1}{\tau^2}$,
\al{
\norm{I - \eta \Phi \Phi^\top }_{\text{op}} \leq 1-\eta \sigma^2. \label{lec14:eqn:g_decomp_op}
}
Why? Let the eigenvalues of $\Phi \Phi^\top$ be (in descending order) $\tau_1^2, \dots , \tau_n^2$. By definition, $\tau_1^2 = \tau^2$ and $\tau_n^2 = \sigma^2$. Now, given the singular value decomposition, $\Phi = U\Sigma V^\top$, we obtain the eigendecomposition: 
\al{
I - \eta \Phi \Phi^\top &= I - \eta U \Sigma^2 U^\top \\
&= U U^\top - \eta U \Sigma^2 U^\top \\ 
&= U(I - \eta \Sigma^2)U^\top \label{lec14:eqn:g_coeff_eigendecomposition}.
}
\eqref{lec14:eqn:g_coeff_eigendecomposition} is the eigendecomposition of $I - \eta \Phi \Phi^\top$, so $I - \eta \Phi \Phi^\top$ has eigenvalues $1 - \eta \tau_1^2, \dots, 1 - \eta \tau_n^2$.
\tnotelong{add more backgrounds about pseudo-inverse and SVD, and linear algebra} Note that assuming $\eta \leq \frac{1}{\tau^2}$ ensures that all eigenvalues of $I - \eta \Phi \Phi^\top$ are non-negative. Thus,
\al{
\norm{I - \eta \Phi \Phi^\top}_\text{op} &\leq \max_j |1-\eta \tau_j^2|\\
&= 1 - \eta \tau_n^2 \label{lec14:eqn:eigenvalue_bound}\\
&= 1 - \eta \sigma^2,
}
where the non-negativity of $1 - \eta \tau_j^2$ for all $j$ implies \eqref{lec14:eqn:eigenvalue_bound}.

Using this result, we obtain our desired result. Namely, assuming $\eta \leq \frac{1}{\tau^2}$,
\al{
\norm{\hat{y}^{t+1} - \vec{y}}_2 &= \norm{I - \eta \Phi \Phi^\top }_\text{op} \cdot \norm{\hat{y}^t - \vec{y}}_2 \\
&\leq (1-\eta\sigma^2)\norm{\hat{y}^t - \vec{y}}_2 \\
&\leq (1 - \eta \sigma^2 )^{t+1}\norm{\hat{y}^0 - \vec{y}}_2.
}
This yields the desired exponential decay in the error. Thus, after $T = O \l( \frac{\log 1/\epsilon}{\eta \sigma^2}\r)$ iterations, 
\al{
\norm{ \hat{y}^T - \vec{y} }_2 \leq \epsilon \norm{\hat{y}^0 - \vec{y}}_2. \label{lec14:eqn:g_exponential_decay}
}

\subsubsec{Optimizing $f_\theta$}
We now transition to an analysis of the optimization of $f_{\theta}$. Our key result is Theorem \ref{lec14:thm:optimization_f}. If we compare it against what we have in \eqref{lec14:eqn:g_exponential_decay}, we see the claimed similarity between $f_\theta$ and $g_\theta$ in error decay under optimization. 

\begin{theorem}
There exists a constant $c_0 \in (0, 1)$ such that for $\frac{\beta}{\sigma^2} \leq \frac{c_0}{n}$ and sufficiently small $\eta$ (which could depend on $\beta, \sigma$, or $p$), $\hat{L}\l(f_{\theta^T}\r) \leq \epsilon$ after $T = O \l(\frac{\log 1/\epsilon}{\eta\sigma^2}\r)$ steps. \label{lec14:thm:optimization_f} 
\end{theorem}

\begin{proof}

(This is actually a proof sketch that elides a few technical details for the sake of a simpler exposition.) Our approach is to follow the preceding analysis of $g_\theta$, making changes where necessary.

Let  
\al{
\Phi^t =
\begin{bmatrix}
\nabla f_{\theta^t}(x^{(1)})^\top \\
\vdots \\
\nabla f_{\theta^t}(x^{(n)})^\top
\end{bmatrix}
\in \R^{n \times p}.
}
To obtain our gradient descent update rule, we find, using the chain rule,
\begin{align}
    \nabla \hat{L}\l(f_{\theta^t}\r) &=  \sum_{i=1}^n\l(f_{\theta^t}\l(x^{(i)}\r) - y^{(i)}\r)\nabla f_{\theta^t}\l(x^{(i)}\r) \\ 
    &=  \sum_{i=1}^n\l(\hat{y}^{(i), t} - y^{(i)}\r)\nabla f_{\theta^t}\l(x^{(i)}\r) \\
    &= (\Phi^t)^\top\l(\hat{y}^t - \vec{y}\r).
\end{align}
This results in the policy
\begin{align}
    \theta^{t+1} &= \theta^t - \eta \nabla \hat{L}\l(f_{\theta^t}\r) \\
    &= \theta^t - \eta (\Phi^t)^\top\l(\hat{y}^t - \vec{y}\r) \\ 
    &= \theta^t - \eta b^t,
\end{align}
where we have let $b^t = (\Phi^t)^\top\l(\hat{y}^t - \vec{y}\r)$. Following our treatment of $g_\theta$, we want to express $\hat{y}^{t+1}$ as a function of $\hat{y}^{t}$. The challenge now is that $f$ is nonlinear. To deal with this, we Taylor expand $f_\theta$ at $\theta_t$:

\begin{align}
   f_{\theta^{t+1}}(x^{(i)}) &= f_{\theta^{t}}(x^{(i)}) + \l<\nabla f_{\theta^t}(x^{(i)}), \theta^{t+1} - \theta^t \r> + \text{high order terms} \\
   &= f_{\theta^{t }}(x^{(i)}) + \l<\nabla f_{\theta^t}(x^{(i)}), -\eta b^t \r> + O\l(\norm{\theta^{t+1} - \theta^t}_2^2\r). \label{lec14:eqn:f_taylor_expansion}
 \end{align}
Since $O\l(\norm{\theta^{t+1} - \theta^t}_2^2\r)$ is $O\l(\eta^2\r)$, we can ignore this term as $\eta \rightarrow 0$. Vectorizing \eqref{lec14:eqn:f_taylor_expansion} without $O\l(\norm{\theta^{t+1} - \theta^t}_2^2\r)$,
\begin{align}
    \hat{y}^{t+1} &= \hat{y}^t - \eta \Phi^t b^t \\
    &= \hat{y}^t + \eta \Phi^t\l(\Phi^t\r)^\top(\vec{y} - \hat{y}^t).
\end{align}
Subtracting $\vec{y}$ and re-arranging,
\begin{align}
    \hat{y}^{t+1} - \vec{y} &= \hat{y}^t - \vec{y} + \eta \Phi^t\l(\Phi^t\r)^\top(\vec{y} - \hat{y}^t) \\ 
    &= \l(I - \eta \Phi^t\l(\Phi^t\r)^\top\r)\l(\hat{y}^t - \vec{y}\r). \label{lec14:eqn:f_decomposition}
\end{align}
Comparing \eqref{lec14:eqn:f_decomposition} with \eqref{lec14:eqn:g_decomp}, we see one difference: in \eqref{lec14:eqn:f_decomposition}, our convergence depends on $\eta \Phi^t\l(\Phi^t\r)^\top$, which is a matrix that changes as we iterate, whereas in \eqref{lec14:eqn:g_decomp}, convergence is controlled by a matrix that is fixed as we iterate. 

To understand the convergence implications of \eqref{lec14:eqn:f_decomposition}, we examine the eigenvalues of $I - \eta \Phi^t \l(\Phi^t\r)^\top$. For now, suppose 
\begin{equation}
    \norm{\theta^t - \theta^0}_2 \leq \sigma/(4\sqrt{n}\beta)
\end{equation} 
at time $t$. This implies that $\norm{\Phi^t - \Phi}_F \leq \frac{\sigma}{4}$ by the Lipschitzness of $\nabla f_\theta(x)$ in $\theta$. Then, we claim that 
\begin{align}
    \sigma_{\text{min}}(\Phi^t) \geq 3\sigma/4. \label{lec14:eqn:phi_t_eigenvalue_bound}
\end{align}
Why does \eqref{lec14:eqn:phi_t_eigenvalue_bound} hold? Observe that
\begin{align}
    \sigma_\text{min}(\Phi^t) &= \underset{\norm{x}_2=1}{\text{min}} x^\top \Phi^tx \\
   &\geq \underset{\norm{x}_2=1}{\text{min}} x^\top (\Phi^t - \Phi)x + \underset{\norm{x}_2=1}{\text{min}}  x^\top \Phi x. \label{lec14:eqn:eigenbound_phi_t}
\end{align}
We can lower bound the first term of \eqref{lec14:eqn:eigenbound_phi_t} as follows:
\begin{align}
    x^\top (\Phi^t - \Phi)x &\geq -|\l<x, (\Phi^t - \Phi)x\r>| \\
    &\geq -\norm{x}_2 \norm{(\Phi^t - \Phi)x}_2 &\text{(Cauchy-Schwarz)}\\ 
    &\geq -\norm{\Phi^t - \Phi}_2 &\text{($\norm{x}_2 = 1$)}\\ 
    &\geq -\sigma/4 &\text{(Lipschitzness of $\Phi$)}. \label{lec14:eqn:termone_eigenbound}
\end{align}
Next, we note that the second term of \eqref{lec14:eqn:eigenbound_phi_t} is lower bounded by $\sigma$ by simplifying and applying the definition of $\sigma$ given in \eqref{lec14:eqn:sigma_def}. Combining this observation with \eqref{lec14:eqn:termone_eigenbound}, we conclude that \eqref{lec14:eqn:phi_t_eigenvalue_bound} must hold.

Applying this lower bound on the eigenvalues of $\Phi^t$, we can use the same argument we used to establish \eqref{lec14:eqn:g_decomp_op} to conclude that
\begin{align}
    \norm{I - \eta \Phi^t \l(\Phi^t\r)^\top}_{\text{op}} \leq 1 - 3\eta \sigma/4 \label{lec14:eqn:op_norm_bound},
\end{align}
and 
\begin{align}
    \norm{\hat{y}^{t+1} - \vec{y}}_{2} \leq \l(1 - 3\eta \sigma/4\r)^{t+1} \norm{\hat{y}^{0} - \vec{y}}_{2}.
\end{align}
So, as desired, we see exponential decay in the error at each iteration and after $T = O \l( \frac{\log 1/\epsilon}{n\sigma^2}\r)$ iterations,
\al{
\hat{L}(f_{\theta^T}) \leq \epsilon.
}

To complete our proof, observe that this argument is predicated upon the assumption that $\norm{\theta^t - \theta^0}_2 \leq \sigma/(4\sqrt{n}\beta)$. This assumption is reasonable, however, given what we have already proven. Recall that in Lemma~\ref{lec13:lma:nearest_minimum}, we proved that 
\begin{align}
    \norm{\Delta \hat{\theta}}_2 = \norm{\hat{\theta} - \theta^{0}}_2 \lesssim \sqrt{n}/\sigma.
\end{align}
Thus, when $\beta/\sigma^2 \rightarrow 0$, eventually, $\sqrt{n}/\sigma \ll \sigma/(4\sqrt{n}\beta)$. To extend this to $\norm{\hat{\theta} - \theta^t}_2$ for arbitrary $t$, we heuristically argue that since the empirical minimizer is within $\sigma/(4\sqrt{n}\beta)$ of $\theta^0$, we would not expect to have traveled more than $\sigma/(4\sqrt{n}\beta)$ from $\theta^0$ at \emph{any} iteration. 

More formally, we claim that for all $t \in \mathbb{N}$, 
\al{
\norm{\hat{y}^t - \vec{y}}_2 \leq \cO (\sqrt{n}).  \label{lec14:eqn:induction}
}
We proceed via induction. For $t=0$, because each element of $\hat{y}$ is of order $1$, we know that: 
\al{
\frac{1}{\sqrt{n}} \norm{\hat{y}^0 - \vec{y}}_2 \leq O (1).
}
Now, suppose that \eqref{lec14:eqn:induction} holds for some $t$. Then, because the errors are monotonically decreasing, (cf. \eqref{lec14:eqn:f_decomposition} and \eqref{lec14:eqn:op_norm_bound}), 
\al{
\frac{1}{\sqrt{n}} \norm{\hat{y}^{t+1} - \vec{y}}_2 \leq \frac{1}{\sqrt{n}}  \norm{\hat{y}^t - \vec{y}}_2 \leq O(1).
}
Thus, \eqref{lec14:eqn:induction} holds for all $t \in \mathbb{N}$. 

Next, applying Lemma~\ref{lec13:lma:accurate_approximation} with $\theta = \theta^t$ and our assumption that $\frac{\beta}{\sigma^2} \lesssim \frac{1}{n}$, we conclude that:
\begin{align}
    \frac{1}{\sqrt{n}} \norm{\Phi \theta^t - \hat{y}^t}_2 \leq O(1)
\end{align}
Using this result and \eqref{lec14:eqn:induction}, we can show that $\frac{1}{\sqrt{n}} \norm{\Phi(\theta^t - \hat{\theta})}_2$ is $O(1)$.
\al{
    \frac{1}{\sqrt{n}} \norm{\Phi(\theta^t - \hat{\theta})}_2 &= \frac{1}{\sqrt{n}} \norm{\Phi \theta^t - \vec{y}}_2 &\text{($\vec{y} = \Phi \hat{\theta}$)} \\
    &= \frac{1}{\sqrt{n}} \norm{\Phi \theta^t - \hat{y}^t + \hat{y}^t - \vec{y}}_2 \\
    &\leq \frac{1}{\sqrt{n}} \norm{\Phi \theta^t - \hat{y}^t}_2 + \frac{1}{\sqrt{n}}\norm{\hat{y}^t - \vec{y}}_2 &\text{(triangle ineq.)} \\
    &\leq O(1).
}
Then, leveraging the definition of $\sigma$ given in \eqref{lec14:eqn:sigma_def} and rearranging, we obtain (nearly) the desired result:
\al{
\norm{\theta^t - \hat{\theta}}_2 \leq \frac{1}{\sigma}\norm{\Phi (\theta^t - \hat{\theta})}_2 &\leq O(\sqrt{n}/\sigma).
}
Recall that in Lemma~\ref{lec13:lma:nearest_minimum}, we proved that 
\al{
\norm{\hat{\theta} - \theta^0}_2 \leq O(\sqrt{n}/\sigma).
}
If $\beta/\sigma^2 \ll 1/n$, we conclude that
\al{
\norm{\theta^t - \theta^0}_2 &\leq \norm{\hat{\theta} - \theta^0}_2 + \norm{\theta^t - \hat{\theta}}_2 &\text{(triangle ineq.)}\\
&\leq O\l (\frac{\sqrt{n}}{\sigma} \r ) \leq \frac{\sigma}{4\sqrt{n}\beta}. 
}
\end{proof}
\subsec{Limitations of NTK}

The NTK approach has its limitations.
\begin{itemize}
    \item Empirically, optimizing $g_\theta(x)$ as described in the theory does not work as well as state-of-the-art (or even standard) deep learning methods. For example, using the NTK approach (i.e., taking the Taylor expansion and optimizing $g_{\theta}(x)$) with a ResNet generally does not perform as well as ResNet with best-tuned hyperparameters.
    
    \item The NTK approach requires a specific initialization scheme and learning rate which may not coincide with what is commonly used in practice.
    
    \item The analysis above was for gradient descent, while stochastic gradient descent is used in practice, introducing noise in the procedure. This means that NTK with stochastic gradient descent requires a small learning rate to stay in the initialization neighborhood. Deviating from the requirements can lead to leaving the initialization neighborhood.
\end{itemize}

One possible explanation for the gap between theory and practice is because NTK effectively requires a fixed kernel, so there is no incentive to select the right features. Furthermore, the minimum $\ell_2$-norm solution is typically dense. This is similar to the difference between sparse and dense combinations of features observed in the $\ell_1$-SVM/two-layer network versus the standard kernel method SVM (or $\ell_2$-SVM) analyzed previously.

To make these ideas more concrete, consider the following example \cite{wei2020regularization}. 
\begin{example}\label{lec12:ex:sparse123}
Let $x \in \R^d$ and $y \in \{-1, 1\}$. Assume that each component of $x$ satisfies $x_i \in \{ -1, 1\}$. Define the output $y = x_1x_2$, that is, $y$ is only a function of the first two components of $x$.

This output function can be described exactly by a neural network consisting of a sparse combination of the features (4 neurons to be exact):
\begin{align}
\hat y &= \frac{1}{2} \left[ \phirelu(x_1 + x_2) + \phirelu(-x_1 - x_2)  - \phirelu(x_1 - x_2) -  \phirelu(x_2 - x_1)  \right] \\
&= \frac{1}{2}\left( |x_1 + x_2| - |x_1 - x_2| \right) \label{lec12:eqn:ex1} \\
&= x_1x_2. \label{lec12:eqn:ex2}
\end{align}
\eqref{lec12:eqn:ex1} follows from the fact that $\phirelu(t) + \phirelu(-t) = |t|$ for all $t$, while \eqref{lec12:eqn:ex2} follows from evaluating the 4 possible values of $(x_1, x_2)$. Thus, we can solve this problem exactly with a very sparse combination of features.

However, if we were to use the NTK approach (kernel method), the network's output will always involve $\sigma(w^\top x)$ where $w$ is random so it includes all components of $x$ (i.e. a dense combination of features), and cannot isolate just the relevant features $x_1$ and $x_2$. This is illustrated in the following informal theorem:
\begin{theorem}
The kernel method with NTK requires $n = \Omega(d^2)$ samples to learn Example \ref{lec12:ex:sparse123} well. In contrast, the neural network regularized by $\sum_{j = 1}^m | u_j| \| w_j\|_2$ only requires $n = O(d)$ samples.
\end{theorem}
\end{example}



%%	\input{collection/07-05-ntk-limitation.tex}
%	
%	\chapter{Implicit/Algorithmic Regularization Effect}
%	\input{collection/08-01-algorithmic.tex}
%	\input{collection/08-02-algorithmic.tex}
%	\metadata{16}{Leah Reeder and Trevor Maxfield}{Nov 10th, 2021}

\sec{From Small to Large Initialization: a Precise Characterization} \tnote{please double check consistency of capitalization in section headers}

We have previously discussed how certain initializations of gradient descent converge to minimum-norm solutions. In the sequel, we characterize the effect of initialization more precisely---we will show that in a variant of the settings in Section~\ref{?} \tnote{section 9.2}, we can precisely compute the corresponding regularizer induced by any initialization. We will see that when the initialization is small, we obtain the bias towards minimum norm solution (in the parameter space used in optimization), whereas when the initialization is large, we are in the NTK regime (Section~\ref{?} \tnote{ntk section})where the implicit bias is towards the min norm solution under the NTK kernel. 

\subsection{Preparation: gradient flow}
To simplify the analysis, we will consider the concept of gradient flow, i.e. gradient descent with an infinitesimal learning rate.  This allows us omit the second order effect from the learning rate and simplify the analysis. 

We begin by recalling the gradient descent update formula. In our previous description of gradient descent, we indexed the updated parameters by $t = 1,2,\dots$. Anticipating our generalization to infinitesimal steps, we will index the updated parameters using parentheses instead of subscripts. In particular, the standard gradient descent update given a loss function $L(w)$ is
\al{
w(t+1) = w(t) - \eta \nabla L(w(t)).
}
If we scale the time by $\eta$ so that each update by gradient descent corresponds to a time step of size $\eta$ (rather than size 1), the update becomes
\al{
w(t + \eta) = w(t) - \eta \nabla L(w(t)).
}
Taking $\eta \to 0$ yields a differential equation, which can be thought of as a continuous process rather than discrete updates:
\al{
w(t+dt) = w(t) - dt \cdot \nabla L(w(t)).
}
This can also be written as:
\al{
\dot{w}(t) = -\nabla L(w(t) \quad \text{ with } \quad \dot{w}(t) = \frac{\partial w(t)}{\partial t}
}
This allows us to ignore the $\eta^2$ term (alternatively the $(dt^2)$ term), which will simplify some of the technical details that follow.

\subsec{Characterizing the implicit bias of initialization}
The results in this section are slight simplification of the recent paper by~\citet{woodworth2020kernel}. The model is a variant of the one we introduced in \eqref{lec13:eqn:hadamard_model_1}. Recalling that $x^{\odot 2} = x \odot x$, let
\al{
f_w(x) = \left(w_+^{\odot 2} - w_-^{\odot 2}\right)^\top x.
}
where $w_+, w_- \in \R^d$. Let $w$ denote the concatenation of the two parameter vectors, i.e. $= (w_+, w_-)$.  In \eqref{lec13:eqn:hadamard_model_1}, we defined $f_\beta(x) = (\beta \odot \beta)^\top x$; this model can only represent positive linear combinations of $x$.  By contrast, $f_w(x)$ can represent any linear model. Moreover, if we choose our initialization for $w$ such that $w_+(0) = w_-(0)$, we obtain $f_{w(0)}(x) \equiv 0$ for all $x$. Similar to our analysis of the NTK, this initialization will simplify the subsequent derivations.

For the loss, we define\tnote{a bit more formal language}
\al{
\hatL(w) = \frac{1}{2} \sum_{i=1}^n \left( y\sp{i} - f_w(x\sp{i})\right)^2
}
and consider the initialization
\al{
w_+(0) = w_-(0) = \alpha \cdot \vec{\mathbf{1}}
}
where $\vec{\mathbf{1}}$ denotes the all-ones vector. The analysis technique still applies to any general initializations as long as all the dimension are initialized to be non-zero, but the the initialization scale is the most important factor, and therefore we chose this simplification for the ease of exposition. 

Note that every $w = (w_+, w_{-})$ corresponds to a de facto linear function of $x$. We denote the resulting linear model as $\theta_w$:
\al{
\theta_w = w_+^{\odot 2} - w_-^{\odot 2}.
}
Note that $\theta_w^\top x = f_w(x)$. 

Let $w(\infty)$ denote the limit of the gradient flow, i.e.
\al{
w(\infty) = \lim_{t \to \infty} w(t).
}
Then, the converged linear model in the $\theta$ space is defined by $\theta_\alpha(\infty) = \theta_{w(\infty)}$---we are interested in understanding its properties.  For simplicity, we will omit the $\infty$ index and refer to this quantity as $\theta_\alpha$. We assume throughout that the limit exists and all corresponding regularity conditions are met.

Let
\al{
X = \begin{bmatrix} x^{(1)^\top} \\ \vdots \\ x^{(n)^\top} \end{bmatrix} \in \R^{n \times d} \quad \text{ and } \quad \hat{y} = \begin{bmatrix} y^{(1)} \\ \vdots \\ y^{(n)} \end{bmatrix}.
}
\tnote{should be $\vec{y}$}
And with this setup we can give the following theorem\tnote{more formal language}:
 % 18:30
\begin{theorem}[Theorem 1 in \cite{woodworth2020kernel}]\tnote{use citet}
	 \label{lec16:thm:interpolatingAlpha}
For any $0 < \alpha < \infty$, assume that we \tnote{/GF with initilaization..} converge to a solution that fits the data exactly: $X \theta_{\alpha} = \vec{y}$.\footnote{This assumption can likely be proved to be true and thus not required. Here we still include the condition because the original paper~\citet{woodworth2020kernel} assumed it.}  Then, the solution satisfies the following notion of minimum complexity:
\al{ 
\theta_\alpha = \argmin_\theta Q_\alpha(\theta)\\
 \quad \textup{ s.t. } \quad X \theta = y \label{lec16:eqn:constrained_complexity}
}
\tnote{$y$ should be $\vec{y}$; check other occurrences?}
where
\al{
Q_\alpha(\theta) = \alpha^2 \cdot \sum_{i=1}^n q\left(\frac{\theta_i}{\alpha^2} \right)
}
and
\al{
q(z) = 2 - \sqrt{4 + z^2} + z \cdot \textup{arcsinh}\left(\frac{z}{2}\right)
}
\end{theorem}
In words, Theorem~\ref{lec16:thm:interpolatingAlpha} claims that $\theta_\alpha$ is the minimum complexity solution for the complexity measure $Q_\alpha$.

%23 minutes.
\begin{remark}
In particular, when $\alpha \to \infty$ we have that 
\begin{align}
    q(\theta_i /\alpha^2) \asymp \theta_i^2/\alpha^4
\end{align}
and so 
\begin{align}
    Q_\alpha(\theta) \asymp \frac{1}{\alpha^2} \Norm{\theta}_2^2.
\end{align}
This means that if $\alpha \to \infty$ than the complexity measure $Q_\alpha$ is the $\ell_2$-norm, $||\theta||_2$.  If $\alpha \to 0$, then the complexity measure becomes
\al{
q\left(\frac{\theta_i}{\alpha^2}\right) &\asymp \frac{\left|\theta_i\right|}{\alpha^2} \log\left(\frac{1}{\alpha^2}\right) \quad\text{(by Taylor expansion)}
}
and so,
\al{
Q_\alpha\left(\theta\right) &\asymp \frac{\Norm{\theta}_1}{\alpha^2} \log\left(\frac{1}{\alpha^2}\right)
}
To summarize, for $\alpha \to \infty$, the constrained minimization problem we solve in \eqref{lec16:eqn:constrained_complexity} yields the minimum $\ell_2$-norm solution of $\theta$ (i.e. the $\ell_4$-norm for $w$).  When $\alpha \to 0$, solving \eqref{lec16:eqn:constrained_complexity} yields the minimum $\ell_1$-norm $\theta$ (which is the $\ell_2$-norm for $w$).  For $0 < \alpha < \infty$, we obtain some interpolation of $\ell_1$ and $\ell_2$ regularization of the optimum.
\end{remark}

%27.30 minutes
\begin{remark}
Note that when $\alpha \to 0$, the intuition is similar to what we had observed in previous analyses; in particular, the solution is the global minimum closest to the initialization.  Note however, that when $\alpha \neq 0$, the solution discovered by gradient descent will not \textit{exactly} correspond to the solution closest to the initialization.
\end{remark}

\begin{remark}
When $\alpha \to \infty$, we claim that the model optimization is in the neural tangent kernel (NTK) regime.  Recall that we had two parameters, $(\sigma, \beta)$, that determined if we could treat the optimization problem as a kernel regression. Further recall that $\sigma$ denotes the minimum singular value of $\Phi$ and $\beta$ is the Lipschitzness of the gradient. Let us now compute $\sigma$ and $\beta$ for large $\alpha$ initializations of our model.

For $w_-(0) = w_+(0) = \alpha \vec{\mathbf{1}}$,
\al{
\nabla f_{w(0)}(x) = 2 \begin{bmatrix} W_{+}(0) \cdot x \\ -W_{-}(0) \odot x \end{bmatrix} = 2 \alpha \begin{bmatrix} x \\ -x \end{bmatrix}
}
\tnote{litter letter for W in equation above; other occurrences?}
by the chain rule.  It is clear then that both $\sigma$ and $\beta$ linearly depend on $\alpha$.  This implies that
\al{
\frac{\beta}{\sigma^2} \to 0 \quad \text{ as } \alpha \to \infty
}
since the denominator is $O(\alpha^2)$, while the numerator is $O(\alpha)$.  In particular, the features used in this kernel method are:
\al{
\phi(x) = \nabla f_{w(0)} (x) = 2 \alpha \begin{bmatrix} x \\ - x \end{bmatrix}
}
The neural tangent kernel perspective then gives an alternative proof of this complexity minimization result for $\alpha \to \infty$. In the NTK regime, the solution (to our convex problem) is always the minimum $\ell_2$-norm solution for the feature matrix, which in this case equals $\begin{bmatrix} x \\ - x \end{bmatrix}$. 

Note that practice tends not to follow the assumptions made here. Often, people either do not use large initializations or do not use infinitesimally small step sizes. But this is a good thing  because we do not want to be in the NTK regime; being in the NTK regime implies that we are doing no different or better than just using a kernel method.
\end{remark}

We can now prove Theorem~\ref{lec16:thm:interpolatingAlpha}, which is similar to the overparametrized linear regression proof of Theorem~\ref{lec13:thm:linear-main}.

This proof follows in two steps:
\begin{enumerate}
\item We find an invariance maintained by the optimizer. In the overparametrized linear regression proof of Theorem~\ref{lec13:thm:linear-main}, we required $\theta \in \text{span}\{x\sp{i}\}$.  For this proof, we will use a slightly more complicated invariance.
\item We characterize the solution using this invariance.  The invariance, which depends on $\alpha$, will tell us which zero error solution the optimization converges to.
\end{enumerate}
Note also that all of these conditions only depend upon the empirically observed samples. The invariance and minimum is not defined with respect to any population quantities.
\begin{proof}  
Let
\al{
\tilde{X} = \begin{bmatrix}x & -x\end{bmatrix} \in \R^{n \times 2d} \quad \text{ and } \quad w(t) = \begin{bmatrix} w_+(t) \\ w_-(t) \end{bmatrix} \in \mathbb{R}^{2d}.
}
\tnote{litter x to capital X in the above equations; could you help check if other occurrences}
Then, the model output on $n$ data points can be described in matrix notation as follows:
\al{
\tilde{X} w(t)^{\odot 2} = \begin{bmatrix}x & -x\end{bmatrix} \begin{bmatrix} w_+(t)^{\odot 2} \\ w_-(t)^{\odot 2} \end{bmatrix} = \begin{bmatrix} f_{w(t)} (x\sp{1}) \\ \vdots \\ f_{w(t)}(x\sp{n})\end{bmatrix} \in \R^n.
}
Given the loss function,
\al{
L(w(t)) = \frac{1}{2} \Norm{\tilde{X} w(t)^{\odot 2} - \vec{y}}_2^2,
}
the gradient of $w(t)$ can be computed as
\al{
\dot{w}(t) &= -\nabla L(w(t)) \\
&= - \nabla \left( \Norm{\tilde{X} w(t)^{\odot 2} - \vec{y}}_2^2 \right) \\
&= \left(\tilde{X}^\top r(t)\right) \odot w(t) \quad \quad \quad \text{(chain rule)}\label{lec16:eqn:Xtrtwt}
}
where $r(t) = \tilde{X} w(t)^{\odot 2} - \vec{y}$ denotes the residual vector.  We see that the $\tilde{X}^\top r(t)$ term in \eqref{lec16:eqn:Xtrtwt} is reminiscent of linear regression for which it would correspond to the gradient, although the $\odot w(t)$ reminds us that this problem is indeed quadratic.

We cannot directly solve this differential equation, but we claim that
\al{ \label{lec16:eqn:w_claim}
w(t) = w(0) \odot \text{exp}\left(-2\tilde{X}^\top \int_0^\top r(s) ds \right) \quad \text{(exp is applied entry-wise)}
}
which is not quite a closed form solution of equation \ref{lec16:eqn:Xtrtwt} since $r(s)$ is still a function of $w(t)$.  To understand how we obtained this ``solution,'' we consider a more abstract setting. Suppose that
\al{
\dot{u}(t) &= v(t) \dot u(t)
}
We can then ``solve'' this differential equation as follows. Rearranging, we observe that
\al{
\frac{\dot{u}(t)}{u(t)} &= v(t) \\
\frac{d \log u(t)}{dt} &= v(t) \quad \text{(chain rule)} \\
\log u(t) - \log u(0) &= \int_0^t v(s) ds \quad \text{(integration)} \\
\frac{u(t)}{u(0)} &= \text{exp} \left( \int_0^t v(s) ds\right)
}
In our problem, $u \leftrightarrow w_i$ and $v \leftrightarrow (\tilde{X}^\top r(t))_i$.

We have characterized $w$, but we want to transform this to a characterization that involves $\theta$.
Recall that \(w_+(0) = \alpha \vec{\mathbf{1}}\) and \(w_-(0) = \alpha \vec{\mathbf{1}}\) so that \(w(0) = \alpha \vec{\mathbf{1}} \in \R^{2d}\). Additionally, we have that \(\theta(t) = w_+(t)^{\odot 2} - w_-(t)^{\odot 2} \).
We can now apply \eqref{lec16:eqn:w_claim} to expand \(w(t)\) and simplify. 
\tnote{I think from here until 9.126, all the little $x$ should be capital $X$}Note that if we have \(\tilde{x}^\top = \begin{bmatrix} x^\top \\ -x^\top \end{bmatrix} \in \R^{2n\times d}\), then for some vector \(v\),
\al{
    \left(\exp(-2\tilde{x}^\top v) \right)^{\odot 2} &=
    \begin{bmatrix}
    \exp(-2x^\top v) \\
    \exp(2x^\top v)
    \end{bmatrix}^{\odot 2} \\
    &= \begin{bmatrix}
    \exp(-4x^\top v) \\
    \exp(4x^\top v)
    \end{bmatrix}.
}
Applying this result for $v = \int_0^T r(s) ds$, we obtain that:
\al{
    \theta(t) &= w_+(t)^{\odot 2} - w_-(t)^{\odot 2} \\
    &= \alpha^2 \left[ \exp \left( -4 x^\top \int_0^t r(s) ds \right) - \exp \left( 4 x^\top \int_0^t r(s) ds \right)\right] \\
    &= 2 \alpha^2 \sinh \left(-4 x^\top \int_0^t r(s) ds \right).
}
Letting $t \to \infty$, we have that
\al{\label{lec16:eqn:theta_infty}
    \theta_\alpha = 2 \alpha^2 \sinh \left(-4x^\top \int_0^\infty r(s) ds \right).
}
Lastly, we also know 
\al{
    X \theta_\alpha = y \label{lec16:eqn:theta_constraint}
 } 
 since this is the assumption by the theorem (which should can be proven because the optimization should converge to a zero-error solution). We next show that \eqref{lec16:eqn:theta_infty} and \eqref{lec16:eqn:theta_constraint} are also sufficient conditions for a solution to the constrained optimization problem given by \eqref{lec16:eqn:constrained_complexity}. In particular, \eqref{lec16:eqn:theta_infty} and \eqref{lec16:eqn:theta_constraint} correspond to the Karush-Kuhn-Tucker (or KKT) conditions of \eqref{lec16:eqn:constrained_complexity}.

A KKT condition is an optimality condition for constrained optimization problems. While these conditions can have a variety of formulations and typically one can invoke some off-the-shelf theorems to use them, we can motivate the conditions we encountered by considering the following general optimization program:
\al{
    \argmin \quad &Q(\theta) \\
    \text{s.t.} \quad &X\theta = y.
}
We say that \(\theta\) satisfies the (first order) KKT conditions if
\begin{align}
    \nabla Q(\theta) &= X^\top \nu \text{ for some } \nu \in \R^n \\
    X\theta &=y
\end{align}
More intuitively, we know that optimality implies that there are no first order local improvements that satisfy the constraint (up to first order). Then, consider a perturbation \(\Delta \theta\). In order to satisfy the constraint, we must enforce the following:
\begin{align}
\Delta \theta \perp \text{row-span}\{X\}  \quad \text{ so } \quad X \Delta \theta = 0
\end{align}
So, if we look at \(\theta + \Delta \theta \) satisfying the constraint, we can use a Taylor expansion to show that
\al{
Q(\theta + \Delta \theta) = Q(\theta) + \langle \Delta \theta, \nabla Q(\theta) \rangle \leq Q(\theta)
}
because if \( \langle \Delta \theta, \nabla Q(\theta) \rangle\) is positive it violates the optimality assumption.
In fact, it is very easy to make the sign flip for \( \langle \Delta \theta, \nabla Q(\theta) \rangle\) because you can flip \(\Delta \theta\) to be the opposite direction. This means that
\al{
    \forall \, \Delta \theta \perp \text{row-span}\{X\}, \quad \langle \Delta \theta, \nabla Q(\theta) \rangle = 0
}
because if it is negative, you can equivalently flip it to be positive which violates optimality.
This means that \(Q(\theta) \subseteq \text{row-span}\{X\}\), or \(Q(\theta) = X^\top \nu\) for some $\nu$.

Returning to our problem, the KKT condition gives
\al{
    \nabla Q(\theta) = X^\top \nu
}
\tnote{little x to big $X$ again}
and the invariance gives us
\al{
    \theta_\alpha &= 2 \alpha^2 \sinh\left(-4x^\top \int_0^\infty r(s) ds \right) \\
    &= 2\alpha^2 \sinh \left( -4x^\top v'\right)
}
where we let \(v' = \int_0^\infty r(s) ds\) for simplicity.
Taking the gradient of \(Q\) gives
\al{
    \nabla Q_\alpha (\theta) = \operatorname{arcsinh}\left(\frac{1}{2\alpha^2} \theta \right)
}
Plugging in \(\theta_\alpha\), we get
\al{
    \nabla Q(\theta_\alpha) = \operatorname{arcsinh}\left (\frac{1}{2\alpha^2} \theta_\alpha \right ) = -4 x^\top v'
}
Thus, \(\theta_\alpha\) satisfies both KKT conditions. Even further, since our optimization problem~\eqref{lec16:eqn:constrained_complexity} is convex (we do not formally argue this), we conclude that \(\theta_\alpha\) is a global minimum.
\end{proof}

\sec{Implicit Regularization Towards Max-margin Solutions In Classification Problems}
We now switch our focus to classification problems. We consider linear models (though these results also apply to nonlinear models with a weaker version of the conclusion). We assume that our data is separable and will prove that gradient descent converges to the max-margin solution. This result holds for any initialization and does not require any additional regularization; we only require the use of gradient descent and the standard logistic loss function.

\tnote{is this backslash parenthesis a latex standard way to replace dollar sign? If not, could you replace them back to dollar sign? It will create overheads for future editing so let's avoid using it now.}
Assume we have data \(\{(x\sp{i}, y\sp{i}) \}_{i=1}^n \), where \(x\sp{i} \in \R^d\) and \(y\sp{i} \in \{\pm 1 \}\). We consider the linear model \( h_w(x) = w^\top x\) and the cross entropy loss function \(\hatL (w) = \sum_{i=1}^n \ell\left(y\sp{i}, h_w\l (x\sp{i} \r )\right)\), where \( \ell(t) = \log(1 + \exp(-t))\) is the logistic loss.

As we have separable data, there can be multiple global minima, as you can trivially take an infinite number of separators. More formally, there are an infinite number of unit vectors \(\bar{w}\) such that $\bar{w}^\top x\sp{i} y\sp{i} > 0$ for all $i$ as one can perturb any strict separator  while still maintaining a separation of classes. Then, we can scale the separator to make the loss arbitrarily small---we have that \( \hatL(\alpha \bar{w}) \to 0\) as \( \alpha \to \infty\). Thus, informally, for any unit vector $\bar{w}$ that separate the data, $\infty \cdot \bar{w}$ is a global minimum.\tnote{this paragraph didn't make much sense before..} %Thus, even if we arbitrarily scale the unit vector, you still have that the loss goes to zero as \(\ell(t)\) approaches zero as \(t\) gets large. Thus, all choices of $w$ correspond to global minima, as the loss function goes to zero for infinite scalings.

We would like to understand which global minimum gradient descent converges to. We will now show that it finds the max-margin solution. Before we can do so, we recall/introduce the following definitions.

\begin{definition}[Margin]
Let \(\{(x\sp{i}, y\sp{i}) \}_{i=1}^n \) be given data. Assuming \(w\) is linearly separable, a \textit{margin} is defined as
\al{
    \min_{i \in [1,...,n]} y\sp{i} w^\top x\sp{i}
}
\end{definition}
\tnote{change $[1,...,n]$ to $[n]$ (it also applies to other occurrences)}

\begin{definition}[Normalized Margin]\label{lec16:def:norm_margin}
Let \(\{(x\sp{i}, y\sp{i}) \}_{i=1}^n \) be given data. Assuming \( w\) is linearly separable\tnote{a classifier cannot be linearly separable}, a \textit{normalized margin} is defined as
\al{
    \gamma(w) = \frac{\min_{i \in [1,...,n]} y\sp{i} w^\top x\sp{i}}{\norm{w}_{2}}
}
\end{definition}

\begin{definition}[Max-Margin Solution]
Using the normalized margin \(\gamma\) defined in Definition~\ref{lec16:def:norm_margin}, we define a \textit{max-margin solution} as
\al{
    \bar{\gamma} = \max_{w} \gamma(w)
}
and let \(w^*\) be the unit-norm maximizer. \tnote{add a footnote on the scale invariance of this notion}
\end{definition}

Using these definitions, we claim the following result.
\begin{theorem} \label{lec16:thm:maxmargin_gd}
Gradient flow converges to the direction of max-margin solution in the sense that
\al{
    \gamma(w(t)) \to \bar{\gamma} \text{  as  } t \to \infty
}
where \(w(t)\) is the iterate at time \(t\).
\end{theorem}

The following observations provide some intuition for Theorem~\ref{lec16:thm:maxmargin_gd}.
\begin{enumerate}
    \item \(\hatL(w(t)) \to 0\) by a standard optimization argument. This is because if our optimization iteration is working, \(w(t)\) at large \(t\) will cause the loss function to go to zero.\tnote{to reword a bit}
    \item Using a Taylor expansion, we can show that \( \ell(z) = \log(1 + \exp(-z)) \approx \exp(-z)\) for large \(z\). Thus, logistic loss is close to exponential loss when \(z\) is very large.
    \item Using observation 1, we see that \(\norm{w(t)}_{2} \to \infty\) because if \(\norm{w(t)}_{2}\) were instead bounded, then the loss \(\hatL (w(t))\) will be bounded below by a constant that is strictly greater than zero, contradicting observation 1. Formally, if
    \(\norm{w(t)}_{2} \leq B,\)
    then
    \al{
        |y\sp{i} w^t x\sp{i}| \leq B \norm{x\sp{i}},
    }
    and therefore we get
    \al{
        \hatL(w(t)) \geq \sum_{i=1}^n \exp\left(-B\norm{x\sp{i}}_{2} \right)> 0.
    }
    \item Suppose we have \(w\) such that \(\norm{w}_{2} = q \) is very big. Then, using observation 2, we see that
    \al{
        \hatL(w) &= \sum_{i=1}^n \ell(y\sp{i} w^\top x\sp{i}) \\
        &\approx \sum_{i=1}^n \exp\left(-y\sp{i} w^\top x\sp{i} \right) \\
        \log \hatL(w) &\approx \log \sum_{i=1}^n \exp\left(-y\sp{i} w^\top x\sp{i} \right) \\
        &= \log \sum_{i=1}^n \exp \left(-q y\sp{i} \bar{w}^\top x\sp{i} \right) \\
        &\approx \max_{i \in [1,2,...,n]} -q y\sp{i} \bar{w}^\top x\sp{i}
    }
    where \( \bar{w} = \frac{w}{\norm{w}_{2}}\) and the last step holds because the log of a sum of exponentials (\textit{log-sum-exp}) is a smooth approximation to the maximum function. To motivate this claim, observe that:  
    \al{
         \log \sum_{i=1}^n \exp(a u_i) &\geq q \max_i u_i  \\
        \log \sum_{i=1}^n \exp(a u_i) &\leq \log \left(n \exp(q \max_i u_i)\right) \\
        &= \log n + q \max_i u_i \\
        &\approx q \max_{i \in [1,2,...,n]} u_i + o(q) \text{ as } q \to \infty
    }
    Thus, minimizing the loss is the same as
    \al{
    \min_w \max_{i \in [1,2,...,n]} -qy\sp{i} \bar{w}^\top x\sp{i}
    }
    which can be reformulated as
    \al{
    \max_w \min_{i \in [1,2,...,n]} qy\sp{i} \bar{w}^\top x\sp{i}
    }

\end{enumerate}

The above observations demonstrate that minimizing the logistic loss with gradient descent is equivalent (in the limit) to maximizing the margins. This constitutes an intuitive proof of how gradient flow converges to the direction of the max-margin solution.

\sec{Implicit Regularization Effect of Noise in SGD}

In the previous section, we discussed implicit regularization via initialization and the implicit regularization of gradient descent for logistic loss-minimizing classifiers. 
%These methods were based on a specific model setup and limited to gradient flow. 
In the sequel, we will move forward to the implicit regularization effect of SGD noise. Starting from the quadratic case, we analyze how the SGD noise will affect the optimization solution, and present (heuristically) a result for non-quadratic loss functions. In particular, the main (heuristic) results are:
\begin{enumerate}
\item On the one-dimensional quadratic function, the iterate can be disentangled into a contraction part and a stochastic part, the latter of which is characterized by the Ornstein–Uhlenbeck (OU) process. The noise makes the iterate bounce around the global minimum.
\item On the multi-dimensional quadratic function, the iterate can be disentangled into multiple separate 1-D OU processes. The noise makes the iterate bounce around the global minimum, while the fluctuation is closely related to the shape of the noise.
\item On non-quadratic functions, SGD with \textit{label noise} on empirical loss $\hat{L}(\theta)$ converges to a stationary point of the regularized loss $\hat{L}(\theta) + \lambda \textup{tr}(\nabla^2\hat{L}(\theta))$, which is mainly due to the accumulation of a third order effect.
\end{enumerate}
 

For the remainder of this section, let $g(x)$ denote the general loss function. Then, the formulation of SGD is: for $t$ in $[0,T]$,
\begin{align}
\theta_{t+1} = x_{t} - \eta(\nabla g(x_{t}) + \xi_t),
\end{align} 
where $\eta > 0$ is the learning rate, $\xi_t$ denotes the SGD noise, and $\Exp[\xi_t] = 0$. Note that in the most general case, $\xi_t$ can depend on $x_t$.
	
\subsec{Warmup: SGD on the one-dimensional quadratic function}
In this section, we consider the one-dimensional function $g(x) = \frac{1}{2} x^2$. Suppose the noise $\xi_t$ are independent Gaussians, i.e. $\xi_t \sim \mathcal{N}(0,1)$,
\begin{align}
x_{t+1} &= x_t - \eta(\nabla g(x_{t}) + \sigma\xi_t)\\
&= x_t - \eta(x_{t} + \sigma\xi_t)\\
&= \underbrace{(1 - \eta)x_t}_{\text{contraction}} - \underbrace{\eta\sigma\xi_t}_{\text{stochastic}}\label{lec17:eqn:ou}.
\end{align}
$(1 - \eta)x_t$ is called the contraction because $\eta > 0$, which means that this term will shrink after each iteration. The random noise term $\eta\sigma\xi_t$ will accumulate over time, and the scale of $\eta\sigma\xi_t$ remains unchanged. When $x_t$ is large, the contraction term will dominate. When $x_t$ is small, the noise term will dominate. Without the noise term, as $x_t$ continues its contraction, we approach the global minimum $x = 0$. However, with the presence of the noise $\sigma\xi_t$, $x_t$ will not stay at $0$, but instead bounce around it. 

To characterize this intuition more precisely, we have 
\begin{align}
x_{t+1} &= (1 - \eta)x_t - \eta\sigma\xi_t\\
&= (1 - \eta) ((1 - \eta) x_{t - 1}  - \eta \sigma \xi_{t - 1}) - \eta \sigma \xi_t \\
&= (1 - \eta)^2 x_{t - 1} - (1 - \eta) \eta \sigma \xi_{t - 1} - \eta \sigma \xi_{t} \\
&= (1 - \eta)^3 x_{t - 2} - (1 - \eta)^2 \eta \sigma \xi_{t - 2} - (1 - \eta) \eta \sigma \xi_{t - 1} - \eta \sigma \xi_t \\
&\quad \vdots \\
&= (1 - \eta)^{t+1} x_0 - \eta\sigma\sum_{k=0}^{t} \xi_{t-k} (1 - \eta)^{k}. \label{lec17:eqn:warmup_expansion}
\end{align}
The first term in \eqref{lec17:eqn:warmup_expansion} becomes negligible when $\eta t \gg 1$ (since $(1 - \eta)^{t} \approx e^{-\eta t}$). The second term in \eqref{lec17:eqn:warmup_expansion} is the accumulation of noise, which is the sum of Gaussians. Leveraging the properties of Gaussian distributions, we know that its variance equals $\eta^2\sigma^2\sum_{k=0}^{t} (1 - \eta)^{2k}$.

From the analysis above, we know that as $t \rightarrow \infty$, $\Var(x_t) \approx \eta^2\sigma^2\sum_{k=0}^{\infty} (1 - \eta)^{2k} = \frac{\eta^2\sigma^2}{2\eta - \eta^2} = {\Theta}(\eta\sigma^2)$. Therefore, as $t \rightarrow \infty$, $x_t \sim \mathcal{N}(0, {\Theta}(\eta\sigma^2))$.

\paragraph{Interpretation.} In the one-dimensional case, the noise only makes it harder to converge to the global minimum. Classical convex optimization tells us: (1) noisy GD leads to a less accurate solution and (2) noisy GD is faster than GD. What we do in practice is achieve a balance between (1) and (2). This does \textit{not} lead to implicit regularization since $\Exp[x_t] \rightarrow 0$ as $t \rightarrow \infty$. However, this case is important for further analysis because \eqref{lec17:eqn:ou} corresponds to the Ornstein–Uhlenbeck (OU) process which we use more extensively in the multi-dimensional cases.

\subsec{SGD on multi-dimensional quadratic functions}
Consider a PSD matrix $A \in \R^{d\times d}$. In this section, $g(x) = \frac{1}{2}x^\top A x$. Suppose $\xi_t \sim \mathcal{N}(0, \Sigma)$. For ease of presentation, assume that $A$ and $\Sigma$ are simultaneously diagonizable (they have the same set of eigenvectors). We use $K$ to denote the span of the eigenvectors of $A$/$\Sigma$. Then, consider the following SGD iterate:
\begin{align}
x_{t+1} &= x_t - \eta(\nabla g(x_{t}) + \xi_t)\\
&= x_t - \eta(Ax_t + \xi_t)\\
&= (I- \eta A)x_t - \eta\xi_t\\
&= \underbrace{(I- \eta A)^{t+1} x_0}_{\text{contraction}} - \underbrace{\eta\sum_{k=0}^{t} (I- \eta A)^{k}\xi_{t-k}}_{\text{noise accumulation}}.
\end{align}
Similar to the analysis in the 1-D case above, we have $x_t \sim \mathcal{N}(0, \eta^2\sum_{k=0}^{\infty} (I- \eta A)^{k}\Sigma (I- \eta A)^{k})$ as $t \rightarrow \infty$. \footnote{For random variable $\xi\in \R^d$, $\Exp[(W\xi)(W\xi)^\top] = W\Exp[\xi\xi^\top]W^\top$}

Since $A$ and $\Sigma$ are simultaneously diagonizable, we can easily disentangle the iterates into d separate OU process in the eigencoordinate system. Concretely, by eigendecomposition, suppose that $A = U^\top \text{diag}(d_i) U$ and $\Sigma = U^\top \text{diag}(\sigma_i^2) U$, where $U$ is the orthogonal matrix consisting of the eigenvectors of $A$ and $\Sigma$. We can express the covariance of the stationary distribution as
\begin{align}
\eta^2\sum_{k=0}^{\infty} (I- \eta A)^{k}\Sigma (I- \eta A)^{k} &= \eta^2 U\text{diag}\left(\sum_{k=0}^{\infty}\sigma_i^2(1-\eta d_i)^{2k}\right)U^\top\\
&= \eta U\text{diag}\left(\frac{\sigma_i^2}{d_i}\right)U^\top.
\end{align}
\paragraph{Interpretation.} Intuitively, $\frac{\sigma_i^2}{d_i}$ here is the iterate fluctuation in the direction of the $i$-th eigenvector. This results tell us that the fluctuation of the iterates depends on the shape of $\Sigma$ and $A$. If $\Sigma$ is not full rank, the fluctuations will be limited to the subspace $K$. Also note that $\Exp[\|x_t\|_2] = \Theta(\sqrt{\eta})$. This reflects the noise accumulation since the scale of noise in each step is $\Theta({\eta})$. However, we still do not have any implicit regularization effect. 

\begin{figure}[ht]
\centering
 \begin{subfigure}[t]{0.45\textwidth}
        \includegraphics[width=\textwidth]{figures/quadratic.pdf}
        \caption{Quadratic functions.}
        \label{lec17:fig:quadratic}
    \end{subfigure}
    \hfill
    \begin{subfigure}[t]{0.45\textwidth}
        \includegraphics[width=\textwidth]{figures/non-quadratic.pdf}
        \caption{Non-quadratic functions.}
        \label{lec17:fig:non-quadratic}
    \end{subfigure}
   
\caption{Comparison of SGD on quadratic functions (a) and non-quadratic functions (b). $K$ is the span of the noise covariance $\Sigma$. In the quadratic case, the iterates will fluctuate in $K$, but remains unchanged in $K^\perp$. When the function is non-quadratic, the third order effect slowly accumulates in $K^\perp$, resulting in implicit regularization. } 
\label{lec17:fig:OU}
\end{figure}

In the sequel, we separately analyze the second order and third order effects of SGD on a general non-quadratic function. The second order effect exactly corresponds to this section's analysis when $A$ equals the Hessian of the general non-quadratic function.

\subsec{SGD on non-quadratic functions}
In this section, we analyze SGD on non-quadratic functions based on \cite{damian2021label}. Due to the complexity of the analysis, we provide heuristic derivations to convey the main insights. 

Without loss of generality, suppose a global minimum of $g(x)$ is $x=0$. Therefore, $\nabla_x g(0) = 0$ and $\nabla_x^2 g(0)$ is PSD. We also  assume the iterates $x_t$ are close to $0$, so we can Taylor expand around $0$.
\begin{align}
x_{t+1} &= x_t - \eta(\nabla g(x_t) + \xi_t)\\
&= x_t - \eta(\nabla g(0) + \nabla^2g(0)(x_t - 0) + \nabla^3g(0)[x_t,x_t] + \text{higher order terms} + \xi_t). \label{lec17:eqn:full_gradient_update}
\end{align}

Let $H = \nabla^2_x g(0)$ and $T = \nabla^3_x g(0)$. Since $T$ is a tensor, we first clarify our notation. First, for $T \in \R^{d\times d\times d}$, $x,y \in \R^{d}$, $T[x,y]\in \R^d$, and 
\begin{align}
    T[x,y]_i \defeq \sum_{j,k\in[d]}T_{ijk}x_jy_k.
\end{align} 
For $S\in \R^{d\times d}$, $T(S)\in \R^d$, and 
\begin{align} 
    T(S)_i \defeq \sum_{j,k\in[d]}T_{ijk}S_{jk}
\end{align} 

Now returning to \eqref{lec17:eqn:full_gradient_update}, after dropping the higher order terms, we obtain the following third-order Taylor expansion:
\begin{align}
x_{t+1} &\approx x_t - \eta Hx_t - \eta\xi_t - \eta T[x_t,x_t]\\
&= (I-\eta H)x_t - \eta \xi_t - \eta T [x_t,x_t].\label{lec17:eqn:iterate}
\end{align}

If we don't consider the third order term $\eta T [x_t,x_t]$, the update reduces to the one we studied in the previous subsection. Next, recall that $\|x_t\|_2 \approx \sqrt{\eta}$. Therefore, $\eta T[x_t,x_t] \approx \eta^2$. This quantity is dominated by both $\eta \xi_t$ and $\eta Hx_t \approx {\eta}^{1.5}$. 

So, when $H$ is positive definite, the third order term can be negligible. However, in overparametrized models, $H$ is typically low-dimensional. For instance, if the NTK matrix is full rank, then the manifold of interpolators has dimension $d-n$. Then, in the direction orthogonal to the span of $H$, the contraction term disappears. Letting $\Pi_{A}$ denote projections onto the subspace $A$, we see that $\eta H \Pi_{K^\perp}(x_t) = 0$ and $T[x_t,x_t] \approx \eta^2$ will dominate the update in that direction.

Consider the case in which both $H$ and $\Sigma$ are not full rank. When the loss is quadratic as in the previous section, we know that the iterate $x_t$ bounces in the subspace $K$ and remains stable in the subspace $K^\perp$. What happens when the loss is not quadratic, i.e. $T[x_t,x_t]$ affects the gradient update? 

To answer this question, we decompose the effect of the update in \eqref{lec17:eqn:iterate} between the two subspaces of interest, $K$ and $K^\perp$. First, observe that $(I-\eta H)x_t - \eta \xi_t$ is working in $K$, and $- \eta T [x_t,x_t]$ is only working in $K^\perp$ because in $K$ the effect of $\eta T [x_t,x_t]$ is dominated by $(I-\eta H)x_t - \eta \xi_t$. In previous section, we already well-characterized the effect of optimization without a third order effect. To refine our analysis of the gradient update, we define an iterate $u_{t+1} = (I - \eta H)y_t - \eta \xi_t$ in which we do not have the third order effect.\footnote{Note that $\xi_t$ is the same for each $u_t$ and $x_t$.} Then, to analyze what the implicit regularization effect is, we study $r_t = x_t - u_t$.
\begin{align*}
r_{t + 1} &= x_{t+1} - u_{t+1}\\
&= (I-\eta H)(x_t - u_t) - \eta T[x_t,x_t]\\
&= (I-\eta H)r_t - \eta T[x_t,x_t]\\
&\approx (I-\eta H)r_t - \eta T[u_t,u_t].
\end{align*}
Note that we only have the contraction and the bias terms for the $r_t$ iterate. The stochasticity term $\eta \xi_t$ is canceled out. 

In the subspace $K = \text{span}(H)$, the effect of $\eta T [x_t,x_t]$ is again dominated by $(I-\eta H)x_t - \eta \xi_t$, so no meaningful regularization occurs. But letting $\Pi_{A}$ denote the projection onto the subspace $A$, we have that in $K^\perp$,
\begin{align}
\Pi_{K^\perp}r_{t+1} &= \Pi_{K^\perp}r_t - \eta \Pi_{K^\perp} T[u_t,u_t]\\
&=\Pi_{K^\perp}r_0 - \eta \sum_{k=0}^{t}\Pi_{K^\perp}T[u_k,u_k].
\end{align}
Namely, the effect of $T[u_k,u_k]$ is slowly accumulating in ${K^\perp}$.

Note that the OU process is a Markov chain and a Gaussian process. Here we assume that $H$ is constructed such that $u_t$ converges to its stationary distribution. Suppose the Markov chain $u_t$ mixes as $t\rightarrow \infty$. Then, $\sum_{k=0}^{t}\Pi_{K^\perp}T[u_k,u_k] \approx tT(S)$, where $S:=\Exp[u_{\infty}u_{\infty}^\top]$ is the covariance of the stationary distribution.

\paragraph{Interpretation.} Intuitively, the direction of the implicit regularization is $T(S) = \nabla_x \left(\langle\nabla_x^2g(0), S\rangle\right)$. In other words, the implicit bias $-T(S)$ is trying to make $\langle\nabla^2g(0), S\rangle$ small. \cite{damian2021label} further prove that $\Sigma = c\nabla^2\hat{L}(\theta)$ for square loss and logistic loss with label noise. In this case, $\langle\nabla^2g(0), S\rangle = \textup{tr}\left({\langle\nabla^2g(0)^\top S}\right) =  \textup{tr}\left(\text{diag}\left(\frac{\sigma_i^2}{d_i}d_i\right)\right) =  \textup{tr}\left(\Sigma\right) = c \textup{tr}\left(\nabla^2\hat{L}(\theta)\right)$, i.e. SGD with label noise on loss $\hat{L}(\theta)$ converges to a stationary point of the regularized loss $\hat{L}(\theta) + \lambda \textup{tr}(\nabla^2\hat{L}(\theta))$.

\paragraph{Relationship to generalization.} Why is $\textup{tr}(\nabla^2\hat{L}(\theta))$ a good regularizer? \cite{wei2019improved} show that the complexity of neural networks can be controlled by its Lipschitzness. $\textup{tr}(\nabla^2\hat{L}(\theta))$ is intimately related to the Lipschitzness of the networks. \cite{foret2020sharpness} also discover empirically that regularizing the sharpness of the local curvature leads to better generalization performance on a wide range of tasks.
%	
%	\part{Unsupervised Learning and Self-supervised Learning}
%	\chapter{Moment Methods and Spectral Clustering}
%	\newcommand{\jnote}[1]{{\color{red}\authnoteimp{JH}{#1}}}

%\metadata{18}{Haoran Xu and Lewis Liu}{Nov 17th, 2021}

We venture into unsupervised learning by first studying classical (and analytically tractable) approaches to unsupervised learning. Classical unsupervised learning usually consists of specifying a latent variable model and fitting using the expectation-maximization (EM) algorithm. However, so far we do not have a comprehensive theoretical analysis for the convergence of EM algorithms because fundamentally analyzing EM algorithms involves understanding non-convex optimization. Most analysis of EM only applies to special cases (e.g., see ~\citet{xu2016global,daskalakis2016ten}) and it is not clear whether any of the results can be extended to more realistic, complex setups, without a fundamentally new technique for understanding nonconvex optimization. 
Instead, we will analyze a family of algorithms which are broadly referred to as spectral methods or tensor methods, which are a particular application of the method of moments~\citep{pearson1894} with the algorithmic technique of tensor decomposition~\citep{anandkumar2015learning}. While the spectral method appears to be not as empirically sample-efficient as EM, it has provable guarantees and arguably is more reliable than EM given the provable guarantees.

%\tnote{this paragraph require updating}
After discussing the basics of classical unsupervised learning, we will move on to modern applications of deep learning. In particular, we'll focus on theoretical interpretations of contrastive learning, which is a class of successful self-supervised learning algorithms in computer vision. 

\sec{Method of Moments for mixture models} \label{sec:method-of-moments}

We begin by formally describing the unsupervised learning problem. First, assume that we are studying a family of distributions $P_{\theta}$ parameterized by $\theta \in \Theta$, where $P_{\theta}$ can be described by a latent variable model. Then, given data $x^{(i)},...,x^{(n)}$ that is sampled i.i.d. from some distribution in $\{P_\theta\}_{\theta \in \Theta}$, our goal is to recover the true $\theta$. 

Perhaps the most well-studied latent variable model in machine learning is the mixture of Gaussians. We consider the following model for the mixture of $k$ $d$-dimensional Gaussians. Let 
\begin{align}
\theta = \l ( (\mu_1, \cdots, \mu_k), (p_1, \cdots, p_k)\r ),
\end{align}
where $\mu_i\in \R^d$ is the mean of the $i$-th component and $p$ is a vector of probabilities belonging to the $k$-simplex, which represents the mixture coefficient for clusters. Formally, for $\Delta(k) \defeq \{ p: \|p\|_1 = 1, p\geq 0, p\in\R^k\}$, 
\begin{align}
    p = (p_1, \cdots, p_k) \in \Delta(k).
\end{align}
We then sample $x \sim P_\theta$ in a two-step approach: 
\begin{align}
    i &\sim \text{categorical}(p), \notag\\
    x &\sim \cN(\mu_i, I).
\end{align}
Here $i$ is called the latent variable since we only observe $x$. Here we assume the covariances of the Gaussians to be identity, but they can also be parameters that are to be learned.

There are many other latent variables that could be defined via a similar generative process, such as Hidden Markov Models, Independent Component Analysis, which we will discuss later. %, and Expectation-Maximization, but here we focus on the so-called Moment Method.

\subsec{Warm-up: mixture of two Gaussians}
We first study a simple case: the mixture of two Gaussians.
In this case, $k=2$, and we assume $p_1=p_2=\frac{1}{2}$. For simplicity, we also assume $\mu_1=-\mu_2$, that is, the means of the two Gaussians are symmetric around the origin. To simplify our notation, let $\mu_1=\mu$ and $\mu_2=-\mu$. These assumptions yield the following model for $x$:
\begin{equation}
    x \sim \frac{1}{2}\mathcal{N}(\mu,I) + \frac{1}{2}\mathcal{N}(-\mu,I).
\end{equation}
To implement the moment method, we need to complete the following two tasks:
\begin{enumerate}
    \item Estimate the moment(s) of $x$ using empirical samples.
    \item Recover parameters from the moment(s) of $x$.
\end{enumerate}

The first moment of $x$ is
\begin{align}
    M_1 &\defeq \Exp [x] \\
    &= \frac{1}{2}\Exp [x|i=1]+\frac{1}{2}\Exp[x|i=2] \\
    &= \frac{1}{2}\mu + \frac{1}{2}(-\mu) \\
    &= 0.
\end{align}
Therefore, the first moment provides no information about $\mu$. We compute the second moment as
\begin{align}
    M_2 &\defeq \Exp[xx^\top] \\
    &= \frac{1}{2}\Exp[xx^\top|i=1]+\frac{1}{2}\Exp[xx^\top|i=2]
\end{align}
To compute these expectations, consider an arbitrary $Z \sim \cN(\mu, I)$. Then,
\begin{align}
    \Exp [ZZ^\top] &= \Exp[Z] \Exp[Z]^\top + \Cov(Z) \\
    &= \mu \mu^\top + I
\end{align}
Recognizing that this second moment calculation is the same for both Gaussians in our mixture, we obtain:
\begin{align}
    M_2 &= \frac{1}{2}(\mu\mu^\top+I)+\frac{1}{2}(\mu\mu^\top+I) \\
    &=\mu\mu^\top+I
\end{align}
Since the second moment provides information about $\mu$, we can complete the two tasks required for the moment method using the second moment.

If we had access to infinite data, then we can compute the exact second moment $M_2=\mu\mu^\top+I$. Then, we can recover $\mu$ by evaluating the top eigenvector and eigenvalue of $M_2$.\footnote{This approach is known as the spectral method.} The top eigenvector and eigenvalue of $M_2$ is $\bar{\mu} \defeq \frac{\mu}{\norm{\mu}_2}$ and $\norm{\mu}_2^2+1$, respectively. 

In practice, however, we do not have infinite data. In that case, we need to estimate the second moment by an empirical average.
\begin{align}
    \widehat{M}_2=\frac{1}{n}\sum_{i=1}^nx\sp{i} {x\sp{i}}^\top
\end{align}
We can then recover $\mu$ by evaluating the top egivenvector and eigenvalue of $\widehat{M}_2$. However, we need this algorithm to be robust to errors, i.e., similar estimates, $\widehat{M}_2$, of the second moment should yield similar estimates of $\mu$. Fortunately, most algorithms we might use for obtaining the top eigenvector and eigenvalue are robust, so we can limit our attention to the infinite data case. Having outlined the moment method approach to the mixture of two Gaussians problem, we study a generalization of this problem in the sequel.

\subsec{Mixture of Gaussians with more components via tensor decomposition}

The general moment method for solving latent variable models is summarized by the following steps.
\begin{enumerate}
    \item Compute $M_1=\Exp[x]$, $M_2=\Exp[xx^\top]$, $M_3=\Exp[x\otimes x\otimes x],$ $M_4 = \cdots$. Note that $x\otimes x\otimes x$ is in $\mathbb{R}^{d\times d\times d}$ and $(x\otimes x\otimes x)_{ijk}=x_i\cdot x_j\cdot x_k$. For example, $M_{3,ijk}=\Exp[x_ix_jx_k]$.
    \item Design as algorithm $A(M_1, M_2, M_3,\dots)$ that outputs $\theta$.
    \item Show that $A$ is robust to errors in our moment estimates, i.e., we apply $A$ to $(\widehat{M}_1,\widehat{M}_2,\widehat{M}_3,...)$ in reality.
\end{enumerate}
In the sequel, we instantiate this paradigm for mixtures of $k$ Gaussians ($k\geq 3$). 

For the simplicity of demonstrating the idea, we assume $p_1 = \cdots = p_k =\frac{1}{k}$, i.e. $i \stackrel{\text{unif}} \sim[k]$, and $x\sim\mathcal{N}(\mu_i,I)$. Equivalently, 
\begin{equation}
    x\sim\frac{1}{k}\sum_{i=1}^k\mathcal{N}(\mu_i,I).
\end{equation}
In this example, we only describe steps (1) and (2) in the general algorithm described above.  

We first evaluate the first and second moments. The first moment follows from
\begin{align}
    M_1 &=\Exp[x] \\
    &=\sum_{i=1}^k\frac{1}{k}\Exp[x|i] \\
    &=\frac{1}{k}\sum_{i=1}^k\mu_i,
\end{align}
and the second moment is computed as
\begin{align}
    M_2 &= \Exp[xx^\top] \\
    &=\sum_{i=1}^k\frac{1}{k}\Exp[xx^\top|i] \\
    &=\sum_{i=1}^k\frac{1}{k}(\mu_i\mu_i^\top+I) \\
    &=\frac{1}{k}\sum_{i=1}^k\mu_i\mu_i^\top + I.
\end{align}

\subsubsection{Second moments do not suffice}
Can we recover $\mu=(\mu_1,...,\mu_k)$ from $\frac{1}{k}\sum_{i=1}^k\mu_i$ and $\frac{1}{k}\sum_{i=1}^k\mu_i\mu_i^\top$? Unfortunately, in most of the cases when $k\geq 3$, the first and second moments are not sufficent to recover $\mu$. 

One reason is the so-called ``missing rotation information'' problem. Let 
\begin{equation}
    U =\begin{bmatrix} \vrule & & \vrule \\ \mu_1 & \cdots & \mu_k \\ \vrule & & \vrule \end{bmatrix} \in\mathbb{R}^{d\times k}
\end{equation}
denote the matrix we aim to recover. Then, consider some rotation matrix $R\in\mathbb{R}^{k\times k}$. We consider $U$ versus $U R$:
\begin{align}
    \frac{1}{k}\sum_{i=1}^k\mu_i\mu_i^\top &= \frac{1}{k}U U^\top \\
    &=\frac{1}{k}(U R)(U R)^\top &\text{($RR^\top=I$)}
\end{align}
This result proves that the second moment is invariant to rotations. To prove a similar claim for the first moment, we also constrain our choice of $R$ such that
\begin{align}
    R\cdot\Vec{1}=\Vec{1}
\end{align}
Then,
\begin{align}
    \frac{1}{k}\sum_{i=1}^k\mu_i&=\frac{1}{k} U \cdot\Vec{1} \\
    &=\frac{1}{k} U R\cdot\Vec{1}
\end{align}
Therefore, the first and second moments of $U$ and $U R$ are indistinguishable, and we must consider the third moment in order to identify $U$.

\subsubsection{Computing the third moment}

The third moment is the tensor $\Exp[x \otimes x \otimes x] \in \mathbb{R}^{d \times d \times d}$. To express this expectation in terms of tractable quantities, we can condition on the Gaussian observed and average:
\begin{align}
	\Exp[x \otimes x \otimes x] = \frac{1}{k} \sum_{i=1}^k \Exp[x \otimes x \otimes x \mid i]
\end{align}

Each term in the sum now corresponds to the third moment for some multivariate Gaussian. Fortunately, Lemma~\ref{lec19:lma:gaussian_third_moment} suggests a formula for estimating its value.
\begin{lemma} \label{lec19:lma:gaussian_third_moment}
Suppose $z \in \cN(v, I)$. Then, 
\begin{align}
	\Exp[z \otimes z \otimes z] = v \otimes v \otimes v +  \sum_{l=1}^d \Exp[z] \otimes e_l \otimes e_l + \sum_{l=1}^d  e_l \otimes \Exp[z] \otimes e_l + \sum_{l=1}^d  e_l \otimes e_l \otimes \Exp[z] 
\end{align}
where $e_1,\dots,e_d$ denote the canonical basis vectors.
\end{lemma}
This lemma suggests that we can compute $v \otimes v \otimes v$ from a linear combination of $\Exp[z \otimes z \otimes z]$  and $\Exp[z]$. Also note that $\Exp[z] = v$. . 

\begin{proof}
We compute the third moment element-wise. That is,
\begin{align}
	(\Exp[z \otimes z \otimes z])_{ijk} &= \Exp[z_i  z_j z_k] \\
	&= \Exp[(v_i + \xi_i)\cdot (v_j + \xi_j) \cdot (v_k + \xi_k)]  &\text{$(z = v + \xi, \xi \sim \cN(0, I))$}\\
	&= v_i v_j v_k + \underbrace{\Exp [v_i v_j \xi_k] +  \Exp [v_i \xi_j v_k] +  \Exp [\xi_i v_j v_k]}_{=0} \nonumber \\ 
	&\quad + \Exp[v_i \xi_j \xi_k] + \Exp[v_j \xi_i \xi_k] + \Exp[v_k \xi_i \xi_j] + \Exp[\xi_i \xi_j \xi_k] \label{lec19:eqn:higher_moments} 
\end{align}
To explicitly compute the last four terms in \eqref{lec19:eqn:higher_moments}, we note that:
\begin{align}
    \Exp[\xi_i \xi_k] = \begin{cases} 0 &\text{if $i \neq k$} \\ \Exp[\xi_i^2] = 1 &\text{if $i = k$} \end{cases}
\end{align} 
and that for any choice of $i, j,$ and $k$,
\begin{align}
    \Exp[\xi_i \xi_j \xi_k] = 0.
\end{align}
Therefore, 
\begin{equation}
	(\Exp[z \otimes z \otimes z])_{ijk} = v_i v_j v_k + v_i \ind{j=k} +  v_j \ind{i=k}  +  v_k \ind{i=j} 
\end{equation}

Since $(\sum_{l=1}^d v \otimes e_l \otimes e_l)_{ijk} = \sum_{l=1}^d v_i  e_{lj}  e_{lk} = v_i \bbI(j=k)$, we have proven that:
\begin{equation}
	\Exp[z \otimes z \otimes z] = v \otimes v \otimes v +  \sum_{l=1}^d v \otimes e_l \otimes e_l + \sum_{l=1}^d  e_l \otimes v \otimes e_l + \sum_{l=1}^d  e_l \otimes e_l \otimes v.
\end{equation} 
\end{proof}

We can now apply Lemma~\ref{lec19:lma:gaussian_third_moment} to compute the third moment of the mixture of $k$ Gaussians. In particular,
\begin{align}
	\Exp[x \otimes x \otimes x] & =  \frac{1}{k} \sum_{i=1}^k \Exp[x \otimes x \otimes x \mid i] \\
	&=  \frac{1}{k} \sum_{i=1}^k \l (\mu_i \otimes \mu_i \otimes \mu_i +  \sum_{l=1}^d \mu_i \otimes e_l \otimes e_l + \sum_{l=1}^d  e_l \otimes \mu_i \otimes e_l + \sum_{l=1}^d  e_l \otimes e_l \otimes \mu_i \r ) \\
	&=  \frac{1}{k} \sum_{i=1}^k \mu_i \otimes \mu_i \otimes \mu_i +  \sum_{l=1}^d \frac{1}{k} \l (\sum_{i=1}^k \mu_i \r ) \otimes e_l \otimes e_l + \sum_{l=1}^d  e_l \otimes \frac{1}{k} \l (\sum_{i=1}^k \mu_i \r ) \otimes e_l \nonumber \\
    &\quad + \sum_{l=1}^d  e_l \otimes e_l \otimes \frac{1}{k} \l (\sum_{i=1}^k \mu_i \r) \\
	& =  \frac{1}{k} \sum_{i=1}^k \mu_i \otimes \mu_i \otimes \mu_i +  \sum_{l=1}^d \Exp[x] \otimes e_l \otimes e_l + \sum_{l=1}^d  e_l \otimes \Exp[x] \otimes e_l + \sum_{l=1}^d  e_l \otimes e_l \otimes \Exp[x]\\
\end{align} 

For notational convenience, let
\begin{equation}
    a^{\otimes 3} \defeq a \otimes a \otimes a.
\end{equation} 
So far, we have shown how to compute $\frac{1}{k} \sum_{i=1}^k \mu_i^{\otimes 3}$ from $\Exp[x^{\otimes 3}]$ and $\Exp[x]$. In the sequel, we will formalize the remaining problem, recovering $\{\mu_i\}_{i = 1}^k$ from $\frac{1}{k} \sum_{i = 1}^k \mu_i^{\otimes 3}$, as the tensor decomposition problem, and discuss efficient algorithms for it.

\paragraph{Tensor decomposition}
Recovering the Gaussian means, $\{\mu_i\}_{i = 1}^k$,  from the third mixture moment, $\frac{1}{k} \sum_{i = 1}^k \mu_i^{\otimes 3}$, is a special case of the general tensor decomposition problem. That problem is set up as follows: Assume that $a_1, \cdots a_k \in \bbR^d$ are unknown. Then, given $\sum_{i=1}^k a_i^{\otimes 3}$, our goal is to reconstruct the $a_i$ vectors. 

Before we present some standard results on tensor decomposition, we first describe some basic facts about tensors. Much like matrices, tensors have an associated rank. For example, $a \otimes b \otimes c$ is a rank-1 tensor. In general, the rank of a tensor $T$ is the minimum $k$ such that $T$ can be decomposed as 
\begin{equation}
    T = \sum_{i=1}^k a_i \otimes b_i \otimes c_i.
\end{equation} 
for some $\{a_i\}_{i =1}^k, \{b_i\}_{i =1}^k, \{c_i\}_{i =1}^k$.
Another difference between tensors and matrices is that the former objects do not have the typical rotational invariance. In particular, consider applying a right rotation $R\in \R^{k\times k}$ to the matrix \begin{equation}
A =\begin{bmatrix} \vrule & & \vrule \\ a_1 & \cdots & a_k \\ \vrule & & \vrule \end{bmatrix} \in\mathbb{R}^{d\times k}
\end{equation}
and get \al{\widetilde{A} = AR = \begin{bmatrix} \vrule & & \vrule \\ \tilde{a}_1 & \cdots & \tilde{a}_k \\ \vrule & & \vrule \end{bmatrix} \in\mathbb{R}^{d\times k}}

Then, 

\al{\sum_{i=1}^k a_ia_i^\top = AA^\top = (AR)\cdot (AR)^\top = \sum_{i=1}^k \tilde{a}_i\tilde{a}_i^\top}
%In particular, 
However, there is no tensor analogue to the rotation invariance result above. 
%\begin{equation}
%    AA^\top = A RR^\top A^\top.
%\end{equation}
But tensors do maintain an interesting (and useful) permutation invariance; that is, $T = \sum_{i = 1}^k a_i^{\otimes 3}$ is invariant to permutations of the indices of $a_i$. Or in other words, let $P\in \R^{k\times k}$ be a permutation matrix suppose, and let \al{
	\widetilde{A} = AP = \begin{bmatrix} \vrule & & \vrule \\ \tilde{a}_1 & \cdots & \tilde{a}_k \\ \vrule & & \vrule \end{bmatrix} \in\mathbb{R}^{d\times k}
}
Then, 
\al{
	\sum_{i=1}^k a_i^{\otimes 3} =  \sum_{i=1}^k \tilde{a}_i^{\otimes 3}
}
The lack of rotation invariance in the sense above and the existence of permutation invariance make tensor decomposition computationally challenging as well as powerful. 

We now summarize some standard results regarding tensor decomposition for $T = \sum_{i = 1}^k a_i^{\otimes 3}$. The results for decomposing the asymmetric version  $T = \sum_{i = 1}^k a_i\otimes b_i \otimes$ are largely analogous. We will not prove these results here.
\begin{enumerate}
\setcounter{enumi}{-1}
\item  In the most general case, recovering the $a_i$'s from $T$ is computationally hard. Any procedure will either fail to find a unique $a_i$ or it fails to find $a_i$ \emph{efficiently}. \tnotelong{Tengyu will add references here}
\item In the orthogonal case, i.e. $a_1,\dots,a_k$ are orthogonal vectors, each $a_i$ is a global maximizer of 
\begin{equation}
    \max_{\|x\|_2 = 1} T(x,x,x) = \sum_{i,j,k} T_{ijk} x_i x_j x_k
\end{equation}
We can heuristically think of $a_i$ as eigenvectors of $T$ and there exists an algorithm to compute $a_i$ in poly-time.
\item In the independent case, i.e. $a_1,\dots,a_k$ are linearly independent. Jennrich's algorithm can be used to efficiently recover $\{a_i\}_{i = 1}^k$.
\end{enumerate} 
Results 1 and 2 above both involve the so-called ``under-complete'' case ($k \leq d$), e.g., when the number of Gaussians in the mixture is smaller than the dimension of the data. Next, we describe certain overcomplete cases for which efficient tensor decomposition is possible.
\begin{enumerate}
\setcounter{enumi}{2}
\item Suppose $a_1^{\otimes2},\dots,a_k^{\otimes2}$ are independent for $k \leq d^2$. Then, applying Result 2, we can recover $a_i$ from $\sum_{i=1}^k (a_i^{\otimes2})^{\otimes3} = \sum_{i=1}^k (a_i^{\otimes6}) \in \bbR^{d^6}$.
\item Excluding an algebraic set of measure $0$, we can use the FOOBI algorithm to recover $a_i$ from the fourth-order tensor $\sum_{i = 1}^k a_i^{\otimes 4}$ when $k \leq d^2$. A robust version of the FOOBI algorithm is described in \citet{ma2016poly}.
\item Assume $a_i$ are \emph{randomly} generated unit vectors. Then, for the third order tensor, $k$ can be large as $d^{1.5}$ \cite{ma2016poly, schrammsteurer17}. 
\end{enumerate}

In summary, the moment method is a recipe in which we first compute high order moments (i.e. tensors), assume that these estimates are noiseless, and decompose these tensors to recover the latent variables. Though we do not discuss these results here, there is an extensive literature analyzing the robustness of the moment method to error in the moment estimates. Last, we note that even though we only explicitly analyze the mixture of Gaussians model here, latent variable models amenable to analysis by the moment method include ICA, Hidden Markov Models, topic models, etc.

\sec{Graph Decomposition and Spectral Clustering}\label{section:spectral_clustering}
Introduced by \citet{shi2000normalized} and \citet{ng2001spectral}, spectral clustering learns a model for the data points using a \emph{pairwise} notion of similarity. Formally, assume that we are given $n$ data points $x\sp{1}, \dots, x\sp{n}$ as well as a similarity matrix $G \in \bbR^{n\times n}$ such that 
\begin{equation}
    G_{ij} = \rho (x\sp{i}, x\sp{j})
\end{equation}
where $\rho$ is some measure of similarity that assigns larger values to more similar pairs of points. 

For example, $x\sp{i}$ could denote images for which $\rho (x\sp{i}, x\sp{j})$ measures the semantic similarity. Alternatively, $x\sp{i}$ might be users of a social network and $\rho (x\sp{i}, x\sp{j}) = 1$ if $x\sp{i}$ and $x\sp{j}$ are friends (hence usually share similar interests / jobs / $\cdots$). 

We note that in moment methods, $\Exp[xx^\top]$ captures pairwise information between coordinates / dimensions, whereas matrix $G$ here captures pairwise information between datapoints.

Our goal is to cluster the data points by viewing $G$ as a graph. For instance, in the social network example, we can naturally view $G$ as the adjacency matrix of an \emph{unweighted} graph, where $G_{ij} \in \{0, 1\}$ defines the edges. Then, the clustering problem is to partition the graph into distinct neighborhoods, i.e., components that are as separate from each other as possible. As we will see repeatedly in the sequel, the eigendecomposition of $G$ is closely related to this graph paritioning / clustering problem.

\begin{figure}[ht]
	\centering
	\includegraphics[width=2in]{figures/ssl1.pdf}
	\caption{A demonstration of graph partitioning. Sets $S_1$ and $S_2$ form a good partition of the graph since there's only one edge between them.}
%	\label{lec15:fig:OLgame}
\end{figure}


\subsec{Stochastic block model} 
In the stochastic block model (SBM), $G$ is assumed to be generated randomly from two hidden communities. Formally, 
\begin{equation}
    \{ 1, \cdots n \} = S \cup \bar{S},
\end{equation}
where $S$ and $\bar{S}$ partition $[n]$. Assume $|S| = \frac{n}{2}$. We then assume the following generative model for $G$. 
If $i,j \in S$ or $i,j \in \bar{S}$, then 
\begin{align}
    G_{ij} = \begin{cases}
        1 &\text{w.p. $p$} \\
        0 &\text{w.p. $1-p$} \end{cases}.
\end{align} 
Otherwise, for $i$ and $j$ in distinct components, 
\begin{align}
    G_{ij} = \begin{cases}
        1 &\text{w.p. $q$} \\
        0 &\text{w.p. $1-q$} \end{cases}
\end{align} 
for $p > q$ (i.e., more likely to be connected if from the same group). For instance, $S$ and $\bar{S}$ could mean two companies, and $i\in[n]$ is a user of a social network. Two users $i, j$ are more likely to know each other if they are in the same company.

\begin{figure}[ht]
	\centering
	\includegraphics[width=2in]{figures/ssl2.pdf}
	\caption{A graph generated by the stochastic block model with $p=\frac{2}{3}$ and $q=\frac{1}{5}$.}
	%	\label{lec15:fig:OLgame}
\end{figure}


Our goal is then to recover $S$ and $\bar{S}$ from $G$; the primary tool we use towards this goal is the eigendecomposition of $G$.

In some trivial cases, it is not necessary to eigendecompose $G$ to recover the two hidden communities. Suppose, for instance, that $p = 0.5$ and $q = 0$. Then, the graph represented by $G$ will contain two connected components that correspond to $S$ and $\bar{S}$.

As a warm-up to motivate our approach, we eigendecompose $\bar{G} = \Exp[G]$. Observe that
\begin{align}
    \bar{G}_{ij} = \begin{cases}
        p &\text{if $i,j$ from the same community} \\
        q &\text{o.w.} \end{cases}.
\end{align}
It is then easy to see that $\bar{G}$ is a matrix of rank $2$:
\begin{align}
    \bar{G} = \left[
        \begin{array}{c|c}
        p \cdots p & q \cdots q \\
        \vdots & \vdots \\
        p \cdots p & q \cdots q\\
        \hline
        q \cdots q & p \cdots p \\
        \vdots & \vdots \\
        q \cdots q & p \cdots p 
        \end{array}
        \right].
\end{align}
\begin{lemma} \label{lec19:lma:sbm_eigen}
Let $\bar{G} = \Exp[G]$ for the stochastic block model. Then, letting $u_i(A)$ denote the $i$-th eigenvector of the matrix $A$,
\begin{align}
    u_1(\bar{G}) &= \vec{1} \label{lec19:eqn:top_eig_G}\\
    u_2(\bar{G}) &= [\underbrace{1, \dots, 1}_{|S|}, \underbrace{-1, \dots, -1}_{|\bar{S}|}]^\top \label{lec19:eqn:second_eig_G}
\end{align}
where $u_2(\bar{G})$ has $|S|$ entries of $1$ and $|\bar{S}|$ entries of $-1$.
\end{lemma}

\begin{proof}
We begin by directly proving \eqref{lec19:eqn:top_eig_G}.
\begin{align}	
	\bar{G} \cdot \vec{1}  &= \begin{bmatrix}
           \frac{pn}{2} + \frac{qn}{2} \\
           \vdots \\
           \frac{pn}{2} + \frac{qn}{2}
         \end{bmatrix} \\
         &= \frac{p+q}{2} \cdot n \cdot \vec{1}.
\end{align}
More generally, $\vec{1}$ is the top eigenvector for any matrix with fixed row sum or any graph with uniform degree (i.e., regular graph). 

Next, we prove \eqref{lec19:eqn:second_eig_G}. Let 
\begin{align}
    G' = \left[
        \begin{array}{c|c}
        r \cdots r \\
        \vdots & \makebox{\text{\huge 0}} \\
        r \cdots r \\
        \hline
        & r \cdots r \\
        \makebox{\text{\huge 0}} & \vdots \\
        & r \cdots r
        \end{array}
        \right]
\end{align}
for $r = p - q$. To precisely define $G'$, we note that $G'$ is block diagonal with two blocks of size $|S|$ and $|\bar{S}|$, respectively. Then, 
\begin{align}	
	\bar{G} &= \vec{1} \vec{1}^\top q + G'. \label{lec19:eqn:barG}
\end{align}
Thus,
\begin{align}
 G' \cdot u &= \left[
    \begin{array}{c|c}
    r \cdots r \\
    \vdots & \makebox{\text{\huge 0}} \\
    r \cdots r \\
    \hline
    & r \cdots r \\
    \makebox{\text{\huge 0}} & \vdots \\
    & r \cdots r
    \end{array}
    \right] \cdot \begin{bmatrix}
           1 \\ \vdots \\ 1\\ -1 \\
           \vdots \\
           -1
         \end{bmatrix} = r \cdot \frac{n}{2} \cdot u. \label{lec19:eqn:gprimeu}
\end{align}
Then, because $u \perp \vec{1}$, we can combine \eqref{lec19:eqn:barG} and \eqref{lec19:eqn:gprimeu} to obtain
\begin{align}
\bar{G} \cdot u =  G' \cdot u =  r \cdot \frac{n}{2} \cdot u  = \frac{p-q}{2} \cdot n \cdot u
\end{align}
Thus, $u$ has eigenvalue $\frac{p-q}{2} \cdot n$. 
\end{proof}

\begin{remark}
    Lemma~\ref{lec19:lma:sbm_eigen} shows that
    \begin{equation}
        \bar{G} = \frac{p + q}{2} \cdot \vec{1} \vec{1}^\top + \frac{p - q}{2} \cdot u u^\top.
    \end{equation}
\end{remark}
More generally, when we have more than two clusters in the graph, $G'$ is block diagonal with more than two blocks. In this setting, the eigenvectors will still align with the blocks. We illustrate this below for a generic block diagonal matrix. Let 
\begin{align}
    A = \left[
        \begin{array}{c|c|c}
        1 \cdots 1 &&\\
        \vdots & \makebox{\text{\huge 0}} & \makebox{\text{\huge 0}} \\
        1 \cdots 1 &&\\
        \hline
        & 1 \cdots 1 \\
        \makebox{\text{\huge 0}} & \vdots & \makebox{\text{\huge 0}} \\
        & 1 \cdots 1 \\
        \hline
        & & 1 \cdots 1 \\
        \makebox{\text{\huge 0}}& \makebox{\text{\huge 0}} & \vdots \\
        & & 1 \cdots 1
        \end{array}
        \right]
\end{align}

Then, the three eigenvectors of $A$ are 
\begin{align}
    \begin{bmatrix}
        1 \\ \vdots \\ 1\\ 0 \\
        \vdots \\
        0 \\ 0 \\ \vdots \\ 0
      \end{bmatrix}, \begin{bmatrix}
        0 \\ \vdots \\ 0\\ 1 \\
        \vdots \\
        1 \\ 0 \\ \vdots \\ 0
      \end{bmatrix}, \begin{bmatrix}
        0 \\ \vdots \\ 0\\ 0 \\
        \vdots \\
        0 \\ 1 \\ \vdots \\ 1
      \end{bmatrix} \label{lec19:eqn:eigenmatrix}
\end{align}
Furthermore, the rows of the matrix formed by the three eigenvectors given by \eqref{lec19:eqn:eigenmatrix} clearly give the cluster IDs of the vertices in $G$. Note also that permutations of $A$ will result in equivalent permutations in the coordinates of each of the three eigenvectors.

Next, we relate this observation to the result in Lemma~\ref{lec19:lma:sbm_eigen}. While there are no negative values in the eigenvectors given in \eqref{lec19:eqn:eigenmatrix}, we observe that any linear combination of eigenvectors is also an eigenvector, so recovering a solution that look more like \eqref{lec19:eqn:second_eig_G} is straightforward. Indeed, taking linear combinations of the eigenvectors defined above shows that there is an alternative eigenbasis that includes the all-ones vector, $\vec{1}$. How for this choice of $A$, the all-ones vector is not the unique top eigenvector. For that to be the case, we require background noise in $\bar{G}$.

In reality, we only observe $G$. In the sequel, we will show that in terms of the spectrum, $G \approx \Exp[G]$. Formally, we will leverage earlier concentration results to prove that $\norm{G - \Exp[G]}_{\text{op}}$ is small. Concretely, then,
\begin{align}
    G &= (G - \Exp[G]) + \Exp[G] \\
    &= (G - \Exp[G]) + \frac{p + q}{2} \cdot \vec{1} \vec{1}^\top + \frac{p - q}{2} \cdot u u^\top
\end{align}
Rearranging, we obtain that:
\begin{align}
    G - \frac{p + q}{2} \cdot \vec{1} \vec{1}^\top &= (G - \Exp[G]) + \frac{p - q}{2} \cdot u u^\top
\end{align}
We then hope that $G - \Exp[G]$ is a small perturbation, so that the top eigenvector of $G - \frac{p + q}{2} \cdot \vec{1} \vec{1}^\top$ is close to $u$. Namely, it suffices to show that 
\begin{equation}
    \norm{G - \Exp[G]}_{\text{op}} \ll \l \|\frac{p - q}{2} \cdot uu^\top\r \|_{\text{op}}.
\end{equation}

\metadata{20}{Miria Feng and Christopher Wolff}{Dec 1st, 2021}

We will start by proving the following lemma.
\begin{lemma}
With high probability,
\begin{align}
    \norm{ G - \Exp[G] }_{\mathrm{op}} \leq O (\sqrt{n \log n} ) \;.
\end{align}
\end{lemma}

Note that this concentration inequality is different from the ones we have seen in the course so far because both $G$ and $\Exp[G]$ are matrices, not scalars. Our goal will be to turn the quantity on the LHS into something that we are familiar with.

\begin{proof}
The key idea is that we can use uniform convergence, after noting that
\begin{align}
    \norm{ G - \Exp[G] }_{\mathrm{op}} &= \max_{\norm{ v }_2 = 1} \left\vert v^\top (G - \Exp[G]) v \right\vert \\
    &= \max_{\norm{ v }_2 = 1} \left\vert v^\top G v - v^\top \Exp[G] v \right\vert \\
    &= \max_{\norm{ v }_2 = 1} \left\vert \sum_{i, j \in [n]} v_i v_j G_{ij} - \Exp \left[ \sum_{i, j \in [n]} v_i v_j G_{ij} \right] \right\vert \;.
\end{align}
Now, the quantity inside the $\max$ is the difference between the sum of independent random variables and their expectation, which we are familiar with. We can use brute force discretization to deal with the $\max$. First, note that for a fixed $v$ with $\norm{ v }_2 = 1$, we can use Hoeffding's inequality to find that
\begin{align}
    \Pr \left( \left\vert \sum_{i, j \in [n]} v_i v_j G_{ij} - \Exp \left[ \sum_{i, j \in [n]} v_i v_j G_{ij} \right] \right\vert \geq \epsilon \right) \leq \exp(-\frac{\epsilon^2}{2}) \;.
\end{align}
Then, we choose $\epsilon = O(\sqrt{n \log n})$, take a discretization of the unit ball with granularity $1/n^{O(1)}$ (which yields a cover of cardinality  $\exp(n \log n)$), and take a union bound over this discretized set to achieve the desired result.
\end{proof}

\begin{remark}
Comparing this bound to $\frac{p - q}{2} \cdot n$, we can deduce that if $p - q \gg \frac{\sqrt{\log n}}{\sqrt{n}}$, then we can recover the vector $u$ approximately. Via a post-processing step that we do not discuss here, $u$ can actually be recovered exactly. 
\end{remark}

\subsec{Clustering the worst-case graph}\label{subsec:clustering_worst_graph}

%\tnote{I wonder whether it's easier to use $A$ for adjacency matrix instead of overloading $G$}

Given a graph $G(V, E)$ where $V$ denotes the set of vertices and $E$ the set of edges, we define the {\it conductance} of a component $S$ as
\begin{align}
    \phi(S) &\defeq \frac{\vert E(S, \bar{S}) \vert}{\operatorname{Vol}(S)}
\end{align}
where $E(S, \bar{S})$ is the total number of edges between $S$ and $\bar{S}$, and $\operatorname{Vol}(S)$ is the total number of edges connecting to $S$. To be precise, let $A$ be the adjacency matrix of $G$, 
\begin{align}
    E(S, \bar{S}) &= \sum_{i \in S, j \in \bar{S}} A_{ij} \\
    \operatorname{Vol}(S) &= \sum_{i \in S, j \in [n]} A_{ij} \;.
\end{align}
Intuitively, conductance captures how separated $S$ and $\bar{S}$ are. 

Since $\operatorname{Vol}(S) \geq E(S, \bar{S})$, it follows that $\phi(S) \leq 1$. Next, observe that $\operatorname{Vol}(S) + \operatorname{Vol}(\bar{S}) = \operatorname{Vol}(V)$. Then, if $\operatorname{Vol}(S) \leq \operatorname{Vol}(V)/2$, it must be the case that $\operatorname{Vol}(S) \leq \operatorname{Vol}(\bar{S})$ and therefore $\phi(S) \geq \phi(\bar{S})$. In some sense, thus suggests that the conductance of a set $\bar{S}$ with volume larger than $\operatorname{Vol}(V)/2$ could be misleading, because the conductance of the other part could be larger. Therefore, typically people only consider the conductance of a smaller part of the partition. 
%In subsequent analysis, we assume without loss of generality that $\operatorname{Vol}(S) \leq \operatorname{Vol}(V)/2$.

Next, we define $\phi(G)$ to be the {\it sparsest cut} of $G$:
\begin{align}
    \phi(G) &= \min_{S: \operatorname{Vol}(S) \leq \operatorname{Vol}(V)/2} \phi(S) \;.
\end{align}
One may wonder why we need to normalize by $\operatorname{Vol}(S)$ in the definition of conductance. The reason is that $E(S, \bar{S})$ itself is typically minimized when $S$ is small. Thus, without this normalization, the sparsest cut would not be very meaningful. For instance, suppose the graph $G$ contains $N$ nodes and can be divided into two halves each containing $N/2$ nodes, and every node is connected to all the other nodes in the same half, but is connected to only $2$ nodes in the other half (as shown in Figure~\ref{fig:ssl1}). Then, we can consider two subsets $S_1$ and $S_2$, where $S_1$ contains half the nodes, and $S_2$ contains two nodes in the same half. It's easy to see that $E(S_1, \bar{S}_1) = \frac{N}{2}\cdot 2 > E(S_2, \bar{S}_2) = \frac{N}{2}$. However, the conductance of $S_1$ is $\phi(S_1) = \frac{E(S_1, \bar{S}_1)}{\operatorname{Vol}(S_1)} = \frac{\frac{N}{2}\cdot 2}{\frac{N}{2}\cdot(\frac{N}{2}-1)+\frac{N}{2}\cdot 2}\approx\frac{4}{N}$, whereas the conductance of $S_2$ is $\phi(S_2) = \frac{E(S_2, \bar{S}_2)}{\operatorname{Vol}(S_2)}=\frac{\frac{N}{2}}{N+2} \approx\frac{1}{2}$. Thus, when $n$ is large, $S_1$ is a sparser cut than $S_2$ under $\phi(\cdot)$. 


\begin{figure}[ht]
	\centering
	\includegraphics[width=2in]{figures/ssl3.pdf}
	\caption{A demonstration of the sparsest cut. $S_1$ is a sparser cut than $S_2$.}
	\label{fig:ssl1}
\end{figure}


Our goal is to find an approximate sparsest cut $\hat{S}$ such that $\phi(\hat{S}) \approx \phi(G)$.\footnote{Finding the exact sparsest cut is a NP-hard problem.} Our main technique is eigendecomposition or spectral clustering, though in the literature much more advanced and better algorithms have been proposed, e.g., the famous ARV algorithm~\cite{arora2009expander}. Let $d_i = \operatorname{Vol}(\{i\})$ be the degree of node $i$, and let $D = \text{diag}(\{d_i\})$ be the diagonal matrix that contains the degrees $d_i$ as entries. Furthermore, let 
\begin{equation}
    \bar{A} = D^{-\frac{1}{2}} A D^{\frac{1}{2}}
\end{equation}
be the normalized adjacency matrix. This is equivalent to normalizing each element $A_{ij}$ of the adjacency matrix by $\frac{1}{\sqrt{d_i d_j}}$ (i.e., $\bar{A}_{ij} = \frac{A_{ij}}{\sqrt{d_i d_j}}$).
In most cases, it suffices to starting with considering $G$ as a regular graph (whose degrees are all the same), because the proof for regular graph can oftentimes extend to general graph easily. Assuming $G$ is a $\kappa$-regular graph, i.e. $d_i = \kappa$; then, this normalization simply scales $A$ by $\frac{1}{\kappa}$.

 Let $L = I - \bar{A}$ be the Laplacian matrix. Note that any eigenvector of $L$ is also an eigenvector of $\bar{A}$. Suppose $L$ has eigenvalues $\lambda_1 \leq \hdots \leq \lambda_n$ with corresponding eigenvectors $u_1 \hdots u_n$, then $\bar{A}$ has eigenvalues $1 - \lambda_1 \geq \hdots \geq 1 - \lambda_n$ with the same eigenvectors.
 
 The following famous Cheeger's inequality relates the eigendecompositions to the graph partitioning. 

\begin{theorem}[Cheeger's inequality]
The second eigenvalue of $L$, namely $\lambda_2$, is related to the sparsest cut $\phi(G)$ as follows:
\begin{align}
    \frac{\lambda_2}{2} \leq \phi(G) \leq \sqrt{2 \lambda_2} \;.
\end{align}
Moreover, we can find $\hat{S}$ such that $\phi(\hat{S}) \leq \sqrt{2 \lambda_2} \leq 2 \sqrt{\phi(G)}$ efficiently by rounding the second eigenvector. Suppose $u_2 = [\beta_1 \cdots \beta_n]^\top \in \bbR^n$ is the second eigenvector of $L$. Then we can choose a threshold $\tau = \beta_i$ and consider $\hat{S}_i = \{ j \in [n] : \beta_j \leq \tau \}$. At least one such $\hat{S}_i$ satisfies $\phi(\hat{S}_i) \leq 2 \sqrt{\phi(G)}$.
\end{theorem}

Note that this can be viewed as a general but weaker version of the theorem that we proved for stochastic block model. There is no randomized assumption; the conclusion is weaker than those for SBM; also the rounding algorithm to recover the cluster is also more complicated---one has to try multiple thresholding instead of using threshold $1/2$. 

We will refer the readers to~\citet{chung2007four} for the proof of the theorem. Here below we give a few basic lemmas that help build up intuitions on why eigendecompositions relate to graph decomposition.

First, one might wonder why this algorithm uses the second eigenvector of $\bar{A}$, but not the first eigenvector. As we have seen in the SBM case, the first eigenvector captures the background in some sense. Here for general graph, we see the same phenomenon. The top eigenvector is generally not that interesting as it only captures the ``background density'' of the graph. For instance, when $A$ is $\kappa$-regular, $\vec{1}$ is the top eigenvector of $A$ and is thus also the top eigenvector of $\bar{A} = \frac{1}{\kappa} \cdot A$. More generally, we have the following lemma:


\begin{lemma}
The top eigenvector of $\bar{A}$ (respectively, the smallest eigenvector of $L$) is $u_1 = [\sqrt{d_1} \cdots \sqrt{d_n}]^\top$.	
\end{lemma}
\begin{proof}
\begin{align}
	(\bar{A} \cdot u_1)_i &= \sum_j \bar{A}_{ij} \sqrt{d}_j \\
	&= \sum_j \frac{A_{ij}}{\sqrt{d_i}\sqrt{d_j}} \sqrt{d}_j \\
	&= \frac{1}{\sqrt{d}_i} \sum_j A_{ij} \\
	&= \frac{d_i}{\sqrt{d}_i} = \sqrt{d}_i.
\end{align}
\end{proof}


To understand why the eigenvectors of the Laplacian are related to the sparsest cut, we examine the quadratic form the Laplacian. In particular, we have the following lemma:
\begin{lemma}\label{lemma:laplacian_quadratic}
	Let $v\in\bbR^N$ be a vector, $L$ is the graph Laplacian. Then, 
	\begin{align}
		v^\top L v = \frac{1}{2} \sum_{(i, j) \in E} \left( \frac{v_i}{\sqrt{d_i}} - \frac{v_j}{\sqrt{d_j}} \right)^2.
	\end{align}	
\end{lemma}
\begin{proof}
\begin{align}
	v^\top L v &= v^\top I v - v^\top \bar{A} v \\
	&= \sum_{i=1}^n v_i^2 - \sum_{i, j = 1}^n v_i v_j \bar{A}_{ij} \\
	&= \sum_{i=1}^n v_i^2 - \sum_{i, j = 1}^n v_i v_j \frac{A_{ij}}{\sqrt{d_i d_j}} \\
	&= \frac{1}{2}\cdot \left(2\sum_{i=1}^n v_i^2 - 2 \sum_{(i, j) \in E} \frac{v_i}{\sqrt{d_i}} \cdot \frac{v_j}{\sqrt{d_j}}\right) \\
	&= \frac{1}{2} \sum_{(i, j) \in E} \left( \frac{v_i}{\sqrt{d_i}} - \frac{v_j}{\sqrt{d_j}} \right)^2 \;.
\end{align}
\end{proof}

If $G$ is $\kappa$-regular, then this becomes $v^\top L v = \frac{1}{2\kappa} \sum_{(i, j) \in E} (v_i - v_j)^2$. Furthermore, suppose $v \in \{0, 1\}$ is a binary vector with support $S = \operatorname{supp}(v)$. Then, 
\begin{align}
    \frac{1}{2\kappa} \sum_{(i, j) \in E} (v_i - v_j)^2 &= \frac{1}{\kappa} E(S, \bar{S}) \\
    &= \frac{1}{\kappa} E(\operatorname{supp}(v), \overline{\operatorname{supp}}(v)) \;.
\end{align}
If $|\operatorname{supp}(v)| \leq n/2$, implying $\operatorname{Vol}(S) \leq \operatorname{Vol}(V)/2$, then
\begin{align}
    \frac{v^\top L v}{\norm{v}^2_2} &= \frac{\frac{1}{\kappa} E(S, \bar{S})}{\frac{1}{\kappa} \operatorname{Vol}(S)} = \phi(S) \;.
\end{align}
The term $\frac{v^\top L v}{\norm{v}^2_2}$ is also called the {\it Rayleigh quotient}. This result nicely connects the abstract linear algebraic approach to the sparsest cut approach. Note that we only achieve an approximate sparsest cut because when we compute eigenvectors, we minimize the Rayleigh quotient \emph{without any constraints on $v$}. By contrast, the sparsest cut minimizes the Rayleigh quotient subject to the constraint that $v \in \{0,1\}^n$. Proving Cheeger inequality essentially involves controlling the difference caused by real $v$ vs binary $v$. We omit the proof of Cheeger's inequality because it's beyond the scope of this notes. 

\subsec{Applying spectral clustering}

%\tnote{it's possible to merge this with the beginning of 10.2.2 if you felt that's good; please use your own assessments}
How do the ideas from the previous sections connect to our previous discussion of spectral clustering? Suppose that we have some raw data $x^{(1)} \cdots x^{(n)}$ that we'd like to group into $k$ clusters. \citet{ng2001spectral} propose to define a weighted graph $G$ such that each element 
\begin{equation}
    G_{ij} = \exp \left( -\frac{\norm{x^{(i)} - x^{(j)}}^2_2}{2\sigma^2} \right)
\end{equation}
represents a distance between two data points. Then, we compute the first $k$ eigenvectors of the Laplacian $L$ and arrange them into the columns of a matrix: 
\begin{equation}
    \begin{bmatrix} \lvert &  & \lvert \\ u_1 & \cdots &  u_k \\ \lvert &  & \lvert \end{bmatrix} \in \bbR^{n \times k}.
\end{equation} 
The $i$-th row of this matrix (which we denote by $v_i$) is then a $k$-dimensional embedding of the $i$-th example. Note that this is usually a much lower-dimensional representation than the raw data. Finally, we can use $k$-means to cluster the embeddings $\{v_1,\dots,v_n\}$.

In high dimensions, the issue with \citet{ng2001spectral}'s approach is that the training data points are all far away from each other. The Euclidean distance between points becomes meaningless, and so our definition of $G$ does not make sense. 

How do we solve this issue? \citet{haochen2021provable} propose a solution. They consider an infinite weighted graph $G(V, w)$, where $w$ are the edge weights, and $V = \cX \subseteq \bbR^n$ is the set of all possible data points. We define $w(x, x')$ to be large only if $x$ and $x'$ are very close in $\ell_2$ distance. Now, the graph is more meaningful, because only data points that are small perturbations of each other have high connection weights. However, we do not have explicit access to this graph, and its eigenvectors are infinite-dimensional. 

Now, suppose we have some eigenvector $u = (u_x)_{x \in \cX} $. Rather than explicitly represent $u_x$, we represent $u_x$ by $f_\theta(x)$ where $f_\theta$ is some parameterized model. Now, the challenge is to find $\theta$ such that $[f_\theta(x)]_{x \in \cX}$ is the second smallest eigenvector of Laplacian of $G$. It turns out solving this problem gives a form of contrastive learning, which we will discuss in Section~\ref{section:analysis_contrastive_learning}.


\chapter{Self-supervised Learning}

\sec{Pretraining / self-supervised learning / foundation model basics}
Self-supervised learning is widely used for pretraining modern models. These large pretrained models are also called foundation models~\cite{bommasani2021opportunities}. One simplified setup / workflow contains the following two stages:

\paragraph{Pretraining on unlabeled, massive data.} Let $\{x^{(1)}, \cdots, x^{(n)}\}$ be a dataset where $x^{(i)}\in\bbR^d$ is sampled from some pretraining data distribution $x^{(i)}\sim P_{\text{pre}}$. The goal is to learn a pretrained model $f_{\theta}: \bbR^d\rightarrow\bbR^k$, where $k$ is the dimension of features / representations / embeddings, and $\theta$ is the model parameter. This model can be learned by minimizing certain pretrained loss function: $\hat{L}_{\text{pre}}(\theta) = \hat{L}_{\text{pre}}(x^{(1)}, \cdots, x^{(n)}; \theta)$, which sometimes is of the form $\hat{L}_{\text{pre}}(\theta)  = \frac{1}{n}\sum_{i=1}^n \ell_{\text{pre}}(x^{(i)}; \theta)$. We use $\hat{\theta} = \argmin_{\theta}\hat{L}_{\text{pre}}(\theta)$ to denote the parameter learned during pretraining.

\paragraph{Adaptation.} During adaptation, we have access to a set of labeled downstream task examples $\{(x^{(1)}_{\text{task}}, y^{(1)}_{\text{task}}), \cdots, (x^{(n_{\text{task}})}_{\text{task}}, y^{(n_{\text{task}})}_{\text{task}})\}$, where usually $n_{\text{task}}\ll n$. One popular adapataion method is \emph{linear probe}, which uses $f_{\hat{\theta}}(x)$ as features / representations / embeddings, and train a linear classifier on downstream tasks. Concretely, the prediction on data $x$ is $w^\top f_{\hat{\theta}}(x)$, where $w$ is the linear head learned from $\min_{w} \hat{L}_{\text{task}}(w) = \frac{1}{n_{\text{task}}} \sum_{i=1}^{n_{\text{task}}} \ell_{\text{task}}(y^{(i)}_{\text{task}}, w^\top f_{\hat{\theta}}(x^{(i)}_{\text{task}}))$. Another popular adaptation method is \emph{finetuning}, which also updates the parameter $\theta$. Concretely, one initializes from $\theta = \hat{\theta}$, and solve $\min_{\theta, w} \hat{L}_{\text{task}}(w, \theta) = \frac{1}{n_{\text{task}}} \sum_{i=1}^{n_{\text{task}}} \ell_{\text{task}}(y^{(i)}_{\text{task}}, w^\top f_{{\theta}}(x^{(i)}_{\text{task}}))$.

Why does pretraining on unlabeled data with an unsupervised (self-supervised) loss help a wide range of downstream prediction tasks? There are many open questions to be answered in this field. For instance,  we may ask: (1) how pretraining helps label efficiency of downstream tasks, (2) why pretraining can give ``universal'' representations, and (3) why does pretraining provide robustness to distribution shift. 

\sec{Analysis of contrastive learning}\label{section:analysis_contrastive_learning}
Let $\bar{X}$ be the set of all natural images in the image domain, $\bar{P}_{\bar{X}}$ be the distribution over $\bar{X}$. Contrastive learning learns $f_{\theta}$ such that representations of augmentations of the same image are close, whereas augmentations of random images are far away (as visualized in Figure~\ref{fig:ssl2}). 

\begin{figure}[ht]
	\centering
	\includegraphics[width=3in]{figures/ssl5.pdf}
	\caption{A demonstration of contrastive learning. Representations of augmentations of the same image are pulled close, whereas augmentations of random images are pushed far away.}
	\label{fig:ssl2}
\end{figure}

\newcommand{\aug}[1]{\mathcal{A}(\cdot|#1)}

Given a natural image $\bar{x}\in\bar{X}$, one can generate augmentations by random cropping, flipping, adding Gaussian noise or color transformation. We use $x\sim\aug{\bar{x}}$ to denote the conditional distribution of augmentations given the natural image. For simplicity, we consider the case where Gaussian blurring is the augmentation, we have
\begin{align}
	x\sim\aug{\bar{x}} \Leftrightarrow x=\bar{x}+\xi \quad\quad\quad\quad \xi\sim\mathcal{N}(0, \sigma^2\mathcal{I}),
\end{align}
where $\norm{\xi}_2\approx\sigma\sqrt{d}$ should be $\ll\norm{\bar{x}}$. 

We say $(x, x^+)$ is a \emph{positive pair} if they are generated as follows: first sample $\bar{x}\sim \bar{P}_{\bar{X}}$, then sample $x\sim\aug{\bar{x}}$ and $x^+\sim\aug{\bar{x}}$ independently. In other words, $(x, x^+)$ are augmentations of the same natural image.

We say $(x, x')$ is a \emph{random pair / negative pair} if they are sampled as: first sample two natural images $\bar{x}\sim \bar{P}_{\bar{X}}$ and $\bar{x}'\sim \bar{P}_{\bar{X}}$, then sample augmentations $x\sim\aug{\bar{x}}$ and $x'\sim\aug{\bar{x}'}$.

The design principle for contrastive learning is to find $\theta$ such that $f_{\theta}(x)$, $f_{\theta}(x^+)$ are close, while $f_{{\theta}}(x), f_{{\theta}}(x')$ are far away~\citep{chen2020simclr, zbontar2021barlow, he2020momentum}. 

One example of contrastive learning is SimCLR~\citep{chen2020simclr}.  Given $B$ positive pairs $(x^{(1)}, x^{(1)+}), \cdots, (x^{(B)}, x^{(B)+})$, note that $(x^{(i)}, x^{(j)+})$ is a random pair if $i\ne j$, SimCLR defines the loss on the $i$-th pair as 
\begin{align}
	\text{loss}_i = -\log\frac{\exp(f_{{\theta}}(x^{(i)})^\top f_{{\theta}}(x^{(i)+}))}{\exp(f_{{\theta}}(x^{(i)})^\top f_{{\theta}}(x^{(i)+}))+ \sum_{j\ne i} \exp(f_{{\theta}}(x^{(i)})^\top f_{{\theta}}(x^{(j)+}))}.
\end{align}
Notice that $-\log\frac{A}{A+B}$ is decreasing in $A$ but increasing in $B$, the loss above encourages $f_{{\theta}}(x^{(i)})^\top f_{{\theta}}(x^{(i)+})$ to be large, while $f_{{\theta}}(x^{(i)})^\top f_{{\theta}}(x^{(j)+})$ to be small.

In the rest of this section, we consider a variant of contrastive loss, proposed in~\cite{haochen2021provable}: 
\begin{align}
	L(\theta)  = -2\Exp_{(x, x^+) \sim\text{positive}} f_{{\theta}}(x)^\top f_{{\theta}}(x^+) + \Exp_{(x, x') \sim\text{random}}  \left(f_{{\theta}}(x)^\top f_{{\theta}}(x')\right)^2.
\end{align}
This contrastive loss follows the same design principle as other contrastive losses in the literature: suppose all representations have the same norm, then the first term encourages the representations of a positive pair to be closer while  the second term encourages the representations of a random pair to be orthogonal to each other (hence far away). ~\cite{haochen2021provable} show that the loss function, though inspired by theoretical derivations, still perform competitively, nearly matching the SOTA methods. 

We can also define the empirical loss on a set of tuples $(x^{(i)}, x^{+(i)}, x'^{(i)})$ sampled i.i.d. as follows: $\bar{x}\sim\bar{P}_{\bar{X}}, x^{(i)}\sim\aug{\bar{x}^{(i)}}, x^{+(i)}\sim\aug{\bar{x}^{(i)}}$, $\bar{x}'\sim\bar{P}_{\bar{X}}$, $x'^{(i)}\sim\aug{\bar{x}'^{(i)}}$. The empirical loss is defined as 
\begin{align}
	\hat{L}(\theta) = \sum_{i=1}^n \left[-2f_{{\theta}}(x^{(i)})^\top f_{{\theta}}(x^{+(i)}) + \left(f_{{\theta}}(x^{(i)})^\top f_{{\theta}}(x'^{(i)})\right)^2\right].
\end{align}
Then the empirically learned parameter is $\hat{\theta} = \min_{\theta} \hat{L}(\theta)$. 

Suppose the downatream task is binary classification with label set $\{1, -1\}$. We define the downstream loss as 
\begin{align}
	\hat{L}_{\text{task}}(w, \theta) = \frac{1}{n_{\text{task}}} \sum_{i=1}^{n_{\text{task}}} \frac{1}{2} \left(y^{(i)}_{\text{task}} - w^\top f_{{\theta}}(x^{(i)}_{\text{task}})\right)^2.
\end{align}

We learn the linear head $\hat{w} = \argmin_w \hat{L}_{\text{task}}(w, \hat{\theta})$, and the evaluate its performance on downstream population data:
\begin{align}
	L_{\text{task}}(\hat{w}, \hat{\theta}) = \Exp\left[\frac{1}{2} \left(y_{\text{task}} - \hat{w}^\top f_{{\theta}}(x_{\text{task}})\right)^2\right].
\end{align}

\noindent {\bf Analysis pipeline. } We give a summary of our analysis pipeline below. The key takeaway is that we only have to focus on the population distribution case (step 3). 
\begin{enumerate}\setcounter{enumi}{-1}
\item{Assume expressivity, i.e., assuming $\exists \theta^*$ such that $L(\theta^*)$ is sufficiently small (the details will be quantified later).}
\item{For large enough $n$ (e.g., $n>\text{Comp}({\mathcal{F}})/\epsilon^2$ where ${\mathcal{F}}=\{f_\theta\}$ is the function class, $\text{Comp}(\cdot)$ is some measure of complexity, $\epsilon$ is the target error), show that $\hat{L}(\theta) = L(\theta) \pm \epsilon$.}
\item{Let $\hat{\theta}$ be the parameter learned on empirical data. Since $\hat{L}(\hat{\theta})=\min_{\theta}\hat{L}(\theta) \le \hat{L}(\theta^*) \le L(\theta^*)+\epsilon$, we have
\begin{align}
	\hat{L}(\hat{\theta})\le\epsilon \Rightarrow L(\hat{\theta}) \le 2\epsilon
\end{align}
}
\item{\textbf{Key step:} (infinite data case) We will prove a theorem (Theorem~\ref{theorem:scl} below as a simplified version, or Theorem 3.8 of~\citet{haochen2021provable}) that shows if $L(\hat{\theta})\le2\epsilon$, then there exists $w$ such that $L_{\text{task}}(\theta, w)\le \delta$, where $\delta$ is a function of $\epsilon$ and data distribution $\bar{P}$.}
\item{When we have enough downstream data $n_{\text{task}}\ge\text{poly}(k, \frac{1}{\epsilon'})$, for any $\theta$, with high probability we have (via uniform convergence) that for any $w$, 
$	\hat{L}_{\text{task}}(w, \theta) \approx L_{\text{task}}(w, \theta)\pm \epsilon'$. }
\item{Using the results in step 3 and step 4, we have $\hat{L}_{\text{task}}(\hat{w}, \hat{\theta}) = \min_{w} \hat{L}_{\text{task}}(w, \hat{\theta}) \le \min_w L_{\text{task}}(w, \hat{\theta}) + \epsilon' \le \delta + \epsilon'$. Thus, the final evaluation loss on the downstream task is $L_{\text{task}}(\hat{w}, \hat{\theta}) \le \hat{L}_{\text{task}}(\hat{w}, \hat{\theta}) +\epsilon' \le \delta + 2\epsilon'$. }
\end{enumerate}

\noindent{\bf Key step: the case with population pretraining and downstream data.} We will now dive into the analysis of step 3, as all the other steps are from standard concentration inequalities. Recall that 
\begin{align}
		L(\theta)  = -2\Exp_{(x, x^+) } f_{{\theta}}(x)^\top f_{{\theta}}(x^+) + \Exp_{(x, x') }  \left(f_{{\theta}}(x)^\top f_{{\theta}}(x')\right)^2.
\end{align}

As expected, the analysis requires structural assumptions on the data. In particular, we will use the graph-theoretic language to describe the assumptions on population data. Let $X$ be the set of all augmented data, $P$ be the distribution of augmented data $x\sim\aug{\bar{x}}$ where $\bar{x}\sim\bar{P}_{\bar{X}}$. Let $p(x, x^+)$ be the probability density of positive pair $(x, x^+)$. We define a graph $G(V, w)$ where vertex set $V=X$ and edge weights $w(x, z) = p(x, z)$ for any $(x, z) \in X\times X$. In general, this graph may be infinitely large. To simplify math and avoid integrals, we assume $|X|=N$ where $N$ is the number of all possible augmented images (which can be infinite or exponential in dimension). 

The degree of node $x$ is $p(x) = \sum_{z\in X} p(x, z)$.  Let $A\in\bbR^{N\times N}$ be the adjacency matrix of this graph defined as $A_{x, z} = p(x, z)$, and let $\bar{A}$ ber the normalized adjacency matrix such that $\bar{A}_{x, z} = \frac{p(x, z)}{\sqrt{p(x)p(z)}}$. 

The following lemma shows that contrastive learning is closely related to the eigendecomposition of $\bar{A}$. 
\begin{lemma}\label{lemma:scl_as_decomposition}
	Let $L(f) = -2\Exp_{(x, x^+) } f(x)^\top f(x^+) + \Exp_{(x, x') } \left(f(x)^\top f(x')\right)^2$.  Suppose $X=\{x_1, \cdots, x_N\}$, let matrix 
	\begin{equation}
	    F = \begin{bmatrix}  p(x_1)^{\frac{1}{2}} f(x_1)^\top  \\ \vdots \\  p(x_N)^{\frac{1}{2}} f(x_N)^\top \end{bmatrix}.
	\end{equation}
	Then,
	\begin{align}
		L(f) = \norm{\bar{A}-FF^\top}_F^2 +\text{const}.
	\end{align}
	Hence, minimizing $L(f)$ w.r.t the variable $f$ is equivalent to eigendecomposition of $\bar{A}$. 
\end{lemma}
\begin{proof}
Directly expanding the Frobenius norm $\norm{\bar{A}-FF^\top}_F^2$ as a sum over entries, we have
\begin{align}
	\norm{\bar{A} - FF^\top}_F^2 &= \sum_{x, z\in X} \left(\frac{p(x, z)}{\sqrt{p(x)}\sqrt{p(z)}} - f(x)^\top f(z) \sqrt{p(x)}\sqrt{p(z)}\right)^2\\
	&= \text{const} -2 \sum_{x, z\in X} p(x, z) f(x)^\top f(z) + \sum_{x, z\in X}p(x)p(z)\left(f(x)^\top f(z)\right)^2\\
	&= \text{const} -2\Exp_{(x, x^+) \sim\text{positive}} f(x)^\top f(x^+) + \Exp_{(x, x') \sim\text{random}}  \left(f(x)^\top f(x')\right)^2,
\end{align}
where the last equation uses the fact that $p(x, z)$ and $p(x)p(z)$ are the probability densities of $(x,z)$ being a positive pair and a random pair, respectively. 
\end{proof}

Standard matrix decomposition results tell us that the minimizer of $\norm{\bar{A}-FF^\top}_F^2 $ satisfies $F = U \cdot \text{diag}(\gamma_i^{\frac{1}{2}})$, where $\gamma_i$'s are the eigenvalues of $\bar{A}$ and $U\in\bbR^{N\times k}$ contains the top $k$ eigenvectors of $\bar{A}$ as its columns. Suppose we use $v_1, \cdots, v_N$ to represent the rows of $U$, i.e., 
	\begin{equation}
	U = \begin{bmatrix}  v_1^\top  \\ \vdots \\  v_N^\top \end{bmatrix}.
\end{equation}
Then we know $f(x_j)= p(x_j)^{-\frac{1}{2}} \cdot \text{diag}(\gamma_i^{\frac{1}{2}}) \cdot v_j$ is the minimizer of the contrastive loss. 

\newcommand{\id}[1]{\mathbbm{1}\left[#1\right]}

One interesting thing is that $f(x_i)$ has the same separability as $v_i$. This is because for any vector $b\in\bbR^k$, we have $\id{b^\top v_i>0} = \id{b^\top \text{diag}(\gamma_i^{-\frac{1}{2}}) f(x) >0}$, suggesting linear head $\text{diag}(\gamma_i^{-\frac{1}{2}}) b$ applied on feature $f(x_i)$ would achieve the same classification accuracy as $v$ applied on $v_i$. Thus, it suffices to analyze $v_i$'s downstream accuracy under linear head.

Since $v_i$ is exactly the feature used by the classic spectral clustering algorithm, we may ask when spectral clustering produces good features. As discussed in Section~\ref{section:spectral_clustering}, spectral clustering is good at graph partitioning in stochastic block models. In this section, we aim to find more general settings where spectral clustering produces good features. For simplicity, let's consider a regular graph where $w(x) = \sum_{x'\in V} w(x, x') = \kappa$.\footnote{It turns out that most, if not all, spectral graph theory tools on regular graph can extend to general graph settings. Therefore, it oftentimes suffices to consider a regular graph. } %\jnote{rewrote this para}

The following lemma shows that suppose the graph roughly contains two clusters, then the spectral clustering features can be used to accurately predict which cluster a node belongs to.
%\tnote{maybe say what the lemma is about in words before stating the lemma}\jnote{added}
\begin{lemma}\label{lemma:ssl1}
	Suppose the graph $G$ can be partitioned into $2$ clusters $S_1$, $S_2$ with size $|S_1| = |S_2| = \frac{N}{2}$, such that $E(S_1, S_2)=\sum_{x\in S_1, z\in S_2} w(x, z) \le \alpha \kappa N$. Furthermore, suppose $G$ cannot be partitioned well into $3$ clusters in the sense that for all partition $ T_1, T_2, T_3$, we have $\max\{\phi(T_1), \phi(T_2), \phi(T_3)\} \ge \rho$. (Figure~\ref{figure:ssl_thm_assumption} gives a demonstration of these assumptions.)
	\begin{figure}[ht]
		\centering
		\includegraphics[width=3in]{figures/ssl4.pdf}
		\caption{A demonstration of the assumptions in Lemma~\ref{lemma:ssl1}. The left half and right half of the graph can be chosen as $S_1$ and $S_2$, since there's at most $\alpha$ proportion of edges between them. Sets $T_1, T_2, T_3$ form a 3-way partition where $\phi(T_1)\ge \rho$.
%\tnote{a bit more caption?} 
%			\tnote{I think $S_1, S_2$ are mentioned twice with different meanings in the lemma 10.10, so the figure is a bit ambiguous. Maybe replace $S_1, S_2, S_3$ by $T_1,..$ and change the figure.}
		}\label{figure:ssl_thm_assumption}
	\end{figure}
	Then, let $g=\mathbbm{1}(S_1)\in \bbR^N$ (i.e., $g_i=1$ if $i\in S_1$), and $k\ge 6$, there exists linear classifier $b$ such that %\jnote{I realize that there should be a $N$ here on RHS.}
	\begin{align}
		\norm{Ub-g}_2^2 \lesssim \frac{N\alpha}{\rho^2},
	\end{align}
where $U$ contains the top $k$ eigenvectors of $\bar{A}$ as its columns.
\end{lemma}

The above lemma essentially says that $\langle v_x, b \rangle\approx g_x$ for all data $x\in X$, where $v_x$ is the $x$-th row of $U$. 

Before proving the above lemma, we first introduce the following higher-order Cheeger inequality, which shows that when the graph cannot be partitioned well into $3$ clusters, the $6$-th smalled eigenvalue of the Laplacian cannot be too small. %\tnote{say briefly what its about}\jnote{added}
\begin{lemma}[Proposition 1.2  in \cite{louis2014approximation}]\label{lemma:higher_order_cheeger}
	Let $G=(V, w)$ be a weight graph.  Suppose the graph cannot be partitioned into $3$ sets $S_1, S_2, S_3$ such that $\max\{\phi(S_1), \phi(S_2), \phi(S_3)\} \le \rho$. Then, we have
	\begin{align*}
		 \lambda_{6} \gtrsim\rho^2.
	\end{align*}
%	\tnote{can we have the other side of the inequality stated here as well? }\jnote{the other side of inequality requires 6-way partition rather than 3-way. Would it be worth of it to introduce that?}
%	\tnote{could you state the contrapositive of this? [I suspect that it's easier to understand the contrapositive in this context]}
\end{lemma}

Now we give a proof of Lemma~\ref{lemma:ssl1}.
\begin{proof}[Proof of Lemma~\ref{lemma:ssl1}]
	By Lemma~\ref{lemma:laplacian_quadratic} we know that 
\begin{align}
	\frac{2}{N} g^\top L g &= \frac{1}{N\kappa} \sum_{x, z} (g_x - g_z)^2 w(x, z)\\
	&= \frac{1}{N\kappa} \left(\sum_{x\in S_1, z\in S_2} w(x, z) + \sum_{x\in S_2, z\in S_1} w(x, z)\right)\\
	&= \frac{2}{N\kappa} E(S_1, S_2) \\
	&\le \alpha.
\end{align}
Thus, $g$ has to be mostly in the smaller eigenspace of $L$.  Suppose $L$ has eigenvalue $0=\lambda_1\le\lambda_2 \le \cdots \le \lambda_N $, with corresponding eigenvectors $u_1, u_2, \cdots u_N$. Define matrix $U=[u_1, \cdots, u_k] \in \bbR^{N\times k}$. Suppose $\sqrt{\frac{2}{N}}g = \sum_{i=1}^N \beta_i u_i$. Since $\norm{\sqrt{\frac{2}{N}}g}=1$, we know $\sum_{i=1}^N \beta_i^2 = 1$.

Since we know $g^\top L g = \sum_{i=1}^N \beta_i^2 \lambda_i \le \frac{N\alpha}{2}$, we can conclude $\sum_{i>k} \beta_i^2 \lambda_i \le \frac{N\alpha}{2}$, which implies that $\sum_{i>k}\beta_i^2 \le \frac{N\alpha}{2\lambda_{k+1}} \lesssim \frac{N\alpha}{\rho^2}$. Here we used the fact $\lambda_6 \gtrsim \rho^2$ by higher-order Cheeger inequality (Lemma~\ref{lemma:higher_order_cheeger}). Thus, we have $\norm{g-\sum_{i=1}^k \beta_i u_i}_2^2 = \norm{\sum_{i>k} \beta_i u_i}_2^2 \lesssim \frac{N\alpha}{\rho^2}$ which finishes the proof. 
\end{proof}

We can combine Lemma~\ref{lemma:scl_as_decomposition} and Lemma~\ref{lemma:ssl1} to prove the following theorem, which shows that when the graph roughly contains $2$ clusters, the feature learned from contrastive learning can be used to predict the cluster membership accurately. 
%\jnote{added the theorem below so that we have something to point to in the "key step" bullet point}
\begin{theorem}\label{theorem:scl}
	Let $L(f) = -2\Exp_{(x, x^+) } f(x)^\top f(x^+) + \Exp_{(x, x') } \left(f(x)^\top f(x')\right)^2$, and $f^*: X\rightarrow \bbR^k$ is a minimizer of $L(f)$ for $k\ge 6$. Suppose the graph $G$ can be partitioned into $2$ clusters $S_1$, $S_2$ with size $|S_1| = |S_2| = \frac{N}{2}$, such that $E(S_1, S_2)=\sum_{x\in S_1, z\in S_2} w(x, z) \le \alpha \kappa N$. Furthermore, suppose $G$ cannot be partitioned well into $3$ clusters in the sense that for all partition $ T_1, T_2, T_3$, we have $\max\{\phi(T_1), \phi(T_2), \phi(T_3)\} \ge \rho$. Let $y(x_i) = \mathbbm{1}(x_i\in S_1)$ (i.e., $y(x_i)=1$ if $x_i\in S_1$, otherwise $y(x_i)=0$). Then, there exists linear classifier $b\in\bbR^k$ such that
	\begin{align}
		\frac{1}{N}\sum_{i\in [N]} \left(f(x_i)^\top b - y(x_i)\right)^2 \lesssim \frac{\alpha}{\rho^2}.
	\end{align}
\end{theorem}
\begin{proof}
	Let $U\in\bbR^{N\times k}$ contains the top $k$ eigenvectors of $\bar{A}$ as its columns. By Lemma~\ref{lemma:ssl1}, we know there exists some $\hat{b}\in\bbR^k$ such that $\norm{U\hat{b}-g}_2^2 \lesssim \frac{N\alpha}{\rho^2}$, where $g\in\bbR^N$ such that $g_i = y(x_i)$. Let $v_1, \cdots, v_N$ be the rows of $U$. According to Lemma~\ref{lemma:scl_as_decomposition}, we know that $f(x_i)= p(x_i)^{-\frac{1}{2}} \cdot \text{diag}(\gamma_j^{\frac{1}{2}}) \cdot v_i = \kappa^{-\frac{1}{2}}\cdot \text{diag}(\gamma_j^{\frac{1}{2}}) \cdot v_i$, where $\gamma_j$ is the $j$-th largest eigenvalue of $\bar{A}$, and $\text{diag}(\gamma_j^{\frac{1}{2}})$ is a diagonal matrix containing $\gamma_1^\frac{1}{2}, \gamma_2^{\frac{1}{2}}, \cdots, \gamma_k^{\frac{1}{2}}$ as its entries. Thus, if we let $b=\sqrt{\kappa}\cdot  \text{diag}(\gamma_j^{-\frac{1}{2}}) \cdot \hat{b}$, we would have 
	\begin{align}
\sum_{i\in[N]} (f(x_i)^\top b - y(x_i))^2 = \sum_{i\in[N]} (v_i^\top \hat{b} - g_i)^2 = \norm{U\hat{b}-g}_2^2 \lesssim \frac{N\alpha}{\rho^2}.
	\end{align}
\end{proof}

%	
%	\part{Online Learning}
%	\chapter{Online learning}\label{chap:OL}
%	\input{collection/10-01-online.tex}
%	\input{collection/10-02-online.tex}
%	
	
	\appendix
	%    Include appendix "chapters" here.
	
	
	\backmatter
	%    Bibliography styles amsplain or harvard are also acceptable.
	\bibliographystyle{plainnat}
	\bibliography{all,bibliography}
	%    See note above about multiple indexes.
	%\printindex
	
\end{document}

%-----------------------------------------------------------------------
% End of amsbook-template.tex
%-----------------------------------------------------------------------
